\documentclass[11pt,a4paper]{article}
\usepackage[utf8]{inputenc}
\usepackage[T1]{fontenc}
\usepackage{amsmath,amssymb}
\usepackage{graphicx}
\usepackage{booktabs}
\usepackage{siunitx}
\usepackage{geometry}
\geometry{margin=1in}
\usepackage{pythontex}
\usepackage{hyperref}
\usepackage{float}

\title{Air Pollution Dispersion\\Gaussian Plume Modeling}
\author{Environmental Engineering Department}
\date{\today}

\begin{document}
\maketitle

\begin{abstract}
Computational analysis of air pollution dispersion using Gaussian plume models and deposition calculations.
\end{abstract}


\section{Introduction}

Air pollution dispersion models predict pollutant concentrations downwind of emission sources.

\begin{pycode}
import numpy as np
import matplotlib.pyplot as plt
plt.rcParams['text.usetex'] = True
plt.rcParams['font.family'] = 'serif'
\end{pycode}

\section{Gaussian Plume Model}

$C(x,y,z) = \frac{Q}{2\pi u \sigma_y \sigma_z} \exp\left(-\frac{y^2}{2\sigma_y^2}\right) \left[\exp\left(-\frac{(z-H)^2}{2\sigma_z^2}\right) + \exp\left(-\frac{(z+H)^2}{2\sigma_z^2}\right)\right]$

\begin{pycode}
def gaussian_plume(x, y, z, Q, u, H, stability='D'):
    # Pasquill-Gifford dispersion parameters
    if stability == 'A':
        a_y, b_y, a_z, b_z = 0.22, 0.894, 0.20, 1.0
    elif stability == 'D':
        a_y, b_y, a_z, b_z = 0.08, 0.894, 0.06, 0.911
    elif stability == 'F':
        a_y, b_y, a_z, b_z = 0.04, 0.894, 0.016, 0.75

    sigma_y = a_y * x**b_y
    sigma_z = a_z * x**b_z

    C = Q / (2 * np.pi * u * sigma_y * sigma_z) * \
        np.exp(-y**2 / (2 * sigma_y**2)) * \
        (np.exp(-(z - H)**2 / (2 * sigma_z**2)) + np.exp(-(z + H)**2 / (2 * sigma_z**2)))
    return C

# Parameters
Q = 100  # g/s emission rate
u = 5    # m/s wind speed
H = 50   # m stack height

x = np.linspace(100, 5000, 200)
C_ground = gaussian_plume(x, 0, 0, Q, u, H)

fig, ax = plt.subplots(figsize=(10, 6))
ax.semilogy(x/1000, C_ground * 1e6, 'b-', linewidth=2)
ax.set_xlabel('Downwind Distance (km)')
ax.set_ylabel('Concentration ($\\mu$g/m$^3$)')
ax.set_title('Ground-Level Concentration')
ax.grid(True, alpha=0.3, which='both')
plt.tight_layout()
plt.savefig('plume_centerline.pdf', dpi=150, bbox_inches='tight')
plt.close()
\end{pycode}

\begin{figure}[H]
\centering
\includegraphics[width=0.85\textwidth]{plume_centerline.pdf}
\caption{Ground-level pollutant concentration along plume centerline.}
\end{figure}

\section{Stability Class Effects}

\begin{pycode}
fig, ax = plt.subplots(figsize=(10, 6))
for stab in ['A', 'D', 'F']:
    C = gaussian_plume(x, 0, 0, Q, u, H, stability=stab)
    ax.semilogy(x/1000, C * 1e6, linewidth=1.5, label=f'Class {stab}')
ax.set_xlabel('Downwind Distance (km)')
ax.set_ylabel('Concentration ($\\mu$g/m$^3$)')
ax.set_title('Effect of Atmospheric Stability')
ax.legend()
ax.grid(True, alpha=0.3, which='both')
plt.tight_layout()
plt.savefig('stability_effects.pdf', dpi=150, bbox_inches='tight')
plt.close()
\end{pycode}

\begin{figure}[H]
\centering
\includegraphics[width=0.85\textwidth]{stability_effects.pdf}
\caption{Concentration for different stability classes.}
\end{figure}

\section{2D Concentration Field}

\begin{pycode}
x_2d = np.linspace(100, 3000, 100)
y_2d = np.linspace(-500, 500, 100)
X, Y = np.meshgrid(x_2d, y_2d)
C_2d = gaussian_plume(X, Y, 0, Q, u, H)

fig, ax = plt.subplots(figsize=(12, 6))
cs = ax.contourf(X/1000, Y, C_2d * 1e6, levels=20, cmap='YlOrRd')
plt.colorbar(cs, label='Concentration ($\\mu$g/m$^3$)')
ax.set_xlabel('Downwind Distance (km)')
ax.set_ylabel('Crosswind Distance (m)')
ax.set_title('Ground-Level Concentration Field')
plt.tight_layout()
plt.savefig('concentration_field.pdf', dpi=150, bbox_inches='tight')
plt.close()
\end{pycode}

\begin{figure}[H]
\centering
\includegraphics[width=0.95\textwidth]{concentration_field.pdf}
\caption{2D ground-level concentration contours.}
\end{figure}

\section{Stack Height Effects}

\begin{pycode}
heights = [30, 50, 100, 150]

fig, ax = plt.subplots(figsize=(10, 6))
for H_s in heights:
    C = gaussian_plume(x, 0, 0, Q, u, H_s)
    ax.semilogy(x/1000, C * 1e6, linewidth=1.5, label=f'H = {H_s} m')
ax.set_xlabel('Downwind Distance (km)')
ax.set_ylabel('Concentration ($\\mu$g/m$^3$)')
ax.set_title('Effect of Stack Height')
ax.legend()
ax.grid(True, alpha=0.3, which='both')
plt.tight_layout()
plt.savefig('stack_height_effects.pdf', dpi=150, bbox_inches='tight')
plt.close()
\end{pycode}

\begin{figure}[H]
\centering
\includegraphics[width=0.85\textwidth]{stack_height_effects.pdf}
\caption{Ground concentration for different stack heights.}
\end{figure}

\section{Maximum Concentration}

\begin{pycode}
x_max = H * np.sqrt(2)  # Approximate location of max
C_max = gaussian_plume(x, 0, 0, Q, u, H)
idx_max = np.argmax(C_max)

fig, ax = plt.subplots(figsize=(10, 6))
ax.plot(x/1000, C_max * 1e6, 'b-', linewidth=2)
ax.plot(x[idx_max]/1000, C_max[idx_max] * 1e6, 'ro', markersize=10)
ax.annotate(f'Max: {C_max[idx_max]*1e6:.1f} $\\mu$g/m$^3$\nat {x[idx_max]/1000:.1f} km',
            xy=(x[idx_max]/1000, C_max[idx_max]*1e6), xytext=(x[idx_max]/1000 + 1, C_max[idx_max]*1e6 * 1.5),
            arrowprops=dict(arrowstyle='->'))
ax.set_xlabel('Downwind Distance (km)')
ax.set_ylabel('Concentration ($\\mu$g/m$^3$)')
ax.set_title('Maximum Ground-Level Concentration')
ax.grid(True, alpha=0.3)
plt.tight_layout()
plt.savefig('maximum_concentration.pdf', dpi=150, bbox_inches='tight')
plt.close()
\end{pycode}

\begin{figure}[H]
\centering
\includegraphics[width=0.85\textwidth]{maximum_concentration.pdf}
\caption{Location and magnitude of maximum concentration.}
\end{figure}

\section{Vertical Profile}

\begin{pycode}
z_profile = np.linspace(0, 200, 100)
x_loc = 1000  # m

C_vertical = gaussian_plume(x_loc, 0, z_profile, Q, u, H)

fig, ax = plt.subplots(figsize=(8, 8))
ax.plot(C_vertical * 1e6, z_profile, 'b-', linewidth=2)
ax.axhline(y=H, color='r', linestyle='--', label=f'Stack height = {H} m')
ax.set_xlabel('Concentration ($\\mu$g/m$^3$)')
ax.set_ylabel('Height (m)')
ax.set_title(f'Vertical Profile at x = {x_loc} m')
ax.legend()
ax.grid(True, alpha=0.3)
plt.tight_layout()
plt.savefig('vertical_profile.pdf', dpi=150, bbox_inches='tight')
plt.close()
\end{pycode}

\begin{figure}[H]
\centering
\includegraphics[width=0.7\textwidth]{vertical_profile.pdf}
\caption{Vertical concentration profile.}
\end{figure}

\section{Results}

\begin{pycode}
x_max_dist = x[idx_max]
C_max_val = C_max[idx_max] * 1e6

print(r'\begin{table}[H]')
print(r'\centering')
print(r'\caption{Dispersion Model Results}')
print(r'\begin{tabular}{@{}lc@{}}')
print(r'\toprule')
print(r'Parameter & Value \\')
print(r'\midrule')
print(f'Emission rate & {Q} g/s \\\\')
print(f'Wind speed & {u} m/s \\\\')
print(f'Stack height & {H} m \\\\')
print(f'Max concentration & {C_max_val:.1f} $\\mu$g/m$^3$ \\\\')
print(f'Distance to max & {x_max_dist:.0f} m \\\\')
print(r'\bottomrule')
print(r'\end{tabular}')
print(r'\end{table}')
\end{pycode}

\section{Conclusions}

Gaussian plume models provide practical estimates of pollutant dispersion for regulatory applications.


\end{document}
