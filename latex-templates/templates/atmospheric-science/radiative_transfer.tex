\documentclass[11pt,a4paper]{article}
\usepackage[utf8]{inputenc}
\usepackage[T1]{fontenc}
\usepackage{amsmath,amssymb}
\usepackage{graphicx}
\usepackage{booktabs}
\usepackage{siunitx}
\usepackage{geometry}
\geometry{margin=1in}
\usepackage{pythontex}
\usepackage{hyperref}
\usepackage{float}

\title{Radiative Transfer\\Beer-Lambert Law and Greenhouse Effect}
\author{Climate Science Division}
\date{\today}

\begin{document}
\maketitle

\begin{abstract}
Analysis of radiative transfer in the atmosphere including absorption, scattering, and the greenhouse effect.
\end{abstract}


\section{Introduction}

Radiative transfer describes how electromagnetic radiation propagates through the atmosphere.

\begin{pycode}
import numpy as np
import matplotlib.pyplot as plt
plt.rcParams['text.usetex'] = True
plt.rcParams['font.family'] = 'serif'

sigma_sb = 5.67e-8  # Stefan-Boltzmann constant
\end{pycode}

\section{Beer-Lambert Law}

$I(z) = I_0 e^{-\tau}$

\begin{pycode}
tau = np.linspace(0, 5, 100)
I_I0 = np.exp(-tau)

fig, ax = plt.subplots(figsize=(10, 6))
ax.plot(tau, I_I0, 'b-', linewidth=2)
ax.set_xlabel('Optical Depth $\\tau$')
ax.set_ylabel('$I/I_0$')
ax.set_title('Beer-Lambert Transmission')
ax.grid(True, alpha=0.3)
ax.set_ylim([0, 1])
plt.tight_layout()
plt.savefig('beer_lambert.pdf', dpi=150, bbox_inches='tight')
plt.close()
\end{pycode}

\begin{figure}[H]
\centering
\includegraphics[width=0.85\textwidth]{beer_lambert.pdf}
\caption{Transmission as function of optical depth.}
\end{figure}

\section{Planck Function}

\begin{pycode}
h = 6.626e-34
c = 3e8
k = 1.381e-23

wavelength = np.linspace(0.1, 50, 500) * 1e-6  # meters

def planck(lam, T):
    return 2*h*c**2/lam**5 / (np.exp(h*c/(lam*k*T)) - 1)

fig, ax = plt.subplots(figsize=(10, 6))
for T in [5800, 300, 255]:
    B = planck(wavelength, T)
    if T == 5800:
        B = B / 1e7  # Scale for visibility
    ax.semilogy(wavelength*1e6, B, label=f'T = {T} K', linewidth=1.5)
ax.set_xlabel('Wavelength ($\\mu$m)')
ax.set_ylabel('Spectral Radiance')
ax.set_title('Planck Function')
ax.legend()
ax.grid(True, alpha=0.3, which='both')
ax.set_xlim([0, 50])
plt.tight_layout()
plt.savefig('planck_function.pdf', dpi=150, bbox_inches='tight')
plt.close()
\end{pycode}

\begin{figure}[H]
\centering
\includegraphics[width=0.85\textwidth]{planck_function.pdf}
\caption{Planck blackbody spectra at different temperatures.}
\end{figure}

\section{Atmospheric Absorption}

\begin{pycode}
lam = np.linspace(0.3, 30, 1000)

# Simplified absorption spectrum
transmission = np.ones_like(lam)
# O3 absorption (UV)
transmission *= 1 - 0.99 * np.exp(-(lam - 0.25)**2/0.01)
# H2O absorption
for center in [1.4, 1.9, 2.7, 6.3]:
    transmission *= 1 - 0.8 * np.exp(-(lam - center)**2/0.1)
# CO2 absorption
for center in [4.3, 15]:
    transmission *= 1 - 0.9 * np.exp(-(lam - center)**2/0.5)

fig, ax = plt.subplots(figsize=(12, 5))
ax.fill_between(lam, 0, transmission, alpha=0.5, color='blue')
ax.plot(lam, transmission, 'b-', linewidth=1)
ax.set_xlabel('Wavelength ($\\mu$m)')
ax.set_ylabel('Transmission')
ax.set_title('Atmospheric Transmission Spectrum')
ax.grid(True, alpha=0.3)
ax.set_xlim([0, 30])
plt.tight_layout()
plt.savefig('atmospheric_transmission.pdf', dpi=150, bbox_inches='tight')
plt.close()
\end{pycode}

\begin{figure}[H]
\centering
\includegraphics[width=0.95\textwidth]{atmospheric_transmission.pdf}
\caption{Atmospheric transmission showing absorption bands.}
\end{figure}

\section{Greenhouse Effect}

\begin{pycode}
# Simple greenhouse model
S = 1361  # Solar constant W/m^2
albedo = 0.3
epsilon = np.linspace(0, 1, 100)  # Atmospheric emissivity

T_surface = ((S * (1 - albedo) / 4) / (sigma_sb * (1 - epsilon/2)))**0.25

fig, ax = plt.subplots(figsize=(10, 6))
ax.plot(epsilon, T_surface - 273.15, 'b-', linewidth=2)
ax.axhline(y=15, color='r', linestyle='--', label='Current Earth')
ax.set_xlabel('Atmospheric Emissivity $\\epsilon$')
ax.set_ylabel('Surface Temperature ($^\\circ$C)')
ax.set_title('Greenhouse Effect')
ax.legend()
ax.grid(True, alpha=0.3)
plt.tight_layout()
plt.savefig('greenhouse_effect.pdf', dpi=150, bbox_inches='tight')
plt.close()
\end{pycode}

\begin{figure}[H]
\centering
\includegraphics[width=0.85\textwidth]{greenhouse_effect.pdf}
\caption{Surface temperature vs atmospheric emissivity.}
\end{figure}

\section{Radiative Forcing}

\begin{pycode}
CO2 = np.linspace(280, 800, 100)  # ppm
RF = 5.35 * np.log(CO2 / 280)  # W/m^2

fig, ax = plt.subplots(figsize=(10, 6))
ax.plot(CO2, RF, 'b-', linewidth=2)
ax.axvline(x=420, color='r', linestyle='--', label='Current (2023)')
ax.set_xlabel('CO$_2$ Concentration (ppm)')
ax.set_ylabel('Radiative Forcing (W/m$^2$)')
ax.set_title('CO$_2$ Radiative Forcing')
ax.legend()
ax.grid(True, alpha=0.3)
plt.tight_layout()
plt.savefig('radiative_forcing.pdf', dpi=150, bbox_inches='tight')
plt.close()
\end{pycode}

\begin{figure}[H]
\centering
\includegraphics[width=0.85\textwidth]{radiative_forcing.pdf}
\caption{Radiative forcing from CO$_2$ increase.}
\end{figure}

\section{Scattering}

\begin{pycode}
# Rayleigh scattering
lam_scat = np.linspace(0.3, 0.8, 100)
sigma_rayleigh = 1 / lam_scat**4
sigma_rayleigh = sigma_rayleigh / sigma_rayleigh[0]

# Mie scattering (simplified)
sigma_mie = np.ones_like(lam_scat) * 0.3

fig, ax = plt.subplots(figsize=(10, 6))
ax.plot(lam_scat, sigma_rayleigh, 'b-', linewidth=2, label='Rayleigh')
ax.plot(lam_scat, sigma_mie, 'r-', linewidth=2, label='Mie')
ax.set_xlabel('Wavelength ($\\mu$m)')
ax.set_ylabel('Relative Scattering Cross-section')
ax.set_title('Atmospheric Scattering')
ax.legend()
ax.grid(True, alpha=0.3)
plt.tight_layout()
plt.savefig('scattering.pdf', dpi=150, bbox_inches='tight')
plt.close()
\end{pycode}

\begin{figure}[H]
\centering
\includegraphics[width=0.85\textwidth]{scattering.pdf}
\caption{Wavelength dependence of atmospheric scattering.}
\end{figure}

\section{Results}

\begin{pycode}
T_no_atm = (S * (1 - albedo) / (4 * sigma_sb))**0.25
T_with_atm = 288  # K
greenhouse_warming = T_with_atm - T_no_atm

print(r'\begin{table}[H]')
print(r'\centering')
print(r'\caption{Earth Energy Balance}')
print(r'\begin{tabular}{@{}lc@{}}')
print(r'\toprule')
print(r'Parameter & Value \\')
print(r'\midrule')
print(f'Solar constant & {S} W/m$^2$ \\\\')
print(f'Planetary albedo & {albedo} \\\\')
print(f'Effective temperature & {T_no_atm:.0f} K \\\\')
print(f'Greenhouse warming & {greenhouse_warming:.0f} K \\\\')
print(r'\bottomrule')
print(r'\end{tabular}')
print(r'\end{table}')
\end{pycode}

\section{Conclusions}

Radiative transfer processes control Earth's energy balance and climate.


\end{document}
