\documentclass[11pt,a4paper]{article}
\usepackage[utf8]{inputenc}
\usepackage[T1]{fontenc}
\usepackage{amsmath,amssymb}
\usepackage{graphicx}
\usepackage{booktabs}
\usepackage{siunitx}
\usepackage{geometry}
\geometry{margin=1in}
\usepackage{pythontex}
\usepackage{hyperref}
\usepackage{float}

\title{Atmospheric Dynamics\\Geostrophic Wind and Thermal Balance}
\author{Department of Atmospheric Sciences}
\date{\today}

\begin{document}
\maketitle

\begin{abstract}
Analysis of atmospheric dynamics including geostrophic balance, thermal wind, and jet stream formation.
\end{abstract}


\section{Introduction}

Atmospheric dynamics governs weather and climate through pressure, temperature, and wind relationships.

\begin{pycode}
import numpy as np
import matplotlib.pyplot as plt
plt.rcParams['text.usetex'] = True
plt.rcParams['font.family'] = 'serif'

# Constants
Omega = 7.292e-5  # Earth rotation rate
R = 287  # Gas constant for air
g = 9.81  # Gravity
\end{pycode}

\section{Geostrophic Wind}

$u_g = -\frac{1}{f\rho}\frac{\partial p}{\partial y}$, $v_g = \frac{1}{f\rho}\frac{\partial p}{\partial x}$

\begin{pycode}
lat = np.linspace(10, 80, 100)
f = 2 * Omega * np.sin(np.radians(lat))

# Pressure gradient (typical mid-latitude)
dp_dy = 1e-3  # Pa/m
rho = 1.2  # kg/m^3

u_g = -dp_dy / (f * rho)

fig, ax = plt.subplots(figsize=(10, 6))
ax.plot(lat, u_g, 'b-', linewidth=2)
ax.set_xlabel('Latitude (degrees)')
ax.set_ylabel('Geostrophic Wind Speed (m/s)')
ax.set_title('Geostrophic Wind vs Latitude')
ax.grid(True, alpha=0.3)
plt.tight_layout()
plt.savefig('geostrophic_wind.pdf', dpi=150, bbox_inches='tight')
plt.close()
\end{pycode}

\begin{figure}[H]
\centering
\includegraphics[width=0.85\textwidth]{geostrophic_wind.pdf}
\caption{Geostrophic wind speed dependence on latitude.}
\end{figure}

\section{Thermal Wind}

\begin{pycode}
p_levels = np.array([1000, 850, 700, 500, 300, 200])
T_profile = np.array([288, 278, 268, 253, 228, 218])

# Temperature gradient
dT_dy = -2e-6  # K/m (typical)
phi = 45  # Latitude
f_45 = 2 * Omega * np.sin(np.radians(phi))

# Thermal wind shear
du_dz = -(g / (f_45 * T_profile)) * dT_dy
z = np.array([0, 1.5, 3, 5.5, 9, 12])  # km

fig, ax = plt.subplots(figsize=(10, 6))
ax.plot(du_dz * 1000, z, 'b-o', linewidth=1.5, markersize=8)
ax.set_xlabel('Wind Shear (m/s per km)')
ax.set_ylabel('Altitude (km)')
ax.set_title('Thermal Wind Shear Profile')
ax.grid(True, alpha=0.3)
plt.tight_layout()
plt.savefig('thermal_wind.pdf', dpi=150, bbox_inches='tight')
plt.close()
\end{pycode}

\begin{figure}[H]
\centering
\includegraphics[width=0.85\textwidth]{thermal_wind.pdf}
\caption{Thermal wind shear as function of altitude.}
\end{figure}

\section{Rossby Number}

$Ro = \frac{U}{fL}$

\begin{pycode}
U = 10  # Typical wind speed m/s
L = np.logspace(3, 7, 100)  # Length scales

Ro_30 = U / (2 * Omega * np.sin(np.radians(30)) * L)
Ro_60 = U / (2 * Omega * np.sin(np.radians(60)) * L)

fig, ax = plt.subplots(figsize=(10, 6))
ax.loglog(L/1000, Ro_30, label='30$^\\circ$N', linewidth=1.5)
ax.loglog(L/1000, Ro_60, label='60$^\\circ$N', linewidth=1.5)
ax.axhline(y=1, color='r', linestyle='--', label='Ro = 1')
ax.set_xlabel('Length Scale (km)')
ax.set_ylabel('Rossby Number')
ax.set_title('Rossby Number vs Length Scale')
ax.legend()
ax.grid(True, alpha=0.3, which='both')
plt.tight_layout()
plt.savefig('rossby_number.pdf', dpi=150, bbox_inches='tight')
plt.close()
\end{pycode}

\begin{figure}[H]
\centering
\includegraphics[width=0.85\textwidth]{rossby_number.pdf}
\caption{Rossby number for different latitudes and scales.}
\end{figure}

\section{Jet Stream}

\begin{pycode}
lat_jet = np.linspace(20, 70, 100)
z_jet = np.linspace(0, 15, 50)
LAT, Z = np.meshgrid(lat_jet, z_jet)

# Simplified jet stream model
U_jet = 40 * np.exp(-((LAT - 45)**2/100)) * np.exp(-((Z - 10)**2/10))

fig, ax = plt.subplots(figsize=(12, 6))
cs = ax.contourf(LAT, Z, U_jet, levels=20, cmap='jet')
plt.colorbar(cs, label='Wind Speed (m/s)')
ax.set_xlabel('Latitude (degrees)')
ax.set_ylabel('Altitude (km)')
ax.set_title('Jet Stream Wind Speed')
plt.tight_layout()
plt.savefig('jet_stream.pdf', dpi=150, bbox_inches='tight')
plt.close()
\end{pycode}

\begin{figure}[H]
\centering
\includegraphics[width=0.9\textwidth]{jet_stream.pdf}
\caption{Cross-section of jet stream wind speed.}
\end{figure}

\section{Potential Vorticity}

\begin{pycode}
theta = np.linspace(280, 350, 100)  # Potential temperature
f_pv = 1e-4
dtheta_dp = -0.1  # K/Pa

PV = -g * f_pv * dtheta_dp * np.ones_like(theta)

fig, ax = plt.subplots(figsize=(10, 6))
ax.plot(theta, PV * 1e6, 'b-', linewidth=2)
ax.set_xlabel('Potential Temperature (K)')
ax.set_ylabel('PV (PVU)')
ax.set_title('Potential Vorticity')
ax.grid(True, alpha=0.3)
plt.tight_layout()
plt.savefig('potential_vorticity.pdf', dpi=150, bbox_inches='tight')
plt.close()
\end{pycode}

\begin{figure}[H]
\centering
\includegraphics[width=0.85\textwidth]{potential_vorticity.pdf}
\caption{Potential vorticity on isentropic surfaces.}
\end{figure}

\section{Ekman Spiral}

\begin{pycode}
z_ek = np.linspace(0, 2000, 100)
K = 10  # Eddy diffusivity
f_ek = 1e-4
delta = np.sqrt(2 * K / f_ek)

u_ek = 10 * (1 - np.exp(-z_ek/delta) * np.cos(z_ek/delta))
v_ek = 10 * np.exp(-z_ek/delta) * np.sin(z_ek/delta)

fig, (ax1, ax2) = plt.subplots(1, 2, figsize=(12, 5))
ax1.plot(u_ek, z_ek, 'b-', linewidth=2, label='u')
ax1.plot(v_ek, z_ek, 'r-', linewidth=2, label='v')
ax1.set_xlabel('Wind Speed (m/s)')
ax1.set_ylabel('Height (m)')
ax1.set_title('Ekman Layer Wind Profile')
ax1.legend()
ax1.grid(True, alpha=0.3)

ax2.plot(u_ek, v_ek, 'b-', linewidth=1.5)
ax2.plot(u_ek[0], v_ek[0], 'go', markersize=10)
ax2.set_xlabel('u (m/s)')
ax2.set_ylabel('v (m/s)')
ax2.set_title('Ekman Spiral')
ax2.grid(True, alpha=0.3)
ax2.axis('equal')
plt.tight_layout()
plt.savefig('ekman_spiral.pdf', dpi=150, bbox_inches='tight')
plt.close()
\end{pycode}

\begin{figure}[H]
\centering
\includegraphics[width=0.95\textwidth]{ekman_spiral.pdf}
\caption{Ekman layer wind profile and spiral.}
\end{figure}

\section{Results}

\begin{pycode}
print(r'\begin{table}[H]')
print(r'\centering')
print(r'\caption{Atmospheric Parameters at 45$^\circ$N}')
print(r'\begin{tabular}{@{}lc@{}}')
print(r'\toprule')
print(r'Parameter & Value \\')
print(r'\midrule')
print(f'Coriolis parameter & {f_45:.2e} s$^{{-1}}$ \\\\')
print(f'Ekman depth & {delta:.0f} m \\\\')
print(f'Rossby deformation radius & $\\sim$1000 km \\\\')
print(r'\bottomrule')
print(r'\end{tabular}')
print(r'\end{table}')
\end{pycode}

\section{Conclusions}

Atmospheric dynamics is governed by the balance between pressure gradient, Coriolis, and frictional forces.


\end{document}
