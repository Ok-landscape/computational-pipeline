\documentclass[11pt,a4paper]{article}

% Document Setup
\usepackage[utf8]{inputenc}
\usepackage[T1]{fontenc}
\usepackage{lmodern}
\usepackage[margin=1in]{geometry}
\usepackage{amsmath,amssymb}
\usepackage{siunitx}
\usepackage{booktabs}
\usepackage{float}
\usepackage{caption}
\usepackage{hyperref}

% PythonTeX Setup
\usepackage[makestderr]{pythontex}

\title{Pipe Flow Analysis: Darcy-Weisbach and Friction Factors}
\author{Mechanical Engineering Laboratory}
\date{\today}

\begin{document}
\maketitle

\begin{abstract}
This report presents computational analysis of pipe flow using the Darcy-Weisbach equation. We examine Reynolds number regimes, friction factor correlations including the Colebrook-White equation, the Moody diagram, minor losses, and pipe network analysis. Python-based computations provide quantitative analysis with dynamic visualization.
\end{abstract}

\tableofcontents
\newpage

\section{Introduction to Pipe Flow}

Internal flow through pipes is fundamental to hydraulic system design. Key applications include:
\begin{itemize}
    \item Water distribution networks
    \item Oil and gas pipelines
    \item HVAC systems
    \item Process piping in chemical plants
\end{itemize}

% Initialize Python environment
\begin{pycode}
import numpy as np
import matplotlib.pyplot as plt
from scipy.optimize import fsolve

plt.rcParams['figure.figsize'] = (8, 5)
plt.rcParams['font.size'] = 10
plt.rcParams['text.usetex'] = True

def save_fig(filename):
    plt.savefig(filename, dpi=150, bbox_inches='tight')
    plt.close()
\end{pycode}

\section{Fundamental Equations}

\subsection{Reynolds Number}

The Reynolds number characterizes flow regime:
\begin{equation}
Re = \frac{\rho V D}{\mu} = \frac{V D}{\nu}
\end{equation}
where $\rho$ is density, $V$ is velocity, $D$ is diameter, and $\mu$ is dynamic viscosity.

\begin{pycode}
# Flow regime visualization
Re_range = np.logspace(2, 6, 500)

fig, ax = plt.subplots(figsize=(10, 5))

# Laminar regime
ax.axvspan(100, 2300, alpha=0.3, color='blue', label='Laminar')
# Transition regime
ax.axvspan(2300, 4000, alpha=0.3, color='yellow', label='Transition')
# Turbulent regime
ax.axvspan(4000, 1e6, alpha=0.3, color='red', label='Turbulent')

ax.axvline(2300, color='black', linestyle='--', linewidth=2)
ax.axvline(4000, color='black', linestyle='--', linewidth=2)

ax.set_xscale('log')
ax.set_xlabel('Reynolds Number')
ax.set_ylabel('Flow Characteristic')
ax.set_title('Flow Regime Classification')
ax.legend(loc='upper left')
ax.set_xlim(100, 1e6)
ax.set_yticks([])

# Add annotations
ax.text(500, 0.5, 'Laminar\\n$Re < 2300$', ha='center', fontsize=12, transform=ax.get_xaxis_transform())
ax.text(3100, 0.5, 'Trans.', ha='center', fontsize=10, transform=ax.get_xaxis_transform())
ax.text(50000, 0.5, 'Turbulent\\n$Re > 4000$', ha='center', fontsize=12, transform=ax.get_xaxis_transform())

save_fig('flow_regimes.pdf')
\end{pycode}

\begin{figure}[H]
\centering
\includegraphics[width=\textwidth]{flow_regimes.pdf}
\caption{Flow regime classification based on Reynolds number.}
\end{figure}

\subsection{Darcy-Weisbach Equation}

Head loss in pipe flow:
\begin{equation}
h_f = f \frac{L}{D} \frac{V^2}{2g}
\end{equation}
or in terms of pressure drop:
\begin{equation}
\Delta P = f \frac{L}{D} \frac{\rho V^2}{2}
\end{equation}

\section{Friction Factor Correlations}

\subsection{Laminar Flow}

For laminar flow ($Re < 2300$):
\begin{equation}
f = \frac{64}{Re}
\end{equation}

\subsection{Turbulent Flow - Colebrook-White Equation}

For turbulent flow:
\begin{equation}
\frac{1}{\sqrt{f}} = -2\log_{10}\left(\frac{\epsilon/D}{3.7} + \frac{2.51}{Re\sqrt{f}}\right)
\end{equation}

\begin{pycode}
# Friction factor calculation functions
def f_laminar(Re):
    return 64 / Re

def f_turbulent(Re, eps_D):
    """Solve Colebrook-White equation iteratively"""
    if Re < 2300:
        return 64 / Re

    # Swamee-Jain initial guess
    f0 = 0.25 / (np.log10(eps_D/3.7 + 5.74/Re**0.9))**2

    # Newton-Raphson iteration
    for _ in range(50):
        sqrt_f = np.sqrt(f0)
        LHS = 1/sqrt_f
        RHS = -2 * np.log10(eps_D/3.7 + 2.51/(Re*sqrt_f))
        f_new = 1/(RHS)**2
        if abs(f_new - f0) < 1e-10:
            break
        f0 = f_new

    return f0

# Calculate friction factors for various Reynolds numbers and roughness
Re_range = np.logspace(3, 8, 200)
eps_D_values = [0, 1e-6, 1e-5, 1e-4, 1e-3, 0.01, 0.05]

fig, ax = plt.subplots(figsize=(12, 8))

# Laminar line
Re_lam = np.linspace(600, 2300, 100)
f_lam = [f_laminar(Re) for Re in Re_lam]
ax.loglog(Re_lam, f_lam, 'b-', linewidth=2, label='Laminar')

# Turbulent lines
colors = plt.cm.viridis(np.linspace(0, 1, len(eps_D_values)))
for eps_D, color in zip(eps_D_values, colors):
    Re_turb = Re_range[Re_range > 2300]
    f_turb = [f_turbulent(Re, eps_D) for Re in Re_turb]
    if eps_D == 0:
        label = 'Smooth'
    else:
        label = f'$\\epsilon/D$ = {eps_D}'
    ax.loglog(Re_turb, f_turb, color=color, linewidth=1.5, label=label)

ax.axvline(2300, color='gray', linestyle='--', alpha=0.5)
ax.set_xlabel('Reynolds Number, Re')
ax.set_ylabel('Friction Factor, f')
ax.set_title('Moody Diagram')
ax.legend(loc='upper right', fontsize=8)
ax.grid(True, which='both', alpha=0.3)
ax.set_xlim(600, 1e8)
ax.set_ylim(0.008, 0.1)

save_fig('moody_diagram.pdf')
\end{pycode}

\begin{figure}[H]
\centering
\includegraphics[width=\textwidth]{moody_diagram.pdf}
\caption{Moody diagram showing friction factor vs Reynolds number for various relative roughness values.}
\end{figure}

\section{Pipe Flow Analysis}

\begin{pycode}
# Design problem: water flow in commercial steel pipe
# Properties
rho = 998  # kg/m^3 (water)
mu = 0.001  # Pa*s
nu = mu / rho

# Pipe parameters
D = 0.1  # m (100 mm diameter)
L = 100  # m
epsilon = 0.045e-3  # m (commercial steel)
eps_D = epsilon / D

# Flow rates
Q_range = np.linspace(0.001, 0.05, 50)  # m^3/s
V_range = Q_range / (np.pi * D**2 / 4)
Re_range = V_range * D / nu

# Calculate friction factors and pressure drops
f_range = [f_turbulent(Re, eps_D) for Re in Re_range]
dP_range = np.array(f_range) * L/D * rho * V_range**2 / 2  # Pa
h_f_range = dP_range / (rho * 9.81)  # m

fig, axes = plt.subplots(2, 2, figsize=(12, 10))

# Velocity vs flow rate
axes[0, 0].plot(Q_range*1000, V_range, 'b-', linewidth=2)
axes[0, 0].set_xlabel('Flow Rate (L/s)')
axes[0, 0].set_ylabel('Velocity (m/s)')
axes[0, 0].set_title('Flow Velocity')
axes[0, 0].grid(True, alpha=0.3)

# Reynolds number
axes[0, 1].plot(Q_range*1000, Re_range/1000, 'r-', linewidth=2)
axes[0, 1].axhline(2.3, color='k', linestyle='--', alpha=0.5, label='Re = 2300')
axes[0, 1].set_xlabel('Flow Rate (L/s)')
axes[0, 1].set_ylabel('Reynolds Number ($\\times 10^3$)')
axes[0, 1].set_title('Reynolds Number')
axes[0, 1].legend()
axes[0, 1].grid(True, alpha=0.3)

# Pressure drop
axes[1, 0].plot(Q_range*1000, dP_range/1000, 'g-', linewidth=2)
axes[1, 0].set_xlabel('Flow Rate (L/s)')
axes[1, 0].set_ylabel('Pressure Drop (kPa)')
axes[1, 0].set_title('Pressure Drop')
axes[1, 0].grid(True, alpha=0.3)

# System curve (head loss vs flow rate)
axes[1, 1].plot(Q_range*1000, h_f_range, 'm-', linewidth=2, label='System curve')
axes[1, 1].set_xlabel('Flow Rate (L/s)')
axes[1, 1].set_ylabel('Head Loss (m)')
axes[1, 1].set_title('System Curve')
axes[1, 1].grid(True, alpha=0.3)

plt.tight_layout()
save_fig('pipe_flow_analysis.pdf')

# Design point calculations
Q_design = 0.02  # m^3/s (20 L/s)
V_design = Q_design / (np.pi * D**2 / 4)
Re_design = V_design * D / nu
f_design = f_turbulent(Re_design, eps_D)
dP_design = f_design * L/D * rho * V_design**2 / 2
h_f_design = dP_design / (rho * 9.81)
\end{pycode}

\begin{figure}[H]
\centering
\includegraphics[width=\textwidth]{pipe_flow_analysis.pdf}
\caption{Pipe flow analysis showing velocity, Reynolds number, pressure drop, and system curve.}
\end{figure}

\subsection{Design Point Results}

For a design flow rate of $Q = 20$ L/s:

\begin{table}[H]
\centering
\caption{Pipe Flow Design Calculations}
\begin{tabular}{lcc}
\toprule
Parameter & Value & Units \\
\midrule
Velocity & \py{f"{V_design:.2f}"} & m/s \\
Reynolds number & \py{f"{Re_design:.0f}"} & -- \\
Friction factor & \py{f"{f_design:.4f}"} & -- \\
Pressure drop & \py{f"{dP_design/1000:.2f}"} & kPa \\
Head loss & \py{f"{h_f_design:.2f}"} & m \\
\bottomrule
\end{tabular}
\end{table}

\section{Minor Losses}

Minor (local) losses from fittings and valves:
\begin{equation}
h_m = K \frac{V^2}{2g}
\end{equation}

\begin{pycode}
# Minor loss coefficients
fittings = {
    '90 elbow (regular)': 0.9,
    '90 elbow (long radius)': 0.6,
    '45 elbow': 0.4,
    'Tee (branch)': 1.8,
    'Gate valve (full open)': 0.2,
    'Globe valve (full open)': 10.0,
    'Check valve': 2.5,
    'Entrance (sharp)': 0.5,
    'Exit': 1.0
}

fig, (ax1, ax2) = plt.subplots(1, 2, figsize=(14, 5))

# Bar chart of K values
names = list(fittings.keys())
K_vals = list(fittings.values())
y_pos = np.arange(len(names))

ax1.barh(y_pos, K_vals, color='steelblue', alpha=0.7)
ax1.set_yticks(y_pos)
ax1.set_yticklabels(names)
ax1.set_xlabel('Loss Coefficient K')
ax1.set_title('Minor Loss Coefficients')
ax1.grid(True, alpha=0.3, axis='x')

# Total system losses with fittings
# Example system: entrance + 4 elbows + gate valve + exit
system_fittings = [
    ('Entrance', 0.5),
    ('4x 90 elbows', 4*0.9),
    ('Gate valve', 0.2),
    ('Exit', 1.0)
]

V = V_design
g = 9.81
minor_losses = []
labels = []
for name, K in system_fittings:
    h_m = K * V**2 / (2*g)
    minor_losses.append(h_m)
    labels.append(name)

# Add major loss
major_loss = h_f_design
labels.append('Major loss')
minor_losses.append(major_loss)

colors = ['lightblue']*4 + ['salmon']
ax2.bar(labels, minor_losses, color=colors, alpha=0.7)
ax2.set_ylabel('Head Loss (m)')
ax2.set_title('System Head Loss Breakdown')
ax2.grid(True, alpha=0.3, axis='y')

total_loss = sum(minor_losses)
ax2.axhline(total_loss/len(minor_losses), color='red', linestyle='--',
            label=f'Total = {total_loss:.2f} m')
ax2.legend()

plt.tight_layout()
save_fig('minor_losses.pdf')

K_total_minor = sum([k for _, k in system_fittings])
h_minor_total = K_total_minor * V_design**2 / (2*g)
\end{pycode}

\begin{figure}[H]
\centering
\includegraphics[width=\textwidth]{minor_losses.pdf}
\caption{Minor loss coefficients and system head loss breakdown.}
\end{figure}

Total minor losses: $h_m = \py{f"{h_minor_total:.2f}"}$ m

\section{Pipe Diameter Effect}

\begin{pycode}
# Effect of pipe diameter on pressure drop
diameters = np.linspace(0.05, 0.3, 100)  # m
Q_fixed = 0.02  # m^3/s

dP_diam = []
V_diam = []

for D in diameters:
    A = np.pi * D**2 / 4
    V = Q_fixed / A
    Re = V * D / nu
    eps_D = epsilon / D
    f = f_turbulent(Re, eps_D)
    dP = f * L/D * rho * V**2 / 2
    dP_diam.append(dP)
    V_diam.append(V)

fig, (ax1, ax2) = plt.subplots(1, 2, figsize=(12, 5))

ax1.semilogy(diameters*1000, np.array(dP_diam)/1000, 'b-', linewidth=2)
ax1.axvline(100, color='r', linestyle='--', label='D = 100 mm')
ax1.set_xlabel('Pipe Diameter (mm)')
ax1.set_ylabel('Pressure Drop (kPa)')
ax1.set_title('Pressure Drop vs Diameter (Q = 20 L/s)')
ax1.legend()
ax1.grid(True, alpha=0.3)

ax2.plot(diameters*1000, V_diam, 'g-', linewidth=2)
ax2.axhline(3, color='r', linestyle='--', alpha=0.5, label='Typical max V')
ax2.set_xlabel('Pipe Diameter (mm)')
ax2.set_ylabel('Flow Velocity (m/s)')
ax2.set_title('Velocity vs Diameter')
ax2.legend()
ax2.grid(True, alpha=0.3)

plt.tight_layout()
save_fig('diameter_effect.pdf')
\end{pycode}

\begin{figure}[H]
\centering
\includegraphics[width=\textwidth]{diameter_effect.pdf}
\caption{Effect of pipe diameter on pressure drop and velocity for constant flow rate.}
\end{figure}

\section{Pipe Network Analysis}

\subsection{Pipes in Series}

For pipes in series, flow rate is constant and head losses add:
\begin{equation}
h_{f,total} = \sum_{i=1}^{n} h_{f,i}
\end{equation}

\subsection{Pipes in Parallel}

For pipes in parallel, head loss is equal and flow rates add:
\begin{equation}
Q_{total} = \sum_{i=1}^{n} Q_i \quad \text{with} \quad h_{f,1} = h_{f,2} = \cdots
\end{equation}

\begin{pycode}
# Parallel pipe analysis
# Two pipes with different diameters
D1 = 0.1  # m
D2 = 0.08  # m
L_both = 100  # m

# Find flow distribution for given total head loss
h_f_target = 5  # m target head loss

def flow_rate_from_head(h_f, D, L, eps):
    """Calculate flow rate given head loss using iteration"""
    eps_D = eps / D
    A = np.pi * D**2 / 4

    # Initial guess using Hazen-Williams approximation
    V = np.sqrt(2 * 9.81 * h_f * D / (0.02 * L))

    for _ in range(50):
        Re = V * D / nu
        f = f_turbulent(Re, eps_D)
        V_new = np.sqrt(2 * 9.81 * h_f * D / (f * L))
        if abs(V_new - V) < 1e-6:
            break
        V = V_new

    return V * A

h_range = np.linspace(1, 10, 50)
Q1_range = [flow_rate_from_head(h, D1, L_both, epsilon) for h in h_range]
Q2_range = [flow_rate_from_head(h, D2, L_both, epsilon) for h in h_range]
Q_total = [q1 + q2 for q1, q2 in zip(Q1_range, Q2_range)]

fig, (ax1, ax2) = plt.subplots(1, 2, figsize=(12, 5))

ax1.plot(np.array(Q1_range)*1000, h_range, 'b-', linewidth=2, label=f'Pipe 1 (D={D1*1000:.0f}mm)')
ax1.plot(np.array(Q2_range)*1000, h_range, 'r-', linewidth=2, label=f'Pipe 2 (D={D2*1000:.0f}mm)')
ax1.plot(np.array(Q_total)*1000, h_range, 'k--', linewidth=2, label='Total')
ax1.set_xlabel('Flow Rate (L/s)')
ax1.set_ylabel('Head Loss (m)')
ax1.set_title('Parallel Pipe System Curves')
ax1.legend()
ax1.grid(True, alpha=0.3)

# Flow distribution at target head loss
Q1_target = flow_rate_from_head(h_f_target, D1, L_both, epsilon)
Q2_target = flow_rate_from_head(h_f_target, D2, L_both, epsilon)
Q_total_target = Q1_target + Q2_target

fractions = [Q1_target/Q_total_target * 100, Q2_target/Q_total_target * 100]
labels = [f'Pipe 1\\n{Q1_target*1000:.1f} L/s', f'Pipe 2\\n{Q2_target*1000:.1f} L/s']
ax2.pie(fractions, labels=labels, autopct='%1.1f%%', colors=['lightblue', 'salmon'])
ax2.set_title(f'Flow Distribution at $h_f$ = {h_f_target} m')

plt.tight_layout()
save_fig('parallel_pipes.pdf')
\end{pycode}

\begin{figure}[H]
\centering
\includegraphics[width=\textwidth]{parallel_pipes.pdf}
\caption{Parallel pipe analysis showing system curves and flow distribution.}
\end{figure}

\section{Pump Selection}

\begin{pycode}
# Pump operating point determination
# System curve: h_sys = h_static + K*Q^2
h_static = 10  # m (elevation difference)
K_sys = 50000  # system constant (s^2/m^5)

Q_sys = np.linspace(0, 0.04, 100)
h_sys = h_static + K_sys * Q_sys**2

# Pump curve (typical centrifugal)
h_shutoff = 25  # m
Q_max = 0.05  # m^3/s
h_pump = h_shutoff * (1 - (Q_sys/Q_max)**2)

# Find operating point
idx_op = np.argmin(np.abs(h_pump - h_sys))
Q_op = Q_sys[idx_op]
h_op = h_pump[idx_op]

# NPSH requirements
NPSH_r = 2 + 5 * (Q_sys/Q_max)**2  # typical requirement curve

fig, (ax1, ax2) = plt.subplots(1, 2, figsize=(12, 5))

ax1.plot(Q_sys*1000, h_sys, 'b-', linewidth=2, label='System curve')
ax1.plot(Q_sys*1000, h_pump, 'r-', linewidth=2, label='Pump curve')
ax1.plot(Q_op*1000, h_op, 'go', markersize=10, label=f'Operating point\\n({Q_op*1000:.1f} L/s, {h_op:.1f} m)')
ax1.set_xlabel('Flow Rate (L/s)')
ax1.set_ylabel('Head (m)')
ax1.set_title('Pump Operating Point')
ax1.legend()
ax1.grid(True, alpha=0.3)

# Efficiency curve
eta = 0.85 * (1 - ((Q_sys - 0.025)/0.025)**2)
eta = np.maximum(eta, 0)
ax2.plot(Q_sys*1000, eta*100, 'g-', linewidth=2, label='Efficiency')
ax2.plot(Q_sys*1000, NPSH_r, 'm--', linewidth=2, label='NPSH$_r$')
ax2.axvline(Q_op*1000, color='k', linestyle='--', alpha=0.5)
ax2.set_xlabel('Flow Rate (L/s)')
ax2.set_ylabel('Efficiency (\\%) / NPSH (m)')
ax2.set_title('Pump Performance')
ax2.legend()
ax2.grid(True, alpha=0.3)

plt.tight_layout()
save_fig('pump_selection.pdf')

# Calculate pump power
eta_op = 0.85 * (1 - ((Q_op - 0.025)/0.025)**2)
P_hydraulic = rho * 9.81 * Q_op * h_op
P_shaft = P_hydraulic / eta_op
\end{pycode}

\begin{figure}[H]
\centering
\includegraphics[width=\textwidth]{pump_selection.pdf}
\caption{Pump selection showing operating point and performance characteristics.}
\end{figure}

Pump power required: $P = \py{f"{P_shaft/1000:.2f}"}$ kW (at $\eta = \py{f"{eta_op*100:.1f}"}$\%)

\section{Conclusions}

This analysis demonstrates key aspects of pipe flow:
\begin{enumerate}
    \item The Darcy-Weisbach equation provides accurate head loss predictions
    \item Friction factors depend on both Reynolds number and relative roughness
    \item The Moody diagram visualizes friction factor correlations
    \item Minor losses from fittings can be significant in short pipe runs
    \item Pipe diameter has a strong effect on pressure drop (varies as $D^{-5}$ for constant Q)
    \item Parallel pipe analysis requires iterative solution for flow distribution
    \item Pump operating point is determined by intersection of system and pump curves
\end{enumerate}

\end{document}
