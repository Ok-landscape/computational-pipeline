\documentclass[11pt,a4paper]{article}

% Document Setup
\usepackage[utf8]{inputenc}
\usepackage[T1]{fontenc}
\usepackage{lmodern}
\usepackage[margin=1in]{geometry}
\usepackage{amsmath,amssymb}
\usepackage{siunitx}
\usepackage{booktabs}
\usepackage{float}
\usepackage{caption}
\usepackage{hyperref}

% PythonTeX Setup
\usepackage[makestderr]{pythontex}

\title{Thermodynamic Cycles: Efficiency Analysis}
\author{Mechanical Engineering Laboratory}
\date{\today}

\begin{document}
\maketitle

\begin{abstract}
This report presents computational analysis of thermodynamic power cycles including Carnot, Otto, Diesel, and Rankine cycles. We examine ideal and actual cycle efficiencies, P-v and T-s diagrams, and parametric studies. Python-based computations provide quantitative analysis with dynamic visualization.
\end{abstract}

\tableofcontents
\newpage

\section{Introduction to Thermodynamic Cycles}

Thermodynamic cycles convert heat into work. The four cycles analyzed here are:
\begin{itemize}
    \item Carnot cycle: Maximum possible efficiency (theoretical ideal)
    \item Otto cycle: Spark-ignition internal combustion engines
    \item Diesel cycle: Compression-ignition engines
    \item Rankine cycle: Steam power plants
\end{itemize}

% Initialize Python environment
\begin{pycode}
import numpy as np
import matplotlib.pyplot as plt
from scipy.optimize import fsolve

plt.rcParams['figure.figsize'] = (8, 5)
plt.rcParams['font.size'] = 10
plt.rcParams['text.usetex'] = True

# Air properties (ideal gas)
gamma = 1.4
cp = 1005  # J/kgK
cv = cp / gamma
R = cp - cv

def save_fig(filename):
    plt.savefig(filename, dpi=150, bbox_inches='tight')
    plt.close()
\end{pycode}

\section{Carnot Cycle}

The Carnot efficiency sets the maximum limit:
\begin{equation}
\eta_{Carnot} = 1 - \frac{T_L}{T_H}
\end{equation}

\begin{pycode}
# Carnot cycle analysis
T_L = 300  # K (cold reservoir)
T_H_range = np.linspace(400, 1500, 100)
eta_carnot = 1 - T_L / T_H_range

# T-s diagram for Carnot cycle
T_H = 1000  # K
s1, s2 = 0, 1  # kJ/kgK

fig, (ax1, ax2) = plt.subplots(1, 2, figsize=(12, 5))

# Efficiency vs temperature
ax1.plot(T_H_range, eta_carnot * 100, 'b-', linewidth=2)
ax1.axhline(50, color='r', linestyle='--', alpha=0.5, label='50\\% efficiency')
ax1.set_xlabel('Hot Reservoir Temperature $T_H$ (K)')
ax1.set_ylabel('Carnot Efficiency (\\%)')
ax1.set_title(f'Carnot Efficiency ($T_L$ = {T_L} K)')
ax1.legend()
ax1.grid(True, alpha=0.3)

# T-s diagram
s_cycle = [s1, s2, s2, s1, s1]
T_cycle = [T_L, T_L, T_H, T_H, T_L]
ax2.plot(s_cycle, T_cycle, 'b-', linewidth=2)
ax2.fill(s_cycle, T_cycle, alpha=0.3)
ax2.set_xlabel('Entropy $s$ (kJ/kg$\\cdot$K)')
ax2.set_ylabel('Temperature $T$ (K)')
ax2.set_title('Carnot Cycle T-s Diagram')
ax2.grid(True, alpha=0.3)

# Label processes
ax2.annotate('1-2: Isothermal expansion', xy=(0.5, T_H), xytext=(0.6, T_H+100),
            arrowprops=dict(arrowstyle='->', color='black'), fontsize=9)
ax2.annotate('3-4: Isothermal compression', xy=(0.5, T_L), xytext=(0.6, T_L-100),
            arrowprops=dict(arrowstyle='->', color='black'), fontsize=9)

plt.tight_layout()
save_fig('carnot_cycle.pdf')

eta_carnot_design = 1 - T_L / T_H
\end{pycode}

\begin{figure}[H]
\centering
\includegraphics[width=\textwidth]{carnot_cycle.pdf}
\caption{Carnot cycle: efficiency dependence on temperature and T-s diagram.}
\end{figure}

Carnot efficiency at $T_H = 1000$ K: $\eta = \py{f"{eta_carnot_design*100:.1f}"}$\%

\section{Otto Cycle}

The Otto cycle efficiency depends on compression ratio:
\begin{equation}
\eta_{Otto} = 1 - \frac{1}{r^{\gamma-1}}
\end{equation}

\begin{pycode}
# Otto cycle analysis
r_range = np.linspace(4, 14, 100)  # Compression ratio
eta_otto = 1 - 1/r_range**(gamma-1)

# P-v diagram for specific compression ratio
r = 10
P1 = 100e3  # Pa
T1 = 300    # K
V1 = 1      # m^3 (normalized)
V2 = V1/r

# State points
T2 = T1 * r**(gamma-1)
P2 = P1 * r**gamma

# Heat addition (constant volume)
q_in = 1000e3  # J/kg
T3 = T2 + q_in/cv
P3 = P2 * T3/T2

# Expansion
T4 = T3 / r**(gamma-1)
P4 = P3 / r**gamma

fig, axes = plt.subplots(2, 2, figsize=(12, 10))

# Efficiency vs compression ratio
axes[0, 0].plot(r_range, eta_otto * 100, 'b-', linewidth=2)
axes[0, 0].axvline(r, color='r', linestyle='--', alpha=0.5, label=f'r = {r}')
axes[0, 0].set_xlabel('Compression Ratio $r$')
axes[0, 0].set_ylabel('Thermal Efficiency (\\%)')
axes[0, 0].set_title('Otto Cycle Efficiency')
axes[0, 0].legend()
axes[0, 0].grid(True, alpha=0.3)

# P-v diagram
V = np.linspace(V2, V1, 100)
P_12 = P1 * (V1/V)**gamma  # Compression (1-2)
P_34 = P3 * (V2/V)**gamma  # Expansion (3-4)

axes[0, 1].plot(V, P_12/1e6, 'b-', linewidth=2)
axes[0, 1].plot(V, P_34/1e6, 'r-', linewidth=2)
axes[0, 1].plot([V2, V2], [P2/1e6, P3/1e6], 'g-', linewidth=2)  # 2-3
axes[0, 1].plot([V1, V1], [P4/1e6, P1/1e6], 'g-', linewidth=2)  # 4-1
axes[0, 1].set_xlabel('Volume $V/V_1$')
axes[0, 1].set_ylabel('Pressure (MPa)')
axes[0, 1].set_title('Otto Cycle P-v Diagram')
axes[0, 1].grid(True, alpha=0.3)

# T-s diagram (approximate)
s = np.array([0, 0, 0.7, 0.7, 0])
T = np.array([T1, T2, T3, T4, T1])
axes[1, 0].plot(s, T, 'b-o', linewidth=2, markersize=8)
axes[1, 0].set_xlabel('Entropy $s$ (kJ/kg$\\cdot$K)')
axes[1, 0].set_ylabel('Temperature (K)')
axes[1, 0].set_title('Otto Cycle T-s Diagram')
axes[1, 0].grid(True, alpha=0.3)
for i, txt in enumerate(['1', '2', '3', '4']):
    axes[1, 0].annotate(txt, (s[i], T[i]), xytext=(5, 5), textcoords='offset points')

# Effect of gamma
gamma_range = [1.2, 1.3, 1.4, 1.5]
for g in gamma_range:
    eta = 1 - 1/r_range**(g-1)
    axes[1, 1].plot(r_range, eta * 100, linewidth=1.5, label=f'$\\gamma$ = {g}')

axes[1, 1].set_xlabel('Compression Ratio $r$')
axes[1, 1].set_ylabel('Thermal Efficiency (\\%)')
axes[1, 1].set_title('Effect of Specific Heat Ratio')
axes[1, 1].legend()
axes[1, 1].grid(True, alpha=0.3)

plt.tight_layout()
save_fig('otto_cycle.pdf')

eta_otto_design = 1 - 1/r**(gamma-1)
W_net = q_in * eta_otto_design
\end{pycode}

\begin{figure}[H]
\centering
\includegraphics[width=\textwidth]{otto_cycle.pdf}
\caption{Otto cycle analysis: efficiency, P-v diagram, T-s diagram, and gamma effect.}
\end{figure}

Otto efficiency at $r = 10$: $\eta = \py{f"{eta_otto_design*100:.1f}"}$\%, Net work = \py{f"{W_net/1000:.0f}"} kJ/kg

\section{Diesel Cycle}

The Diesel cycle efficiency includes the cutoff ratio:
\begin{equation}
\eta_{Diesel} = 1 - \frac{1}{r^{\gamma-1}} \cdot \frac{r_c^{\gamma} - 1}{\gamma(r_c - 1)}
\end{equation}

\begin{pycode}
# Diesel cycle analysis
r = 20  # Compression ratio (higher than Otto)
r_c_range = np.linspace(1.5, 4, 100)  # Cutoff ratio

eta_diesel = 1 - (1/r**(gamma-1)) * (r_c_range**gamma - 1)/(gamma*(r_c_range - 1))

# Compare with Otto at same compression ratio
eta_otto_r20 = 1 - 1/r**(gamma-1)

fig, (ax1, ax2) = plt.subplots(1, 2, figsize=(12, 5))

# Efficiency vs cutoff ratio
ax1.plot(r_c_range, eta_diesel * 100, 'b-', linewidth=2, label='Diesel')
ax1.axhline(eta_otto_r20 * 100, color='r', linestyle='--', linewidth=2, label=f'Otto (r={r})')
ax1.set_xlabel('Cutoff Ratio $r_c$')
ax1.set_ylabel('Thermal Efficiency (\\%)')
ax1.set_title(f'Diesel Cycle Efficiency (r = {r})')
ax1.legend()
ax1.grid(True, alpha=0.3)

# Comparison of Otto and Diesel
r_comp = np.linspace(8, 24, 100)
eta_otto_comp = 1 - 1/r_comp**(gamma-1)
r_c = 2.5  # Fixed cutoff ratio
eta_diesel_comp = 1 - (1/r_comp**(gamma-1)) * (r_c**gamma - 1)/(gamma*(r_c - 1))

ax2.plot(r_comp, eta_otto_comp * 100, 'b-', linewidth=2, label='Otto')
ax2.plot(r_comp, eta_diesel_comp * 100, 'r-', linewidth=2, label=f'Diesel ($r_c$={r_c})')
ax2.set_xlabel('Compression Ratio $r$')
ax2.set_ylabel('Thermal Efficiency (\\%)')
ax2.set_title('Otto vs Diesel Cycle Comparison')
ax2.legend()
ax2.grid(True, alpha=0.3)

plt.tight_layout()
save_fig('diesel_cycle.pdf')

r_c_design = 2.5
eta_diesel_design = 1 - (1/r**(gamma-1)) * (r_c_design**gamma - 1)/(gamma*(r_c_design - 1))
\end{pycode}

\begin{figure}[H]
\centering
\includegraphics[width=\textwidth]{diesel_cycle.pdf}
\caption{Diesel cycle: efficiency dependence on cutoff ratio and comparison with Otto.}
\end{figure}

Diesel efficiency at $r = 20$, $r_c = 2.5$: $\eta = \py{f"{eta_diesel_design*100:.1f}"}$\%

\section{Rankine Cycle}

The Rankine cycle uses phase change for higher efficiency:

\begin{pycode}
# Simplified Rankine cycle analysis
# Using approximate steam properties

# Operating conditions
P_boiler = 10e6  # Pa (10 MPa)
P_condenser = 10e3  # Pa (10 kPa)
T_boiler = 500 + 273  # K

# Simplified enthalpy calculations (kJ/kg)
h1 = 191.8  # Saturated liquid at condenser pressure
h2 = h1 + 0.00101 * (P_boiler - P_condenser)/1000  # Pump work (approximate)
h3 = 3373.6  # Superheated steam at boiler conditions
h4s = 2345  # Isentropic expansion to condenser pressure

# Actual expansion with turbine efficiency
eta_turbine = 0.85
h4 = h3 - eta_turbine * (h3 - h4s)

# Cycle efficiency
W_turbine = h3 - h4
W_pump = h2 - h1
Q_in = h3 - h2
eta_rankine = (W_turbine - W_pump) / Q_in

# Effect of boiler pressure
P_boiler_range = np.linspace(1, 20, 50)  # MPa
eta_range = []

for P in P_boiler_range:
    # Simplified model: efficiency increases with pressure
    h3_approx = 2800 + 50*P  # Approximate
    h4s_approx = 2400 - 5*P
    h4_approx = h3_approx - 0.85 * (h3_approx - h4s_approx)
    Q_approx = h3_approx - 200
    W_approx = h3_approx - h4_approx
    eta_range.append(W_approx / Q_approx * 100)

fig, axes = plt.subplots(2, 2, figsize=(12, 10))

# T-s diagram
s = [0.6, 0.7, 6.5, 7.0]  # Approximate entropy values
T = [319, 584, 773, 319]  # Temperature in K

axes[0, 0].plot(s, T, 'b-o', linewidth=2, markersize=8)
axes[0, 0].fill(s, T, alpha=0.3)
axes[0, 0].set_xlabel('Entropy $s$ (kJ/kg$\\cdot$K)')
axes[0, 0].set_ylabel('Temperature $T$ (K)')
axes[0, 0].set_title('Rankine Cycle T-s Diagram')
axes[0, 0].grid(True, alpha=0.3)
for i, txt in enumerate(['1', '2', '3', '4']):
    axes[0, 0].annotate(txt, (s[i], T[i]), xytext=(5, 5), textcoords='offset points')

# Efficiency vs boiler pressure
axes[0, 1].plot(P_boiler_range, eta_range, 'b-', linewidth=2)
axes[0, 1].set_xlabel('Boiler Pressure (MPa)')
axes[0, 1].set_ylabel('Thermal Efficiency (\\%)')
axes[0, 1].set_title('Rankine Efficiency vs Boiler Pressure')
axes[0, 1].grid(True, alpha=0.3)

# Energy flow diagram
components = ['Turbine Work', 'Pump Work', 'Net Work']
values = [W_turbine, W_pump, W_turbine - W_pump]
colors = ['green', 'red', 'blue']

axes[1, 0].bar(components, values, color=colors, alpha=0.7)
axes[1, 0].set_ylabel('Energy (kJ/kg)')
axes[1, 0].set_title('Rankine Cycle Energy Balance')
axes[1, 0].grid(True, alpha=0.3, axis='y')

# Effect of reheat
# Simplified: reheat increases efficiency by ~3-5%
base_eta = np.array(eta_range)
reheat_eta = base_eta * 1.04

axes[1, 1].plot(P_boiler_range, base_eta, 'b-', linewidth=2, label='Simple')
axes[1, 1].plot(P_boiler_range, reheat_eta, 'r--', linewidth=2, label='With Reheat')
axes[1, 1].set_xlabel('Boiler Pressure (MPa)')
axes[1, 1].set_ylabel('Thermal Efficiency (\\%)')
axes[1, 1].set_title('Effect of Reheat on Efficiency')
axes[1, 1].legend()
axes[1, 1].grid(True, alpha=0.3)

plt.tight_layout()
save_fig('rankine_cycle.pdf')
\end{pycode}

\begin{figure}[H]
\centering
\includegraphics[width=\textwidth]{rankine_cycle.pdf}
\caption{Rankine cycle: T-s diagram, pressure effect, energy balance, and reheat improvement.}
\end{figure}

Rankine efficiency: $\eta = \py{f"{eta_rankine*100:.1f}"}$\%, Net work = \py{f"{W_turbine - W_pump:.0f}"} kJ/kg

\section{Cycle Comparison}

\begin{pycode}
# Compare all cycles
cycles = ['Carnot', 'Otto (r=10)', 'Diesel (r=20)', 'Rankine']
efficiencies = [
    (1 - 300/1000) * 100,
    (1 - 1/10**(gamma-1)) * 100,
    (1 - (1/20**(gamma-1)) * (2.5**gamma - 1)/(gamma*(2.5 - 1))) * 100,
    eta_rankine * 100
]

fig, ax = plt.subplots(figsize=(10, 6))

colors = ['gold', 'blue', 'green', 'red']
bars = ax.bar(cycles, efficiencies, color=colors, alpha=0.7)

ax.set_ylabel('Thermal Efficiency (\\%)')
ax.set_title('Comparison of Thermodynamic Cycles')
ax.grid(True, alpha=0.3, axis='y')

for bar, eff in zip(bars, efficiencies):
    ax.text(bar.get_x() + bar.get_width()/2, bar.get_height() + 1,
            f'{eff:.1f}\\%', ha='center', fontsize=10)

save_fig('cycle_comparison.pdf')
\end{pycode}

\begin{figure}[H]
\centering
\includegraphics[width=0.8\textwidth]{cycle_comparison.pdf}
\caption{Comparison of thermal efficiencies for different thermodynamic cycles.}
\end{figure}

\section{Summary Table}

\begin{table}[H]
\centering
\caption{Thermodynamic Cycle Parameters}
\begin{tabular}{lcccc}
\toprule
Cycle & Key Parameter & Efficiency Formula & Typical $\eta$ & Application \\
\midrule
Carnot & $T_H/T_L$ & $1 - T_L/T_H$ & 70\% & Theoretical \\
Otto & $r$ & $1 - r^{1-\gamma}$ & 60\% & Gasoline engines \\
Diesel & $r$, $r_c$ & Complex & 55\% & Diesel engines \\
Rankine & $P_{boiler}$ & Energy balance & 35\% & Power plants \\
\bottomrule
\end{tabular}
\end{table}

\section{Conclusions}

This analysis demonstrates key aspects of thermodynamic cycles:
\begin{enumerate}
    \item Carnot efficiency sets the theoretical maximum for any heat engine
    \item Otto efficiency increases with compression ratio but is limited by knock
    \item Diesel cycles achieve higher compression ratios but lower peak efficiency
    \item Rankine cycles use phase change for effective heat addition
    \item Reheat and regeneration improve Rankine cycle efficiency
    \item Actual efficiencies are lower due to irreversibilities
\end{enumerate}

\end{document}
