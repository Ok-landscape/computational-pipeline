\documentclass[11pt,a4paper]{article}

% Document Setup
\usepackage[utf8]{inputenc}
\usepackage[T1]{fontenc}
\usepackage{lmodern}
\usepackage[margin=1in]{geometry}
\usepackage{amsmath,amssymb}
\usepackage{siunitx}
\usepackage{booktabs}
\usepackage{float}
\usepackage{caption}
\usepackage{hyperref}

% PythonTeX Setup
\usepackage[makestderr]{pythontex}

\title{Heat Transfer Analysis: Conduction, Convection, and Fins}
\author{Mechanical Engineering Laboratory}
\date{\today}

\begin{document}
\maketitle

\begin{abstract}
This report presents computational analysis of heat transfer mechanisms including conduction through composite walls, convection correlations, fin analysis, and heat exchanger design. Python-based computations provide quantitative analysis with dynamic visualization of temperature distributions and heat flux.
\end{abstract}

\tableofcontents
\newpage

\section{Introduction to Heat Transfer}

Heat transfer is the thermal energy in transit due to a temperature difference. The three modes are:
\begin{itemize}
    \item Conduction: Energy transfer through molecular interactions
    \item Convection: Energy transfer by fluid motion
    \item Radiation: Energy transfer by electromagnetic waves
\end{itemize}

% Initialize Python environment
\begin{pycode}
import numpy as np
import matplotlib.pyplot as plt
from scipy.optimize import fsolve

plt.rcParams['figure.figsize'] = (8, 5)
plt.rcParams['font.size'] = 10
plt.rcParams['text.usetex'] = True

def save_fig(filename):
    plt.savefig(filename, dpi=150, bbox_inches='tight')
    plt.close()
\end{pycode}

\section{Conduction Heat Transfer}

\subsection{Fourier's Law}

The rate of heat conduction is proportional to the temperature gradient:
\begin{equation}
q = -k \frac{dT}{dx}
\end{equation}
where $k$ is thermal conductivity (\si{\watt\per\meter\per\kelvin}).

\subsection{Composite Wall Analysis}

For a composite wall with convection on both sides:
\begin{equation}
q = \frac{T_{i} - T_{o}}{R_{total}} = \frac{T_{i} - T_{o}}{\frac{1}{h_i A} + \sum\frac{L_j}{k_j A} + \frac{1}{h_o A}}
\end{equation}

\begin{pycode}
# Composite wall analysis
# Three-layer wall: brick + insulation + plaster
materials = [
    {'name': 'Brick', 'k': 0.72, 'L': 0.23},  # W/mK, m
    {'name': 'Insulation', 'k': 0.038, 'L': 0.08},
    {'name': 'Plaster', 'k': 0.48, 'L': 0.02}
]

h_i = 10  # W/m2K (inside convection)
h_o = 25  # W/m2K (outside convection)
T_i = 22  # C (inside temperature)
T_o = -5  # C (outside temperature)

# Calculate thermal resistances (per unit area)
R_i = 1/h_i
R_o = 1/h_o
R_total = R_i + R_o

x_positions = [0]
T_positions = [T_i]

# Calculate temperature at each interface
q = 0  # Will be calculated
current_T = T_i - (T_i - T_o) * R_i  # Temperature at inner surface

R_materials = []
for mat in materials:
    R = mat['L'] / mat['k']
    R_materials.append(R)
    R_total += R

# Heat flux
q = (T_i - T_o) / R_total

# Temperature profile
T_positions = [T_i]
x_positions = [0]

# Inner convection boundary
T_s1 = T_i - q * R_i
T_positions.append(T_s1)
x = 0
x_positions.append(x)

# Through each material layer
for mat, R in zip(materials, R_materials):
    x += mat['L']
    T_next = T_positions[-1] - q * R
    x_positions.append(x)
    T_positions.append(T_next)

# Outer surface
T_positions.append(T_o)
x_positions.append(x_positions[-1])

fig, (ax1, ax2) = plt.subplots(1, 2, figsize=(14, 5))

# Temperature profile
ax1.plot(np.array(x_positions)*100, T_positions, 'b-o', linewidth=2, markersize=8)
ax1.axhline(0, color='k', linestyle='--', alpha=0.3)

# Shade regions
colors = ['#FFB6C1', '#90EE90', '#ADD8E6']
x_start = 0
for i, (mat, color) in enumerate(zip(materials, colors)):
    ax1.axvspan(x_start*100, (x_start + mat['L'])*100, alpha=0.3, color=color, label=mat['name'])
    x_start += mat['L']

ax1.set_xlabel('Position (cm)')
ax1.set_ylabel('Temperature ($^\\circ$C)')
ax1.set_title('Temperature Profile Through Composite Wall')
ax1.legend(loc='upper right')
ax1.grid(True, alpha=0.3)

# Thermal resistance breakdown
resistances = [R_i] + R_materials + [R_o]
labels = ['Conv (in)'] + [m['name'] for m in materials] + ['Conv (out)']
colors_bar = ['yellow'] + colors + ['yellow']

ax2.bar(labels, resistances, color=colors_bar, alpha=0.7)
ax2.set_ylabel('Thermal Resistance (m$^2$K/W)')
ax2.set_title(f'Resistance Breakdown (Total = {R_total:.3f} m$^2$K/W)')
ax2.grid(True, alpha=0.3, axis='y')

plt.tight_layout()
save_fig('composite_wall.pdf')
\end{pycode}

\begin{figure}[H]
\centering
\includegraphics[width=\textwidth]{composite_wall.pdf}
\caption{Composite wall analysis: temperature profile and thermal resistance breakdown.}
\end{figure}

Heat flux through wall: $q = \py{f"{q:.1f}"}$ \si{\watt\per\square\meter}

\section{Convection Heat Transfer}

\subsection{Convection Correlations}

The heat transfer coefficient depends on the Nusselt number:
\begin{equation}
h = \frac{Nu \cdot k}{L_c}
\end{equation}

\begin{pycode}
# Convection correlations for different geometries
def Nu_flat_plate_laminar(Re, Pr):
    """Laminar flow over flat plate"""
    return 0.664 * Re**0.5 * Pr**(1/3)

def Nu_flat_plate_turbulent(Re, Pr):
    """Turbulent flow over flat plate"""
    return 0.037 * Re**0.8 * Pr**(1/3)

def Nu_cylinder_crossflow(Re, Pr):
    """Cross-flow over cylinder (Churchill-Bernstein)"""
    return 0.3 + (0.62 * Re**0.5 * Pr**(1/3)) / (1 + (0.4/Pr)**(2/3))**0.25 * \
           (1 + (Re/282000)**(5/8))**(4/5)

def Nu_pipe_turbulent(Re, Pr):
    """Fully developed turbulent flow in pipe (Dittus-Boelter)"""
    return 0.023 * Re**0.8 * Pr**0.4

# Calculate for air flow
Pr = 0.71  # Prandtl number for air
Re_range = np.logspace(3, 6, 100)

fig, (ax1, ax2) = plt.subplots(1, 2, figsize=(12, 5))

# External flow correlations
Nu_plate_lam = [Nu_flat_plate_laminar(Re, Pr) for Re in Re_range if Re < 5e5]
Nu_plate_turb = [Nu_flat_plate_turbulent(Re, Pr) for Re in Re_range if Re >= 5e5]
Nu_cyl = [Nu_cylinder_crossflow(Re, Pr) for Re in Re_range]

ax1.loglog(Re_range[Re_range < 5e5], Nu_plate_lam, 'b-', linewidth=2, label='Flat plate (laminar)')
ax1.loglog(Re_range[Re_range >= 5e5], Nu_plate_turb, 'b--', linewidth=2, label='Flat plate (turbulent)')
ax1.loglog(Re_range, Nu_cyl, 'r-', linewidth=2, label='Cylinder cross-flow')
ax1.axvline(5e5, color='k', linestyle=':', alpha=0.5)
ax1.set_xlabel('Reynolds Number')
ax1.set_ylabel('Nusselt Number')
ax1.set_title('External Flow Correlations')
ax1.legend()
ax1.grid(True, which='both', alpha=0.3)

# Internal flow correlation
Re_pipe = np.logspace(4, 6, 50)
Nu_pipe = [Nu_pipe_turbulent(Re, Pr) for Re in Re_pipe]

ax2.loglog(Re_pipe, Nu_pipe, 'g-', linewidth=2, label='Dittus-Boelter (heating)')
ax2.set_xlabel('Reynolds Number')
ax2.set_ylabel('Nusselt Number')
ax2.set_title('Internal Flow Correlation (Turbulent Pipe)')
ax2.legend()
ax2.grid(True, which='both', alpha=0.3)

plt.tight_layout()
save_fig('convection_correlations.pdf')
\end{pycode}

\begin{figure}[H]
\centering
\includegraphics[width=\textwidth]{convection_correlations.pdf}
\caption{Convection heat transfer correlations for external and internal flows.}
\end{figure}

\section{Extended Surfaces (Fins)}

\subsection{Fin Temperature Distribution}

For a fin with adiabatic tip:
\begin{equation}
\frac{\theta}{\theta_b} = \frac{\cosh[m(L-x)]}{\cosh(mL)}
\end{equation}
where $m = \sqrt{\frac{hP}{kA_c}}$ and $\theta = T - T_{\infty}$.

\begin{pycode}
# Fin analysis
k_fin = 200  # W/mK (aluminum)
h_fin = 50   # W/m2K
L_fin = 0.1  # m (fin length)
t_fin = 0.005  # m (fin thickness)
w_fin = 0.05  # m (fin width)

# Cross-section parameters
P = 2 * (t_fin + w_fin)  # perimeter
A_c = t_fin * w_fin  # cross-sectional area
m = np.sqrt(h_fin * P / (k_fin * A_c))

T_b = 100  # C (base temperature)
T_inf = 25  # C (ambient temperature)
theta_b = T_b - T_inf

# Temperature distribution
x = np.linspace(0, L_fin, 100)
theta = theta_b * np.cosh(m * (L_fin - x)) / np.cosh(m * L_fin)
T_fin = theta + T_inf

# Heat transfer rate
q_fin = np.sqrt(h_fin * P * k_fin * A_c) * theta_b * np.tanh(m * L_fin)

# Maximum possible heat transfer (if entire fin at base temperature)
A_fin = P * L_fin  # fin surface area
q_max = h_fin * A_fin * theta_b

# Fin efficiency
eta_fin = q_fin / q_max

fig, axes = plt.subplots(2, 2, figsize=(12, 10))

# Temperature distribution
axes[0, 0].plot(x*100, T_fin, 'b-', linewidth=2)
axes[0, 0].axhline(T_inf, color='r', linestyle='--', label=f'$T_\\infty$ = {T_inf}$^\\circ$C')
axes[0, 0].set_xlabel('Distance from Base (cm)')
axes[0, 0].set_ylabel('Temperature ($^\\circ$C)')
axes[0, 0].set_title('Fin Temperature Distribution')
axes[0, 0].legend()
axes[0, 0].grid(True, alpha=0.3)

# Temperature ratio
axes[0, 1].plot(x*100, theta/theta_b * 100, 'g-', linewidth=2)
axes[0, 1].set_xlabel('Distance from Base (cm)')
axes[0, 1].set_ylabel('Temperature Ratio $\\theta/\\theta_b$ (\\%)')
axes[0, 1].set_title('Normalized Temperature Distribution')
axes[0, 1].grid(True, alpha=0.3)

# Effect of fin length
L_range = np.linspace(0.01, 0.2, 50)
eta_range = []
q_range = []
for L in L_range:
    mL = m * L
    q = np.sqrt(h_fin * P * k_fin * A_c) * theta_b * np.tanh(mL)
    A_s = P * L
    q_max_L = h_fin * A_s * theta_b
    eta = q / q_max_L
    eta_range.append(eta * 100)
    q_range.append(q)

axes[1, 0].plot(L_range*100, eta_range, 'r-', linewidth=2)
axes[1, 0].axvline(L_fin*100, color='k', linestyle='--', alpha=0.5)
axes[1, 0].set_xlabel('Fin Length (cm)')
axes[1, 0].set_ylabel('Fin Efficiency (\\%)')
axes[1, 0].set_title('Fin Efficiency vs Length')
axes[1, 0].grid(True, alpha=0.3)

# Heat transfer rate
axes[1, 1].plot(L_range*100, q_range, 'm-', linewidth=2)
axes[1, 1].axvline(L_fin*100, color='k', linestyle='--', alpha=0.5)
axes[1, 1].set_xlabel('Fin Length (cm)')
axes[1, 1].set_ylabel('Heat Transfer Rate (W)')
axes[1, 1].set_title('Fin Heat Transfer vs Length')
axes[1, 1].grid(True, alpha=0.3)

plt.tight_layout()
save_fig('fin_analysis.pdf')
\end{pycode}

\begin{figure}[H]
\centering
\includegraphics[width=\textwidth]{fin_analysis.pdf}
\caption{Fin analysis: temperature distribution, efficiency, and heat transfer rate.}
\end{figure}

\begin{table}[H]
\centering
\caption{Fin Performance Parameters}
\begin{tabular}{lcc}
\toprule
Parameter & Value & Units \\
\midrule
Fin parameter $m$ & \py{f"{m:.2f}"} & 1/m \\
Product $mL$ & \py{f"{m*L_fin:.2f}"} & -- \\
Fin efficiency & \py{f"{eta_fin*100:.1f}"} & \% \\
Heat transfer rate & \py{f"{q_fin:.1f}"} & W \\
\bottomrule
\end{tabular}
\end{table}

\section{Heat Exchanger Analysis}

\subsection{LMTD Method}

For counter-flow heat exchangers:
\begin{equation}
\Delta T_{lm} = \frac{\Delta T_1 - \Delta T_2}{\ln(\Delta T_1/\Delta T_2)}
\end{equation}

\begin{pycode}
# Counter-flow heat exchanger analysis
# Hot fluid (oil): inlet 150C, outlet 90C
# Cold fluid (water): inlet 20C, outlet 70C

T_h_in = 150  # C
T_h_out = 90  # C
T_c_in = 20   # C
T_c_out = 70  # C

# Temperature differences
dT_1 = T_h_in - T_c_out  # at hot inlet
dT_2 = T_h_out - T_c_in  # at hot outlet

# LMTD
if dT_1 != dT_2:
    LMTD = (dT_1 - dT_2) / np.log(dT_1/dT_2)
else:
    LMTD = dT_1

# Assume heat transfer and calculate UA
m_dot_h = 0.5  # kg/s
c_p_h = 2000   # J/kgK (oil)
Q = m_dot_h * c_p_h * (T_h_in - T_h_out)
UA = Q / LMTD

# Temperature profiles along heat exchanger
x = np.linspace(0, 1, 100)

# Counter-flow
T_h_counter = T_h_in - (T_h_in - T_h_out) * x
T_c_counter = T_c_out - (T_c_out - T_c_in) * x

# Parallel flow (for comparison)
# Need to recalculate outlet temps for same UA
dT_1_parallel = T_h_in - T_c_in
C_h = m_dot_h * c_p_h
m_dot_c = Q / (4186 * (T_c_out - T_c_in))
C_c = m_dot_c * 4186
C_min = min(C_h, C_c)
NTU = UA / C_min
C_r = C_min / max(C_h, C_c)
eps_counter = (1 - np.exp(-NTU * (1 - C_r))) / (1 - C_r * np.exp(-NTU * (1 - C_r)))
eps_parallel = (1 - np.exp(-NTU * (1 + C_r))) / (1 + C_r)

fig, (ax1, ax2) = plt.subplots(1, 2, figsize=(12, 5))

# Counter-flow temperature profiles
ax1.plot(x*100, T_h_counter, 'r-', linewidth=2, label='Hot fluid')
ax1.plot(x*100, T_c_counter, 'b-', linewidth=2, label='Cold fluid')
ax1.fill_between(x*100, T_h_counter, T_c_counter, alpha=0.2, color='green')
ax1.set_xlabel('Position (\\% of length)')
ax1.set_ylabel('Temperature ($^\\circ$C)')
ax1.set_title(f'Counter-Flow HX (LMTD = {LMTD:.1f}$^\\circ$C)')
ax1.legend()
ax1.grid(True, alpha=0.3)

# Effectiveness-NTU curves
NTU_range = np.linspace(0.1, 5, 100)
C_r_values = [0, 0.25, 0.5, 0.75, 1.0]

for C_r in C_r_values:
    if C_r == 1:
        eps = NTU_range / (1 + NTU_range)
    else:
        eps = (1 - np.exp(-NTU_range * (1 - C_r))) / (1 - C_r * np.exp(-NTU_range * (1 - C_r)))
    ax2.plot(NTU_range, eps, linewidth=1.5, label=f'$C_r$ = {C_r}')

ax2.set_xlabel('NTU')
ax2.set_ylabel('Effectiveness $\\varepsilon$')
ax2.set_title('Counter-Flow HX Effectiveness')
ax2.legend()
ax2.grid(True, alpha=0.3)

plt.tight_layout()
save_fig('heat_exchanger.pdf')
\end{pycode}

\begin{figure}[H]
\centering
\includegraphics[width=\textwidth]{heat_exchanger.pdf}
\caption{Heat exchanger analysis: temperature profiles and effectiveness-NTU curves.}
\end{figure}

Heat duty: $Q = \py{f"{Q/1000:.1f}"}$ kW, $UA = \py{f"{UA:.0f}"}$ W/K

\section{Transient Conduction}

\subsection{Lumped Capacitance Method}

When $Bi = hL_c/k < 0.1$:
\begin{equation}
\frac{T - T_{\infty}}{T_i - T_{\infty}} = \exp\left(-\frac{hA_s}{\rho V c_p}t\right) = \exp\left(-\frac{t}{\tau}\right)
\end{equation}

\begin{pycode}
# Transient cooling of a sphere
D = 0.05  # m diameter
rho = 2700  # kg/m3 (aluminum)
c_p = 900  # J/kgK
k = 200    # W/mK
h = 100    # W/m2K

V = (4/3) * np.pi * (D/2)**3
A_s = 4 * np.pi * (D/2)**2
L_c = V / A_s

# Biot number
Bi = h * L_c / k

# Time constant
tau = rho * V * c_p / (h * A_s)

T_i = 200  # C initial
T_inf = 25  # C ambient

# Temperature history
t = np.linspace(0, 600, 200)
theta_ratio = np.exp(-t/tau)
T = T_inf + (T_i - T_inf) * theta_ratio

# Heat transfer rate
q_t = h * A_s * (T - T_inf)
Q_total = rho * V * c_p * (T_i - T_inf)  # total energy
Q_transferred = Q_total * (1 - theta_ratio)

fig, axes = plt.subplots(2, 2, figsize=(12, 10))

# Temperature history
axes[0, 0].plot(t, T, 'b-', linewidth=2)
axes[0, 0].axhline(T_inf, color='r', linestyle='--', label=f'$T_\\infty$ = {T_inf}$^\\circ$C')
axes[0, 0].axvline(tau, color='k', linestyle=':', alpha=0.5, label=f'$\\tau$ = {tau:.0f} s')
axes[0, 0].set_xlabel('Time (s)')
axes[0, 0].set_ylabel('Temperature ($^\\circ$C)')
axes[0, 0].set_title('Temperature History (Lumped Capacitance)')
axes[0, 0].legend()
axes[0, 0].grid(True, alpha=0.3)

# Temperature ratio (semi-log)
axes[0, 1].semilogy(t, theta_ratio, 'g-', linewidth=2)
axes[0, 1].axhline(0.368, color='k', linestyle='--', alpha=0.5, label='$e^{-1}$')
axes[0, 1].axvline(tau, color='k', linestyle=':', alpha=0.5)
axes[0, 1].set_xlabel('Time (s)')
axes[0, 1].set_ylabel('$(T - T_\\infty)/(T_i - T_\\infty)$')
axes[0, 1].set_title('Normalized Temperature')
axes[0, 1].legend()
axes[0, 1].grid(True, which='both', alpha=0.3)

# Instantaneous heat transfer rate
axes[1, 0].plot(t, q_t, 'r-', linewidth=2)
axes[1, 0].set_xlabel('Time (s)')
axes[1, 0].set_ylabel('Heat Transfer Rate (W)')
axes[1, 0].set_title('Instantaneous Heat Transfer')
axes[1, 0].grid(True, alpha=0.3)

# Cumulative energy transferred
axes[1, 1].plot(t, Q_transferred/1000, 'm-', linewidth=2)
axes[1, 1].axhline(Q_total/1000, color='k', linestyle='--', alpha=0.5, label=f'Total = {Q_total/1000:.2f} kJ')
axes[1, 1].set_xlabel('Time (s)')
axes[1, 1].set_ylabel('Energy Transferred (kJ)')
axes[1, 1].set_title('Cumulative Energy Transfer')
axes[1, 1].legend()
axes[1, 1].grid(True, alpha=0.3)

plt.tight_layout()
save_fig('transient_conduction.pdf')
\end{pycode}

\begin{figure}[H]
\centering
\includegraphics[width=\textwidth]{transient_conduction.pdf}
\caption{Transient conduction analysis using lumped capacitance method.}
\end{figure}

Biot number: $Bi = \py{f"{Bi:.3f}"}$ (lumped model valid since $Bi < 0.1$)

\section{Radiation Heat Transfer}

\begin{pycode}
# Radiation between surfaces
sigma = 5.67e-8  # Stefan-Boltzmann constant

# Enclosure with two surfaces
T1 = 500 + 273  # K (hot surface)
T2 = 300 + 273  # K (cold surface)
eps1 = 0.8
eps2 = 0.6
A1 = 2  # m2
F12 = 0.5  # view factor

# Net radiation heat transfer
q_rad = sigma * A1 * F12 * (T1**4 - T2**4) / (1/eps1 + A1/(A1*F12) * (1/eps2 - 1))

# Blackbody radiation
T_range = np.linspace(300, 1500, 100)
E_b = sigma * T_range**4

fig, (ax1, ax2) = plt.subplots(1, 2, figsize=(12, 5))

# Blackbody emissive power
ax1.plot(T_range - 273, E_b/1000, 'r-', linewidth=2)
ax1.set_xlabel('Temperature ($^\\circ$C)')
ax1.set_ylabel('Emissive Power (kW/m$^2$)')
ax1.set_title('Blackbody Emissive Power')
ax1.grid(True, alpha=0.3)

# Effect of emissivity on net radiation
eps_range = np.linspace(0.1, 1.0, 50)
q_eps = []
for eps in eps_range:
    q = sigma * A1 * F12 * (T1**4 - T2**4) * eps  # simplified
    q_eps.append(q/1000)

ax2.plot(eps_range, q_eps, 'b-', linewidth=2)
ax2.set_xlabel('Emissivity')
ax2.set_ylabel('Heat Transfer (kW)')
ax2.set_title('Effect of Emissivity on Radiation')
ax2.grid(True, alpha=0.3)

plt.tight_layout()
save_fig('radiation.pdf')
\end{pycode}

\begin{figure}[H]
\centering
\includegraphics[width=\textwidth]{radiation.pdf}
\caption{Radiation heat transfer: blackbody emission and emissivity effects.}
\end{figure}

\section{Conclusions}

This analysis demonstrates key aspects of heat transfer:
\begin{enumerate}
    \item Composite walls require thermal resistance network analysis
    \item Convection correlations depend on flow geometry and regime
    \item Fin efficiency decreases with length but total heat transfer increases
    \item Heat exchanger design uses LMTD or effectiveness-NTU methods
    \item Lumped capacitance applies when $Bi < 0.1$
    \item Radiation becomes dominant at high temperatures
\end{enumerate}

\end{document}
