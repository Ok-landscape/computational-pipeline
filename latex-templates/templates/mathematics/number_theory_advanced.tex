\documentclass[12pt]{article}

% Encoding and Fonts
\usepackage[utf8]{inputenc}
\usepackage[T1]{fontenc}
\usepackage{lmodern}

% Mathematics and Physics
\usepackage{amsmath}
\usepackage{amssymb}
\usepackage{amsthm}
\usepackage{physics}
\usepackage{siunitx}

% Graphics and Colors
\usepackage{graphicx}
\usepackage{xcolor}
\usepackage{float}

% PythonTeX Setup
\usepackage[makestderr, pyfuture=all]{pythontex}

% Page Layout
\usepackage{geometry}
\geometry{a4paper, margin=1in}

% Hyperlinks
\usepackage{hyperref}
\hypersetup{
    colorlinks=true,
    linkcolor=blue,
    citecolor=blue,
    urlcolor=blue
}

% Theorem Environments
\newtheorem{theorem}{Theorem}[section]
\newtheorem{lemma}[theorem]{Lemma}
\newtheorem{proposition}[theorem]{Proposition}
\newtheorem{corollary}[theorem]{Corollary}
\theoremstyle{definition}
\newtheorem{definition}[theorem]{Definition}
\newtheorem{example}[theorem]{Example}

\title{Advanced Number Theory}
\author{Computational Pipeline}
\date{\today}

\begin{document}
\maketitle

\begin{abstract}
This document provides a computational exploration of advanced number theory using PythonTeX for dynamic calculations and visualizations. All computations are performed inline, ensuring reproducibility and consistency between text and results.
\end{abstract}

\section{Introduction}

This template demonstrates the integration of Python computation with LaTeX typesetting for advanced number theory. The document includes:

\begin{itemize}
    \item Mathematical formulations and derivations
    \item Computational implementations using Python
    \item Dynamic visualizations with Matplotlib
    \item Numerical analysis and verification
\end{itemize}

\section{Theoretical Background}

\subsection{Mathematical Framework}

We begin with the fundamental equations and theoretical foundations relevant to this topic.

\begin{pycode}
import numpy as np
import matplotlib.pyplot as plt
from scipy import optimize, integrate
import sys

# Configure plotting
plt.rc('text', usetex=True)
plt.rc('font', family='serif', size=10)

# Set random seed for reproducibility
np.random.seed(42)

# Global parameters
print(f"Python version: {sys.version.split()[0]}")
print(f"NumPy version: {np.__version__}")
\end{pycode}

\section{Computational Implementation}

\subsection{Numerical Methods}

We implement the core algorithms using Python and NumPy:

\begin{pycode}
# Define key parameters
n_points = 100
x = np.linspace(0, 10, n_points)

# Sample computation
y = np.sin(x) * np.exp(-x/10)

# Statistical analysis
mean_val = np.mean(y)
std_val = np.std(y)
max_val = np.max(y)
min_val = np.min(y)

print(f"Mean value: {mean_val:.6f}")
print(f"Standard deviation: {std_val:.6f}")
print(f"Maximum value: {max_val:.6f}")
print(f"Minimum value: {min_val:.6f}")
\end{pycode}

The analysis shows that the mean value is \py{f"{mean_val:.4f}"} with a standard deviation of \py{f"{std_val:.4f}"}.

\subsection{Visualization}

\begin{pycode}
fig, (ax1, ax2) = plt.subplots(1, 2, figsize=(10, 4))

# Plot 1: Primary visualization
ax1.plot(x, y, 'b-', linewidth=2, label='Function')
ax1.axhline(mean_val, color='r', linestyle='--', label=f'Mean: {mean_val:.3f}')
ax1.fill_between(x, mean_val-std_val, mean_val+std_val, alpha=0.3, color='red', label=r'$\pm 1\sigma$')
ax1.set_xlabel('x')
ax1.set_ylabel('y')
ax1.set_title('Primary Analysis')
ax1.legend()
ax1.grid(True, alpha=0.3)

# Plot 2: Distribution analysis
ax2.hist(y, bins=20, density=True, alpha=0.7, color='blue', edgecolor='black')
ax2.axvline(mean_val, color='r', linestyle='--', linewidth=2, label='Mean')
ax2.set_xlabel('Value')
ax2.set_ylabel('Density')
ax2.set_title('Distribution')
ax2.legend()
ax2.grid(True, alpha=0.3)

plt.tight_layout()
plt.savefig('number_theory_advanced_analysis.pdf', bbox_inches='tight')
plt.close()

print(r'\begin{figure}[H]')
print(r'\centering')
print(r'\includegraphics[width=0.95\textwidth]{number_theory_advanced_analysis.pdf}')
print(r'\caption{Computational analysis results}')
print(r'\label{fig:analysis}')
print(r'\end{figure}')
\end{pycode}

\section{Results and Discussion}

\subsection{Quantitative Results}

The computational analysis yields the following key findings:

\begin{pycode}
# Additional analysis
results = {
    'Data points': n_points,
    'Range': f'[{min_val:.4f}, {max_val:.4f}]',
    'Variance': f'{np.var(y):.6f}',
    'Peak-to-peak': f'{max_val - min_val:.6f}'
}

print(r'\begin{table}[H]')
print(r'\centering')
print(r'\begin{tabular}{ll}')
print(r'\hline')
print(r'\textbf{Metric} & \textbf{Value} \\')
print(r'\hline')
for key, value in results.items():
    print(r'{} & {} \\'.format(key, value))
print(r'\hline')
print(r'\end{tabular}')
print(r'\caption{Summary of computational results}')
print(r'\label{tab:results}')
print(r'\end{table}')
\end{pycode}

\subsection{Interpretation}

The results demonstrate the computational approach to analyzing advanced number theory. The integration of Python computation within LaTeX ensures that all numerical values, figures, and tables are automatically generated from the code, maintaining consistency and reproducibility.

\section{Conclusion}

This document demonstrates a reproducible computational workflow for advanced number theory. Key advantages include:

\begin{enumerate}
    \item \textbf{Reproducibility}: All computations can be re-executed
    \item \textbf{Consistency}: Numbers in text match calculations exactly
    \item \textbf{Efficiency}: Updates to parameters automatically propagate
    \item \textbf{Transparency}: Complete methodology is documented
\end{enumerate}

\subsection{Future Directions}

Potential extensions of this work include:
\begin{itemize}
    \item Advanced numerical methods
    \item Parameter sensitivity analysis
    \item Optimization techniques
    \item Comparison with analytical solutions
\end{itemize}

\section{Computational Environment}

\begin{pycode}
import platform
print(r'\begin{itemize}')
print(f'\item Operating System: {platform.system()} {platform.release()}')
print(f'\item Python: {sys.version.split()[0]}')
print(f'\item NumPy: {np.__version__}')
print(f'\item Matplotlib: {plt.matplotlib.__version__}')
print(r'\end{itemize}')
\end{pycode}

\end{document}
