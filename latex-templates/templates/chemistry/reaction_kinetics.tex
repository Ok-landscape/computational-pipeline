% Reaction Kinetics Template
% Topics: Rate laws, Arrhenius equation, reaction mechanisms, catalysis
% Style: Research article with experimental data analysis

\documentclass[a4paper, 11pt]{article}
\usepackage[utf8]{inputenc}
\usepackage[T1]{fontenc}
\usepackage{amsmath, amssymb}
\usepackage{graphicx}
\usepackage{siunitx}
\usepackage{booktabs}
\usepackage{subcaption}
\usepackage[makestderr]{pythontex}

% Theorem environments
\newtheorem{definition}{Definition}[section]
\newtheorem{theorem}{Theorem}[section]
\newtheorem{example}{Example}[section]
\newtheorem{remark}{Remark}[section]

\title{Chemical Reaction Kinetics: Rate Laws, Mechanisms, and Catalysis}
\author{Physical Chemistry Laboratory}
\date{\today}

\begin{document}
\maketitle

\begin{abstract}
This study presents a comprehensive analysis of chemical reaction kinetics, examining
rate laws for reactions of different orders, temperature dependence through the
Arrhenius equation, and the effects of catalysis on reaction rates. We analyze
experimental concentration-time data to determine rate constants, activation energies,
and pre-exponential factors. Computational analysis demonstrates the integrated rate
laws, half-life relationships, and mechanistic interpretation of kinetic data.
\end{abstract}

\section{Introduction}

Chemical kinetics describes the rates of chemical reactions and the factors that affect
them. Understanding reaction kinetics is essential for reaction mechanism elucidation,
industrial process optimization, and pharmaceutical drug stability studies.

\begin{definition}[Rate Law]
For a reaction $aA + bB \rightarrow$ products, the rate law has the general form:
\begin{equation}
\text{Rate} = -\frac{1}{a}\frac{d[A]}{dt} = k[A]^m[B]^n
\end{equation}
where $k$ is the rate constant, and $m$, $n$ are the reaction orders.
\end{definition}

\section{Theoretical Framework}

\subsection{Integrated Rate Laws}

\begin{theorem}[Integrated Rate Laws]
For a reaction $A \rightarrow$ products with initial concentration $[A]_0$:
\begin{itemize}
\item \textbf{Zero-order}: $[A] = [A]_0 - kt$, \quad $t_{1/2} = \frac{[A]_0}{2k}$
\item \textbf{First-order}: $\ln[A] = \ln[A]_0 - kt$, \quad $t_{1/2} = \frac{\ln 2}{k}$
\item \textbf{Second-order}: $\frac{1}{[A]} = \frac{1}{[A]_0} + kt$, \quad $t_{1/2} = \frac{1}{k[A]_0}$
\end{itemize}
\end{theorem}

\subsection{Temperature Dependence}

\begin{definition}[Arrhenius Equation]
The temperature dependence of rate constants is described by:
\begin{equation}
k = A e^{-E_a/RT}
\end{equation}
where $A$ is the pre-exponential factor, $E_a$ is the activation energy, $R$ is the gas
constant, and $T$ is absolute temperature. The linearized form is:
\begin{equation}
\ln k = \ln A - \frac{E_a}{R} \cdot \frac{1}{T}
\end{equation}
\end{definition}

\begin{theorem}[Eyring Equation]
Transition state theory gives:
\begin{equation}
k = \frac{k_B T}{h} e^{-\Delta G^\ddagger/RT} = \frac{k_B T}{h} e^{\Delta S^\ddagger/R} e^{-\Delta H^\ddagger/RT}
\end{equation}
where $\Delta G^\ddagger$, $\Delta H^\ddagger$, and $\Delta S^\ddagger$ are the activation parameters.
\end{theorem}

\subsection{Catalysis}

\begin{definition}[Catalytic Effect]
A catalyst provides an alternative reaction pathway with lower activation energy $E_a^{cat} < E_a^{uncat}$.
The rate enhancement factor is:
\begin{equation}
\frac{k_{cat}}{k_{uncat}} = e^{(E_a^{uncat} - E_a^{cat})/RT}
\end{equation}
\end{definition}

\begin{remark}[Enzyme Catalysis]
Enzymes are biological catalysts that follow Michaelis-Menten kinetics:
\begin{equation}
v = \frac{V_{max}[S]}{K_m + [S]}
\end{equation}
\end{remark}

\section{Computational Analysis}

\begin{pycode}
import numpy as np
import matplotlib.pyplot as plt
from scipy.optimize import curve_fit
from scipy.stats import linregress

np.random.seed(42)

# Integrated rate laws
def zero_order(t, A0, k):
    return np.maximum(A0 - k * t, 0)

def first_order(t, A0, k):
    return A0 * np.exp(-k * t)

def second_order(t, A0, k):
    return A0 / (1 + A0 * k * t)

# True parameters
A0_true = 1.0  # mol/L
k_zero = 0.02  # mol/(L·s)
k_first = 0.05  # 1/s
k_second = 0.1  # L/(mol·s)

# Time array
t = np.linspace(0, 50, 100)

# Generate concentration data with noise
noise = 0.02
conc_zero = zero_order(t, A0_true, k_zero) * (1 + noise * np.random.randn(len(t)))
conc_first = first_order(t, A0_true, k_first) * (1 + noise * np.random.randn(len(t)))
conc_second = second_order(t, A0_true, k_second) * (1 + noise * np.random.randn(len(t)))
conc_zero = np.maximum(conc_zero, 0.01)
conc_first = np.maximum(conc_first, 0.01)
conc_second = np.maximum(conc_second, 0.01)

# Fit first-order data
ln_conc = np.log(conc_first)
slope_first, intercept_first, r_first, _, _ = linregress(t, ln_conc)
k_first_fit = -slope_first
A0_first_fit = np.exp(intercept_first)

# Fit second-order data
inv_conc = 1 / conc_second
slope_second, intercept_second, r_second, _, _ = linregress(t, inv_conc)
k_second_fit = slope_second
A0_second_fit = 1 / intercept_second

# Arrhenius analysis
T = np.array([280, 290, 300, 310, 320, 330, 340])  # K
Ea_true = 50000  # J/mol
A_true = 1e10  # 1/s
R = 8.314  # J/(mol·K)
k_arr = A_true * np.exp(-Ea_true / (R * T))
k_arr_exp = k_arr * (1 + 0.05 * np.random.randn(len(T)))

# Arrhenius fit
inv_T = 1 / T
ln_k = np.log(k_arr_exp)
slope_arr, intercept_arr, r_arr, _, _ = linregress(inv_T, ln_k)
Ea_fit = -slope_arr * R
A_fit = np.exp(intercept_arr)

# Catalysis comparison
Ea_uncat = 75000  # J/mol
Ea_cat = 45000  # J/mol
T_cat = np.linspace(250, 400, 100)
k_uncat = 1e13 * np.exp(-Ea_uncat / (R * T_cat))
k_cat = 1e13 * np.exp(-Ea_cat / (R * T_cat))
enhancement = k_cat / k_uncat

# Half-life comparison
t_range = np.linspace(0, 100, 500)
half_lives = []
for order, k in [(0, k_zero), (1, k_first), (2, k_second)]:
    if order == 0:
        t_half = A0_true / (2 * k)
    elif order == 1:
        t_half = np.log(2) / k
    else:
        t_half = 1 / (k * A0_true)
    half_lives.append(t_half)

# Create figure
fig = plt.figure(figsize=(14, 12))

# Plot 1: Concentration vs time for different orders
ax1 = fig.add_subplot(3, 3, 1)
ax1.plot(t, conc_zero, 'b-', linewidth=2, label='Zero-order')
ax1.plot(t, conc_first, 'g-', linewidth=2, label='First-order')
ax1.plot(t, conc_second, 'r-', linewidth=2, label='Second-order')
ax1.set_xlabel('Time (s)')
ax1.set_ylabel('[A] (mol/L)')
ax1.set_title('Concentration vs Time')
ax1.legend(fontsize=8)
ax1.set_ylim([0, 1.2])

# Plot 2: First-order linearization
ax2 = fig.add_subplot(3, 3, 2)
ax2.scatter(t, ln_conc, s=20, c='green', edgecolor='black', alpha=0.7)
ax2.plot(t, slope_first * t + intercept_first, 'r-', linewidth=2)
ax2.set_xlabel('Time (s)')
ax2.set_ylabel('ln[A]')
ax2.set_title(f'First-Order Plot ($R^2 = {r_first**2:.4f}$)')

# Plot 3: Second-order linearization
ax3 = fig.add_subplot(3, 3, 3)
ax3.scatter(t, inv_conc, s=20, c='red', edgecolor='black', alpha=0.7)
ax3.plot(t, slope_second * t + intercept_second, 'b-', linewidth=2)
ax3.set_xlabel('Time (s)')
ax3.set_ylabel('1/[A] (L/mol)')
ax3.set_title(f'Second-Order Plot ($R^2 = {r_second**2:.4f}$)')

# Plot 4: Arrhenius plot
ax4 = fig.add_subplot(3, 3, 4)
ax4.scatter(1000/T, ln_k, s=60, c='blue', edgecolor='black')
inv_T_fine = np.linspace(1000/350, 1000/270, 100)
ax4.plot(inv_T_fine, slope_arr * inv_T_fine/1000 + intercept_arr, 'r-', linewidth=2)
ax4.set_xlabel('1000/T (K$^{-1}$)')
ax4.set_ylabel('ln k')
ax4.set_title('Arrhenius Plot')

# Plot 5: Rate constant vs temperature
ax5 = fig.add_subplot(3, 3, 5)
ax5.semilogy(T, k_arr_exp, 'bo', markersize=8)
T_fine = np.linspace(275, 345, 100)
ax5.semilogy(T_fine, A_fit * np.exp(-Ea_fit / (R * T_fine)), 'r-', linewidth=2)
ax5.set_xlabel('Temperature (K)')
ax5.set_ylabel('k (s$^{-1}$)')
ax5.set_title('Rate Constant vs Temperature')

# Plot 6: Catalysis effect
ax6 = fig.add_subplot(3, 3, 6)
ax6.semilogy(T_cat, k_uncat, 'b-', linewidth=2, label='Uncatalyzed')
ax6.semilogy(T_cat, k_cat, 'r-', linewidth=2, label='Catalyzed')
ax6.set_xlabel('Temperature (K)')
ax6.set_ylabel('k (s$^{-1}$)')
ax6.set_title('Effect of Catalysis')
ax6.legend(fontsize=8)

# Plot 7: Rate enhancement factor
ax7 = fig.add_subplot(3, 3, 7)
ax7.semilogy(T_cat, enhancement, 'purple', linewidth=2)
ax7.set_xlabel('Temperature (K)')
ax7.set_ylabel('$k_{cat}/k_{uncat}$')
ax7.set_title('Rate Enhancement Factor')

# Plot 8: Half-life comparison
ax8 = fig.add_subplot(3, 3, 8)
A0_range = np.linspace(0.1, 2.0, 100)
t_half_zero = A0_range / (2 * k_zero)
t_half_first = np.log(2) / k_first * np.ones_like(A0_range)
t_half_second = 1 / (k_second * A0_range)
ax8.plot(A0_range, t_half_zero, 'b-', linewidth=2, label='Zero-order')
ax8.plot(A0_range, t_half_first, 'g-', linewidth=2, label='First-order')
ax8.plot(A0_range, t_half_second, 'r-', linewidth=2, label='Second-order')
ax8.set_xlabel('$[A]_0$ (mol/L)')
ax8.set_ylabel('$t_{1/2}$ (s)')
ax8.set_title('Half-Life vs Initial Concentration')
ax8.legend(fontsize=8)
ax8.set_ylim([0, 50])

# Plot 9: Energy diagram
ax9 = fig.add_subplot(3, 3, 9)
reaction_coord = np.linspace(0, 1, 100)
# Gaussian-like barrier
E_react = 0
E_prod = -30
E_ts_uncat = 50
E_ts_cat = 30
E_uncat = E_react + (E_ts_uncat - E_react) * np.exp(-((reaction_coord - 0.4)**2) / 0.02)
E_uncat[reaction_coord > 0.5] = E_ts_uncat - (E_ts_uncat - E_prod) * (reaction_coord[reaction_coord > 0.5] - 0.4) / 0.6
E_cat = E_react + (E_ts_cat - E_react) * np.exp(-((reaction_coord - 0.4)**2) / 0.02)
E_cat[reaction_coord > 0.5] = E_ts_cat - (E_ts_cat - E_prod) * (reaction_coord[reaction_coord > 0.5] - 0.4) / 0.6
ax9.plot(reaction_coord, E_uncat, 'b-', linewidth=2, label='Uncatalyzed')
ax9.plot(reaction_coord, E_cat, 'r--', linewidth=2, label='Catalyzed')
ax9.axhline(y=0, color='gray', linestyle=':', alpha=0.5)
ax9.set_xlabel('Reaction Coordinate')
ax9.set_ylabel('Energy (kJ/mol)')
ax9.set_title('Energy Diagram')
ax9.legend(fontsize=8)

plt.tight_layout()
plt.savefig('reaction_kinetics_analysis.pdf', dpi=150, bbox_inches='tight')
plt.close()
\end{pycode}

\begin{figure}[htbp]
\centering
\includegraphics[width=\textwidth]{reaction_kinetics_analysis.pdf}
\caption{Reaction kinetics analysis: (a) Concentration decay for different reaction orders;
(b-c) Linearization plots for first and second-order reactions; (d-e) Arrhenius analysis
for temperature dependence; (f-g) Catalysis effects on rate constants; (h) Half-life
dependence on initial concentration; (i) Potential energy diagram with and without catalyst.}
\label{fig:kinetics}
\end{figure}

\section{Results}

\subsection{Kinetic Parameters}

\begin{pycode}
print(r"\begin{table}[htbp]")
print(r"\centering")
print(r"\caption{Fitted Rate Constants and Kinetic Parameters}")
print(r"\begin{tabular}{lccc}")
print(r"\toprule")
print(r"Parameter & True Value & Fitted Value & Units \\")
print(r"\midrule")
print(f"$k_{{first}}$ & {k_first} & {k_first_fit:.4f} & s$^{{-1}}$ \\\\")
print(f"$k_{{second}}$ & {k_second} & {k_second_fit:.4f} & L mol$^{{-1}}$ s$^{{-1}}$ \\\\")
print(f"$E_a$ & {Ea_true/1000:.1f} & {Ea_fit/1000:.1f} & kJ/mol \\\\")
print(f"$A$ & {A_true:.2e} & {A_fit:.2e} & s$^{{-1}}$ \\\\")
print(r"\bottomrule")
print(r"\end{tabular}")
print(r"\label{tab:parameters}")
print(r"\end{table}")
\end{pycode}

\subsection{Half-Lives}

\begin{pycode}
print(r"\begin{table}[htbp]")
print(r"\centering")
print(r"\caption{Half-Lives for Different Reaction Orders}")
print(r"\begin{tabular}{lccc}")
print(r"\toprule")
print(r"Order & Formula & Value (s) & Dependence on $[A]_0$ \\")
print(r"\midrule")
print(f"Zero & $[A]_0/(2k)$ & {half_lives[0]:.1f} & Proportional \\\\")
print(f"First & $\\ln 2/k$ & {half_lives[1]:.1f} & Independent \\\\")
print(f"Second & $1/(k[A]_0)$ & {half_lives[2]:.1f} & Inversely proportional \\\\")
print(r"\bottomrule")
print(r"\end{tabular}")
print(r"\label{tab:halflives}")
print(r"\end{table}")
\end{pycode}

\section{Discussion}

\begin{example}[Determining Reaction Order]
The reaction order is determined by finding which linearization plot gives the best fit:
\begin{itemize}
\item Zero-order: $[A]$ vs $t$ is linear
\item First-order: $\ln[A]$ vs $t$ is linear
\item Second-order: $1/[A]$ vs $t$ is linear
\end{itemize}
The first-order plot has $R^2 = \py{f"{r_first**2:.4f}"}$ for the first-order data.
\end{example}

\begin{remark}[Activation Energy Interpretation]
The fitted activation energy of $\py{f"{Ea_fit/1000:.1f}"}$ kJ/mol indicates:
\begin{itemize}
\item $E_a < 40$ kJ/mol: Diffusion-controlled reaction
\item $40 < E_a < 120$ kJ/mol: Typical chemical reaction
\item $E_a > 120$ kJ/mol: High barrier, slow reaction
\end{itemize}
\end{remark}

\begin{example}[Catalytic Enhancement]
At 300 K, the rate enhancement due to catalysis is:
\begin{equation}
\frac{k_{cat}}{k_{uncat}} = e^{(75000 - 45000)/(8.314 \times 300)} = \py{f"{np.exp(30000/(8.314*300)):.0f}"}
\end{equation}
This enormous enhancement explains the importance of catalysts in industrial chemistry.
\end{example}

\section{Conclusions}

This analysis demonstrates the fundamental principles of chemical kinetics:
\begin{enumerate}
\item First-order rate constant: $k = \py{f"{k_first_fit:.4f}"}$ s$^{-1}$ with $t_{1/2} = \py{f"{half_lives[1]:.1f}"}$ s
\item Activation energy from Arrhenius plot: $E_a = \py{f"{Ea_fit/1000:.1f}"}$ kJ/mol
\item Catalysis reduces activation energy by $\py{f"{(Ea_uncat-Ea_cat)/1000:.0f}"}$ kJ/mol
\item Half-life dependence on $[A]_0$ distinguishes reaction orders
\item Linearization methods enable determination of rate laws from experimental data
\end{enumerate}

\section*{Further Reading}

\begin{itemize}
\item Atkins, P. \& de Paula, J. \textit{Physical Chemistry}, 11th ed. Oxford, 2018.
\item Houston, P.L. \textit{Chemical Kinetics and Reaction Dynamics}. Dover, 2006.
\item Steinfeld, J.I. et al. \textit{Chemical Kinetics and Dynamics}, 2nd ed. Prentice Hall, 1998.
\end{itemize}

\end{document}
