\documentclass[a4paper, 11pt]{article}
\usepackage[utf8]{inputenc}
\usepackage[T1]{fontenc}
\usepackage{amsmath, amssymb, amsthm}
\usepackage{graphicx}
\usepackage{booktabs}
\usepackage{siunitx}
\usepackage{subcaption}
\usepackage[makestderr]{pythontex}

\newtheorem{definition}{Definition}
\newtheorem{theorem}{Theorem}
\newtheorem{example}{Example}
\newtheorem{remark}{Remark}

\title{Standard Model Physics: Coupling Evolution and Grand Unification}
\author{Theoretical Physics Computation Laboratory}
\date{\today}

\begin{document}
\maketitle

\begin{abstract}
This technical report presents comprehensive computational analysis of the Standard Model gauge couplings and their renormalization group evolution. We implement one-loop and two-loop running of the electromagnetic, weak, and strong coupling constants, analyze gauge unification scenarios in the MSSM, and compute threshold corrections. The analysis addresses hierarchy problems, proton decay constraints, and predictions for beyond-Standard-Model physics.
\end{abstract}

\section{Theoretical Framework}

\begin{definition}[Gauge Couplings]
The Standard Model contains three gauge groups $SU(3)_C \times SU(2)_L \times U(1)_Y$ with couplings:
\begin{itemize}
    \item $g_1$ (hypercharge): $\alpha_1 = g_1^2/(4\pi)$
    \item $g_2$ (weak isospin): $\alpha_2 = g_2^2/(4\pi)$
    \item $g_3$ (color): $\alpha_3 = g_3^2/(4\pi)$
\end{itemize}
\end{definition}

\begin{theorem}[Renormalization Group Equations]
At one-loop, the coupling constants evolve as:
\begin{equation}
\frac{d\alpha_i^{-1}}{d\ln\mu} = -\frac{b_i}{2\pi}
\end{equation}
where the beta function coefficients are:
\begin{align}
b_1 &= \frac{41}{10}, \quad b_2 = -\frac{19}{6}, \quad b_3 = -7 \quad \text{(SM)}
\end{align}
\end{theorem}

\subsection{GUT Normalization}

\begin{definition}[SU(5) Normalization]
For embedding in SU(5), the hypercharge coupling is rescaled:
\begin{equation}
\alpha_1^{GUT} = \frac{5}{3} \alpha_1 = \frac{5}{3} \frac{\alpha_{em}}{\cos^2\theta_W}
\end{equation}
\end{definition}

\begin{example}[Weak Mixing Angle]
The electroweak mixing angle at $M_Z$ relates couplings:
\begin{equation}
\sin^2\theta_W = \frac{g_1^2}{g_1^2 + g_2^2} = \frac{\alpha_{em}}{\alpha_2}
\end{equation}
Measured value: $\sin^2\theta_W(M_Z) = 0.2312$.
\end{example}

\section{Computational Analysis}

\begin{pycode}
import numpy as np
from scipy.integrate import odeint
import matplotlib.pyplot as plt
plt.rc('text', usetex=True)
plt.rc('font', family='serif', size=10)

# Physical constants
M_Z = 91.1876  # Z boson mass (GeV)
alpha_em_MZ = 1/127.9  # Electromagnetic coupling at M_Z
sin2_theta_W = 0.23122  # Weinberg angle squared
alpha_s_MZ = 0.1179  # Strong coupling at M_Z

# Convert to GUT normalization
alpha_1_MZ = (5/3) * alpha_em_MZ / (1 - sin2_theta_W)
alpha_2_MZ = alpha_em_MZ / sin2_theta_W
alpha_3_MZ = alpha_s_MZ

# Beta function coefficients
# Standard Model
b_SM = np.array([41/10, -19/6, -7])

# MSSM (Minimal Supersymmetric Standard Model)
b_MSSM = np.array([33/5, 1, -3])

# Two-loop coefficients (SM)
b2_SM = np.array([
    [199/50, 27/10, 44/5],
    [9/10, 35/6, 12],
    [11/10, 9/2, -26]
])

def run_1loop(alpha_mZ, b, mu, mu0=M_Z):
    """One-loop running of coupling constant."""
    return 1 / (1/alpha_mZ - b/(2*np.pi) * np.log(mu/mu0))

def rge_2loop(y, t, b1, b2):
    """Two-loop RGE system."""
    alpha_inv = y
    alpha = 1 / alpha_inv

    dalpha_inv = np.zeros(3)
    for i in range(3):
        dalpha_inv[i] = -b1[i]/(2*np.pi)
        for j in range(3):
            dalpha_inv[i] -= b2[i,j]/(8*np.pi**2) * alpha[j]

    return dalpha_inv

# Energy scale range
log_mu = np.linspace(np.log10(M_Z), 17, 500)
mu = 10**log_mu

# One-loop SM running
alpha_1_SM = run_1loop(alpha_1_MZ, b_SM[0], mu)
alpha_2_SM = run_1loop(alpha_2_MZ, b_SM[1], mu)
alpha_3_SM = run_1loop(alpha_3_MZ, b_SM[2], mu)

# MSSM running (assuming SUSY scale = 1 TeV)
M_SUSY = 1000  # GeV

def run_mssm(alpha_mZ, b_sm, b_mssm, mu, mu0=M_Z, mu_susy=M_SUSY):
    """Running with SUSY threshold."""
    alpha = np.zeros_like(mu)

    # Below SUSY scale: SM running
    below = mu <= mu_susy
    alpha[below] = run_1loop(alpha_mZ, b_sm, mu[below], mu0)

    # Above SUSY scale: MSSM running
    alpha_susy = run_1loop(alpha_mZ, b_sm, mu_susy, mu0)
    above = mu > mu_susy
    alpha[above] = run_1loop(alpha_susy, b_mssm, mu[above], mu_susy)

    return alpha

alpha_1_MSSM = run_mssm(alpha_1_MZ, b_SM[0], b_MSSM[0], mu)
alpha_2_MSSM = run_mssm(alpha_2_MZ, b_SM[1], b_MSSM[1], mu)
alpha_3_MSSM = run_mssm(alpha_3_MZ, b_SM[2], b_MSSM[2], mu)

# Find unification scale (MSSM)
diff_12 = np.abs(1/alpha_1_MSSM - 1/alpha_2_MSSM)
unif_idx = np.argmin(diff_12)
M_GUT = mu[unif_idx]
alpha_GUT = alpha_1_MSSM[unif_idx]

# Strong coupling running at lower scales
mu_low = 10**np.linspace(0, 3, 100)
alpha_s_low = run_1loop(alpha_s_MZ, b_SM[2], mu_low)
alpha_s_low[alpha_s_low > 1] = np.nan  # Perturbation theory fails

# Particle masses
particle_masses = {
    'e': 0.000511,
    r'$\mu$': 0.106,
    r'$\tau$': 1.777,
    'u': 0.002,
    'd': 0.005,
    's': 0.095,
    'c': 1.27,
    'b': 4.18,
    't': 173.0,
    'W': 80.4,
    'Z': 91.2,
    'H': 125.1
}

# Yukawa couplings
def yukawa_coupling(m_fermion, v=246):
    """Yukawa coupling from fermion mass."""
    return np.sqrt(2) * m_fermion / v

y_t = yukawa_coupling(173.0)  # Top Yukawa

# Higgs self-coupling running (simplified)
lambda_H = 0.13  # At electroweak scale
lambda_H_running = lambda_H * (1 + 3*y_t**4/(8*np.pi**2) * np.log(mu/M_Z))

# Threshold corrections estimate
def threshold_correction(M_SUSY, M_GUT):
    """Approximate threshold corrections."""
    return 0.1 * np.log(M_GUT / M_SUSY)

# Proton lifetime estimate (SU(5))
def proton_lifetime(M_GUT, alpha_GUT):
    """Estimate proton lifetime in years."""
    M_p = 0.938  # Proton mass (GeV)
    tau = M_GUT**4 / (alpha_GUT**2 * M_p**5) * 1e-24 / (3.15e7)  # Convert to years
    return tau

tau_proton = proton_lifetime(M_GUT * 1e9, alpha_GUT)

# Create visualization
fig = plt.figure(figsize=(12, 10))
gs = fig.add_gridspec(3, 3, hspace=0.35, wspace=0.35)

# Plot 1: SM coupling evolution
ax1 = fig.add_subplot(gs[0, 0])
ax1.plot(log_mu, 1/alpha_1_SM, 'b-', lw=2, label=r'$\alpha_1^{-1}$')
ax1.plot(log_mu, 1/alpha_2_SM, 'g-', lw=2, label=r'$\alpha_2^{-1}$')
ax1.plot(log_mu, 1/alpha_3_SM, 'r-', lw=2, label=r'$\alpha_3^{-1}$')
ax1.set_xlabel(r'$\log_{10}(\mu/\mathrm{GeV})$')
ax1.set_ylabel(r'$\alpha_i^{-1}$')
ax1.set_title('SM Coupling Evolution')
ax1.legend(fontsize=7)
ax1.grid(True, alpha=0.3)
ax1.set_xlim([2, 17])
ax1.set_ylim([0, 70])

# Plot 2: MSSM unification
ax2 = fig.add_subplot(gs[0, 1])
ax2.plot(log_mu, 1/alpha_1_MSSM, 'b-', lw=2, label=r'$\alpha_1^{-1}$')
ax2.plot(log_mu, 1/alpha_2_MSSM, 'g-', lw=2, label=r'$\alpha_2^{-1}$')
ax2.plot(log_mu, 1/alpha_3_MSSM, 'r-', lw=2, label=r'$\alpha_3^{-1}$')
ax2.axvline(x=np.log10(M_SUSY), color='gray', ls='--', alpha=0.5)
ax2.axvline(x=np.log10(M_GUT), color='orange', ls='--', alpha=0.7, label='GUT')
ax2.set_xlabel(r'$\log_{10}(\mu/\mathrm{GeV})$')
ax2.set_ylabel(r'$\alpha_i^{-1}$')
ax2.set_title('MSSM Gauge Unification')
ax2.legend(fontsize=7, loc='upper right')
ax2.grid(True, alpha=0.3)
ax2.set_xlim([2, 17])
ax2.set_ylim([0, 70])

# Plot 3: Strong coupling detail
ax3 = fig.add_subplot(gs[0, 2])
ax3.plot(np.log10(mu_low), alpha_s_low, 'r-', lw=2)
ax3.axhline(y=0.1179, color='gray', ls='--', alpha=0.5, label=r'$\alpha_s(M_Z)$')
ax3.set_xlabel(r'$\log_{10}(\mu/\mathrm{GeV})$')
ax3.set_ylabel(r'$\alpha_s(\mu)$')
ax3.set_title('Strong Coupling Running')
ax3.legend(fontsize=8)
ax3.grid(True, alpha=0.3)
ax3.set_ylim([0, 0.5])

# Plot 4: Particle mass hierarchy
ax4 = fig.add_subplot(gs[1, 0])
names = list(particle_masses.keys())
masses = list(particle_masses.values())
colors = ['blue']*3 + ['red']*6 + ['green']*3
ax4.barh(names, np.log10(masses), color=colors, alpha=0.7)
ax4.set_xlabel(r'$\log_{10}(m/\mathrm{GeV})$')
ax4.set_title('SM Particle Masses')
ax4.grid(True, alpha=0.3, axis='x')

# Plot 5: Electroweak mixing angle running
ax5 = fig.add_subplot(gs[1, 1])
sin2_running = alpha_em_MZ / (alpha_em_MZ + alpha_2_SM * (1 - sin2_theta_W) / sin2_theta_W)
ax5.plot(log_mu, sin2_running, 'purple', lw=2)
ax5.axhline(y=0.25, color='gray', ls='--', alpha=0.5, label='SU(5) prediction')
ax5.set_xlabel(r'$\log_{10}(\mu/\mathrm{GeV})$')
ax5.set_ylabel(r'$\sin^2\theta_W$')
ax5.set_title('Weinberg Angle Running')
ax5.legend(fontsize=8)
ax5.grid(True, alpha=0.3)

# Plot 6: Higgs self-coupling (stability)
ax6 = fig.add_subplot(gs[1, 2])
ax6.plot(log_mu, lambda_H_running, 'brown', lw=2)
ax6.axhline(y=0, color='r', ls='--', alpha=0.5, label='Stability bound')
ax6.set_xlabel(r'$\log_{10}(\mu/\mathrm{GeV})$')
ax6.set_ylabel(r'$\lambda_H$')
ax6.set_title('Higgs Self-Coupling')
ax6.legend(fontsize=8)
ax6.grid(True, alpha=0.3)

# Plot 7: Beta function coefficients
ax7 = fig.add_subplot(gs[2, 0])
x_pos = [0, 1, 2]
width = 0.35
ax7.bar([x - width/2 for x in x_pos], b_SM, width, label='SM', alpha=0.7)
ax7.bar([x + width/2 for x in x_pos], b_MSSM, width, label='MSSM', alpha=0.7)
ax7.set_xticks(x_pos)
ax7.set_xticklabels([r'$b_1$', r'$b_2$', r'$b_3$'])
ax7.set_ylabel('Beta Function Coefficient')
ax7.set_title('Beta Coefficients')
ax7.legend(fontsize=8)
ax7.grid(True, alpha=0.3, axis='y')
ax7.axhline(y=0, color='gray', ls='-', alpha=0.3)

# Plot 8: Unification scale vs SUSY scale
ax8 = fig.add_subplot(gs[2, 1])
M_SUSY_range = np.logspace(2, 5, 50)
M_GUT_range = []

for M_S in M_SUSY_range:
    alpha_1_temp = run_mssm(alpha_1_MZ, b_SM[0], b_MSSM[0], mu, mu_susy=M_S)
    alpha_2_temp = run_mssm(alpha_2_MZ, b_SM[1], b_MSSM[1], mu, mu_susy=M_S)
    diff = np.abs(1/alpha_1_temp - 1/alpha_2_temp)
    idx = np.argmin(diff)
    M_GUT_range.append(mu[idx])

ax8.loglog(M_SUSY_range, M_GUT_range, 'b-', lw=2)
ax8.set_xlabel(r'$M_{SUSY}$ (GeV)')
ax8.set_ylabel(r'$M_{GUT}$ (GeV)')
ax8.set_title('GUT Scale vs SUSY Scale')
ax8.grid(True, alpha=0.3, which='both')

# Plot 9: Gauge coupling ratios
ax9 = fig.add_subplot(gs[2, 2])
ratio_12 = alpha_1_MSSM / alpha_2_MSSM
ratio_23 = alpha_2_MSSM / alpha_3_MSSM

ax9.plot(log_mu, ratio_12, 'b-', lw=1.5, label=r'$\alpha_1/\alpha_2$')
ax9.plot(log_mu, ratio_23, 'r--', lw=1.5, label=r'$\alpha_2/\alpha_3$')
ax9.axhline(y=1, color='gray', ls='--', alpha=0.5)
ax9.set_xlabel(r'$\log_{10}(\mu/\mathrm{GeV})$')
ax9.set_ylabel('Coupling Ratio')
ax9.set_title('Gauge Coupling Ratios')
ax9.legend(fontsize=7)
ax9.grid(True, alpha=0.3)

plt.savefig('standard_model_plot.pdf', bbox_inches='tight', dpi=150)
print(r'\begin{center}')
print(r'\includegraphics[width=\textwidth]{standard_model_plot.pdf}')
print(r'\end{center}')
plt.close()
\end{pycode}

\section{Results and Analysis}

\subsection{Coupling Constants at $M_Z$}

\begin{pycode}
print(r'\begin{table}[htbp]')
print(r'\centering')
print(r'\caption{Gauge Couplings at $M_Z = 91.2$ GeV}')
print(r'\begin{tabular}{lccc}')
print(r'\toprule')
print(r'Coupling & Value & $\alpha_i^{-1}$ & Physical Origin \\')
print(r'\midrule')
print(f'$\\alpha_1$ (GUT norm.) & {alpha_1_MZ:.4f} & {1/alpha_1_MZ:.1f} & U(1)$_Y$ hypercharge \\\\')
print(f'$\\alpha_2$ & {alpha_2_MZ:.4f} & {1/alpha_2_MZ:.1f} & SU(2)$_L$ weak isospin \\\\')
print(f'$\\alpha_3$ & {alpha_3_MZ:.4f} & {1/alpha_3_MZ:.1f} & SU(3)$_C$ color \\\\')
print(f'$\\alpha_{{em}}$ & {alpha_em_MZ:.5f} & {1/alpha_em_MZ:.1f} & U(1)$_{{em}}$ \\\\')
print(r'\bottomrule')
print(r'\end{tabular}')
print(r'\end{table}')
\end{pycode}

\subsection{Unification Parameters}

\begin{pycode}
print(r'\begin{table}[htbp]')
print(r'\centering')
print(r'\caption{MSSM Unification Parameters}')
print(r'\begin{tabular}{lcc}')
print(r'\toprule')
print(r'Parameter & Value & Units \\')
print(r'\midrule')
print(f'SUSY scale & {M_SUSY:.0f} & GeV \\\\')
print(f'GUT scale & {M_GUT:.1e} & GeV \\\\')
print(f'Unified coupling $\\alpha_{{GUT}}$ & {alpha_GUT:.4f} & -- \\\\')
print(f'$\\alpha_{{GUT}}^{{-1}}$ & {1/alpha_GUT:.1f} & -- \\\\')
print(f'Proton lifetime (est.) & {tau_proton:.1e} & years \\\\')
print(r'\bottomrule')
print(r'\end{tabular}')
print(r'\end{table}')
\end{pycode}

\begin{remark}
In the Standard Model, the three couplings do not meet at a single point. Supersymmetry modifies the beta functions above $M_{SUSY}$, enabling gauge unification at $M_{GUT} \sim 10^{16}$ GeV.
\end{remark}

\subsection{Beta Function Comparison}

\begin{pycode}
print(r'\begin{table}[htbp]')
print(r'\centering')
print(r'\caption{Beta Function Coefficients}')
print(r'\begin{tabular}{lccc}')
print(r'\toprule')
print(r'Model & $b_1$ & $b_2$ & $b_3$ \\')
print(r'\midrule')
print(f'Standard Model & {b_SM[0]:.2f} & {b_SM[1]:.2f} & {b_SM[2]:.2f} \\\\')
print(f'MSSM & {b_MSSM[0]:.2f} & {b_MSSM[1]:.2f} & {b_MSSM[2]:.2f} \\\\')
print(r'\bottomrule')
print(r'\end{tabular}')
print(r'\end{table}')
\end{pycode}

\section{Beyond the Standard Model}

\begin{theorem}[Proton Decay]
In SU(5) GUT, proton decay via $X$ boson exchange gives lifetime:
\begin{equation}
\tau_p \propto \frac{M_X^4}{\alpha_{GUT}^2 m_p^5}
\end{equation}
Current experimental limit: $\tau_p > 10^{34}$ years constrains $M_{GUT}$.
\end{theorem}

\begin{example}[Hierarchy Problem]
The Higgs mass receives quadratically divergent corrections:
\begin{equation}
\delta m_H^2 \sim \frac{\Lambda^2}{16\pi^2}
\end{equation}
SUSY cancels these via boson-fermion loop contributions.
\end{example}

\section{Discussion}

The Standard Model analysis reveals:

\begin{enumerate}
    \item \textbf{Asymptotic freedom}: $\alpha_3$ decreases at high energies ($b_3 < 0$).
    \item \textbf{Incomplete unification}: SM couplings miss at high scale.
    \item \textbf{SUSY solution}: MSSM achieves unification with $M_{GUT} \sim 10^{16}$ GeV.
    \item \textbf{Threshold effects}: Particle masses near SUSY scale affect running.
    \item \textbf{Vacuum stability}: Top Yukawa drives $\lambda_H$ potentially negative.
\end{enumerate}

\section{Conclusions}

This computational analysis demonstrates:
\begin{itemize}
    \item Strong coupling at $M_Z$: $\alpha_s = \py{f"{alpha_s_MZ:.4f}"}$
    \item Weinberg angle: $\sin^2\theta_W = \py{f"{sin2_theta_W:.4f}"}$
    \item MSSM GUT scale: $\py{f"{M_GUT:.1e}"}$ GeV
    \item Unified coupling: $\alpha_{GUT}^{-1} \approx \py{f"{1/alpha_GUT:.0f}"}$
\end{itemize}

The running of gauge couplings provides crucial tests of the Standard Model and guides searches for new physics at the energy frontier.

\section{Further Reading}
\begin{itemize}
    \item Georgi, H., Quinn, H.R., Weinberg, S., Hierarchy of interactions in unified gauge theories, \textit{Phys. Rev. Lett.} 33, 451 (1974)
    \item Langacker, P., Grand unified theories and proton decay, \textit{Phys. Rep.} 72, 185 (1981)
    \item Martin, S.P., A Supersymmetry Primer, \textit{Adv. Ser. Direct. High Energy Phys.} 21, 1 (2010)
\end{itemize}

\end{document}
