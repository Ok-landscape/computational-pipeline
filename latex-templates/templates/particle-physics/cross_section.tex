\documentclass[a4paper, 11pt]{article}
\usepackage[utf8]{inputenc}
\usepackage[T1]{fontenc}
\usepackage{amsmath, amssymb, amsthm}
\usepackage{graphicx}
\usepackage{booktabs}
\usepackage{siunitx}
\usepackage{subcaption}
\usepackage[makestderr]{pythontex}

\newtheorem{definition}{Definition}
\newtheorem{theorem}{Theorem}
\newtheorem{example}{Example}
\newtheorem{remark}{Remark}

\title{Particle Scattering Cross Sections: Rutherford, Mott, and Form Factor Analysis}
\author{High Energy Physics Computation Laboratory}
\date{\today}

\begin{document}
\maketitle

\begin{abstract}
This technical report presents comprehensive computational analysis of particle scattering cross sections. We implement the Rutherford formula for classical Coulomb scattering, Mott cross section with relativistic and spin corrections, and nuclear form factors for extended charge distributions. The analysis covers differential and integrated cross sections, structure functions, and momentum transfer dependence essential for understanding particle interactions and nuclear structure.
\end{abstract}

\section{Theoretical Framework}

\begin{definition}[Differential Cross Section]
The differential cross section $d\sigma/d\Omega$ gives the probability per unit solid angle for scattering into direction $(\theta, \phi)$:
\begin{equation}
\frac{dN}{dt} = I_0 n \frac{d\sigma}{d\Omega} d\Omega
\end{equation}
where $I_0$ is incident flux and $n$ is target number density.
\end{definition}

\begin{theorem}[Rutherford Scattering]
For Coulomb scattering of a particle with charge $Z_1 e$ from a nucleus with charge $Z_2 e$:
\begin{equation}
\frac{d\sigma}{d\Omega}_{Ruth} = \left(\frac{Z_1 Z_2 \alpha \hbar c}{4 E_k \sin^2(\theta/2)}\right)^2
\end{equation}
where $\alpha \approx 1/137$ is the fine structure constant.
\end{theorem}

\subsection{Relativistic and Quantum Corrections}

\begin{theorem}[Mott Cross Section]
For relativistic electron scattering, the Mott formula includes spin effects:
\begin{equation}
\frac{d\sigma}{d\Omega}_{Mott} = \frac{d\sigma}{d\Omega}_{Ruth} \left(1 - \beta^2 \sin^2(\theta/2)\right)
\end{equation}
where $\beta = v/c$.
\end{theorem}

\begin{example}[Nuclear Form Factor]
For extended nuclear charge distribution $\rho(r)$:
\begin{equation}
\frac{d\sigma}{d\Omega} = \frac{d\sigma}{d\Omega}_{Mott} |F(q)|^2
\end{equation}
where $F(q) = \int \rho(r) e^{i\mathbf{q}\cdot\mathbf{r}} d^3r$ is the form factor and $q = 2k\sin(\theta/2)$ is momentum transfer.
\end{example}

\section{Computational Analysis}

\begin{pycode}
import numpy as np
from scipy.special import spherical_jn
import matplotlib.pyplot as plt
plt.rc('text', usetex=True)
plt.rc('font', family='serif', size=10)

# Physical constants
alpha = 1/137.036  # Fine structure constant
hbar_c = 197.327   # MeV fm
m_e = 0.511        # Electron mass (MeV/c^2)
fm_to_barn = 100   # fm^2 to barn

def rutherford_cross_section(theta, E_lab, Z1, Z2):
    """Rutherford differential cross section (fm^2/sr)."""
    sin2 = np.sin(theta/2)**2
    sin2 = np.maximum(sin2, 1e-10)  # Avoid singularity
    return (Z1 * Z2 * alpha * hbar_c / (4 * E_lab * sin2))**2

def mott_cross_section(theta, E_lab, Z1, Z2, m_proj):
    """Mott differential cross section with relativistic correction."""
    gamma = E_lab / m_proj
    beta = np.sqrt(1 - 1/gamma**2) if gamma > 1 else 0
    ruth = rutherford_cross_section(theta, E_lab, Z1, Z2)
    return ruth * (1 - beta**2 * np.sin(theta/2)**2)

def momentum_transfer(theta, E_lab, m_proj):
    """Calculate momentum transfer q in fm^-1."""
    gamma = E_lab / m_proj
    p = np.sqrt(E_lab**2 - m_proj**2)  # MeV/c
    q = 2 * p * np.sin(theta/2) / hbar_c  # fm^-1
    return q

# Nuclear form factors
def uniform_sphere_ff(q, R):
    """Form factor for uniform sphere charge distribution."""
    qR = q * R
    # Avoid numerical issues at qR = 0
    qR = np.maximum(qR, 1e-10)
    return 3 * (np.sin(qR) - qR * np.cos(qR)) / qR**3

def gaussian_ff(q, a):
    """Form factor for Gaussian charge distribution."""
    return np.exp(-q**2 * a**2 / 6)

def fermi_ff(q, c, a):
    """Form factor for Woods-Saxon (Fermi) distribution."""
    # Approximate form factor using parametrization
    R_rms = np.sqrt(0.6) * c  # RMS radius approximation
    return uniform_sphere_ff(q, R_rms)

# Target parameters
Z_target = 79  # Gold
A_target = 197
R_Au = 1.2 * A_target**(1/3)  # Nuclear radius (fm)
R_rms = 5.33  # RMS charge radius of Au (fm)

# Carbon for form factor analysis
Z_C = 6
A_C = 12
R_C = 1.2 * A_C**(1/3)

# Scattering angles
theta = np.linspace(0.01, np.pi, 500)

# Beam energies (MeV)
energies = [50, 100, 200, 500]

# Calculate cross sections for different energies
cs_ruth = []
cs_mott = []
for E in energies:
    ruth = rutherford_cross_section(theta, E, 1, Z_target)
    mott = mott_cross_section(theta, E, 1, Z_target, m_e)
    cs_ruth.append(ruth)
    cs_mott.append(mott)

# Form factor analysis
E_form = 500  # MeV
q_values = momentum_transfer(theta, E_form, m_e)

# Different form factors
ff_uniform = uniform_sphere_ff(q_values, R_C)
ff_gaussian = gaussian_ff(q_values, R_C / np.sqrt(5/3))

# Cross section with form factor
cs_mott_ff = mott_cross_section(theta, E_form, 1, Z_C, m_e)
cs_with_ff = cs_mott_ff * ff_uniform**2

# Integrated cross section (above minimum angle)
def integrated_cross_section(theta_min, E_lab, Z1, Z2, m_proj):
    """Integrated cross section for theta > theta_min."""
    theta_int = np.linspace(theta_min, np.pi, 1000)
    dsigma = mott_cross_section(theta_int, E_lab, Z1, Z2, m_proj)
    # Integrate over solid angle
    sigma = 2 * np.pi * np.trapz(dsigma * np.sin(theta_int), theta_int)
    return sigma

# Total cross section vs energy
E_range = np.logspace(1, 4, 100)
theta_min = np.deg2rad(1)  # Minimum angle
sigma_total = [integrated_cross_section(theta_min, E, 1, Z_target, m_e) for E in E_range]

# Mandelstam variables
def mandelstam_t(theta, E_lab, m_proj, m_target):
    """Calculate Mandelstam t variable."""
    s = m_target**2 + 2 * E_lab * m_target  # Assuming target at rest
    p_cm_sq = (s - (m_proj + m_target)**2) * (s - (m_proj - m_target)**2) / (4*s)
    t = -2 * p_cm_sq * (1 - np.cos(theta))
    return t

# Create visualization
fig = plt.figure(figsize=(12, 10))
gs = fig.add_gridspec(3, 3, hspace=0.35, wspace=0.35)

# Plot 1: Rutherford cross section at different energies
ax1 = fig.add_subplot(gs[0, 0])
colors = plt.cm.viridis(np.linspace(0.2, 0.8, len(energies)))
for E, cs, color in zip(energies, cs_ruth, colors):
    ax1.semilogy(np.rad2deg(theta), cs * fm_to_barn, color=color,
                 lw=1.5, label=f'{E} MeV')
ax1.set_xlabel('Scattering Angle (deg)')
ax1.set_ylabel(r'$d\sigma/d\Omega$ (barn/sr)')
ax1.set_title('Rutherford Cross Section (Au)')
ax1.legend(fontsize=7)
ax1.grid(True, alpha=0.3, which='both')
ax1.set_xlim([0, 180])

# Plot 2: Mott vs Rutherford
ax2 = fig.add_subplot(gs[0, 1])
E_compare = 200
ruth_200 = rutherford_cross_section(theta, E_compare, 1, Z_target)
mott_200 = mott_cross_section(theta, E_compare, 1, Z_target, m_e)

ax2.semilogy(np.rad2deg(theta), ruth_200 * fm_to_barn, 'b-', lw=2, label='Rutherford')
ax2.semilogy(np.rad2deg(theta), mott_200 * fm_to_barn, 'r--', lw=2, label='Mott')
ax2.set_xlabel('Scattering Angle (deg)')
ax2.set_ylabel(r'$d\sigma/d\Omega$ (barn/sr)')
ax2.set_title(f'Rutherford vs Mott ({E_compare} MeV)')
ax2.legend(fontsize=8)
ax2.grid(True, alpha=0.3, which='both')

# Plot 3: Mott/Rutherford ratio
ax3 = fig.add_subplot(gs[0, 2])
ratio = mott_200 / ruth_200
ax3.plot(np.rad2deg(theta), ratio, 'g-', lw=2)
ax3.set_xlabel('Scattering Angle (deg)')
ax3.set_ylabel('Mott / Rutherford')
ax3.set_title('Relativistic Correction')
ax3.grid(True, alpha=0.3)
ax3.set_ylim([0, 1.1])
ax3.set_xlim([0, 180])

# Plot 4: Form factors
ax4 = fig.add_subplot(gs[1, 0])
ax4.plot(q_values, np.abs(ff_uniform)**2, 'b-', lw=2, label='Uniform sphere')
ax4.plot(q_values, np.abs(ff_gaussian)**2, 'r--', lw=1.5, label='Gaussian')
ax4.set_xlabel('Momentum Transfer $q$ (fm$^{-1}$)')
ax4.set_ylabel('$|F(q)|^2$')
ax4.set_title('Nuclear Form Factors (C-12)')
ax4.legend(fontsize=8)
ax4.grid(True, alpha=0.3)
ax4.set_xlim([0, 4])
ax4.set_ylim([0, 1.1])

# Plot 5: Cross section with form factor
ax5 = fig.add_subplot(gs[1, 1])
ax5.semilogy(np.rad2deg(theta), cs_mott_ff * fm_to_barn, 'b-', lw=2, label='Point nucleus')
ax5.semilogy(np.rad2deg(theta), cs_with_ff * fm_to_barn, 'r-', lw=2, label='With $F(q)$')
ax5.set_xlabel('Scattering Angle (deg)')
ax5.set_ylabel(r'$d\sigma/d\Omega$ (barn/sr)')
ax5.set_title(f'Form Factor Effect (C-12, {E_form} MeV)')
ax5.legend(fontsize=8)
ax5.grid(True, alpha=0.3, which='both')

# Plot 6: Cross section vs q^2
ax6 = fig.add_subplot(gs[1, 2])
q2 = q_values**2
ax6.semilogy(q2, cs_with_ff * fm_to_barn, 'b-', lw=2)
ax6.set_xlabel('$q^2$ (fm$^{-2}$)')
ax6.set_ylabel(r'$d\sigma/d\Omega$ (barn/sr)')
ax6.set_title('Cross Section vs $q^2$')
ax6.grid(True, alpha=0.3, which='both')
ax6.set_xlim([0, 4])

# Plot 7: Integrated cross section vs energy
ax7 = fig.add_subplot(gs[2, 0])
ax7.loglog(E_range, np.array(sigma_total) * fm_to_barn, 'b-', lw=2)
ax7.set_xlabel('Beam Energy (MeV)')
ax7.set_ylabel(r'$\sigma$ (barn)')
ax7.set_title(f'Integrated Cross Section ($\\theta > 1^\\circ$)')
ax7.grid(True, alpha=0.3, which='both')

# Plot 8: Angular distribution (polar plot)
ax8 = fig.add_subplot(gs[2, 1], projection='polar')
for E, cs, color in zip([100, 500], [cs_mott[1], cs_mott[3]], ['blue', 'red']):
    cs_norm = cs / np.max(cs)
    ax8.plot(theta, cs_norm, color=color, lw=1.5, label=f'{E} MeV')
ax8.set_title('Angular Distribution')
ax8.legend(fontsize=7, loc='lower right')

# Plot 9: Mandelstam t distribution
ax9 = fig.add_subplot(gs[2, 2])
m_target_Au = A_target * 931.5  # MeV
t_200 = mandelstam_t(theta, 200, m_e, m_target_Au)
ax9.semilogy(np.rad2deg(theta), np.abs(t_200), 'purple', lw=2)
ax9.set_xlabel('Scattering Angle (deg)')
ax9.set_ylabel('$|t|$ (MeV$^2$)')
ax9.set_title('Momentum Transfer Squared')
ax9.grid(True, alpha=0.3, which='both')

plt.savefig('cross_section_plot.pdf', bbox_inches='tight', dpi=150)
print(r'\begin{center}')
print(r'\includegraphics[width=\textwidth]{cross_section_plot.pdf}')
print(r'\end{center}')
plt.close()

# Summary calculations
dsigma_90_ruth = rutherford_cross_section(np.pi/2, 100, 1, Z_target)
dsigma_90_mott = mott_cross_section(np.pi/2, 100, 1, Z_target, m_e)
ratio_90 = dsigma_90_mott / dsigma_90_ruth
ff_first_min = np.deg2rad(180 * 4.49 / (q_values[-1] * R_C))  # First zero of j_1
\end{pycode}

\section{Results and Analysis}

\subsection{Cross Section Scaling}

\begin{pycode}
print(r'\begin{table}[htbp]')
print(r'\centering')
print(r'\caption{Differential Cross Sections at 90$^\circ$ (Gold Target)}')
print(r'\begin{tabular}{ccccc}')
print(r'\toprule')
print(r'Energy (MeV) & Rutherford & Mott & Ratio & $\beta$ \\')
print(r'\midrule')

for E in energies:
    ruth = rutherford_cross_section(np.pi/2, E, 1, Z_target) * fm_to_barn
    mott = mott_cross_section(np.pi/2, E, 1, Z_target, m_e) * fm_to_barn
    gamma = E / m_e
    beta = np.sqrt(1 - 1/gamma**2) if gamma > 1 else 0
    r = mott/ruth
    print(f'{E} & {ruth:.2e} & {mott:.2e} & {r:.3f} & {beta:.3f} \\\\')

print(r'\bottomrule')
print(r'\end{tabular}')
print(r'\end{table}')
\end{pycode}

\subsection{Key Cross Section Features}

The analysis reveals:
\begin{itemize}
    \item Rutherford formula: $d\sigma/d\Omega \propto \sin^{-4}(\theta/2)$ singularity at $\theta = 0$
    \item Mott correction: reduces backscattering by factor $(1 - \beta^2)$ at 180$^\circ$
    \item Energy scaling: $\sigma \propto E^{-2}$ for Coulomb scattering
    \item Form factor: creates diffraction minima at $q \cdot R \approx 4.49$
\end{itemize}

\begin{remark}
The Mott/Rutherford ratio at 90$^\circ$ for 100 MeV electrons is \py{f"{ratio_90:.3f}"}, showing significant relativistic suppression.
\end{remark}

\subsection{Form Factor Analysis}

\begin{pycode}
print(r'\begin{table}[htbp]')
print(r'\centering')
print(r'\caption{Nuclear Form Factor Parameters}')
print(r'\begin{tabular}{lccc}')
print(r'\toprule')
print(r'Nucleus & $R$ (fm) & $R_{rms}$ (fm) & First minimum ($q$) \\')
print(r'\midrule')
print(f'C-12 & {R_C:.2f} & {R_C * np.sqrt(3/5):.2f} & {4.49/R_C:.2f} fm$^{{-1}}$ \\\\')
print(f'Au-197 & {R_Au:.2f} & {R_rms:.2f} & {4.49/R_Au:.2f} fm$^{{-1}}$ \\\\')
print(r'\bottomrule')
print(r'\end{tabular}')
print(r'\end{table}')
\end{pycode}

\section{Applications}

\begin{example}[Nuclear Size Measurement]
Electron scattering experiments determine nuclear charge radii through form factor analysis. The position of diffraction minima in $d\sigma/d\Omega$ directly gives the nuclear radius: $R \approx 4.49/q_{min}$.
\end{example}

\begin{example}[Deep Inelastic Scattering]
At high $Q^2 = -t$, structure functions $F_1(x, Q^2)$ and $F_2(x, Q^2)$ probe quark distributions inside nucleons, revealing parton dynamics.
\end{example}

\begin{theorem}[Optical Theorem]
The total cross section relates to the forward scattering amplitude:
\begin{equation}
\sigma_{tot} = \frac{4\pi}{k} \text{Im}[f(0)]
\end{equation}
This connects elastic and inelastic scattering processes.
\end{theorem}

\section{Discussion}

The cross section analysis demonstrates:

\begin{enumerate}
    \item \textbf{Classical limit}: Rutherford formula recovers Coulomb scattering with $d\sigma \propto Z^2/E^2$.
    \item \textbf{Relativistic effects}: Mott correction accounts for electron spin-orbit coupling.
    \item \textbf{Nuclear structure}: Form factors encode charge distribution information.
    \item \textbf{Diffraction}: Wave nature of matter creates interference patterns in angular distribution.
    \item \textbf{Scale dependence}: Different energies probe different length scales via $\lambda = \hbar c/E$.
\end{enumerate}

\section{Conclusions}

This computational analysis demonstrates:
\begin{itemize}
    \item Cross section at 90$^\circ$ (100 MeV, Au): \py{f"{dsigma_90_mott * fm_to_barn:.2e}"} barn/sr
    \item Mott/Rutherford ratio (100 MeV): \py{f"{ratio_90:.3f}"}
    \item Gold nuclear radius: \py{f"{R_Au:.2f}"} fm
    \item Carbon nuclear radius: \py{f"{R_C:.2f}"} fm
\end{itemize}

Cross section measurements form the foundation of particle physics experiments, enabling precise determination of fundamental interactions and nuclear structure.

\section{Further Reading}
\begin{itemize}
    \item Povh, B., Rith, K., Scholz, C., Zetsche, F., \textit{Particles and Nuclei}, 7th Ed., Springer, 2015
    \item Hofstadter, R., Electron scattering and nuclear structure, \textit{Rev. Mod. Phys.} 28, 214 (1956)
    \item Halzen, F., Martin, A.D., \textit{Quarks and Leptons}, Wiley, 1984
\end{itemize}

\end{document}
