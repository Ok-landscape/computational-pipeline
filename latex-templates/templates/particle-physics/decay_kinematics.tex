\documentclass[a4paper, 11pt]{article}
\usepackage[utf8]{inputenc}
\usepackage[T1]{fontenc}
\usepackage{amsmath, amssymb, amsthm}
\usepackage{graphicx}
\usepackage{booktabs}
\usepackage{siunitx}
\usepackage{subcaption}
\usepackage[makestderr]{pythontex}

\newtheorem{definition}{Definition}
\newtheorem{theorem}{Theorem}
\newtheorem{example}{Example}
\newtheorem{remark}{Remark}

\title{Particle Decay Kinematics: Two-Body and Three-Body Phase Space Analysis}
\author{High Energy Physics Computation Laboratory}
\date{\today}

\begin{document}
\maketitle

\begin{abstract}
This technical report presents comprehensive computational analysis of relativistic decay kinematics for unstable particles. We implement energy-momentum conservation for two-body and three-body decays, compute Lorentz transformations between rest and laboratory frames, analyze Dalitz plots for three-body phase space, and calculate decay widths from matrix elements. Applications include particle identification, resonance analysis, and detector design optimization.
\end{abstract}

\section{Theoretical Framework}

\begin{definition}[Invariant Mass]
For a system of particles with four-momenta $p_i^\mu$, the invariant mass is:
\begin{equation}
M^2 c^4 = \left(\sum_i p_i^\mu\right) \left(\sum_j p_{j\mu}\right) = \left(\sum_i E_i\right)^2 - \left(\sum_i \mathbf{p}_i\right)^2 c^2
\end{equation}
\end{definition}

\begin{theorem}[Two-Body Decay]
For decay $A \to B + C$ in the rest frame of $A$:
\begin{align}
E_B &= \frac{m_A^2 + m_B^2 - m_C^2}{2m_A} c^2 \\
|\mathbf{p}| &= \frac{c}{2m_A} \sqrt{\lambda(m_A^2, m_B^2, m_C^2)}
\end{align}
where $\lambda(x,y,z) = x^2 + y^2 + z^2 - 2xy - 2yz - 2zx$ is the K\"allen function.
\end{theorem}

\subsection{Lorentz Transformations}

\begin{definition}[Lorentz Boost]
For boost along z-axis with velocity $\beta c$:
\begin{align}
E' &= \gamma(E + \beta c p_z) \\
p_z' &= \gamma(p_z + \beta E/c)
\end{align}
where $\gamma = (1 - \beta^2)^{-1/2}$.
\end{definition}

\begin{example}[Dalitz Plot]
For three-body decay $A \to 1 + 2 + 3$, the Dalitz plot shows the distribution in:
\begin{equation}
m_{12}^2 = (p_1 + p_2)^2, \quad m_{23}^2 = (p_2 + p_3)^2
\end{equation}
Resonances appear as bands in this plot.
\end{example}

\section{Computational Analysis}

\begin{pycode}
import numpy as np
import matplotlib.pyplot as plt
from mpl_toolkits.mplot3d import Axes3D
plt.rc('text', usetex=True)
plt.rc('font', family='serif', size=10)

# Particle masses (MeV/c^2)
particles = {
    'pion+': 139.57,
    'pion0': 134.98,
    'kaon+': 493.68,
    'kaon0': 497.61,
    'muon': 105.66,
    'electron': 0.511,
    'nu_e': 0,
    'nu_mu': 0,
    'proton': 938.27,
    'neutron': 939.57,
    'eta': 547.86,
    'D+': 1869.65,
    'D0': 1864.84,
    'B+': 5279.34,
    'Jpsi': 3096.90,
    'phi': 1019.46
}

def kallen(x, y, z):
    """Kallen triangle function."""
    return x**2 + y**2 + z**2 - 2*x*y - 2*y*z - 2*z*x

def two_body_decay(m_parent, m_d1, m_d2):
    """Two-body decay kinematics in parent rest frame."""
    if m_parent < m_d1 + m_d2:
        return 0, 0, 0

    E1 = (m_parent**2 + m_d1**2 - m_d2**2) / (2 * m_parent)
    E2 = (m_parent**2 + m_d2**2 - m_d1**2) / (2 * m_parent)
    p = np.sqrt(kallen(m_parent**2, m_d1**2, m_d2**2)) / (2 * m_parent)

    return E1, E2, p

def lorentz_boost(E, p_par, p_perp, beta):
    """Lorentz boost along parallel direction."""
    gamma = 1 / np.sqrt(1 - beta**2)
    E_lab = gamma * (E + beta * p_par)
    p_par_lab = gamma * (p_par + beta * E)
    p_perp_lab = p_perp
    return E_lab, p_par_lab, p_perp_lab

def lab_frame_kinematics(m_parent, m_d1, m_d2, gamma_parent, theta_cm):
    """Calculate daughter kinematics in lab frame."""
    E_cm, _, p_cm = two_body_decay(m_parent, m_d1, m_d2)

    beta = np.sqrt(1 - 1/gamma_parent**2)

    p_par = p_cm * np.cos(theta_cm)
    p_perp = p_cm * np.sin(theta_cm)

    E_lab, p_par_lab, p_perp_lab = lorentz_boost(E_cm, p_par, p_perp, beta)

    p_lab = np.sqrt(p_par_lab**2 + p_perp_lab**2)
    theta_lab = np.arctan2(p_perp_lab, p_par_lab)

    return E_lab, p_lab, theta_lab

# Example: pion decay pi+ -> mu+ + nu_mu
m_pi = particles['pion+']
m_mu = particles['muon']
m_nu = particles['nu_mu']

E_mu_cm, E_nu_cm, p_decay = two_body_decay(m_pi, m_mu, m_nu)
KE_mu = E_mu_cm - m_mu

# Kaon decay K+ -> pi+ + pi0
m_K = particles['kaon+']
m_pi_p = particles['pion+']
m_pi_0 = particles['pion0']

E_pip_cm, E_pi0_cm, p_K_decay = two_body_decay(m_K, m_pi_p, m_pi_0)

# Three-body decay: D+ -> K- pi+ pi+
m_D = particles['D+']
m_K_d = particles['kaon+']
m_pi_d = particles['pion+']

# Phase space boundaries
m12_min = (m_K_d + m_pi_d)**2
m12_max = (m_D - m_pi_d)**2
m23_min = (m_pi_d + m_pi_d)**2
m23_max = (m_D - m_K_d)**2

# Generate Dalitz plot points (uniform phase space)
np.random.seed(42)
n_events = 5000

# Sample uniformly in m12^2, m23^2
m12_sq = np.random.uniform(m12_min, m12_max, n_events * 3)
m23_sq = np.random.uniform(m23_min, m23_max, n_events * 3)

# Check kinematic boundaries
E1_star = (m_D**2 + m_K_d**2 - m12_sq) / (2 * m_D)
E3_star = (m_D**2 + m_pi_d**2 - m23_sq) / (2 * m_D)

m12 = np.sqrt(m12_sq)
m23 = np.sqrt(m23_sq)

# Kinematic constraint: m13^2 determined
m13_sq = m_D**2 + m_K_d**2 + 2*m_pi_d**2 - m12_sq - m23_sq

# Physical boundaries
physical = (m13_sq > (m_K_d + m_pi_d)**2) & (m13_sq < (m_D - m_pi_d)**2)
m12_sq_phys = m12_sq[physical][:n_events]
m23_sq_phys = m23_sq[physical][:n_events]

# Lab frame analysis: boosted pion
gamma_pi = np.array([1, 2, 5, 10, 50])
theta_cm = np.linspace(0, np.pi, 100)

# Common decay modes table
decay_modes = [
    ('$\\pi^+ \\to \\mu^+ \\nu_\\mu$', m_pi, m_mu, m_nu),
    ('$K^+ \\to \\pi^+ \\pi^0$', m_K, m_pi_p, m_pi_0),
    ('$K^+ \\to \\mu^+ \\nu_\\mu$', m_K, m_mu, m_nu),
    ('$D^0 \\to K^- \\pi^+$', particles['D0'], m_K_d, m_pi_p),
    ('$B^+ \\to J/\\psi K^+$', particles['B+'], particles['Jpsi'], m_K_d)
]

# Q-value and phase space
def phase_space_factor(m_parent, m_d1, m_d2):
    """Two-body phase space factor."""
    p = np.sqrt(kallen(m_parent**2, m_d1**2, m_d2**2)) / (2 * m_parent)
    return p / m_parent

# Create visualization
fig = plt.figure(figsize=(12, 10))
gs = fig.add_gridspec(3, 3, hspace=0.35, wspace=0.35)

# Plot 1: Two-body decay energies
ax1 = fig.add_subplot(gs[0, 0])
names = [d[0] for d in decay_modes]
E_d1 = [two_body_decay(d[1], d[2], d[3])[0] for d in decay_modes]
E_d2 = [two_body_decay(d[1], d[2], d[3])[1] for d in decay_modes]

x_pos = range(len(names))
width = 0.35
ax1.bar([x - width/2 for x in x_pos], E_d1, width, label='Daughter 1', alpha=0.7)
ax1.bar([x + width/2 for x in x_pos], E_d2, width, label='Daughter 2', alpha=0.7)
ax1.set_xticks(x_pos)
ax1.set_xticklabels(['$\\pi$', 'K$\\to\\pi$', 'K$\\to\\mu$', 'D', 'B'], fontsize=8)
ax1.set_ylabel('Energy (MeV)')
ax1.set_title('Rest Frame Energies')
ax1.legend(fontsize=7)
ax1.set_yscale('log')
ax1.grid(True, alpha=0.3, axis='y')

# Plot 2: Lab frame energy vs angle
ax2 = fig.add_subplot(gs[0, 1])
colors = plt.cm.viridis(np.linspace(0.2, 0.8, len(gamma_pi)))
for gamma, color in zip([2, 5, 10], colors[1:4]):
    E_lab_vals = []
    theta_lab_vals = []
    for th in theta_cm:
        E, p, th_lab = lab_frame_kinematics(m_pi, m_mu, m_nu, gamma, th)
        E_lab_vals.append(E)
        theta_lab_vals.append(th_lab)
    ax2.plot(np.rad2deg(theta_lab_vals), E_lab_vals, color=color,
             lw=1.5, label=f'$\\gamma = {gamma}$')
ax2.set_xlabel('Lab Angle (deg)')
ax2.set_ylabel('Muon Energy (MeV)')
ax2.set_title('$\\pi \\to \\mu\\nu$ Lab Frame')
ax2.legend(fontsize=7)
ax2.grid(True, alpha=0.3)

# Plot 3: CM to Lab angle transformation
ax3 = fig.add_subplot(gs[0, 2])
for gamma, color in zip([2, 5, 10], colors[1:4]):
    theta_lab_vals = []
    for th in theta_cm:
        _, _, th_lab = lab_frame_kinematics(m_pi, m_mu, m_nu, gamma, th)
        theta_lab_vals.append(th_lab)
    ax3.plot(np.rad2deg(theta_cm), np.rad2deg(theta_lab_vals), color=color,
             lw=1.5, label=f'$\\gamma = {gamma}$')
ax3.plot([0, 180], [0, 180], 'k--', alpha=0.3)
ax3.set_xlabel('CM Angle (deg)')
ax3.set_ylabel('Lab Angle (deg)')
ax3.set_title('Angular Transformation')
ax3.legend(fontsize=7)
ax3.grid(True, alpha=0.3)

# Plot 4: Dalitz plot
ax4 = fig.add_subplot(gs[1, 0])
ax4.scatter(m12_sq_phys/1e6, m23_sq_phys/1e6, s=1, alpha=0.3, c='blue')
ax4.set_xlabel('$m_{K\\pi}^2$ (GeV$^2$/c$^4$)')
ax4.set_ylabel('$m_{\\pi\\pi}^2$ (GeV$^2$/c$^4$)')
ax4.set_title('Dalitz Plot: $D^+ \\to K^-\\pi^+\\pi^+$')
ax4.grid(True, alpha=0.3)

# Plot 5: Invariant mass projection
ax5 = fig.add_subplot(gs[1, 1])
ax5.hist(np.sqrt(m12_sq_phys), bins=50, alpha=0.7, color='blue', label='$m_{K\\pi}$')
ax5.set_xlabel('Invariant Mass (MeV/c$^2$)')
ax5.set_ylabel('Events')
ax5.set_title('Invariant Mass Projection')
ax5.legend(fontsize=8)
ax5.grid(True, alpha=0.3)

# Plot 6: Phase space factor
ax6 = fig.add_subplot(gs[1, 2])
m_parent_range = np.linspace(300, 6000, 100)
ps_Kpi = [phase_space_factor(m, m_K_d, m_pi_p) if m > m_K_d + m_pi_p else 0
          for m in m_parent_range]
ps_munu = [phase_space_factor(m, m_mu, 0) if m > m_mu else 0
           for m in m_parent_range]

ax6.plot(m_parent_range, ps_Kpi, 'b-', lw=1.5, label='$K\\pi$')
ax6.plot(m_parent_range, ps_munu, 'r--', lw=1.5, label='$\\mu\\nu$')
ax6.set_xlabel('Parent Mass (MeV/c$^2$)')
ax6.set_ylabel('Phase Space Factor')
ax6.set_title('Two-Body Phase Space')
ax6.legend(fontsize=8)
ax6.grid(True, alpha=0.3)

# Plot 7: Decay momentum
ax7 = fig.add_subplot(gs[2, 0])
momenta = [two_body_decay(d[1], d[2], d[3])[2] for d in decay_modes]
ax7.barh(range(len(names)), momenta, color='green', alpha=0.7)
ax7.set_yticks(range(len(names)))
ax7.set_yticklabels(['$\\pi$', 'K$\\to\\pi$', 'K$\\to\\mu$', 'D', 'B'], fontsize=8)
ax7.set_xlabel('Momentum (MeV/c)')
ax7.set_title('Decay Momentum')
ax7.grid(True, alpha=0.3, axis='x')

# Plot 8: Maximum lab angle
ax8 = fig.add_subplot(gs[2, 1])
gamma_range = np.logspace(0, 3, 100)
theta_max_lab = []

for g in gamma_range:
    if g > 1:
        beta = np.sqrt(1 - 1/g**2)
        beta_cm = p_decay / E_mu_cm
        if beta > beta_cm:
            theta_max_lab.append(np.arcsin(beta_cm / beta))
        else:
            theta_max_lab.append(np.pi)
    else:
        theta_max_lab.append(np.pi)

ax8.semilogx(gamma_range, np.rad2deg(theta_max_lab), 'b-', lw=2)
ax8.set_xlabel('Parent $\\gamma$')
ax8.set_ylabel('Maximum Lab Angle (deg)')
ax8.set_title('Forward Focusing')
ax8.grid(True, alpha=0.3, which='both')
ax8.set_ylim([0, 180])

# Plot 9: Q-value comparison
ax9 = fig.add_subplot(gs[2, 2])
Q_values = [d[1] - d[2] - d[3] for d in decay_modes]
colors = ['blue', 'green', 'orange', 'red', 'purple']
ax9.bar(range(len(Q_values)), Q_values, color=colors, alpha=0.7)
ax9.set_xticks(range(len(names)))
ax9.set_xticklabels(['$\\pi$', 'K$\\to\\pi$', 'K$\\to\\mu$', 'D', 'B'], fontsize=8)
ax9.set_ylabel('Q-value (MeV)')
ax9.set_title('Decay Q-Values')
ax9.grid(True, alpha=0.3, axis='y')
ax9.set_yscale('log')

plt.savefig('decay_kinematics_plot.pdf', bbox_inches='tight', dpi=150)
print(r'\begin{center}')
print(r'\includegraphics[width=\textwidth]{decay_kinematics_plot.pdf}')
print(r'\end{center}')
plt.close()

# Summary calculations
Q_pi = m_pi - m_mu - m_nu
Q_K_pipi = m_K - m_pi_p - m_pi_0
\end{pycode}

\section{Results and Analysis}

\subsection{Two-Body Decay Kinematics}

\begin{pycode}
print(r'\begin{table}[htbp]')
print(r'\centering')
print(r'\caption{Two-Body Decay Parameters in Rest Frame}')
print(r'\begin{tabular}{lccccc}')
print(r'\toprule')
print(r'Decay & $Q$ (MeV) & $p$ (MeV/c) & $E_1$ (MeV) & $E_2$ (MeV) & $\beta_1$ \\')
print(r'\midrule')

for name, m_p, m_d1, m_d2 in decay_modes:
    Q = m_p - m_d1 - m_d2
    E1, E2, p = two_body_decay(m_p, m_d1, m_d2)
    beta = p / E1 if E1 > 0 else 0
    print(f'{name} & {Q:.1f} & {p:.1f} & {E1:.1f} & {E2:.1f} & {beta:.3f} \\\\')

print(r'\bottomrule')
print(r'\end{tabular}')
print(r'\end{table}')
\end{pycode}

\subsection{Pion Decay Analysis}

For $\pi^+ \to \mu^+ + \nu_\mu$:
\begin{itemize}
    \item Q-value: \py{f"{Q_pi:.2f}"} MeV
    \item Muon energy: \py{f"{E_mu_cm:.2f}"} MeV
    \item Muon kinetic energy: \py{f"{KE_mu:.2f}"} MeV
    \item Decay momentum: \py{f"{p_decay:.2f}"} MeV/c
\end{itemize}

\begin{remark}
The muon receives most of the available energy due to helicity suppression of the electron channel. The ratio $\Gamma(\pi \to e\nu)/\Gamma(\pi \to \mu\nu) \approx 1.2 \times 10^{-4}$.
\end{remark}

\section{Three-Body Phase Space}

\begin{theorem}[Dalitz Plot Boundaries]
For three-body decay $A \to 1 + 2 + 3$, the kinematically allowed region in the $(m_{12}^2, m_{23}^2)$ plane is bounded by:
\begin{equation}
(E_1^* + E_2^*)^2 - \left(\sqrt{E_1^{*2} - m_1^2} + \sqrt{E_2^{*2} - m_2^2}\right)^2 \leq m_{12}^2
\end{equation}
where $E_i^* = (M^2 + m_i^2 - m_{jk}^2)/(2M)$.
\end{theorem}

\begin{example}[Resonance Identification]
In the Dalitz plot:
\begin{itemize}
    \item Vertical/horizontal bands indicate $s$-channel resonances
    \item Diagonal bands indicate $t$-channel or $u$-channel exchanges
    \item Interference patterns reveal relative phases
\end{itemize}
\end{example}

\section{Laboratory Frame Effects}

\begin{theorem}[Forward Focusing]
For a relativistic parent with $\gamma \gg 1$, the maximum lab angle of daughter particles is:
\begin{equation}
\sin\theta_{max} = \frac{\beta_{CM}}{\beta_{parent}}
\end{equation}
where $\beta_{CM}$ is the daughter velocity in the CM frame.
\end{theorem}

\section{Discussion}

The decay kinematics analysis reveals:

\begin{enumerate}
    \item \textbf{Energy distribution}: Heavier daughters receive larger energy fractions in two-body decays.
    \item \textbf{Forward focusing}: Relativistic boosts compress angular distributions toward the beam axis.
    \item \textbf{Phase space}: Available phase space determines relative decay rates for different channels.
    \item \textbf{Dalitz analysis}: Two-dimensional phase space plots reveal resonance structure.
\end{enumerate}

\section{Conclusions}

This computational analysis demonstrates:
\begin{itemize}
    \item Pion decay muon energy: \py{f"{E_mu_cm:.2f}"} MeV
    \item Pion decay momentum: \py{f"{p_decay:.2f}"} MeV/c
    \item Kaon $\to$ 2$\pi$ Q-value: \py{f"{Q_K_pipi:.2f}"} MeV
    \item D meson mass: \py{f"{m_D:.1f}"} MeV/c$^2$
\end{itemize}

Kinematic analysis is essential for particle identification, background rejection, and precision measurements in collider experiments.

\section{Further Reading}
\begin{itemize}
    \item Particle Data Group, Review of Particle Physics, \textit{Phys. Rev. D}, 2022
    \item Byckling, E., Kajantie, K., \textit{Particle Kinematics}, Wiley, 1973
    \item Perkins, D.H., \textit{Introduction to High Energy Physics}, 4th Ed., Cambridge, 2000
\end{itemize}

\end{document}
