\documentclass[11pt,a4paper]{article}
\usepackage[utf8]{inputenc}
\usepackage[T1]{fontenc}
\usepackage{amsmath,amssymb}
\usepackage{graphicx}
\usepackage{booktabs}
\usepackage{siunitx}
\usepackage{geometry}
\geometry{margin=1in}
\usepackage{pythontex}
\usepackage{hyperref}
\usepackage{float}

\title{Pharmacokinetics\\Compartment Models and Drug Concentration}
\author{Clinical Pharmacology Division}
\date{\today}

\begin{document}
\maketitle

\begin{abstract}
Computational modeling of drug absorption, distribution, metabolism, and elimination using compartment models.
\end{abstract}


\section{Introduction}

Pharmacokinetics describes the time course of drug concentration in the body.

\begin{pycode}
import numpy as np
import matplotlib.pyplot as plt
from scipy.integrate import odeint
plt.rcParams['text.usetex'] = True
plt.rcParams['font.family'] = 'serif'
\end{pycode}

\section{One-Compartment Model}

$\frac{dC}{dt} = -k_e C$

\begin{pycode}
# IV bolus
C0 = 100  # Initial concentration (mg/L)
k_e = 0.1  # Elimination rate constant (1/h)
t = np.linspace(0, 48, 200)

C = C0 * np.exp(-k_e * t)
t_half = np.log(2) / k_e

fig, ax = plt.subplots(figsize=(10, 6))
ax.semilogy(t, C, 'b-', linewidth=2)
ax.axhline(y=C0/2, color='r', linestyle='--', label=f'$t_{{1/2}}$ = {t_half:.1f} h')
ax.axvline(x=t_half, color='r', linestyle='--')
ax.set_xlabel('Time (h)')
ax.set_ylabel('Concentration (mg/L)')
ax.set_title('One-Compartment Model: IV Bolus')
ax.legend()
ax.grid(True, alpha=0.3, which='both')
plt.tight_layout()
plt.savefig('one_compartment.pdf', dpi=150, bbox_inches='tight')
plt.close()
\end{pycode}

\begin{figure}[H]
\centering
\includegraphics[width=0.85\textwidth]{one_compartment.pdf}
\caption{Drug concentration after IV bolus administration.}
\end{figure}

\section{Oral Administration}

\begin{pycode}
k_a = 1.0  # Absorption rate constant
F = 0.8    # Bioavailability
D = 500    # Dose (mg)
V = 50     # Volume of distribution (L)

C_oral = (F * D * k_a / (V * (k_a - k_e))) * (np.exp(-k_e * t) - np.exp(-k_a * t))

fig, ax = plt.subplots(figsize=(10, 6))
ax.plot(t, C_oral, 'b-', linewidth=2)
t_max = np.log(k_a / k_e) / (k_a - k_e)
C_max = (F * D * k_a / (V * (k_a - k_e))) * (np.exp(-k_e * t_max) - np.exp(-k_a * t_max))
ax.plot(t_max, C_max, 'ro', markersize=10)
ax.annotate(f'$C_{{max}}$ = {C_max:.1f} mg/L\n$t_{{max}}$ = {t_max:.1f} h',
            xy=(t_max, C_max), xytext=(t_max + 5, C_max))
ax.set_xlabel('Time (h)')
ax.set_ylabel('Concentration (mg/L)')
ax.set_title('Oral Administration')
ax.grid(True, alpha=0.3)
plt.tight_layout()
plt.savefig('oral_admin.pdf', dpi=150, bbox_inches='tight')
plt.close()
\end{pycode}

\begin{figure}[H]
\centering
\includegraphics[width=0.85\textwidth]{oral_admin.pdf}
\caption{Drug concentration after oral administration.}
\end{figure}

\section{Two-Compartment Model}

\begin{pycode}
def two_compartment(y, t, k12, k21, k10):
    C1, C2 = y
    dC1dt = -k12 * C1 + k21 * C2 - k10 * C1
    dC2dt = k12 * C1 - k21 * C2
    return [dC1dt, dC2dt]

k12 = 0.5
k21 = 0.2
k10 = 0.1
y0 = [100, 0]

sol = odeint(two_compartment, y0, t, args=(k12, k21, k10))

fig, ax = plt.subplots(figsize=(10, 6))
ax.semilogy(t, sol[:, 0], 'b-', linewidth=2, label='Central')
ax.semilogy(t, sol[:, 1], 'r-', linewidth=2, label='Peripheral')
ax.set_xlabel('Time (h)')
ax.set_ylabel('Concentration (mg/L)')
ax.set_title('Two-Compartment Model')
ax.legend()
ax.grid(True, alpha=0.3, which='both')
plt.tight_layout()
plt.savefig('two_compartment.pdf', dpi=150, bbox_inches='tight')
plt.close()
\end{pycode}

\begin{figure}[H]
\centering
\includegraphics[width=0.85\textwidth]{two_compartment.pdf}
\caption{Two-compartment model showing distribution phase.}
\end{figure}

\section{Multiple Dosing}

\begin{pycode}
tau = 8  # Dosing interval (h)
n_doses = 6
t_multi = np.linspace(0, n_doses * tau, 500)
C_multi = np.zeros_like(t_multi)

for i in range(n_doses):
    t_dose = i * tau
    mask = t_multi >= t_dose
    C_multi[mask] += C0 * np.exp(-k_e * (t_multi[mask] - t_dose))

fig, ax = plt.subplots(figsize=(10, 6))
ax.plot(t_multi, C_multi, 'b-', linewidth=2)
ax.set_xlabel('Time (h)')
ax.set_ylabel('Concentration (mg/L)')
ax.set_title('Multiple Dosing Regimen')
ax.grid(True, alpha=0.3)

# Steady state
C_ss_max = C0 / (1 - np.exp(-k_e * tau))
C_ss_min = C0 * np.exp(-k_e * tau) / (1 - np.exp(-k_e * tau))
ax.axhline(y=C_ss_max, color='r', linestyle='--', alpha=0.5)
ax.axhline(y=C_ss_min, color='g', linestyle='--', alpha=0.5)
plt.tight_layout()
plt.savefig('multiple_dosing.pdf', dpi=150, bbox_inches='tight')
plt.close()
\end{pycode}

\begin{figure}[H]
\centering
\includegraphics[width=0.85\textwidth]{multiple_dosing.pdf}
\caption{Drug accumulation with repeated dosing.}
\end{figure}

\section{Continuous Infusion}

\begin{pycode}
R = 50  # Infusion rate (mg/h)
CL = k_e * V  # Clearance

C_infusion = (R / CL) * (1 - np.exp(-k_e * t))
C_ss = R / CL

fig, ax = plt.subplots(figsize=(10, 6))
ax.plot(t, C_infusion, 'b-', linewidth=2)
ax.axhline(y=C_ss, color='r', linestyle='--', label=f'$C_{{ss}}$ = {C_ss:.1f} mg/L')
ax.set_xlabel('Time (h)')
ax.set_ylabel('Concentration (mg/L)')
ax.set_title('Continuous IV Infusion')
ax.legend()
ax.grid(True, alpha=0.3)
plt.tight_layout()
plt.savefig('infusion.pdf', dpi=150, bbox_inches='tight')
plt.close()
\end{pycode}

\begin{figure}[H]
\centering
\includegraphics[width=0.85\textwidth]{infusion.pdf}
\caption{Drug concentration during continuous infusion.}
\end{figure}

\section{Loading Dose}

\begin{pycode}
D_loading = C_ss * V
t_loading = np.linspace(0, 24, 200)

# With loading dose
C_with_load = C_ss + (D_loading/V - C_ss) * np.exp(-k_e * t_loading)
# Without loading dose
C_without_load = (R / CL) * (1 - np.exp(-k_e * t_loading))

fig, ax = plt.subplots(figsize=(10, 6))
ax.plot(t_loading, C_with_load, 'b-', linewidth=2, label='With loading dose')
ax.plot(t_loading, C_without_load, 'r--', linewidth=2, label='Without loading dose')
ax.axhline(y=C_ss, color='k', linestyle=':', alpha=0.5)
ax.set_xlabel('Time (h)')
ax.set_ylabel('Concentration (mg/L)')
ax.set_title('Effect of Loading Dose')
ax.legend()
ax.grid(True, alpha=0.3)
plt.tight_layout()
plt.savefig('loading_dose.pdf', dpi=150, bbox_inches='tight')
plt.close()
\end{pycode}

\begin{figure}[H]
\centering
\includegraphics[width=0.85\textwidth]{loading_dose.pdf}
\caption{Comparison with and without loading dose.}
\end{figure}

\section{Results}

\begin{pycode}
print(r'\begin{table}[H]')
print(r'\centering')
print(r'\caption{Pharmacokinetic Parameters}')
print(r'\begin{tabular}{@{}lc@{}}')
print(r'\toprule')
print(r'Parameter & Value \\')
print(r'\midrule')
print(f'Half-life & {t_half:.1f} h \\\\')
print(f'Volume of distribution & {V} L \\\\')
print(f'Clearance & {CL:.1f} L/h \\\\')
print(f'Steady-state concentration & {C_ss:.1f} mg/L \\\\')
print(r'\bottomrule')
print(r'\end{tabular}')
print(r'\end{table}')
\end{pycode}

\section{Conclusions}

Compartment models provide essential tools for drug dosing optimization and therapeutic monitoring.


\end{document}
