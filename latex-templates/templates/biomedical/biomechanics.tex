\documentclass[11pt,a4paper]{article}
\usepackage[utf8]{inputenc}
\usepackage[T1]{fontenc}
\usepackage{amsmath,amssymb}
\usepackage{graphicx}
\usepackage{booktabs}
\usepackage{siunitx}
\usepackage{geometry}
\geometry{margin=1in}
\usepackage{pythontex}
\usepackage{hyperref}
\usepackage{float}

\title{Biomechanics\\Tissue Mechanics and Viscoelasticity}
\author{Biomedical Engineering Department}
\date{\today}

\begin{document}
\maketitle

\begin{abstract}
Analysis of biological tissue mechanics including stress-strain relationships, viscoelasticity, and bone mechanics.
\end{abstract}


\section{Introduction}

Biomechanics applies mechanical principles to biological systems.

\begin{pycode}
import numpy as np
import matplotlib.pyplot as plt
plt.rcParams['text.usetex'] = True
plt.rcParams['font.family'] = 'serif'
\end{pycode}

\section{Stress-Strain Curves}

\begin{pycode}
strain = np.linspace(0, 0.3, 100)

# Different tissue types
bone_stress = 20e9 * strain * (strain < 0.02) + (20e9 * 0.02) * (strain >= 0.02)
tendon_stress = 1.5e9 * strain**1.5
skin_stress = 1e6 * (np.exp(10*strain) - 1)

fig, ax = plt.subplots(figsize=(10, 6))
ax.plot(strain * 100, bone_stress / 1e6, label='Bone', linewidth=1.5)
ax.plot(strain * 100, tendon_stress / 1e6, label='Tendon', linewidth=1.5)
ax.plot(strain * 100, skin_stress / 1e6, label='Skin', linewidth=1.5)
ax.set_xlabel('Strain (\\%)')
ax.set_ylabel('Stress (MPa)')
ax.set_title('Stress-Strain Curves for Biological Tissues')
ax.legend()
ax.grid(True, alpha=0.3)
ax.set_xlim([0, 30])
ax.set_ylim([0, 500])
plt.tight_layout()
plt.savefig('tissue_stress_strain.pdf', dpi=150, bbox_inches='tight')
plt.close()
\end{pycode}

\begin{figure}[H]
\centering
\includegraphics[width=0.85\textwidth]{tissue_stress_strain.pdf}
\caption{Stress-strain behavior of different tissues.}
\end{figure}

\section{Viscoelastic Models}

Standard Linear Solid: $\sigma + \tau_\sigma \dot{\sigma} = E_R \epsilon + \tau_\epsilon E_R \dot{\epsilon}$

\begin{pycode}
# Stress relaxation
t = np.linspace(0, 10, 100)
E_0 = 10  # Initial modulus
E_inf = 2  # Equilibrium modulus
tau = 2   # Relaxation time

G_t = E_inf + (E_0 - E_inf) * np.exp(-t/tau)

fig, ax = plt.subplots(figsize=(10, 6))
ax.plot(t, G_t, 'b-', linewidth=2)
ax.axhline(y=E_inf, color='r', linestyle='--', label='$E_\\infty$')
ax.set_xlabel('Time (s)')
ax.set_ylabel('Relaxation Modulus (MPa)')
ax.set_title('Stress Relaxation')
ax.legend()
ax.grid(True, alpha=0.3)
plt.tight_layout()
plt.savefig('stress_relaxation.pdf', dpi=150, bbox_inches='tight')
plt.close()
\end{pycode}

\begin{figure}[H]
\centering
\includegraphics[width=0.85\textwidth]{stress_relaxation.pdf}
\caption{Viscoelastic stress relaxation.}
\end{figure}

\section{Creep Response}

\begin{pycode}
J_0 = 0.1  # Initial compliance
J_inf = 0.5  # Equilibrium compliance

J_t = J_inf - (J_inf - J_0) * np.exp(-t/tau)

fig, ax = plt.subplots(figsize=(10, 6))
ax.plot(t, J_t, 'b-', linewidth=2)
ax.set_xlabel('Time (s)')
ax.set_ylabel('Creep Compliance (1/MPa)')
ax.set_title('Creep Response')
ax.grid(True, alpha=0.3)
plt.tight_layout()
plt.savefig('creep_response.pdf', dpi=150, bbox_inches='tight')
plt.close()
\end{pycode}

\begin{figure}[H]
\centering
\includegraphics[width=0.85\textwidth]{creep_response.pdf}
\caption{Viscoelastic creep compliance.}
\end{figure}

\section{Dynamic Modulus}

\begin{pycode}
omega = np.logspace(-2, 2, 100)

# Storage and loss moduli
E_storage = E_inf + (E_0 - E_inf) * (omega * tau)**2 / (1 + (omega * tau)**2)
E_loss = (E_0 - E_inf) * omega * tau / (1 + (omega * tau)**2)

fig, ax = plt.subplots(figsize=(10, 6))
ax.loglog(omega, E_storage, 'b-', linewidth=2, label="$E'$ (Storage)")
ax.loglog(omega, E_loss, 'r-', linewidth=2, label="$E''$ (Loss)")
ax.set_xlabel('Frequency (rad/s)')
ax.set_ylabel('Modulus (MPa)')
ax.set_title('Dynamic Mechanical Properties')
ax.legend()
ax.grid(True, alpha=0.3, which='both')
plt.tight_layout()
plt.savefig('dynamic_modulus.pdf', dpi=150, bbox_inches='tight')
plt.close()
\end{pycode}

\begin{figure}[H]
\centering
\includegraphics[width=0.85\textwidth]{dynamic_modulus.pdf}
\caption{Storage and loss moduli vs frequency.}
\end{figure}

\section{Bone Remodeling}

\begin{pycode}
# Wolff's law simulation
rho_0 = 1500  # Initial density
stimulus = np.linspace(0, 2, 100)  # Mechanical stimulus

# Remodeling response
drho_dt = 0.1 * (stimulus - 1) * rho_0

fig, ax = plt.subplots(figsize=(10, 6))
ax.plot(stimulus, drho_dt, 'b-', linewidth=2)
ax.axhline(y=0, color='k', linewidth=0.5)
ax.axvline(x=1, color='r', linestyle='--', label='Equilibrium')
ax.set_xlabel('Mechanical Stimulus (normalized)')
ax.set_ylabel('Remodeling Rate')
ax.set_title("Bone Remodeling (Wolff's Law)")
ax.legend()
ax.grid(True, alpha=0.3)
plt.tight_layout()
plt.savefig('bone_remodeling.pdf', dpi=150, bbox_inches='tight')
plt.close()
\end{pycode}

\begin{figure}[H]
\centering
\includegraphics[width=0.85\textwidth]{bone_remodeling.pdf}
\caption{Bone remodeling rate vs mechanical stimulus.}
\end{figure}

\section{Hyperelastic Model}

\begin{pycode}
# Neo-Hookean model
lambda_stretch = np.linspace(1, 2, 100)
mu = 0.5  # Shear modulus (MPa)

# Cauchy stress
sigma_nh = mu * (lambda_stretch - 1/lambda_stretch**2)

fig, ax = plt.subplots(figsize=(10, 6))
ax.plot(lambda_stretch, sigma_nh, 'b-', linewidth=2)
ax.set_xlabel('Stretch Ratio $\\lambda$')
ax.set_ylabel('Cauchy Stress (MPa)')
ax.set_title('Neo-Hookean Hyperelastic Model')
ax.grid(True, alpha=0.3)
plt.tight_layout()
plt.savefig('hyperelastic.pdf', dpi=150, bbox_inches='tight')
plt.close()
\end{pycode}

\begin{figure}[H]
\centering
\includegraphics[width=0.85\textwidth]{hyperelastic.pdf}
\caption{Neo-Hookean stress-stretch relationship.}
\end{figure}

\section{Results}

\begin{pycode}
print(r'\begin{table}[H]')
print(r'\centering')
print(r'\caption{Tissue Mechanical Properties}')
print(r'\begin{tabular}{@{}lcc@{}}')
print(r'\toprule')
print(r'Tissue & Elastic Modulus & Ultimate Stress \\')
print(r'\midrule')
print(r'Cortical Bone & 15-20 GPa & 100-150 MPa \\')
print(r'Tendon & 1-2 GPa & 50-100 MPa \\')
print(r'Articular Cartilage & 1-10 MPa & 10-40 MPa \\')
print(r'Skin & 0.1-1 MPa & 2-20 MPa \\')
print(r'\bottomrule')
print(r'\end{tabular}')
print(r'\end{table}')
\end{pycode}

\section{Conclusions}

Biological tissues exhibit complex nonlinear and time-dependent mechanical behavior.


\end{document}
