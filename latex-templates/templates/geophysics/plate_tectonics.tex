\documentclass[a4paper, 11pt]{article}
\usepackage[utf8]{inputenc}
\usepackage[T1]{fontenc}
\usepackage{amsmath, amssymb, amsthm}
\usepackage{graphicx}
\usepackage{booktabs}
\usepackage{siunitx}
\usepackage{subcaption}
\usepackage[makestderr]{pythontex}

\newtheorem{definition}{Definition}
\newtheorem{theorem}{Theorem}
\newtheorem{example}{Example}
\newtheorem{remark}{Remark}

\title{Plate Tectonics: Thermal Evolution, Plate Motion, and Mantle Convection}
\author{Computational Geophysics Laboratory}
\date{\today}

\begin{document}
\maketitle

\begin{abstract}
This technical report presents comprehensive computational analysis of plate tectonic processes including lithospheric cooling, seafloor subsidence, heat flow evolution, and plate kinematics. We implement the half-space and plate cooling models, analyze Euler pole rotation kinematics, and model mantle convection using Rayleigh-Benard theory. The analysis quantifies lithospheric thickness, thermal age relationships, and driving forces of plate motion.
\end{abstract}

\section{Theoretical Framework}

\begin{definition}[Thermal Diffusion]
Heat conduction in the lithosphere follows the diffusion equation:
\begin{equation}
\frac{\partial T}{\partial t} = \kappa \nabla^2 T
\end{equation}
where $\kappa = k/(\rho c_p)$ is thermal diffusivity ($\sim 10^{-6}$ m$^2$/s).
\end{definition}

\begin{theorem}[Half-Space Cooling Model]
For lithosphere cooling from initial mantle temperature $T_m$, the temperature profile is:
\begin{equation}
T(z, t) = T_s + (T_m - T_s) \text{erf}\left(\frac{z}{2\sqrt{\kappa t}}\right)
\end{equation}
where $\text{erf}$ is the error function and $T_s$ is surface temperature.
\end{theorem}

\subsection{Seafloor Subsidence}

Thermal contraction causes seafloor deepening with age:
\begin{equation}
d(t) = d_r + \frac{2\rho_m \alpha_V (T_m - T_s)}{\rho_m - \rho_w} \sqrt{\frac{\kappa t}{\pi}}
\end{equation}

where $d_r$ is ridge depth, $\alpha_V$ is volumetric thermal expansion, and $\rho_m$, $\rho_w$ are mantle and water densities.

\begin{example}[Heat Flow]
Surface heat flow decreases with age:
\begin{equation}
q(t) = \frac{k(T_m - T_s)}{\sqrt{\pi \kappa t}}
\end{equation}
Typical values range from $>$200 mW/m$^2$ at ridges to $<$50 mW/m$^2$ on old oceanic crust.
\end{example}

\subsection{Plate Kinematics}

Plate motion on a sphere follows Euler's theorem---rotation about a fixed pole:
\begin{equation}
\mathbf{v} = \boldsymbol{\omega} \times \mathbf{r}
\end{equation}
where $\boldsymbol{\omega}$ is the angular velocity vector and $\mathbf{r}$ is the position vector.

\section{Computational Analysis}

\begin{pycode}
import numpy as np
from scipy.special import erf, erfc
import matplotlib.pyplot as plt
plt.rc('text', usetex=True)
plt.rc('font', family='serif', size=10)

# Physical parameters
kappa = 1e-6      # Thermal diffusivity (m^2/s)
k = 3.3           # Thermal conductivity (W/m/K)
Tm = 1350         # Mantle temperature (C)
Ts = 0            # Surface temperature (C)
alpha_V = 3e-5    # Thermal expansion (1/K)
rho_m = 3300      # Mantle density (kg/m^3)
rho_w = 1030      # Seawater density (kg/m^3)
rho_a = 3200      # Asthenosphere density (kg/m^3)
dr = 2500         # Ridge depth (m)
g = 9.81          # Gravity (m/s^2)

# Time conversion
Ma_to_s = 1e6 * 365.25 * 24 * 3600

# Age range
age_Ma = np.linspace(0.1, 200, 200)
age_s = age_Ma * Ma_to_s

# Half-space cooling model
def halfspace_depth(age_s):
    """Seafloor depth from half-space model (m)."""
    return dr + 2 * rho_m * alpha_V * (Tm - Ts) / (rho_m - rho_w) * np.sqrt(kappa * age_s / np.pi)

def halfspace_heatflow(age_s):
    """Surface heat flow from half-space model (mW/m^2)."""
    return k * (Tm - Ts) / np.sqrt(np.pi * kappa * age_s) * 1000

def lithosphere_thickness(age_s, T_threshold=1100):
    """Thermal lithosphere thickness (m)."""
    return 2.32 * np.sqrt(kappa * age_s)  # Depth to T = 0.9*Tm

# Plate model parameters
a_plate = 125000  # Plate thickness (m)
tau = a_plate**2 / (np.pi**2 * kappa)  # Thermal time constant

def plate_depth(age_s):
    """Seafloor depth from plate model (m)."""
    d_inf = dr + rho_m * alpha_V * (Tm - Ts) * a_plate / (rho_m - rho_w)
    return d_inf - (d_inf - dr) * (8/np.pi**2) * np.exp(-age_s/tau)

# Temperature profiles at different ages
ages_profile = [10, 50, 100, 150]  # Ma
z = np.linspace(0, 150000, 300)  # Depth in m

# Plate velocities (mm/yr) - major plates
plates = {
    'Pacific': 75,
    'Australian': 60,
    'Nazca': 55,
    'Cocos': 52,
    'Philippine': 45,
    'N American': 23,
    'S American': 17,
    'Eurasian': 21,
    'African': 22,
    'Antarctic': 17
}

# Euler pole example (Pacific-North American)
euler_lat = 48.7
euler_lon = -78.2
omega = 0.78  # deg/Ma

# Driving forces (relative magnitudes)
forces = {
    'Ridge Push': 2.5,
    'Slab Pull': 10,
    'Basal Drag': 3,
    'Mantle Suction': 1.5
}

# Rayleigh number for mantle convection
Ra_crit = 1000  # Critical Rayleigh number
deltaT = 2500   # Temperature difference (K)
d_mantle = 2900e3  # Mantle thickness (m)
nu = 1e21       # Viscosity (Pa*s)
alpha = 3e-5    # Thermal expansion

Ra = rho_m * g * alpha * deltaT * d_mantle**3 / (kappa * nu)

# Calculate depths and heat flow
depth_halfspace = halfspace_depth(age_s)
depth_plate = plate_depth(age_s)
heatflow = halfspace_heatflow(age_s)
lith_thickness = lithosphere_thickness(age_s)

# Subsidence rate
subsidence_rate = np.gradient(depth_halfspace, age_Ma) / 1000  # km/Ma

# Create visualization
fig = plt.figure(figsize=(12, 10))
gs = fig.add_gridspec(3, 3, hspace=0.35, wspace=0.35)

# Plot 1: Seafloor depth vs age
ax1 = fig.add_subplot(gs[0, 0])
ax1.plot(age_Ma, depth_halfspace/1000, 'b-', lw=2, label='Half-space')
ax1.plot(age_Ma, depth_plate/1000, 'r--', lw=2, label='Plate')
ax1.plot(np.sqrt(age_Ma), depth_halfspace/1000, 'g:', lw=1, alpha=0)  # Hidden
ax1.set_xlabel('Age (Ma)')
ax1.set_ylabel('Depth (km)')
ax1.set_title('Seafloor Subsidence')
ax1.legend(fontsize=8)
ax1.invert_yaxis()
ax1.grid(True, alpha=0.3)

# Plot 2: Heat flow vs age
ax2 = fig.add_subplot(gs[0, 1])
ax2.plot(age_Ma, heatflow, 'r-', lw=2)
ax2.axhline(y=65, color='gray', ls='--', alpha=0.5, label='Continental avg')
ax2.set_xlabel('Age (Ma)')
ax2.set_ylabel('Heat Flow (mW/m$^2$)')
ax2.set_title('Surface Heat Flow')
ax2.legend(fontsize=8)
ax2.grid(True, alpha=0.3)
ax2.set_ylim([0, 500])

# Plot 3: Temperature profiles
ax3 = fig.add_subplot(gs[0, 2])
colors = plt.cm.viridis(np.linspace(0.2, 0.8, len(ages_profile)))
for age, color in zip(ages_profile, colors):
    age_sec = age * Ma_to_s
    T = Ts + (Tm - Ts) * erf(z / (2 * np.sqrt(kappa * age_sec)))
    ax3.plot(T, z/1000, color=color, lw=2, label=f'{age} Ma')

ax3.axhline(y=lith_thickness[np.argmin(np.abs(age_Ma - 100))]/1000,
            color='gray', ls='--', alpha=0.5)
ax3.set_xlabel('Temperature ($^\\circ$C)')
ax3.set_ylabel('Depth (km)')
ax3.set_title('Geotherm Evolution')
ax3.legend(fontsize=7)
ax3.invert_yaxis()
ax3.grid(True, alpha=0.3)
ax3.set_xlim([0, 1400])

# Plot 4: Lithosphere thickness
ax4 = fig.add_subplot(gs[1, 0])
ax4.plot(age_Ma, lith_thickness/1000, 'b-', lw=2)
ax4.plot(age_Ma, np.sqrt(age_Ma) * 10, 'r--', lw=1.5, alpha=0.7,
         label='$\\propto \\sqrt{t}$')
ax4.set_xlabel('Age (Ma)')
ax4.set_ylabel('Thickness (km)')
ax4.set_title('Thermal Lithosphere Thickness')
ax4.legend(fontsize=8)
ax4.grid(True, alpha=0.3)

# Plot 5: Plate velocities
ax5 = fig.add_subplot(gs[1, 1])
names = list(plates.keys())
velocities = list(plates.values())
colors = plt.cm.plasma(np.linspace(0.2, 0.8, len(plates)))
y_pos = range(len(names))
ax5.barh(y_pos, velocities, color=colors, alpha=0.8)
ax5.set_yticks(y_pos)
ax5.set_yticklabels(names, fontsize=8)
ax5.set_xlabel('Velocity (mm/yr)')
ax5.set_title('Major Plate Velocities')
ax5.grid(True, alpha=0.3, axis='x')

# Plot 6: Depth vs sqrt(age)
ax6 = fig.add_subplot(gs[1, 2])
sqrt_age = np.sqrt(age_Ma)
ax6.plot(sqrt_age, depth_halfspace/1000, 'b-', lw=2)
# Linear fit
coeffs = np.polyfit(sqrt_age[10:], depth_halfspace[10:]/1000, 1)
ax6.plot(sqrt_age, np.polyval(coeffs, sqrt_age), 'r--', lw=1.5,
         label=f'Slope = {coeffs[0]:.0f} m/Ma$^{{1/2}}$')
ax6.set_xlabel('$\\sqrt{\\mathrm{Age}}$ (Ma$^{1/2}$)')
ax6.set_ylabel('Depth (km)')
ax6.set_title('Depth vs $\\sqrt{t}$ (Parsons-Sclater)')
ax6.legend(fontsize=8)
ax6.invert_yaxis()
ax6.grid(True, alpha=0.3)

# Plot 7: Driving forces
ax7 = fig.add_subplot(gs[2, 0])
force_names = list(forces.keys())
force_vals = list(forces.values())
colors = ['blue', 'red', 'green', 'orange']
ax7.bar(range(len(forces)), force_vals, color=colors, alpha=0.7)
ax7.set_xticks(range(len(forces)))
ax7.set_xticklabels(force_names, fontsize=8, rotation=15)
ax7.set_ylabel('Relative Magnitude')
ax7.set_title('Plate Driving Forces')
ax7.grid(True, alpha=0.3, axis='y')

# Plot 8: Subsidence rate
ax8 = fig.add_subplot(gs[2, 1])
ax8.plot(age_Ma[1:], -subsidence_rate[1:] * 1000, 'purple', lw=2)
ax8.set_xlabel('Age (Ma)')
ax8.set_ylabel('Subsidence Rate (m/Ma)')
ax8.set_title('Subsidence Rate vs Age')
ax8.grid(True, alpha=0.3)
ax8.set_xlim([0, 200])

# Plot 9: Rayleigh-Benard stability
ax9 = fig.add_subplot(gs[2, 2])
Ra_range = np.logspace(2, 8, 100)
Nu = np.ones_like(Ra_range)
Nu[Ra_range > Ra_crit] = 0.28 * Ra_range[Ra_range > Ra_crit]**0.21

ax9.loglog(Ra_range, Nu, 'b-', lw=2)
ax9.axvline(x=Ra_crit, color='r', ls='--', label=f'Ra$_c$ = {Ra_crit}')
ax9.axvline(x=Ra, color='g', ls='--', alpha=0.7, label=f'Mantle Ra')
ax9.set_xlabel('Rayleigh Number')
ax9.set_ylabel('Nusselt Number')
ax9.set_title('Convection Heat Transfer')
ax9.legend(fontsize=8)
ax9.grid(True, alpha=0.3, which='both')

plt.savefig('plate_tectonics_plot.pdf', bbox_inches='tight', dpi=150)
print(r'\begin{center}')
print(r'\includegraphics[width=\textwidth]{plate_tectonics_plot.pdf}')
print(r'\end{center}')
plt.close()

# Summary calculations
depth_100Ma = halfspace_depth(100 * Ma_to_s)
heatflow_10Ma = halfspace_heatflow(10 * Ma_to_s)
lith_100Ma = lithosphere_thickness(100 * Ma_to_s)
max_velocity = max(velocities)
parsons_slope = coeffs[0]
\end{pycode}

\section{Results and Analysis}

\subsection{Thermal Evolution}

\begin{pycode}
print(r'\begin{table}[htbp]')
print(r'\centering')
print(r'\caption{Oceanic Lithosphere Properties vs Age}')
print(r'\begin{tabular}{ccccc}')
print(r'\toprule')
print(r'Age (Ma) & Depth (km) & Heat Flow (mW/m$^2$) & Lith. Thick. (km) & Subsidence (m/Ma) \\')
print(r'\midrule')

ages_table = [1, 10, 25, 50, 100, 150, 200]
for age in ages_table:
    idx = np.argmin(np.abs(age_Ma - age))
    d = depth_halfspace[idx]/1000
    q = heatflow[idx]
    l = lith_thickness[idx]/1000
    s = -subsidence_rate[idx] * 1000 if idx > 0 else np.nan
    s_str = f'{s:.0f}' if not np.isnan(s) else '--'
    print(f'{age} & {d:.2f} & {q:.0f} & {l:.0f} & {s_str} \\\\')

print(r'\bottomrule')
print(r'\end{tabular}')
print(r'\end{table}')
\end{pycode}

\subsection{Model Comparison}

The half-space and plate models diverge for old lithosphere:
\begin{itemize}
    \item Half-space predicts continuous deepening: $d \propto \sqrt{t}$
    \item Plate model predicts asymptotic depth for $t > \py{f"{tau/Ma_to_s:.0f}"}$ Ma
    \item Observed data favor plate model for ages $>$ 80 Ma
    \item Parsons-Sclater slope: \py{f"{parsons_slope:.0f}"} m/Ma$^{1/2}$
\end{itemize}

\begin{remark}
The $\sqrt{t}$ dependence of seafloor depth is a diagnostic signature of conductive cooling. Deviations indicate convective or compositional effects.
\end{remark}

\subsection{Plate Kinematics}

\begin{pycode}
print(r'\begin{table}[htbp]')
print(r'\centering')
print(r'\caption{Plate Motion Statistics}')
print(r'\begin{tabular}{lcc}')
print(r'\toprule')
print(r'Statistic & Value & Units \\')
print(r'\midrule')
print(f'Maximum plate velocity & {max_velocity} & mm/yr \\\\')
print(f'Mean velocity & {np.mean(velocities):.1f} & mm/yr \\\\')
print(f'Fastest plate & Pacific & -- \\\\')
print(f'Slowest plate & Antarctic & -- \\\\')
print(f'Euler pole (Pac-NAm) & ({euler_lat}$^\\circ$N, {euler_lon}$^\\circ$E) & -- \\\\')
print(f'Angular velocity & {omega} & deg/Ma \\\\')
print(r'\bottomrule')
print(r'\end{tabular}')
print(r'\end{table}')
\end{pycode}

\section{Physical Processes}

\begin{example}[Ridge Push Force]
The elevation of mid-ocean ridges creates a gravitational driving force:
\begin{equation}
F_{RP} = g\rho_m \alpha_V (T_m - T_s) \kappa t \approx 2-3 \times 10^{12} \text{ N/m}
\end{equation}
This force acts throughout the lithosphere volume.
\end{example}

\begin{example}[Slab Pull Force]
Subducting lithosphere is denser than surrounding mantle:
\begin{equation}
F_{SP} = \Delta\rho \cdot g \cdot L \cdot h \approx 10^{13} \text{ N/m}
\end{equation}
where $L$ is slab length and $h$ is thickness. This is the dominant driving force.
\end{example}

\begin{theorem}[Mantle Convection]
Convection occurs when the Rayleigh number exceeds the critical value:
\begin{equation}
Ra = \frac{\rho g \alpha \Delta T d^3}{\kappa \nu} > Ra_c \approx 1000
\end{equation}
For Earth's mantle, $Ra \approx \py{f"{Ra:.1e}"}$, indicating vigorous convection.
\end{theorem}

\section{Discussion}

The analysis reveals several key features of plate tectonics:

\begin{enumerate}
    \item \textbf{Thermal control}: Lithospheric properties are primarily controlled by conductive cooling from the mantle.
    \item \textbf{Depth-age relationship}: The $\sqrt{t}$ subsidence law holds for ages $<$ 80 Ma; older lithosphere approaches thermal equilibrium.
    \item \textbf{Velocity distribution}: Oceanic plates with attached slabs move faster than continental plates.
    \item \textbf{Force balance}: Slab pull dominates over ridge push by a factor of $\sim$4.
    \item \textbf{Vigorous convection}: The mantle Rayleigh number far exceeds critical, enabling plate recycling.
\end{enumerate}

\section{Conclusions}

This computational analysis demonstrates:
\begin{itemize}
    \item Seafloor depth at 100 Ma: \py{f"{depth_100Ma/1000:.2f}"} km
    \item Heat flow at 10 Ma: \py{f"{heatflow_10Ma:.0f}"} mW/m$^2$
    \item Lithosphere thickness at 100 Ma: \py{f"{lith_100Ma/1000:.0f}"} km
    \item Maximum plate velocity: \py{f"{max_velocity}"} mm/yr (Pacific)
    \item Mantle Rayleigh number: \py{f"{Ra:.1e}"}
\end{itemize}

The thermal evolution of oceanic lithosphere provides fundamental constraints on mantle convection and the driving forces of plate tectonics.

\section{Further Reading}
\begin{itemize}
    \item Turcotte, D.L., Schubert, G., \textit{Geodynamics}, 3rd Edition, Cambridge University Press, 2014
    \item Fowler, C.M.R., \textit{The Solid Earth}, 2nd Edition, Cambridge University Press, 2004
    \item Parsons, B., Sclater, J.G., An analysis of the variation of ocean floor bathymetry and heat flow with age, \textit{J. Geophys. Res.}, 1977
\end{itemize}

\end{document}
