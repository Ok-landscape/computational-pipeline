\documentclass[a4paper, 11pt]{report}
\usepackage[utf8]{inputenc}
\usepackage[T1]{fontenc}
\usepackage{amsmath, amssymb}
\usepackage{graphicx}
\usepackage{siunitx}
\usepackage{booktabs}
\usepackage{xcolor}
\usepackage[makestderr]{pythontex}

\definecolor{parent}{RGB}{231, 76, 60}
\definecolor{daughter}{RGB}{46, 204, 113}
\definecolor{isochron}{RGB}{52, 152, 219}

\title{Radiometric Dating:\\
Isotope Geochronology and Age Determination}
\author{Department of Earth Science\\Technical Report ES-2024-001}
\date{\today}

\begin{document}
\maketitle

\begin{abstract}
This report presents a comprehensive analysis of radiometric dating methods. We examine radioactive decay kinetics, implement isochron dating for Rb-Sr and U-Pb systems, analyze concordia-discordia relationships, calculate closure temperatures, and demonstrate carbon-14 dating for recent samples. All computations use PythonTeX for reproducibility.
\end{abstract}

\tableofcontents

\chapter{Introduction}

Radiometric dating determines absolute ages of geological materials using radioactive decay:
\begin{equation}
N(t) = N_0 e^{-\lambda t}
\end{equation}
where $N_0$ is the initial number of parent atoms, $\lambda$ is the decay constant, and $t$ is time.

\section{Decay Constant and Half-Life}
The relationship between decay constant and half-life:
\begin{equation}
\lambda = \frac{\ln 2}{t_{1/2}}
\end{equation}

\section{Age Equation}
From the parent-daughter relationship:
\begin{equation}
D = D_0 + N(e^{\lambda t} - 1)
\end{equation}

\begin{pycode}
import numpy as np
import matplotlib.pyplot as plt
from scipy.optimize import curve_fit
from scipy.stats import linregress
plt.rc('text', usetex=True)
plt.rc('font', family='serif')

np.random.seed(42)

# Isotope systems database
isotope_systems = {
    'U-238/Pb-206': {'half_life': 4.468e9, 'lambda': np.log(2)/4.468e9},
    'U-235/Pb-207': {'half_life': 7.038e8, 'lambda': np.log(2)/7.038e8},
    'Rb-87/Sr-87': {'half_life': 4.88e10, 'lambda': np.log(2)/4.88e10},
    'K-40/Ar-40': {'half_life': 1.248e9, 'lambda': np.log(2)/1.248e9},
    'Sm-147/Nd-143': {'half_life': 1.06e11, 'lambda': np.log(2)/1.06e11},
    'C-14/N-14': {'half_life': 5730, 'lambda': np.log(2)/5730}
}

def decay_curve(t, half_life):
    return np.exp(-np.log(2) * t / half_life)

def daughter_growth(t, half_life):
    return 1 - np.exp(-np.log(2) * t / half_life)

def calculate_age(D_star, N, lambda_val):
    return np.log(1 + D_star/N) / lambda_val

def isochron_age(slope, lambda_val):
    return np.log(slope + 1) / lambda_val
\end{pycode}

\chapter{Radioactive Decay Kinetics}

\begin{pycode}
fig, axes = plt.subplots(2, 3, figsize=(14, 8))

# Decay curves for different systems
ax = axes[0, 0]
systems = ['U-238/Pb-206', 'U-235/Pb-207', 'Rb-87/Sr-87', 'K-40/Ar-40']
colors = ['#e74c3c', '#3498db', '#2ecc71', '#f39c12']
for system, color in zip(systems, colors):
    t_max = 5 * isotope_systems[system]['half_life']
    t = np.linspace(0, t_max, 1000)
    N_N0 = decay_curve(t, isotope_systems[system]['half_life'])
    ax.plot(t/1e9, N_N0, label=system.split('/')[0], color=color, linewidth=2)

ax.set_xlabel('Time (Ga)')
ax.set_ylabel('$N/N_0$')
ax.set_title('Parent Isotope Decay')
ax.legend(fontsize=8)
ax.grid(True, alpha=0.3)
ax.set_xlim(0, 20)

# Daughter growth
ax = axes[0, 1]
for system, color in zip(systems, colors):
    t_max = 5 * isotope_systems[system]['half_life']
    t = np.linspace(0, t_max, 1000)
    D_D_inf = daughter_growth(t, isotope_systems[system]['half_life'])
    ax.plot(t/1e9, D_D_inf, label=system.split('/')[1], color=color, linewidth=2)

ax.set_xlabel('Time (Ga)')
ax.set_ylabel('$D^*/D^*_\\infty$')
ax.set_title('Daughter Isotope Growth')
ax.legend(fontsize=8)
ax.grid(True, alpha=0.3)
ax.set_xlim(0, 20)

# Half-life comparison
ax = axes[0, 2]
systems_all = list(isotope_systems.keys())
half_lives = [isotope_systems[s]['half_life'] for s in systems_all]
colors_all = ['#e74c3c', '#3498db', '#2ecc71', '#f39c12', '#9b59b6', '#1abc9c']
bars = ax.barh(range(len(systems_all)), np.log10(half_lives), color=colors_all, alpha=0.7)
ax.set_yticks(range(len(systems_all)))
ax.set_yticklabels([s.split('/')[0] for s in systems_all])
ax.set_xlabel('$\\log_{10}(t_{1/2}$ in years)')
ax.set_title('Half-life Comparison')
ax.grid(True, alpha=0.3, axis='x')

# Rb-Sr isochron
ax = axes[1, 0]
true_age_RbSr = 3.8e9
lambda_Rb = isotope_systems['Rb-87/Sr-87']['lambda']
Rb87_Sr86 = np.array([0.3, 0.8, 1.5, 2.5, 4.0, 6.0])
Sr87_Sr86_initial = 0.7045
Sr87_Sr86 = Sr87_Sr86_initial + Rb87_Sr86 * (np.exp(lambda_Rb * true_age_RbSr) - 1)
Sr87_Sr86_err = np.random.normal(0, 0.001, len(Sr87_Sr86))
Sr87_Sr86_meas = Sr87_Sr86 + Sr87_Sr86_err

slope_Rb, intercept_Rb, r_Rb, p_Rb, se_Rb = linregress(Rb87_Sr86, Sr87_Sr86_meas)
age_RbSr = isochron_age(slope_Rb, lambda_Rb)

ax.errorbar(Rb87_Sr86, Sr87_Sr86_meas, yerr=0.002, fmt='o', color='#2ecc71',
            markersize=8, capsize=3, label='Samples')
fit_x = np.linspace(0, 7, 100)
ax.plot(fit_x, slope_Rb * fit_x + intercept_Rb, 'r-', linewidth=2, label='Isochron')
ax.set_xlabel('$^{87}$Rb/$^{86}$Sr')
ax.set_ylabel('$^{87}$Sr/$^{86}$Sr')
ax.set_title(f'Rb-Sr Isochron: {age_RbSr/1e9:.2f} Ga')
ax.legend()
ax.grid(True, alpha=0.3)

# K-Ar age spectrum
ax = axes[1, 1]
steps = np.arange(1, 11)
cumulative_Ar = np.cumsum(np.random.exponential(10, 10))
cumulative_Ar = cumulative_Ar / cumulative_Ar[-1] * 100
plateau_age = 2.5e9
ages = plateau_age + np.random.normal(0, 0.1e9, 10)
ages[0:2] *= 0.8  # Low-T steps often younger
ages[8:10] *= 1.1  # High-T steps may be older

ax.step(cumulative_Ar, ages/1e9, where='mid', linewidth=2, color='#f39c12')
ax.axhline(y=plateau_age/1e9, color='r', linestyle='--', alpha=0.7, label='Plateau')
ax.fill_between([20, 80], [plateau_age/1e9-0.1]*2, [plateau_age/1e9+0.1]*2,
                alpha=0.2, color='red')
ax.set_xlabel('Cumulative $^{39}$Ar Released (\\%)')
ax.set_ylabel('Apparent Age (Ga)')
ax.set_title('Ar-Ar Age Spectrum')
ax.legend()
ax.grid(True, alpha=0.3)
ax.set_xlim(0, 100)

# C-14 calibration curve
ax = axes[1, 2]
C14_age = np.linspace(0, 50000, 500)
cal_age = C14_age * (1 + 0.05 * np.sin(C14_age/5000))  # Simplified calibration
cal_age += np.random.normal(0, 200, len(C14_age))

ax.plot(cal_age, C14_age, 'b-', linewidth=1.5, alpha=0.7)
ax.plot([0, 50000], [0, 50000], 'r--', alpha=0.5, label='1:1 line')
ax.set_xlabel('Calendar Age (years BP)')
ax.set_ylabel('$^{14}$C Age (years BP)')
ax.set_title('Radiocarbon Calibration')
ax.legend()
ax.grid(True, alpha=0.3)

plt.tight_layout()
plt.savefig('decay_kinetics.pdf', dpi=150, bbox_inches='tight')
plt.close()
\end{pycode}

\begin{figure}[htbp]
\centering
\includegraphics[width=0.95\textwidth]{decay_kinetics.pdf}
\caption{Radioactive decay: (a) parent decay curves, (b) daughter growth, (c) half-life comparison, (d) Rb-Sr isochron, (e) Ar-Ar spectrum, (f) C-14 calibration.}
\end{figure}

\chapter{U-Pb Concordia Dating}

\section{Concordia Equation}
The U-Pb concordia curve represents concordant ages:
\begin{align}
\frac{^{206}\text{Pb}^*}{^{238}\text{U}} &= e^{\lambda_{238} t} - 1 \\
\frac{^{207}\text{Pb}^*}{^{235}\text{U}} &= e^{\lambda_{235} t} - 1
\end{align}

\begin{pycode}
# U-Pb Concordia analysis
fig, axes = plt.subplots(2, 2, figsize=(12, 10))

lambda_238 = isotope_systems['U-238/Pb-206']['lambda']
lambda_235 = isotope_systems['U-235/Pb-207']['lambda']

# Concordia curve
t_conc = np.linspace(0, 4.5e9, 1000)
Pb206_U238 = np.exp(lambda_238 * t_conc) - 1
Pb207_U235 = np.exp(lambda_235 * t_conc) - 1

ax = axes[0, 0]
ax.plot(Pb207_U235, Pb206_U238, 'k-', linewidth=2, label='Concordia')

# Age markers
ages_mark = [0.5e9, 1e9, 2e9, 3e9, 4e9]
for age in ages_mark:
    x = np.exp(lambda_235 * age) - 1
    y = np.exp(lambda_238 * age) - 1
    ax.plot(x, y, 'ko', markersize=8)
    ax.annotate(f'{age/1e9:.1f} Ga', xy=(x, y), xytext=(x+2, y+0.05),
                fontsize=8)

# Discordant samples
upper_intercept = 3.5e9
lower_intercept = 0.5e9
n_samples = 6
fractions = np.linspace(0.1, 0.9, n_samples)
sample_ages = lower_intercept + fractions * (upper_intercept - lower_intercept)
sample_x = np.exp(lambda_235 * sample_ages) - 1 + np.random.normal(0, 0.5, n_samples)
sample_y = np.exp(lambda_238 * sample_ages) - 1 + np.random.normal(0, 0.02, n_samples)

ax.scatter(sample_x, sample_y, c='red', s=60, zorder=5, label='Zircons')

# Discordia line
slope_disc, intercept_disc = np.polyfit(sample_x, sample_y, 1)
disc_x = np.linspace(0, 80, 100)
ax.plot(disc_x, slope_disc * disc_x + intercept_disc, 'r--', linewidth=1.5,
        alpha=0.7, label='Discordia')

ax.set_xlabel('$^{207}$Pb*/$^{235}$U')
ax.set_ylabel('$^{206}$Pb*/$^{238}$U')
ax.set_title('Concordia-Discordia Diagram')
ax.legend(loc='lower right')
ax.grid(True, alpha=0.3)
ax.set_xlim(0, 80)
ax.set_ylim(0, 1.2)

# Wetherill concordia (zoomed)
ax = axes[0, 1]
t_zoom = np.linspace(3e9, 4e9, 500)
Pb206_zoom = np.exp(lambda_238 * t_zoom) - 1
Pb207_zoom = np.exp(lambda_235 * t_zoom) - 1

ax.plot(Pb207_zoom, Pb206_zoom, 'k-', linewidth=2, label='Concordia')

# Concordant zircons
concordant_ages = np.array([3.2e9, 3.4e9, 3.5e9, 3.6e9, 3.8e9])
conc_x = np.exp(lambda_235 * concordant_ages) - 1 + np.random.normal(0, 0.2, 5)
conc_y = np.exp(lambda_238 * concordant_ages) - 1 + np.random.normal(0, 0.005, 5)
ax.scatter(conc_x, conc_y, c='green', s=60, zorder=5, label='Concordant')

# Error ellipses (simplified)
for x, y in zip(conc_x, conc_y):
    ellipse = plt.Circle((x, y), 0.3, fill=False, color='green', alpha=0.5)
    ax.add_patch(ellipse)

ax.set_xlabel('$^{207}$Pb*/$^{235}$U')
ax.set_ylabel('$^{206}$Pb*/$^{238}$U')
ax.set_title('Concordant Zircons (3-4 Ga)')
ax.legend()
ax.grid(True, alpha=0.3)

# Tera-Wasserburg diagram
ax = axes[1, 0]
U238_Pb206 = 1 / Pb206_U238[1:]  # Avoid division by zero
Pb207_Pb206 = Pb207_U235[1:] / Pb206_U238[1:] * (1/137.88)  # Using U isotope ratio

ax.plot(U238_Pb206, Pb207_Pb206, 'k-', linewidth=2)
for age in ages_mark:
    if age > 0:
        x = 1 / (np.exp(lambda_238 * age) - 1)
        y = (np.exp(lambda_235 * age) - 1) / (np.exp(lambda_238 * age) - 1) / 137.88
        ax.plot(x, y, 'ko', markersize=6)

ax.set_xlabel('$^{238}$U/$^{206}$Pb*')
ax.set_ylabel('$^{207}$Pb*/$^{206}$Pb*')
ax.set_title('Tera-Wasserburg Diagram')
ax.set_xlim(0, 20)
ax.grid(True, alpha=0.3)

# Pb-Pb isochron
ax = axes[1, 1]
mu_values = np.array([8, 9, 10, 11, 12])  # Different U/Pb sources
t_earth = 4.55e9
Pb204_initial = 1.0

Pb206_Pb204 = 9.307 + mu_values * (np.exp(lambda_238 * t_earth) - 1)
Pb207_Pb204 = 10.294 + mu_values/137.88 * (np.exp(lambda_235 * t_earth) - 1)
Pb206_Pb204 += np.random.normal(0, 0.1, len(mu_values))
Pb207_Pb204 += np.random.normal(0, 0.02, len(mu_values))

ax.scatter(Pb206_Pb204, Pb207_Pb204, c='#9b59b6', s=60)
slope_Pb, intercept_Pb = np.polyfit(Pb206_Pb204, Pb207_Pb204, 1)
fit_x = np.linspace(14, 22, 100)
ax.plot(fit_x, slope_Pb * fit_x + intercept_Pb, 'r-', linewidth=2)

# Calculate model age
Pb_age = (1/lambda_235) * np.log(1 + slope_Pb * 137.88 *
         (np.exp(lambda_235 * t_earth) - 1) / (np.exp(lambda_238 * t_earth) - 1))

ax.set_xlabel('$^{206}$Pb/$^{204}$Pb')
ax.set_ylabel('$^{207}$Pb/$^{204}$Pb')
ax.set_title('Pb-Pb Isochron')
ax.grid(True, alpha=0.3)

plt.tight_layout()
plt.savefig('UPb_concordia.pdf', dpi=150, bbox_inches='tight')
plt.close()

# Store results
upper_age = upper_intercept
lower_age = lower_intercept
\end{pycode}

\begin{figure}[htbp]
\centering
\includegraphics[width=0.95\textwidth]{UPb_concordia.pdf}
\caption{U-Pb dating: (a) concordia-discordia, (b) concordant zircons, (c) Tera-Wasserburg, (d) Pb-Pb isochron.}
\end{figure}

\chapter{Closure Temperature}

\section{Dodson Equation}
The closure temperature $T_c$ depends on diffusion parameters:
\begin{equation}
T_c = \frac{E_a/R}{\ln\left(\frac{A R T_c^2 D_0/a^2}{E_a \cdot dT/dt}\right)}
\end{equation}

\begin{pycode}
# Closure temperature analysis
fig, axes = plt.subplots(1, 3, figsize=(14, 4))

# Mineral closure temperatures
minerals = ['Zircon (U-Pb)', 'Monazite', 'Titanite', 'Hornblende',
            'Muscovite', 'Biotite', 'K-feldspar', 'Apatite']
Tc = [900, 750, 650, 550, 400, 350, 200, 70]
colors_min = plt.cm.RdYlBu(np.linspace(0.9, 0.1, len(minerals)))

ax = axes[0]
bars = ax.barh(range(len(minerals)), Tc, color=colors_min, alpha=0.8)
ax.set_yticks(range(len(minerals)))
ax.set_yticklabels(minerals)
ax.set_xlabel('Closure Temperature ($^\\circ$C)')
ax.set_title('Mineral Closure Temperatures')
ax.grid(True, alpha=0.3, axis='x')

# Cooling path
ax = axes[1]
time = np.linspace(0, 100, 100)  # Ma
T_path = 900 * np.exp(-time/30)  # Exponential cooling

ax.plot(time, T_path, 'b-', linewidth=2)
for mineral, tc in zip(['Zircon', 'Hornblende', 'Biotite', 'Apatite'],
                       [900, 550, 350, 70]):
    t_closure = -30 * np.log(tc/900)
    if t_closure > 0:
        ax.plot(t_closure, tc, 'ro', markersize=8)
        ax.annotate(mineral, xy=(t_closure, tc), xytext=(t_closure+5, tc+30),
                    fontsize=8, arrowprops=dict(arrowstyle='->', alpha=0.5))

ax.set_xlabel('Time (Ma after crystallization)')
ax.set_ylabel('Temperature ($^\\circ$C)')
ax.set_title('Cooling Path and Closure Events')
ax.grid(True, alpha=0.3)

# Multi-chronometer cooling history
ax = axes[2]
systems_cool = ['U-Pb Zrn', 'Ar-Ar Hbl', 'Rb-Sr Ms', 'K-Ar Bt', 'FT Ap']
ages_cool = [100, 85, 60, 45, 20]  # Ma
Tc_cool = [900, 550, 400, 350, 70]

ax.scatter(ages_cool, Tc_cool, s=100, c='#e74c3c', zorder=5)
for sys, age, tc in zip(systems_cool, ages_cool, Tc_cool):
    ax.annotate(sys, xy=(age, tc), xytext=(age-10, tc+50),
                fontsize=8, arrowprops=dict(arrowstyle='->', alpha=0.5))

# Fit cooling curve
from scipy.optimize import curve_fit
def cooling(t, T0, tau, T_amb):
    return T_amb + (T0 - T_amb) * np.exp(-(100-t)/tau)

popt, _ = curve_fit(cooling, ages_cool, Tc_cool, p0=[1000, 30, 0], maxfev=5000)
t_fit = np.linspace(0, 100, 100)
ax.plot(t_fit, cooling(t_fit, *popt), 'b-', linewidth=2, alpha=0.7)

ax.set_xlabel('Age (Ma)')
ax.set_ylabel('Temperature ($^\\circ$C)')
ax.set_title('Thermochronology')
ax.grid(True, alpha=0.3)
ax.invert_xaxis()

plt.tight_layout()
plt.savefig('closure_temperature.pdf', dpi=150, bbox_inches='tight')
plt.close()

# Cooling rate
cooling_rate = (Tc_cool[0] - Tc_cool[-1]) / (ages_cool[0] - ages_cool[-1])
\end{pycode}

\begin{figure}[htbp]
\centering
\includegraphics[width=0.95\textwidth]{closure_temperature.pdf}
\caption{Closure temperature: (a) mineral comparison, (b) cooling path, (c) multi-system thermochronology.}
\end{figure}

\chapter{Numerical Results}

\begin{pycode}
# Compile results
results_table = [
    ('Rb-Sr isochron age', f'{age_RbSr/1e9:.3f}', 'Ga'),
    ('Initial $^{87}$Sr/$^{86}$Sr', f'{intercept_Rb:.4f}', ''),
    ('Isochron R$^2$', f'{r_Rb**2:.4f}', ''),
    ('U-Pb upper intercept', f'{upper_age/1e9:.2f}', 'Ga'),
    ('U-Pb lower intercept', f'{lower_age/1e9:.2f}', 'Ga'),
    ('Average cooling rate', f'{abs(cooling_rate):.1f}', '$^\\circ$C/Ma'),
]
\end{pycode}

\begin{table}[htbp]
\centering
\caption{Radiometric dating results}
\begin{tabular}{@{}lcc@{}}
\toprule
Parameter & Value & Units \\
\midrule
\py{results_table[0][0]} & \py{results_table[0][1]} & \py{results_table[0][2]} \\
\py{results_table[1][0]} & \py{results_table[1][1]} & \py{results_table[1][2]} \\
\py{results_table[2][0]} & \py{results_table[2][1]} & \py{results_table[2][2]} \\
\py{results_table[3][0]} & \py{results_table[3][1]} & \py{results_table[3][2]} \\
\py{results_table[4][0]} & \py{results_table[4][1]} & \py{results_table[4][2]} \\
\py{results_table[5][0]} & \py{results_table[5][1]} & \py{results_table[5][2]} \\
\bottomrule
\end{tabular}
\end{table}

\chapter{Conclusions}

\begin{enumerate}
    \item Isochron dating provides both age and initial ratios
    \item U-Pb concordia reveals open-system behavior
    \item Multiple isotope systems enable cross-validation
    \item Closure temperature controls age interpretation
    \item Thermochronology constrains cooling histories
    \item Radiocarbon requires calibration for calendar ages
\end{enumerate}

\end{document}
