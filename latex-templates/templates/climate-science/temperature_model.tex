% Temperature Model Template
% Topics: Energy balance, radiative forcing, climate sensitivity, feedback analysis
% Style: Research article with model validation

\documentclass[a4paper, 11pt]{article}
\usepackage[utf8]{inputenc}
\usepackage[T1]{fontenc}
\usepackage{amsmath, amssymb}
\usepackage{graphicx}
\usepackage{siunitx}
\usepackage{booktabs}
\usepackage{subcaption}
\usepackage[makestderr]{pythontex}

% Theorem environments
\newtheorem{definition}{Definition}[section]
\newtheorem{theorem}{Theorem}[section]
\newtheorem{example}{Example}[section]
\newtheorem{remark}{Remark}[section]

\title{Global Temperature Modeling: Energy Balance and Climate Sensitivity}
\author{Climate Dynamics Research Group}
\date{\today}

\begin{document}
\maketitle

\begin{abstract}
This study presents energy balance models for global mean surface temperature,
examining radiative forcing from greenhouse gases and the response of the climate system.
We analyze zero-dimensional and one-dimensional models, calculate climate sensitivity
from different feedback mechanisms, and compare model projections with observations.
The analysis quantifies transient and equilibrium climate response to CO$_2$ forcing.
\end{abstract}

\section{Introduction}

Earth's global mean surface temperature is determined by the balance between incoming
solar radiation and outgoing longwave radiation. Perturbations to this balance from
greenhouse gas increases lead to warming until a new equilibrium is reached.

\begin{definition}[Radiative Forcing]
Radiative forcing $F$ is the change in net radiative flux at the tropopause due to
a change in an external driver. For CO$_2$:
\begin{equation}
F = 5.35 \ln\left(\frac{C}{C_0}\right) \quad \text{W/m}^2
\end{equation}
where $C$ is CO$_2$ concentration and $C_0$ is the reference (pre-industrial) value.
\end{definition}

\section{Theoretical Framework}

\subsection{Zero-Dimensional Energy Balance}

\begin{theorem}[Planetary Energy Balance]
The rate of change of Earth's heat content is:
\begin{equation}
C_p \frac{dT}{dt} = F - \lambda \Delta T
\end{equation}
where $C_p$ is heat capacity, $F$ is radiative forcing, $\lambda$ is the climate
feedback parameter, and $\Delta T = T - T_0$ is the temperature anomaly.
\end{theorem}

\begin{definition}[Climate Sensitivity]
Equilibrium climate sensitivity (ECS) is the warming for doubled CO$_2$:
\begin{equation}
\text{ECS} = \frac{F_{2\times\text{CO}_2}}{\lambda} = \frac{3.7}{\lambda} \quad \text{K}
\end{equation}
With $\lambda \approx 1.2$ W/(m$^2$K), ECS $\approx$ 3 K.
\end{definition}

\subsection{Climate Feedbacks}

\begin{theorem}[Feedback Analysis]
The total feedback parameter is the sum of individual feedbacks:
\begin{equation}
\lambda = \lambda_0 + \lambda_{WV} + \lambda_{LR} + \lambda_A + \lambda_C
\end{equation}
where $\lambda_0$ is the Planck response (no feedbacks), and other terms are water
vapor, lapse rate, albedo, and cloud feedbacks.
\end{theorem}

\begin{remark}[Feedback Values]
Typical feedback values (W m$^{-2}$ K$^{-1}$):
\begin{itemize}
\item Planck (blackbody): $\lambda_0 \approx 3.2$ (negative, stabilizing)
\item Water vapor: $\lambda_{WV} \approx -1.8$ (positive, amplifying)
\item Lapse rate: $\lambda_{LR} \approx 0.6$ (negative)
\item Albedo: $\lambda_A \approx -0.3$ (positive)
\item Cloud: $\lambda_C \approx -0.5$ (positive, uncertain)
\end{itemize}
Net: $\lambda \approx 1.2$ W/(m$^2$K)
\end{remark}

\subsection{Transient Climate Response}

\begin{definition}[TCR and ECS]
\begin{itemize}
\item \textbf{Transient Climate Response (TCR)}: Warming at time of CO$_2$ doubling
under 1\%/yr increase ($\sim$70 years)
\item \textbf{Equilibrium Climate Sensitivity (ECS)}: Final equilibrium warming
for doubled CO$_2$
\end{itemize}
Typically TCR/ECS $\approx$ 0.5--0.7 due to ocean heat uptake.
\end{definition}

\section{Computational Analysis}

\begin{pycode}
import numpy as np
import matplotlib.pyplot as plt
from scipy.integrate import odeint

np.random.seed(42)

# Energy balance model
def ebm_single(T, t, F_func, C_eff, lambda_val):
    F = F_func(t)
    dTdt = (F - lambda_val * T) / C_eff
    return dTdt

# Two-box model (surface + deep ocean)
def ebm_twobox(y, t, F_func, C_s, C_d, lambda_val, gamma):
    T_s, T_d = y
    F = F_func(t)
    dTs = (F - lambda_val * T_s - gamma * (T_s - T_d)) / C_s
    dTd = gamma * (T_s - T_d) / C_d
    return [dTs, dTd]

# Parameters
C_eff = 10.0  # Effective heat capacity (W yr m^-2 K^-1)
C_s = 8.0     # Surface layer
C_d = 100.0   # Deep ocean
lambda_val = 1.2  # Climate feedback (W m^-2 K^-1)
gamma = 0.5   # Ocean heat exchange

# Forcing scenarios
CO2_pi = 280  # Pre-industrial CO2 (ppm)

def forcing_step(t):
    """Step doubling of CO2"""
    return 3.7 if t > 0 else 0

def forcing_ramp(t):
    """1%/yr CO2 increase"""
    if t < 0:
        return 0
    elif t < 70:
        return 5.35 * np.log(1.01**t)
    else:
        return 5.35 * np.log(2)

def forcing_historical(t):
    """Simplified historical forcing"""
    if t < 1850:
        return 0
    elif t < 2020:
        CO2 = 280 * np.exp(0.004 * (t - 1850))
        return 5.35 * np.log(CO2 / 280)
    else:
        return 3.0

def forcing_rcp(t, scenario='4.5'):
    """RCP scenario forcing"""
    F_2020 = 3.0
    if t < 2020:
        return forcing_historical(t)
    if scenario == '2.6':
        return F_2020 + 0.5 * (1 - np.exp(-0.05 * (t - 2020)))
    elif scenario == '4.5':
        return F_2020 + 1.5 * (1 - np.exp(-0.03 * (t - 2020)))
    else:  # 8.5
        return F_2020 + 0.08 * (t - 2020)

# Time arrays
t_step = np.linspace(-10, 200, 500)
t_ramp = np.linspace(-10, 140, 500)
t_hist = np.linspace(1850, 2100, 251)

# Solve step response
T_step = odeint(ebm_single, 0, t_step, args=(forcing_step, C_eff, lambda_val))[:, 0]
y0_2box = [0, 0]
T_2box_step = odeint(ebm_twobox, y0_2box, t_step,
                     args=(forcing_step, C_s, C_d, lambda_val, gamma))

# Solve ramp response
T_ramp = odeint(ebm_single, 0, t_ramp, args=(forcing_ramp, C_eff, lambda_val))[:, 0]
T_2box_ramp = odeint(ebm_twobox, y0_2box, t_ramp,
                     args=(forcing_ramp, C_s, C_d, lambda_val, gamma))

# Historical simulation
T_hist = odeint(ebm_single, 0, t_hist,
                args=(lambda t: forcing_historical(t), C_eff, lambda_val))[:, 0]

# RCP scenarios
T_rcp26 = odeint(ebm_single, 0, t_hist,
                 args=(lambda t: forcing_rcp(t, '2.6'), C_eff, lambda_val))[:, 0]
T_rcp45 = odeint(ebm_single, 0, t_hist,
                 args=(lambda t: forcing_rcp(t, '4.5'), C_eff, lambda_val))[:, 0]
T_rcp85 = odeint(ebm_single, 0, t_hist,
                 args=(lambda t: forcing_rcp(t, '8.5'), C_eff, lambda_val))[:, 0]

# Climate sensitivity calculation
ECS = 3.7 / lambda_val
TCR = T_ramp[np.argmin(np.abs(t_ramp - 70))]

# Feedback analysis
lambda_0 = 3.2   # Planck
lambda_wv = -1.8  # Water vapor
lambda_lr = 0.6   # Lapse rate
lambda_a = -0.3   # Albedo
lambda_c = -0.5   # Cloud
feedbacks = {
    'Planck': lambda_0,
    'Water vapor': lambda_wv,
    'Lapse rate': lambda_lr,
    'Albedo': lambda_a,
    'Cloud': lambda_c
}
lambda_total = sum(feedbacks.values())

# ECS uncertainty range
lambda_range = np.linspace(0.8, 2.0, 100)
ECS_range = 3.7 / lambda_range

# Create figure
fig = plt.figure(figsize=(14, 12))

# Plot 1: Step response
ax1 = fig.add_subplot(3, 3, 1)
ax1.plot(t_step, T_step, 'b-', linewidth=2, label='1-box')
ax1.plot(t_step, T_2box_step[:, 0], 'r-', linewidth=2, label='2-box surface')
ax1.plot(t_step, T_2box_step[:, 1], 'r--', linewidth=1.5, label='2-box deep')
ax1.axhline(y=ECS, color='gray', linestyle='--', alpha=0.7, label=f'ECS = {ECS:.1f} K')
ax1.set_xlabel('Time (years)')
ax1.set_ylabel('$\\Delta T$ (K)')
ax1.set_title('Step Response (2xCO$_2$)')
ax1.legend(fontsize=7)
ax1.set_xlim([-10, 200])

# Plot 2: Ramp response
ax2 = fig.add_subplot(3, 3, 2)
ax2.plot(t_ramp, T_ramp, 'b-', linewidth=2, label='Temperature')
forcing_ramp_arr = np.array([forcing_ramp(t) for t in t_ramp])
ax2_twin = ax2.twinx()
ax2_twin.plot(t_ramp, forcing_ramp_arr, 'r--', linewidth=1.5, label='Forcing')
ax2.axvline(x=70, color='gray', linestyle=':', alpha=0.7)
ax2.scatter([70], [TCR], s=100, c='green', zorder=5, label=f'TCR = {TCR:.2f} K')
ax2.set_xlabel('Time (years)')
ax2.set_ylabel('$\\Delta T$ (K)', color='b')
ax2_twin.set_ylabel('Forcing (W/m$^2$)', color='r')
ax2.set_title('1\\%/yr CO$_2$ Increase')
ax2.legend(fontsize=8, loc='upper left')

# Plot 3: Historical + projections
ax3 = fig.add_subplot(3, 3, 3)
ax3.plot(t_hist, T_hist, 'k-', linewidth=2, label='Historical')
ax3.plot(t_hist, T_rcp26, 'g-', linewidth=2, label='RCP 2.6')
ax3.plot(t_hist, T_rcp45, 'b-', linewidth=2, label='RCP 4.5')
ax3.plot(t_hist, T_rcp85, 'r-', linewidth=2, label='RCP 8.5')
ax3.axhline(y=1.5, color='orange', linestyle=':', label='1.5 K')
ax3.axhline(y=2.0, color='red', linestyle=':', label='2.0 K')
ax3.set_xlabel('Year')
ax3.set_ylabel('$\\Delta T$ (K)')
ax3.set_title('Temperature Projections')
ax3.legend(fontsize=7, loc='upper left')

# Plot 4: Feedback analysis
ax4 = fig.add_subplot(3, 3, 4)
names = list(feedbacks.keys())
values = list(feedbacks.values())
colors = ['gray', 'blue', 'cyan', 'orange', 'purple']
bars = ax4.barh(names, values, color=colors)
ax4.axvline(x=0, color='black', linewidth=0.5)
ax4.set_xlabel('Feedback (W m$^{-2}$ K$^{-1}$)')
ax4.set_title(f'Feedback Components ($\\lambda_{{total}}$ = {lambda_total:.1f})')

# Plot 5: ECS distribution
ax5 = fig.add_subplot(3, 3, 5)
ax5.plot(lambda_range, ECS_range, 'b-', linewidth=2)
ax5.axhline(y=3.0, color='red', linestyle='--', alpha=0.7, label='Best estimate')
ax5.fill_between([1.0, 1.5], 0, 6, alpha=0.2, color='green', label='Likely range')
ax5.set_xlabel('$\\lambda$ (W m$^{-2}$ K$^{-1}$)')
ax5.set_ylabel('ECS (K)')
ax5.set_title('ECS vs Feedback Parameter')
ax5.legend(fontsize=8)
ax5.set_ylim([0, 6])

# Plot 6: Warming rate
ax6 = fig.add_subplot(3, 3, 6)
dT_dt = np.diff(T_hist) / np.diff(t_hist)
ax6.plot(t_hist[1:], dT_dt * 10, 'b-', linewidth=1.5)  # per decade
ax6.set_xlabel('Year')
ax6.set_ylabel('Warming rate (K/decade)')
ax6.set_title('Rate of Temperature Change')

# Plot 7: Ocean heat uptake
ax7 = fig.add_subplot(3, 3, 7)
ax7.plot(t_step, T_2box_step[:, 0] - T_2box_step[:, 1], 'purple', linewidth=2)
ax7.set_xlabel('Time (years)')
ax7.set_ylabel('$T_s - T_d$ (K)')
ax7.set_title('Surface-Deep Ocean Temperature Difference')

# Plot 8: Forcing components
ax8 = fig.add_subplot(3, 3, 8)
# Different forcing agents (simplified)
t_forcing = np.linspace(1850, 2020, 100)
CO2_forcing = 5.35 * np.log(280 * np.exp(0.004 * (t_forcing - 1850)) / 280)
CH4_forcing = 0.3 * (1 - np.exp(-0.01 * (t_forcing - 1850)))
aerosol_forcing = -0.5 * (1 - np.exp(-0.02 * (t_forcing - 1850)))
ax8.plot(t_forcing, CO2_forcing, 'r-', linewidth=2, label='CO$_2$')
ax8.plot(t_forcing, CH4_forcing, 'g-', linewidth=2, label='CH$_4$')
ax8.plot(t_forcing, aerosol_forcing, 'b-', linewidth=2, label='Aerosols')
ax8.plot(t_forcing, CO2_forcing + CH4_forcing + aerosol_forcing, 'k--',
         linewidth=2, label='Total')
ax8.set_xlabel('Year')
ax8.set_ylabel('Forcing (W/m$^2$)')
ax8.set_title('Radiative Forcing Components')
ax8.legend(fontsize=8)

# Plot 9: Energy imbalance
ax9 = fig.add_subplot(3, 3, 9)
F_arr = np.array([forcing_historical(t) for t in t_hist])
N = F_arr - lambda_val * T_hist  # Net energy imbalance
ax9.plot(t_hist, N, 'b-', linewidth=2)
ax9.axhline(y=0, color='black', linewidth=0.5)
ax9.set_xlabel('Year')
ax9.set_ylabel('N (W/m$^2$)')
ax9.set_title('Planetary Energy Imbalance')
ax9.fill_between(t_hist, 0, N, where=N > 0, alpha=0.3, color='red', label='Heating')
ax9.legend(fontsize=8)

plt.tight_layout()
plt.savefig('temperature_model_analysis.pdf', dpi=150, bbox_inches='tight')
plt.close()

# Key results
T_2020 = T_hist[np.argmin(np.abs(t_hist - 2020))]
T_2100_rcp45 = T_rcp45[-1]
\end{pycode}

\begin{figure}[htbp]
\centering
\includegraphics[width=\textwidth]{temperature_model_analysis.pdf}
\caption{Climate temperature modeling: (a) Step response to CO$_2$ doubling; (b) Ramp
response showing TCR; (c) Historical and projected temperatures; (d) Feedback component
analysis; (e) ECS dependence on feedback parameter; (f) Warming rate over time;
(g) Ocean heat uptake dynamics; (h) Radiative forcing components; (i) Planetary energy
imbalance.}
\label{fig:temperature}
\end{figure}

\section{Results}

\subsection{Climate Sensitivity}

\begin{pycode}
print(r"\begin{table}[htbp]")
print(r"\centering")
print(r"\caption{Climate Sensitivity Parameters}")
print(r"\begin{tabular}{lcc}")
print(r"\toprule")
print(r"Parameter & Value & Units \\")
print(r"\midrule")
print(f"Feedback parameter $\\lambda$ & {lambda_val} & W m$^{{-2}}$ K$^{{-1}}$ \\\\")
print(f"Equilibrium Climate Sensitivity & {ECS:.1f} & K \\\\")
print(f"Transient Climate Response & {TCR:.2f} & K \\\\")
print(f"TCR/ECS ratio & {TCR/ECS:.2f} & --- \\\\")
print(r"\bottomrule")
print(r"\end{tabular}")
print(r"\label{tab:sensitivity}")
print(r"\end{table}")
\end{pycode}

\subsection{Temperature Projections}

\begin{pycode}
print(r"\begin{table}[htbp]")
print(r"\centering")
print(r"\caption{Projected Temperature Changes}")
print(r"\begin{tabular}{lcc}")
print(r"\toprule")
print(r"Scenario & 2100 $\Delta T$ (K) & Exceeds 2 K? \\")
print(r"\midrule")
print(f"RCP 2.6 & {T_rcp26[-1]:.1f} & {'No' if T_rcp26[-1] < 2 else 'Yes'} \\\\")
print(f"RCP 4.5 & {T_rcp45[-1]:.1f} & {'No' if T_rcp45[-1] < 2 else 'Yes'} \\\\")
print(f"RCP 8.5 & {T_rcp85[-1]:.1f} & {'No' if T_rcp85[-1] < 2 else 'Yes'} \\\\")
print(r"\bottomrule")
print(r"\end{tabular}")
print(r"\label{tab:projections}")
print(r"\end{table}")
\end{pycode}

\section{Discussion}

\begin{example}[Equilibrium vs Transient Response]
The TCR of $\py{f"{TCR:.2f}"}$ K is smaller than the ECS of $\py{f"{ECS:.1f}"}$ K because
the deep ocean absorbs heat. The ratio TCR/ECS = $\py{f"{TCR/ECS:.2f}"}$ indicates that
only $\sim$\py{f"{TCR/ECS*100:.0f}"}\% of equilibrium warming is realized at the time
of CO$_2$ doubling.
\end{example}

\begin{remark}[Feedback Uncertainty]
Cloud feedback remains the largest source of uncertainty in ECS estimates. Positive
cloud feedback (reduced low clouds with warming) increases ECS, while negative feedback
decreases it. Current estimates range from $-0.2$ to $-1.2$ W m$^{-2}$ K$^{-1}$.
\end{remark}

\begin{example}[Committed Warming]
Even if emissions stop, warming continues due to:
\begin{itemize}
\item Ocean thermal inertia (decades to equilibrate)
\item Reduction in aerosol cooling (immediate)
\item Carbon cycle feedbacks (decades to centuries)
\end{itemize}
This "committed warming" adds $\sim$0.5 K to current warming.
\end{example}

\section{Conclusions}

This temperature modeling analysis demonstrates:
\begin{enumerate}
\item ECS is $\py{f"{ECS:.1f}"}$ K with feedback parameter $\lambda = \py{f"{lambda_val}"}$ W m$^{-2}$ K$^{-1}$
\item TCR is $\py{f"{TCR:.2f}"}$ K, approximately \py{f"{TCR/ECS*100:.0f}"}\% of ECS
\item Current warming is $\sim$\py{f"{T_2020:.1f}"} K above pre-industrial
\item RCP 4.5 projects $\py{f"{T_2100_rcp45:.1f}"}$ K warming by 2100
\item Water vapor and cloud feedbacks dominate sensitivity uncertainty
\end{enumerate}

\section*{Further Reading}

\begin{itemize}
\item Hartmann, D.L. \textit{Global Physical Climatology}, 2nd ed. Elsevier, 2016.
\item Held, I.M. \& Soden, B.J. Water vapor feedback and global warming. \textit{Annu. Rev. Energy Environ.} 25, 441--475, 2000.
\item Sherwood, S.C. et al. An assessment of Earth's climate sensitivity using multiple lines of evidence. \textit{Rev. Geophys.} 58, e2019RG000678, 2020.
\end{itemize}

\end{document}
