% Carbon Cycle Model Template
% Topics: Box models, carbon reservoirs, anthropogenic perturbation, feedback loops
% Style: Technical report with policy implications

\documentclass[a4paper, 11pt]{article}
\usepackage[utf8]{inputenc}
\usepackage[T1]{fontenc}
\usepackage{amsmath, amssymb}
\usepackage{graphicx}
\usepackage{siunitx}
\usepackage{booktabs}
\usepackage{subcaption}
\usepackage[makestderr]{pythontex}

% Theorem environments
\newtheorem{definition}{Definition}[section]
\newtheorem{theorem}{Theorem}[section]
\newtheorem{example}{Example}[section]
\newtheorem{remark}{Remark}[section]

\title{Global Carbon Cycle Modeling: Reservoirs, Fluxes, and Anthropogenic Perturbation}
\author{Earth System Science Research}
\date{\today}

\begin{document}
\maketitle

\begin{abstract}
This technical report presents a comprehensive analysis of the global carbon cycle using
box models to represent carbon exchange between atmosphere, ocean, and terrestrial biosphere.
We examine natural carbon fluxes, anthropogenic emissions, and the resulting changes in
atmospheric CO$_2$ concentration. The model explores the airborne fraction of emissions,
ocean uptake dynamics, and climate-carbon feedbacks. Projections under different emission
scenarios illustrate the long-term implications for atmospheric carbon levels.
\end{abstract}

\section{Introduction}

The global carbon cycle plays a central role in Earth's climate system. Anthropogenic
emissions have perturbed this cycle, leading to rising atmospheric CO$_2$ concentrations
and associated climate change.

\begin{definition}[Carbon Reservoirs]
The major carbon reservoirs are:
\begin{itemize}
\item \textbf{Atmosphere}: $\sim$850 PgC (pre-industrial: 590 PgC)
\item \textbf{Ocean}: $\sim$38,000 PgC (surface + deep)
\item \textbf{Terrestrial biosphere}: $\sim$2,000 PgC (vegetation + soil)
\item \textbf{Fossil fuels}: $\sim$4,000 PgC
\end{itemize}
(1 PgC = $10^{15}$ g carbon = 1 GtC)
\end{definition}

\section{Theoretical Framework}

\subsection{Box Model Equations}

\begin{theorem}[Three-Box Carbon Cycle Model]
The evolution of carbon in atmosphere ($C_A$), surface ocean ($C_O$), and biosphere ($C_B$) is:
\begin{align}
\frac{dC_A}{dt} &= -k_{AO}(C_A - C_A^{eq}) - k_{AB}(C_A - C_A^{eq}) + E(t) \\
\frac{dC_O}{dt} &= k_{AO}(C_A - C_A^{eq}) - k_{OD}(C_O - C_O^{eq}) \\
\frac{dC_B}{dt} &= k_{AB}(C_A - C_A^{eq}) - k_{BR}C_B
\end{align}
where $k$ values are exchange coefficients and $E(t)$ is anthropogenic emission.
\end{theorem}

\subsection{CO$_2$ Concentration and Carbon Mass}

\begin{definition}[Conversion Factors]
Atmospheric CO$_2$ concentration (ppm) relates to carbon mass (PgC):
\begin{equation}
C_A \text{ (PgC)} = \frac{\text{CO}_2 \text{ (ppm)}}{2.13}
\end{equation}
Thus 1 ppm CO$_2 \approx 2.13$ PgC.
\end{definition}

\subsection{Airborne Fraction}

\begin{definition}[Airborne Fraction]
The fraction of anthropogenic emissions remaining in the atmosphere:
\begin{equation}
f_{airborne} = \frac{\Delta C_A}{\sum E(t)}
\end{equation}
Currently $f_{airborne} \approx 0.44$ (the ocean and biosphere absorb $\sim$56\%).
\end{definition}

\begin{remark}[Ocean Chemistry]
Ocean CO$_2$ uptake is limited by the Revelle factor:
\begin{equation}
R = \frac{\Delta[\text{CO}_2]/[\text{CO}_2]}{\Delta\text{DIC}/\text{DIC}} \approx 10
\end{equation}
Only 1/R of absorbed CO$_2$ remains as dissolved CO$_2$; the rest converts to bicarbonate.
\end{remark}

\section{Computational Analysis}

\begin{pycode}
import numpy as np
import matplotlib.pyplot as plt
from scipy.integrate import odeint

np.random.seed(42)

# Carbon cycle box model
def carbon_cycle(y, t, params, emission_func):
    C_A, C_O, C_B = y
    k_AO, k_AB, k_OD, k_BR, C_A_eq, C_O_eq = params

    E = emission_func(t)

    dC_A = -k_AO * (C_A - C_A_eq) - k_AB * (C_A - C_A_eq) + E
    dC_O = k_AO * (C_A - C_A_eq) - k_OD * (C_O - C_O_eq)
    dC_B = k_AB * (C_A - C_A_eq) - k_BR * C_B

    return [dC_A, dC_O, dC_B]

# Parameters (based on simplified IPCC values)
k_AO = 0.1   # Atmosphere-ocean exchange (1/yr)
k_AB = 0.05  # Atmosphere-biosphere exchange (1/yr)
k_OD = 0.01  # Surface-deep ocean exchange (1/yr)
k_BR = 0.02  # Biosphere respiration (1/yr)
C_A_eq = 590  # Pre-industrial atmospheric C (PgC)
C_O_eq = 1000  # Surface ocean equilibrium (PgC)
params = (k_AO, k_AB, k_OD, k_BR, C_A_eq, C_O_eq)

# Initial conditions (year 1850)
C_A_0 = 590  # PgC
C_O_0 = 1000  # PgC
C_B_0 = 550  # PgC
y0 = [C_A_0, C_O_0, C_B_0]

# Emission scenarios
def emissions_historical(t):
    # Simplified historical emissions (PgC/yr)
    if t < 1900:
        return 0.5 * np.exp(0.02 * (t - 1850))
    elif t < 2000:
        return 1.0 * np.exp(0.025 * (t - 1900))
    else:
        return 10.0

def emissions_rcp26(t):
    # RCP 2.6 - aggressive mitigation
    if t < 2020:
        return emissions_historical(t)
    elif t < 2050:
        return 10.0 * (1 - 0.8 * (t - 2020) / 30)
    else:
        return 2.0 * np.exp(-0.05 * (t - 2050))

def emissions_rcp45(t):
    # RCP 4.5 - moderate mitigation
    if t < 2040:
        return emissions_historical(t) if t < 2020 else 10.0 + 0.05 * (t - 2020)
    elif t < 2080:
        return 11.0 - 0.2 * (t - 2040)
    else:
        return 3.0

def emissions_rcp85(t):
    # RCP 8.5 - business as usual
    if t < 2020:
        return emissions_historical(t)
    else:
        return 10.0 * np.exp(0.01 * (t - 2020))

# Time arrays
t_hist = np.linspace(1850, 2020, 171)
t_proj = np.linspace(2020, 2100, 81)
t_full = np.linspace(1850, 2100, 251)

# Solve for historical period
sol_hist = odeint(carbon_cycle, y0, t_hist, args=(params, emissions_historical))

# Solve for different scenarios
y0_2020 = sol_hist[-1]
sol_rcp26 = odeint(carbon_cycle, y0_2020, t_proj, args=(params, emissions_rcp26))
sol_rcp45 = odeint(carbon_cycle, y0_2020, t_proj, args=(params, emissions_rcp45))
sol_rcp85 = odeint(carbon_cycle, y0_2020, t_proj, args=(params, emissions_rcp85))

# Convert to CO2 concentration (ppm)
ppm_conversion = 2.13
CO2_hist = sol_hist[:, 0] / ppm_conversion
CO2_rcp26 = sol_rcp26[:, 0] / ppm_conversion
CO2_rcp45 = sol_rcp45[:, 0] / ppm_conversion
CO2_rcp85 = sol_rcp85[:, 0] / ppm_conversion

# Calculate cumulative emissions
cumulative_hist = np.cumsum([emissions_historical(t) for t in t_hist]) * (t_hist[1] - t_hist[0])

# Calculate airborne fraction
delta_C_A = sol_hist[-1, 0] - C_A_0
airborne_fraction = delta_C_A / cumulative_hist[-1]

# Ocean and land uptake
ocean_uptake = sol_hist[:, 1] - C_O_0
land_uptake = sol_hist[:, 2] - C_B_0

# Create annual emissions array
emissions_array_hist = np.array([emissions_historical(t) for t in t_hist])
emissions_rcp26_arr = np.array([emissions_rcp26(t) for t in t_proj])
emissions_rcp45_arr = np.array([emissions_rcp45(t) for t in t_proj])
emissions_rcp85_arr = np.array([emissions_rcp85(t) for t in t_proj])

# Create figure
fig = plt.figure(figsize=(14, 12))

# Plot 1: Historical CO2 concentration
ax1 = fig.add_subplot(3, 3, 1)
ax1.plot(t_hist, CO2_hist, 'b-', linewidth=2)
ax1.axhline(y=280, color='gray', linestyle='--', alpha=0.7, label='Pre-industrial')
ax1.set_xlabel('Year')
ax1.set_ylabel('CO$_2$ (ppm)')
ax1.set_title('Historical Atmospheric CO$_2$')
ax1.legend(fontsize=8)

# Plot 2: Emission scenarios
ax2 = fig.add_subplot(3, 3, 2)
ax2.plot(t_hist, emissions_array_hist, 'k-', linewidth=2, label='Historical')
ax2.plot(t_proj, emissions_rcp26_arr, 'g-', linewidth=2, label='RCP 2.6')
ax2.plot(t_proj, emissions_rcp45_arr, 'b-', linewidth=2, label='RCP 4.5')
ax2.plot(t_proj, emissions_rcp85_arr, 'r-', linewidth=2, label='RCP 8.5')
ax2.set_xlabel('Year')
ax2.set_ylabel('Emissions (PgC/yr)')
ax2.set_title('CO$_2$ Emission Scenarios')
ax2.legend(fontsize=8)

# Plot 3: Projected CO2 concentrations
ax3 = fig.add_subplot(3, 3, 3)
ax3.plot(t_hist, CO2_hist, 'k-', linewidth=2, label='Historical')
ax3.plot(t_proj, CO2_rcp26, 'g-', linewidth=2, label='RCP 2.6')
ax3.plot(t_proj, CO2_rcp45, 'b-', linewidth=2, label='RCP 4.5')
ax3.plot(t_proj, CO2_rcp85, 'r-', linewidth=2, label='RCP 8.5')
ax3.axhline(y=450, color='orange', linestyle=':', alpha=0.7)
ax3.set_xlabel('Year')
ax3.set_ylabel('CO$_2$ (ppm)')
ax3.set_title('Projected CO$_2$ Concentrations')
ax3.legend(fontsize=8)

# Plot 4: Carbon reservoir changes
ax4 = fig.add_subplot(3, 3, 4)
ax4.plot(t_hist, sol_hist[:, 0] - C_A_0, 'r-', linewidth=2, label='Atmosphere')
ax4.plot(t_hist, ocean_uptake, 'b-', linewidth=2, label='Ocean')
ax4.plot(t_hist, land_uptake, 'g-', linewidth=2, label='Land')
ax4.set_xlabel('Year')
ax4.set_ylabel('$\\Delta$C (PgC)')
ax4.set_title('Carbon Reservoir Changes')
ax4.legend(fontsize=8)

# Plot 5: Cumulative emissions and uptake
ax5 = fig.add_subplot(3, 3, 5)
ax5.fill_between(t_hist, 0, cumulative_hist, alpha=0.3, label='Total emissions')
ax5.plot(t_hist, sol_hist[:, 0] - C_A_0, 'r-', linewidth=2, label='Atmosphere')
ax5.set_xlabel('Year')
ax5.set_ylabel('Cumulative C (PgC)')
ax5.set_title(f'Airborne Fraction = {airborne_fraction:.2f}')
ax5.legend(fontsize=8)

# Plot 6: Sink efficiency over time
ax6 = fig.add_subplot(3, 3, 6)
cumsum = np.cumsum(emissions_array_hist)
atm_increase = sol_hist[:, 0] - C_A_0
cumsum[cumsum == 0] = 1  # Avoid division by zero
airborne_time = atm_increase / cumsum
ax6.plot(t_hist[10:], airborne_time[10:], 'purple', linewidth=2)
ax6.axhline(y=0.5, color='gray', linestyle='--', alpha=0.7)
ax6.set_xlabel('Year')
ax6.set_ylabel('Airborne fraction')
ax6.set_title('Sink Efficiency Over Time')
ax6.set_ylim([0.3, 0.7])

# Plot 7: CO2 growth rate
ax7 = fig.add_subplot(3, 3, 7)
growth_rate = np.diff(CO2_hist) / np.diff(t_hist)
ax7.plot(t_hist[1:], growth_rate, 'b-', linewidth=1.5)
ax7.set_xlabel('Year')
ax7.set_ylabel('CO$_2$ growth (ppm/yr)')
ax7.set_title('Atmospheric CO$_2$ Growth Rate')

# Plot 8: Temperature proxy (simplified)
ax8 = fig.add_subplot(3, 3, 8)
# Climate sensitivity ~3 K per doubling
ECS = 3.0  # K
dT_hist = ECS * np.log(CO2_hist / 280) / np.log(2)
dT_rcp26 = ECS * np.log(CO2_rcp26 / 280) / np.log(2)
dT_rcp45 = ECS * np.log(CO2_rcp45 / 280) / np.log(2)
dT_rcp85 = ECS * np.log(CO2_rcp85 / 280) / np.log(2)
ax8.plot(t_hist, dT_hist, 'k-', linewidth=2, label='Historical')
ax8.plot(t_proj, dT_rcp26, 'g-', linewidth=2, label='RCP 2.6')
ax8.plot(t_proj, dT_rcp45, 'b-', linewidth=2, label='RCP 4.5')
ax8.plot(t_proj, dT_rcp85, 'r-', linewidth=2, label='RCP 8.5')
ax8.axhline(y=1.5, color='orange', linestyle=':', label='1.5 K target')
ax8.axhline(y=2.0, color='red', linestyle=':', label='2.0 K limit')
ax8.set_xlabel('Year')
ax8.set_ylabel('$\\Delta T$ (K)')
ax8.set_title('Implied Temperature Change')
ax8.legend(fontsize=7)

# Plot 9: Carbon budget
ax9 = fig.add_subplot(3, 3, 9)
budget_15 = 420  # PgC remaining for 1.5 K
budget_20 = 1170  # PgC remaining for 2.0 K
emissions_cum_proj = np.cumsum(emissions_rcp45_arr) * (t_proj[1] - t_proj[0])
ax9.bar(['1.5 K', '2.0 K'], [budget_15, budget_20], color=['orange', 'red'], alpha=0.7)
ax9.axhline(y=emissions_cum_proj[-1], color='blue', linestyle='--',
            label=f'RCP 4.5 emissions: {emissions_cum_proj[-1]:.0f} PgC')
ax9.set_ylabel('Carbon budget (PgC)')
ax9.set_title('Remaining Carbon Budget')
ax9.legend(fontsize=8)

plt.tight_layout()
plt.savefig('carbon_cycle_analysis.pdf', dpi=150, bbox_inches='tight')
plt.close()

# Final values
CO2_2020 = CO2_hist[-1]
CO2_2100_rcp26 = CO2_rcp26[-1]
CO2_2100_rcp85 = CO2_rcp85[-1]
\end{pycode}

\begin{figure}[htbp]
\centering
\includegraphics[width=\textwidth]{carbon_cycle_analysis.pdf}
\caption{Global carbon cycle analysis: (a) Historical CO$_2$ rise; (b) Emission scenarios;
(c) Projected CO$_2$ concentrations; (d) Carbon reservoir changes; (e) Cumulative emissions
and airborne fraction; (f) Sink efficiency evolution; (g) CO$_2$ growth rate; (h) Implied
temperature change; (i) Remaining carbon budget for climate targets.}
\label{fig:carbon}
\end{figure}

\section{Results}

\subsection{Model Parameters}

\begin{pycode}
print(r"\begin{table}[htbp]")
print(r"\centering")
print(r"\caption{Carbon Cycle Model Parameters}")
print(r"\begin{tabular}{lcc}")
print(r"\toprule")
print(r"Parameter & Value & Units \\")
print(r"\midrule")
print(f"Atmosphere-ocean exchange & {k_AO} & yr$^{{-1}}$ \\\\")
print(f"Atmosphere-biosphere exchange & {k_AB} & yr$^{{-1}}$ \\\\")
print(f"Surface-deep ocean exchange & {k_OD} & yr$^{{-1}}$ \\\\")
print(f"Biosphere respiration & {k_BR} & yr$^{{-1}}$ \\\\")
print(f"Pre-industrial CO$_2$ & {C_A_eq/ppm_conversion:.0f} & ppm \\\\")
print(r"\bottomrule")
print(r"\end{tabular}")
print(r"\label{tab:parameters}")
print(r"\end{table}")
\end{pycode}

\subsection{Scenario Projections}

\begin{pycode}
print(r"\begin{table}[htbp]")
print(r"\centering")
print(r"\caption{Projected CO$_2$ Concentrations in 2100}")
print(r"\begin{tabular}{lcc}")
print(r"\toprule")
print(r"Scenario & CO$_2$ (ppm) & $\Delta T$ (K) \\")
print(r"\midrule")
print(f"RCP 2.6 & {CO2_2100_rcp26:.0f} & {ECS * np.log(CO2_2100_rcp26/280)/np.log(2):.1f} \\\\")
print(f"RCP 4.5 & {CO2_rcp45[-1]:.0f} & {ECS * np.log(CO2_rcp45[-1]/280)/np.log(2):.1f} \\\\")
print(f"RCP 8.5 & {CO2_2100_rcp85:.0f} & {ECS * np.log(CO2_2100_rcp85/280)/np.log(2):.1f} \\\\")
print(r"\bottomrule")
print(r"\end{tabular}")
print(r"\label{tab:projections}")
print(r"\end{table}")
\end{pycode}

\section{Discussion}

\begin{example}[Airborne Fraction]
The airborne fraction of $\py{f"{airborne_fraction:.2f}"}$ means that about \py{f"{(1-airborne_fraction)*100:.0f}"}\%
of emitted carbon is absorbed by natural sinks. The ocean absorbs $\sim$25\% and the
land biosphere $\sim$30\%.
\end{example}

\begin{remark}[Climate-Carbon Feedbacks]
This simple model neglects important feedbacks:
\begin{itemize}
\item \textbf{Ocean warming}: Reduces CO$_2$ solubility
\item \textbf{Permafrost thaw}: Releases stored carbon
\item \textbf{Forest dieback}: Amazon could become a source
\item \textbf{Ocean acidification}: Reduces carbonate buffering
\end{itemize}
These feedbacks would increase the airborne fraction.
\end{remark}

\section{Conclusions}

This carbon cycle analysis demonstrates:
\begin{enumerate}
\item Current CO$_2$ concentration is $\sim$\py{f"{CO2_2020:.0f}"} ppm (2020)
\item Airborne fraction is $\py{f"{airborne_fraction:.2f}"}$
\item RCP 8.5 leads to $\sim$\py{f"{CO2_2100_rcp85:.0f}"} ppm by 2100
\item Only RCP 2.6 scenario keeps warming below 2 K
\item Carbon budget for 1.5 K target is rapidly depleting
\end{enumerate}

\section*{Further Reading}

\begin{itemize}
\item IPCC. \textit{Climate Change 2021: The Physical Science Basis}. Cambridge, 2021.
\item Archer, D. \textit{The Global Carbon Cycle}. Princeton University Press, 2010.
\item Sarmiento, J.L. \& Gruber, N. \textit{Ocean Biogeochemical Dynamics}. Princeton, 2006.
\end{itemize}

\end{document}
