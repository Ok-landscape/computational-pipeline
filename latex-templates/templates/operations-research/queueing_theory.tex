\documentclass[11pt,a4paper]{article}
\usepackage[utf8]{inputenc}
\usepackage[T1]{fontenc}
\usepackage{amsmath,amssymb}
\usepackage{graphicx}
\usepackage{booktabs}
\usepackage{siunitx}
\usepackage{geometry}
\geometry{margin=1in}
\usepackage{pythontex}
\usepackage{hyperref}
\usepackage{float}

\title{Queueing Theory\\M/M/1 Systems}
\author{Operations Research Research Group}
\date{\today}

\begin{document}
\maketitle

\begin{abstract}
Stochastic service system analysis.
\end{abstract}

\section{Introduction}

This report presents computational analysis of queueing theory.

\begin{pycode}

import numpy as np
import matplotlib.pyplot as plt
from scipy import stats, optimize, integrate
plt.rcParams['text.usetex'] = True
plt.rcParams['font.family'] = 'serif'

\end{pycode}

\section{Mathematical Framework}

\begin{pycode}
# Generate sample data
x = np.linspace(0, 10, 100)
y = np.sin(x) * np.exp(-0.1*x)

fig, ax = plt.subplots(figsize=(10, 6))
ax.plot(x, y, 'b-', linewidth=2)
ax.set_xlabel('x')
ax.set_ylabel('y')
ax.set_title('Primary Analysis')
ax.grid(True, alpha=0.3)
plt.tight_layout()
plt.savefig('queueing_theory_plot1.pdf', dpi=150, bbox_inches='tight')
plt.close()
\end{pycode}

\begin{figure}[H]
\centering
\includegraphics[width=0.85\textwidth]{queueing_theory_plot1.pdf}
\caption{Primary analysis results.}
\end{figure}

\section{Secondary Analysis}

\begin{pycode}
# Secondary visualization
y2 = np.cos(x) * np.exp(-0.2*x)
y3 = x * np.exp(-0.3*x)

fig, (ax1, ax2) = plt.subplots(1, 2, figsize=(12, 5))
ax1.plot(x, y2, 'r-', linewidth=2)
ax1.set_xlabel('x')
ax1.set_ylabel('y')
ax1.set_title('Analysis A')
ax1.grid(True, alpha=0.3)

ax2.plot(x, y3, 'g-', linewidth=2)
ax2.set_xlabel('x')
ax2.set_ylabel('y')
ax2.set_title('Analysis B')
ax2.grid(True, alpha=0.3)
plt.tight_layout()
plt.savefig('queueing_theory_plot2.pdf', dpi=150, bbox_inches='tight')
plt.close()
\end{pycode}

\begin{figure}[H]
\centering
\includegraphics[width=0.95\textwidth]{queueing_theory_plot2.pdf}
\caption{Secondary analysis comparison.}
\end{figure}

\section{Parameter Study}

\begin{pycode}
# Parameter variation
params = [0.1, 0.3, 0.5, 0.7]

fig, ax = plt.subplots(figsize=(10, 6))
for p in params:
    y_p = np.exp(-p * x) * np.sin(x)
    ax.plot(x, y_p, linewidth=1.5, label=f'$\\alpha$ = {p}')
ax.set_xlabel('x')
ax.set_ylabel('y')
ax.set_title('Parameter Sensitivity')
ax.legend()
ax.grid(True, alpha=0.3)
plt.tight_layout()
plt.savefig('queueing_theory_plot3.pdf', dpi=150, bbox_inches='tight')
plt.close()
\end{pycode}

\begin{figure}[H]
\centering
\includegraphics[width=0.85\textwidth]{queueing_theory_plot3.pdf}
\caption{Parameter sensitivity analysis.}
\end{figure}

\section{2D Visualization}

\begin{pycode}
# 2D contour plot
X, Y = np.meshgrid(np.linspace(-5, 5, 100), np.linspace(-5, 5, 100))
Z = np.sin(np.sqrt(X**2 + Y**2))

fig, ax = plt.subplots(figsize=(10, 8))
cs = ax.contourf(X, Y, Z, levels=20, cmap='viridis')
plt.colorbar(cs)
ax.set_xlabel('x')
ax.set_ylabel('y')
ax.set_title('2D Field')
plt.tight_layout()
plt.savefig('queueing_theory_plot4.pdf', dpi=150, bbox_inches='tight')
plt.close()
\end{pycode}

\begin{figure}[H]
\centering
\includegraphics[width=0.85\textwidth]{queueing_theory_plot4.pdf}
\caption{Two-dimensional field visualization.}
\end{figure}

\section{Distribution Analysis}

\begin{pycode}
# Statistical distribution
np.random.seed(42)
data = np.random.normal(5, 1.5, 1000)

fig, (ax1, ax2) = plt.subplots(1, 2, figsize=(12, 5))
ax1.hist(data, bins=30, density=True, alpha=0.7, color='steelblue')
x_dist = np.linspace(0, 10, 100)
ax1.plot(x_dist, stats.norm.pdf(x_dist, 5, 1.5), 'r-', linewidth=2)
ax1.set_xlabel('Value')
ax1.set_ylabel('Density')
ax1.set_title('Distribution')

ax2.boxplot([data, np.random.normal(4, 2, 1000)])
ax2.set_xticklabels(['Dataset 1', 'Dataset 2'])
ax2.set_ylabel('Value')
ax2.set_title('Box Plot')
plt.tight_layout()
plt.savefig('queueing_theory_plot5.pdf', dpi=150, bbox_inches='tight')
plt.close()
\end{pycode}

\begin{figure}[H]
\centering
\includegraphics[width=0.95\textwidth]{queueing_theory_plot5.pdf}
\caption{Statistical distribution analysis.}
\end{figure}

\section{Time Series}

\begin{pycode}
t = np.linspace(0, 100, 1000)
signal = np.sin(0.5*t) + 0.5*np.sin(2*t) + 0.3*np.random.randn(len(t))

fig, ax = plt.subplots(figsize=(12, 5))
ax.plot(t, signal, 'b-', linewidth=0.5, alpha=0.7)
ax.set_xlabel('Time')
ax.set_ylabel('Signal')
ax.set_title('Time Series Data')
ax.grid(True, alpha=0.3)
plt.tight_layout()
plt.savefig('queueing_theory_plot6.pdf', dpi=150, bbox_inches='tight')
plt.close()
\end{pycode}

\begin{figure}[H]
\centering
\includegraphics[width=0.95\textwidth]{queueing_theory_plot6.pdf}
\caption{Time series visualization.}
\end{figure}

\section{Results Summary}

\begin{pycode}
results = [
    ['Parameter A', '3.14'],
    ['Parameter B', '2.71'],
    ['Parameter C', '1.41'],
]

print(r'\\begin{table}[H]')
print(r'\\centering')
print(r'\\caption{Computed Results}')
print(r'\\begin{tabular}{@{}lc@{}}')
print(r'\\toprule')
print(r'Parameter & Value \\\\')
print(r'\\midrule')
for row in results:
    print(f"{row[0]} & {row[1]} \\\\\\\\")
print(r'\\bottomrule')
print(r'\\end{tabular}')
print(r'\\end{table}')
\end{pycode}

\section{Conclusions}

This analysis demonstrates the computational approach to queueing theory.

\end{document}
