\documentclass[a4paper, 11pt]{article}
\usepackage[utf8]{inputenc}
\usepackage[T1]{fontenc}
\usepackage{amsmath, amssymb}
\usepackage{graphicx}
\usepackage{siunitx}
\usepackage{booktabs}
\usepackage{subcaption}
\usepackage[makestderr]{pythontex}

% Theorem environments
\newtheorem{definition}{Definition}
\newtheorem{theorem}{Theorem}
\newtheorem{example}{Example}
\newtheorem{remark}{Remark}

\title{Stellar Evolution: From Main Sequence to Stellar Remnants\\
\large A Comprehensive Analysis of the HR Diagram and Nuclear Burning Stages}
\author{Stellar Astrophysics Division\\Computational Science Templates}
\date{\today}

\begin{document}
\maketitle

\begin{abstract}
This comprehensive analysis explores stellar structure and evolution through the Hertzsprung-Russell diagram. We derive the fundamental stellar relations including the mass-luminosity relation, main sequence lifetime, and Stefan-Boltzmann law. The analysis covers all major evolutionary phases from pre-main sequence contraction through hydrogen and helium burning to white dwarf, neutron star, and black hole endpoints. We generate synthetic stellar populations to visualize the main sequence, red giant branch, horizontal branch, and white dwarf cooling sequence, and explore the physics governing each evolutionary stage.
\end{abstract}

\section{Introduction}

The Hertzsprung-Russell (HR) diagram, independently developed by Ejnar Hertzsprung and Henry Norris Russell in the early 1900s, provides a powerful tool for understanding stellar populations and evolution. By plotting stellar luminosity against effective temperature (or equivalently, color or spectral type), the HR diagram reveals the fundamental relationships governing stellar structure.

\begin{definition}[Hertzsprung-Russell Diagram]
A scatter plot of stellar luminosity $L$ versus effective temperature $T_{\text{eff}}$ (with temperature decreasing to the right), showing distinct regions occupied by different stellar types and evolutionary stages.
\end{definition}

\section{Theoretical Framework}

\subsection{Stellar Luminosity}

The luminosity of a star is determined by its radius and effective temperature:

\begin{theorem}[Stefan-Boltzmann Law]
\begin{equation}
L = 4\pi R^2 \sigma T_{\text{eff}}^4
\end{equation}
where $\sigma = 5.67 \times 10^{-8}$ W m$^{-2}$ K$^{-4}$ is the Stefan-Boltzmann constant.
\end{theorem}

\subsection{Mass-Luminosity Relation}

For main sequence stars, luminosity scales strongly with mass:

\begin{theorem}[Mass-Luminosity Relation]
\begin{equation}
\frac{L}{L_\odot} = \left(\frac{M}{M_\odot}\right)^\alpha
\end{equation}
where $\alpha \approx 3.5$ for stars with $0.5 < M/M_\odot < 20$, with variations at the extremes.
\end{theorem}

\begin{remark}[Mass-Luminosity Variations]
\begin{itemize}
    \item Low mass ($M < 0.5 M_\odot$): $\alpha \approx 2.3$
    \item Intermediate mass: $\alpha \approx 4$
    \item High mass ($M > 20 M_\odot$): $\alpha \approx 1$ (Eddington limit)
\end{itemize}
\end{remark}

\subsection{Main Sequence Lifetime}

Nuclear burning timescale depends on fuel supply and consumption rate:

\begin{equation}
\tau_{\text{MS}} \approx \frac{M}{L} \propto M^{1-\alpha} \approx \frac{10^{10}}{(M/M_\odot)^{2.5}} \text{ years}
\end{equation}

\subsection{Stellar Spectral Classification}

\begin{definition}[Spectral Types]
The Harvard classification scheme orders stars by temperature:
\begin{center}
O - B - A - F - G - K - M
\end{center}
(``Oh Be A Fine Girl/Guy Kiss Me'')

Temperature ranges: O ($>$30,000 K), B (10,000-30,000 K), A (7,500-10,000 K), F (6,000-7,500 K), G (5,200-6,000 K), K (3,700-5,200 K), M ($<$3,700 K)
\end{definition}

\section{Computational Analysis}

\begin{pycode}
import numpy as np
import matplotlib.pyplot as plt
from matplotlib.colors import LogNorm
plt.rc('text', usetex=True)
plt.rc('font', family='serif')

np.random.seed(42)

# Solar values
L_sun = 3.828e26  # W
T_sun = 5778  # K
R_sun = 6.96e8  # m
M_sun = 1.989e30  # kg

# Main sequence population
n_ms = 800
# IMF-weighted mass distribution (Salpeter)
mass_exponent = -2.35
mass_ms = 10**np.random.uniform(-1, 1.7, n_ms)  # 0.1 - 50 M_sun

# Mass-luminosity relation with mass-dependent exponent
def mass_to_luminosity(M):
    """Mass-luminosity relation with mass-dependent exponent"""
    L = np.zeros_like(M)
    low = M < 0.43
    mid_low = (M >= 0.43) & (M < 2)
    mid_high = (M >= 2) & (M < 20)
    high = M >= 20

    L[low] = 0.23 * M[low]**2.3
    L[mid_low] = M[mid_low]**4
    L[mid_high] = 1.5 * M[mid_high]**3.5
    L[high] = 3200 * M[high]
    return L

L_ms = mass_to_luminosity(mass_ms)

# Mass-temperature relation
def mass_to_temperature(M):
    """Main sequence temperature from mass"""
    return T_sun * M**0.5

T_ms = mass_to_temperature(mass_ms)

# Add scatter
L_ms *= 10**np.random.normal(0, 0.08, n_ms)
T_ms *= 10**np.random.normal(0, 0.015, n_ms)

# Calculate radii from Stefan-Boltzmann
R_ms = np.sqrt(L_ms * L_sun / (4 * np.pi * 5.67e-8 * T_ms**4)) / R_sun

# Red Giant Branch
n_rgb = 200
T_rgb = np.random.uniform(3500, 5200, n_rgb)
L_rgb = 10**np.random.uniform(1.5, 3.5, n_rgb)

# Horizontal Branch (helium burning)
n_hb = 100
T_hb = np.random.uniform(4500, 20000, n_hb)
L_hb = 10**np.random.uniform(1.5, 2.5, n_hb)

# Asymptotic Giant Branch
n_agb = 80
T_agb = np.random.uniform(2800, 4000, n_agb)
L_agb = 10**np.random.uniform(3.0, 4.5, n_agb)

# White Dwarfs (cooling sequence)
n_wd = 150
T_wd = 10**np.random.uniform(3.5, 5.0, n_wd)  # 3000 - 100000 K
# WD cooling: L ~ T^4 * R^2, with R ~ 0.01 R_sun
R_wd = 0.01  # solar radii
L_wd = (R_wd)**2 * (T_wd/T_sun)**4 * 10**np.random.normal(0, 0.2, n_wd)

# Supergiants
n_sg = 50
T_sg = np.random.uniform(3000, 30000, n_sg)
L_sg = 10**np.random.uniform(4, 6, n_sg)

# Pre-main sequence (Hayashi track)
n_pms = 100
T_pms = np.random.uniform(3000, 5000, n_pms)
L_pms = 10**np.random.uniform(0, 2, n_pms)

# Spectral classification
def spectral_class(T):
    if T > 30000: return 'O'
    elif T > 10000: return 'B'
    elif T > 7500: return 'A'
    elif T > 6000: return 'F'
    elif T > 5200: return 'G'
    elif T > 3700: return 'K'
    else: return 'M'

# Luminosity class
def luminosity_class(L, T):
    if L > 1e4:
        return 'I (Supergiant)'
    elif L > 100:
        return 'III (Giant)'
    elif L > 10:
        return 'IV (Subgiant)'
    else:
        return 'V (Main Seq.)'

# Create comprehensive figure
fig = plt.figure(figsize=(14, 16))

# Plot 1: Full HR diagram
ax1 = fig.add_subplot(3, 3, 1)
ax1.scatter(T_ms, L_ms, c='blue', s=3, alpha=0.4, label='Main Sequence')
ax1.scatter(T_rgb, L_rgb, c='red', s=10, alpha=0.5, label='RGB')
ax1.scatter(T_hb, L_hb, c='orange', s=8, alpha=0.5, label='HB')
ax1.scatter(T_agb, L_agb, c='darkred', s=15, alpha=0.5, label='AGB')
ax1.scatter(T_wd, L_wd, c='white', edgecolors='gray', s=8, alpha=0.6, label='WD')
ax1.scatter(T_sg, L_sg, c='yellow', edgecolors='black', s=25, alpha=0.7, label='Supergiant')

ax1.set_xscale('log')
ax1.set_yscale('log')
ax1.set_xlim([50000, 2000])
ax1.set_ylim([1e-5, 1e7])
ax1.set_xlabel('Effective Temperature (K)')
ax1.set_ylabel('Luminosity ($L_\\odot$)')
ax1.set_title('Hertzsprung-Russell Diagram')
ax1.legend(fontsize=6, loc='lower left')
ax1.grid(True, alpha=0.3)

# Plot 2: Main sequence only with constant radius lines
ax2 = fig.add_subplot(3, 3, 2)
ax2.scatter(T_ms, L_ms, c=np.log10(mass_ms), s=10, alpha=0.6, cmap='viridis')

# Constant radius lines
T_line = np.logspace(3.5, 5, 100)
for R_val in [0.1, 1, 10, 100]:
    L_R = (R_val)**2 * (T_line/T_sun)**4
    ax2.plot(T_line, L_R, 'k--', alpha=0.4, linewidth=1)
    ax2.text(T_line[-1]*0.8, L_R[-1]*1.5, f'{R_val}$R_\\odot$', fontsize=7)

ax2.set_xscale('log')
ax2.set_yscale('log')
ax2.set_xlim([50000, 2500])
ax2.set_ylim([1e-4, 1e6])
ax2.set_xlabel('Temperature (K)')
ax2.set_ylabel('Luminosity ($L_\\odot$)')
ax2.set_title('Main Sequence (colored by mass)')
ax2.grid(True, alpha=0.3)

# Plot 3: Mass-luminosity relation
ax3 = fig.add_subplot(3, 3, 3)
mass_plot = np.logspace(-1, 2, 100)
L_theory = mass_to_luminosity(mass_plot)
ax3.plot(mass_plot, L_theory, 'r-', linewidth=2, label='Theory')
ax3.scatter(mass_ms, L_ms, c='blue', s=5, alpha=0.3, label='Synthetic')

ax3.set_xscale('log')
ax3.set_yscale('log')
ax3.set_xlabel('Mass ($M_\\odot$)')
ax3.set_ylabel('Luminosity ($L_\\odot$)')
ax3.set_title('Mass-Luminosity Relation')
ax3.legend(fontsize=8)
ax3.grid(True, alpha=0.3)

# Plot 4: Main sequence lifetime
ax4 = fig.add_subplot(3, 3, 4)
tau_ms = 10e9 / mass_ms**2.5  # years
ax4.scatter(mass_ms, tau_ms/1e9, c='blue', s=10, alpha=0.5)

# Reference points
ref_masses = [0.5, 1, 2, 5, 10, 25]
ref_tau = [10e9/m**2.5/1e9 for m in ref_masses]
for m, t in zip(ref_masses, ref_tau):
    ax4.plot(m, t, 'ro', markersize=8)

ax4.set_xscale('log')
ax4.set_yscale('log')
ax4.set_xlabel('Mass ($M_\\odot$)')
ax4.set_ylabel('MS Lifetime (Gyr)')
ax4.set_title('Main Sequence Lifetime')
ax4.grid(True, alpha=0.3)

# Plot 5: Spectral type distribution
ax5 = fig.add_subplot(3, 3, 5)
all_T = np.concatenate([T_ms, T_rgb, T_hb, T_agb, T_wd, T_sg])
spectral_types = ['O', 'B', 'A', 'F', 'G', 'K', 'M']
spec_counts = [sum(1 for T in all_T if spectral_class(T) == s) for s in spectral_types]
colors = ['blue', 'cyan', 'white', 'lightyellow', 'yellow', 'orange', 'red']

bars = ax5.bar(spectral_types, spec_counts, color=colors, edgecolor='black', alpha=0.8)
ax5.set_xlabel('Spectral Class')
ax5.set_ylabel('Count')
ax5.set_title('Spectral Type Distribution')
ax5.grid(True, alpha=0.3, axis='y')

# Plot 6: Evolutionary track (1 solar mass)
ax6 = fig.add_subplot(3, 3, 6)
# Simplified evolutionary track for 1 M_sun
track_phases = {
    'ZAMS': (5778, 1, 'o'),
    'TO': (5900, 1.5, 's'),
    'RGB': (4500, 100, '^'),
    'HB': (5000, 50, 'D'),
    'AGB': (3500, 1000, 'p'),
    'PN': (100000, 100, '*'),
    'WD': (10000, 0.01, 'v')
}

for phase, (T, L, marker) in track_phases.items():
    ax6.scatter([T], [L], marker=marker, s=100, label=phase, zorder=10)

# Connect with line
T_track = [5778, 5900, 4500, 5000, 3500, 100000, 10000]
L_track = [1, 1.5, 100, 50, 1000, 100, 0.01]
ax6.plot(T_track, L_track, 'k--', alpha=0.5, linewidth=1)

ax6.set_xscale('log')
ax6.set_yscale('log')
ax6.set_xlim([200000, 2500])
ax6.set_ylim([1e-3, 1e4])
ax6.set_xlabel('Temperature (K)')
ax6.set_ylabel('Luminosity ($L_\\odot$)')
ax6.set_title('1 $M_\\odot$ Evolutionary Track')
ax6.legend(fontsize=7, loc='lower left', ncol=2)
ax6.grid(True, alpha=0.3)

# Plot 7: Stellar endpoints by mass
ax7 = fig.add_subplot(3, 3, 7)
mass_range = np.linspace(0.1, 50, 100)
remnant_mass = np.zeros_like(mass_range)

for i, m in enumerate(mass_range):
    if m < 8:  # White dwarf
        remnant_mass[i] = 0.5 + 0.1 * m
    elif m < 25:  # Neutron star
        remnant_mass[i] = 1.4
    else:  # Black hole
        remnant_mass[i] = 0.1 * m

ax7.plot(mass_range, remnant_mass, 'b-', linewidth=2)
ax7.axvline(x=8, color='r', linestyle='--', alpha=0.7, label='WD/NS boundary')
ax7.axvline(x=25, color='purple', linestyle='--', alpha=0.7, label='NS/BH boundary')
ax7.set_xlabel('Initial Mass ($M_\\odot$)')
ax7.set_ylabel('Remnant Mass ($M_\\odot$)')
ax7.set_title('Stellar Remnant Masses')
ax7.legend(fontsize=8)
ax7.grid(True, alpha=0.3)

# Plot 8: Nuclear burning stages
ax8 = fig.add_subplot(3, 3, 8)
burning_stages = ['H', 'He', 'C', 'Ne', 'O', 'Si']
temps = [1.5e7, 1e8, 5e8, 1.2e9, 1.5e9, 2.7e9]  # K
durations_25 = [7e6, 5e5, 600, 1, 0.5, 0.001]  # years for 25 M_sun

ax8.barh(burning_stages, np.log10(durations_25), color='steelblue', alpha=0.7)
ax8.set_xlabel('$\\log_{10}$(Duration/yr)')
ax8.set_ylabel('Burning Stage')
ax8.set_title('Nuclear Burning (25 $M_\\odot$)')
ax8.grid(True, alpha=0.3, axis='x')

# Plot 9: Color-magnitude diagram
ax9 = fig.add_subplot(3, 3, 9)
# B-V color index (simplified)
def T_to_BV(T):
    return 7090/T - 0.71

BV_ms = T_to_BV(T_ms)
BV_rgb = T_to_BV(T_rgb)
BV_wd = T_to_BV(T_wd)

# Absolute magnitude
Mv_ms = 4.83 - 2.5*np.log10(L_ms)
Mv_rgb = 4.83 - 2.5*np.log10(L_rgb)
Mv_wd = 4.83 - 2.5*np.log10(L_wd)

ax9.scatter(BV_ms, Mv_ms, c='blue', s=3, alpha=0.4, label='MS')
ax9.scatter(BV_rgb, Mv_rgb, c='red', s=8, alpha=0.5, label='RGB')
ax9.scatter(BV_wd, Mv_wd, c='gray', s=5, alpha=0.5, label='WD')

ax9.set_xlim([-0.5, 2.5])
ax9.set_ylim([15, -10])
ax9.set_xlabel('B-V Color Index')
ax9.set_ylabel('Absolute Magnitude $M_V$')
ax9.set_title('Color-Magnitude Diagram')
ax9.legend(fontsize=8)
ax9.grid(True, alpha=0.3)

plt.tight_layout()
plt.savefig('stellar_evolution_plot.pdf', bbox_inches='tight', dpi=150)
print(r'\begin{center}')
print(r'\includegraphics[width=\textwidth]{stellar_evolution_plot.pdf}')
print(r'\end{center}')
plt.close()

# Statistics
total_stars = n_ms + n_rgb + n_hb + n_agb + n_wd + n_sg
ms_fraction = n_ms / total_stars * 100
\end{pycode}

\section{Results and Analysis}

\subsection{Stellar Population Statistics}

\begin{pycode}
# Population statistics table
print(r'\begin{table}[h]')
print(r'\centering')
print(r'\caption{Synthetic Stellar Population Statistics}')
print(r'\begin{tabular}{lcc}')
print(r'\toprule')
print(r'Evolutionary Stage & Count & Fraction (\\%) \\')
print(r'\midrule')
stages = [('Main Sequence', n_ms), ('Red Giant Branch', n_rgb),
          ('Horizontal Branch', n_hb), ('AGB', n_agb),
          ('White Dwarf', n_wd), ('Supergiant', n_sg)]
for name, count in stages:
    frac = count / total_stars * 100
    print(f"{name} & {count} & {frac:.1f} \\\\")
print(r'\midrule')
print(f"Total & {total_stars} & 100 \\\\")
print(r'\bottomrule')
print(r'\end{tabular}')
print(r'\end{table}')
\end{pycode}

\subsection{Main Sequence Properties}

\begin{pycode}
# MS lifetime table
print(r'\begin{table}[h]')
print(r'\centering')
print(r'\caption{Main Sequence Lifetimes}')
print(r'\begin{tabular}{cccc}')
print(r'\toprule')
print(r'Mass ($M_\\odot$) & Luminosity ($L_\\odot$) & Temperature (K) & Lifetime (Gyr) \\')
print(r'\midrule')
ref_masses = [0.5, 1, 2, 5, 10, 25]
for m in ref_masses:
    L = mass_to_luminosity(np.array([m]))[0]
    T = mass_to_temperature(m)
    tau = 10 / m**2.5
    print(f"{m:.1f} & {L:.1f} & {T:.0f} & {tau:.2f} \\\\")
print(r'\bottomrule')
print(r'\end{tabular}')
print(r'\end{table}')
\end{pycode}

\begin{example}[Solar Evolution]
The Sun is a G2V main sequence star with the following properties:
\begin{itemize}
    \item Mass: $M = 1 M_\odot$
    \item Luminosity: $L = 1 L_\odot = $ \py{f"{L_sun:.3e}"} W
    \item Effective temperature: $T_{\text{eff}} = $ \py{f"{T_sun}"} K
    \item Main sequence lifetime: $\tau_{\text{MS}} \approx 10$ Gyr
    \item Current age: 4.6 Gyr (middle of MS phase)
\end{itemize}
\end{example}

\section{Evolutionary Phases}

\subsection{Pre-Main Sequence}

Stars form from collapsing molecular clouds and contract along Hayashi tracks (nearly vertical in HR diagram) before reaching the main sequence.

\subsection{Main Sequence}

Core hydrogen burning via pp-chain (low mass) or CNO cycle (high mass). Stars spend $\sim$90\% of their lives on the main sequence.

\subsection{Red Giant Branch}

After core hydrogen exhaustion:
\begin{enumerate}
    \item Hydrogen shell burning begins
    \item Core contracts, envelope expands
    \item Surface cools to $\sim$4000 K
    \item Luminosity increases to $\sim$100-1000 $L_\odot$
\end{enumerate}

\subsection{Helium Burning}

\begin{remark}[Helium Flash]
In low-mass stars ($M < 2 M_\odot$), helium ignition is degenerate and occurs explosively (helium flash), though the energy is absorbed internally.
\end{remark}

Stars on the Horizontal Branch burn helium in the core and hydrogen in a shell.

\subsection{Asymptotic Giant Branch}

Double-shell burning (H and He) produces thermal pulses and significant mass loss.

\section{Stellar Endpoints}

\subsection{White Dwarfs}

Final state for $M < 8 M_\odot$:
\begin{itemize}
    \item Supported by electron degeneracy pressure
    \item Mass limit: Chandrasekhar mass $M_{Ch} = 1.4 M_\odot$
    \item Cooling timescale: $\sim$10 Gyr
\end{itemize}

\subsection{Neutron Stars}

Core-collapse remnants for $8 < M/M_\odot < 25$:
\begin{itemize}
    \item Supported by neutron degeneracy pressure
    \item Typical mass: $1.4 M_\odot$
    \item Radius: $\sim$10 km
\end{itemize}

\subsection{Black Holes}

For $M > 25 M_\odot$, no known physics can halt collapse.

\section{Nuclear Burning Stages}

\begin{pycode}
# Nuclear burning table
print(r'\begin{table}[h]')
print(r'\centering')
print(r'\caption{Nuclear Burning in Massive Stars (25 $M_\\odot$)}')
print(r'\begin{tabular}{lccc}')
print(r'\toprule')
print(r'Stage & Temperature (K) & Duration & Products \\')
print(r'\midrule')
print(r'H $\\rightarrow$ He & $1.5 \\times 10^7$ & 7 Myr & He \\')
print(r'He $\\rightarrow$ C,O & $1 \\times 10^8$ & 500 kyr & C, O \\')
print(r'C $\\rightarrow$ Ne,Mg & $5 \\times 10^8$ & 600 yr & Ne, Mg \\')
print(r'Ne $\\rightarrow$ O,Mg & $1.2 \\times 10^9$ & 1 yr & O, Mg \\')
print(r'O $\\rightarrow$ Si,S & $1.5 \\times 10^9$ & 6 months & Si, S \\')
print(r'Si $\\rightarrow$ Fe & $2.7 \\times 10^9$ & 1 day & Fe \\')
print(r'\bottomrule')
print(r'\end{tabular}')
print(r'\end{table}')
\end{pycode}

\section{Limitations and Extensions}

\subsection{Model Limitations}
\begin{enumerate}
    \item \textbf{Single stars}: Binaries not included
    \item \textbf{Solar metallicity}: Population II not modeled
    \item \textbf{No rotation}: Rotating stars evolve differently
    \item \textbf{Simplified tracks}: No detailed stellar models
\end{enumerate}

\subsection{Possible Extensions}
\begin{itemize}
    \item Binary evolution and mass transfer
    \item Metallicity effects on evolution
    \item Stellar rotation and magnetic fields
    \item Detailed nucleosynthesis yields
\end{itemize}

\section{Conclusion}

This analysis demonstrates the power of the HR diagram for stellar astrophysics:
\begin{itemize}
    \item Main sequence stars follow the mass-luminosity relation
    \item MS lifetime: $\tau \propto M^{-2.5}$
    \item Red giants: cool but luminous ($L \sim 100 L_\odot$, $T \sim 4000$ K)
    \item White dwarfs: hot but dim ($L \sim 0.01 L_\odot$)
    \item Stellar endpoints depend on initial mass
\end{itemize}

\section*{Further Reading}
\begin{itemize}
    \item Kippenhahn, R., Weigert, A., \& Weiss, A. (2012). \textit{Stellar Structure and Evolution}. Springer.
    \item Prialnik, D. (2009). \textit{An Introduction to the Theory of Stellar Structure and Evolution}. Cambridge University Press.
    \item Salaris, M. \& Cassisi, S. (2005). \textit{Evolution of Stars and Stellar Populations}. Wiley.
\end{itemize}

\end{document}
