\documentclass[a4paper, 11pt]{article}
\usepackage[utf8]{inputenc}
\usepackage[T1]{fontenc}
\usepackage{amsmath, amssymb, amsthm}
\usepackage{graphicx}
\usepackage{booktabs}
\usepackage{siunitx}
\usepackage{subcaption}
\usepackage[makestderr]{pythontex}

\newtheorem{definition}{Definition}
\newtheorem{theorem}{Theorem}
\newtheorem{example}{Example}
\newtheorem{remark}{Remark}

\title{Radioactive Decay Chains: Bateman Equations and Activity Calculations}
\author{Nuclear Physics Computation Laboratory}
\date{\today}

\begin{document}
\maketitle

\begin{abstract}
This technical report presents comprehensive computational analysis of radioactive decay chains using the Bateman equations. We implement solutions for sequential decay series, compute activities as functions of time, and analyze equilibrium conditions including secular and transient equilibrium. Applications include medical isotope production, nuclear waste management, radiometric dating, and radiation dosimetry.
\end{abstract}

\section{Theoretical Framework}

\begin{definition}[Radioactive Decay Law]
The decay of radioactive nuclei follows first-order kinetics:
\begin{equation}
\frac{dN}{dt} = -\lambda N
\end{equation}
where $\lambda = \ln(2)/t_{1/2}$ is the decay constant and $t_{1/2}$ is the half-life.
\end{definition}

\begin{theorem}[Bateman Equations]
For a decay chain $N_1 \to N_2 \to N_3 \to \cdots$, the populations evolve as:
\begin{align}
\frac{dN_1}{dt} &= -\lambda_1 N_1 \\
\frac{dN_2}{dt} &= \lambda_1 N_1 - \lambda_2 N_2 \\
\frac{dN_n}{dt} &= \lambda_{n-1} N_{n-1} - \lambda_n N_n
\end{align}
with solution:
\begin{equation}
N_n(t) = N_1(0) \prod_{i=1}^{n-1} \lambda_i \sum_{i=1}^{n} \frac{e^{-\lambda_i t}}{\prod_{j \neq i}(\lambda_j - \lambda_i)}
\end{equation}
\end{theorem}

\subsection{Activity and Equilibrium}

\begin{definition}[Activity]
The activity $A = \lambda N$ is the number of decays per unit time, measured in Becquerels (Bq = decays/s) or Curies (1 Ci = $3.7 \times 10^{10}$ Bq).
\end{definition}

\begin{example}[Equilibrium Conditions]
For a two-member chain with $\lambda_1 \ll \lambda_2$:
\begin{itemize}
    \item \textbf{Secular equilibrium}: $A_2 \approx A_1$ (daughter activity equals parent)
    \item \textbf{Transient equilibrium}: $A_2/A_1 = \lambda_2/(\lambda_2 - \lambda_1)$
\end{itemize}
\end{example}

\section{Computational Analysis}

\begin{pycode}
import numpy as np
from scipy.integrate import odeint, solve_ivp
import matplotlib.pyplot as plt
plt.rc('text', usetex=True)
plt.rc('font', family='serif', size=10)

# Time unit conversions
sec_per_hour = 3600
sec_per_day = 86400
sec_per_year = 365.25 * sec_per_day

# Decay chain class
class DecayChain:
    def __init__(self, half_lives, names):
        self.names = names
        self.half_lives = np.array(half_lives)
        self.lambdas = np.log(2) / self.half_lives
        self.n = len(half_lives)

    def equations(self, N, t):
        dNdt = np.zeros(self.n)
        dNdt[0] = -self.lambdas[0] * N[0]
        for i in range(1, self.n):
            if self.lambdas[i] > 0:  # Not stable
                dNdt[i] = self.lambdas[i-1] * N[i-1] - self.lambdas[i] * N[i]
            else:
                dNdt[i] = self.lambdas[i-1] * N[i-1]
        return dNdt

    def solve(self, N0, t):
        return odeint(self.equations, N0, t)

    def activity(self, N):
        return N * self.lambdas

# Mo-99/Tc-99m generator (medical isotopes)
mo_tc_chain = DecayChain(
    half_lives=[66.0 * sec_per_hour, 6.0 * sec_per_hour, 1e20],
    names=['Mo-99', 'Tc-99m', 'Tc-99']
)

# Initial conditions (pure Mo-99)
N0_mo = [1.0, 0.0, 0.0]
t_mo = np.linspace(0, 200 * sec_per_hour, 1000)
sol_mo = mo_tc_chain.solve(N0_mo, t_mo)
A_mo = mo_tc_chain.activity(sol_mo)

# Find optimal milking time
tc99m_activity = A_mo[:, 1]
max_idx = np.argmax(tc99m_activity)
t_max_tc = t_mo[max_idx] / sec_per_hour
A_max_tc = tc99m_activity[max_idx]

# Equilibrium ratio
lambda_1, lambda_2 = mo_tc_chain.lambdas[0], mo_tc_chain.lambdas[1]
equilibrium_ratio = lambda_2 / (lambda_2 - lambda_1)

# Uranium-238 decay series (simplified)
# U-238 -> Th-234 -> Pa-234 -> U-234 -> ...
u238_chain = DecayChain(
    half_lives=[4.47e9 * sec_per_year, 24.1 * sec_per_day,
                1.17 / 60 * sec_per_hour, 2.45e5 * sec_per_year],
    names=['U-238', 'Th-234', 'Pa-234', 'U-234']
)

N0_u = [1.0, 0.0, 0.0, 0.0]
t_u = np.linspace(0, 365 * sec_per_day, 1000)  # 1 year
sol_u = u238_chain.solve(N0_u, t_u)
A_u = u238_chain.activity(sol_u)

# Branching decay example: Bi-212
# Bi-212 -> Po-212 (64%) or Tl-208 (36%)
bi212_hl = 60.6 * 60  # seconds
po212_hl = 0.3e-6     # microseconds
tl208_hl = 3.05 * 60  # seconds

branch_alpha = 0.64
branch_beta = 0.36

# Radon-222 and daughters for dosimetry
rn_chain = DecayChain(
    half_lives=[3.8 * sec_per_day, 3.05 * 60, 26.8 * 60,
                1e-4, 22.3 * sec_per_year],
    names=['Rn-222', 'Po-218', 'Pb-214', 'Bi-214', 'Pb-210']
)

# Carbon-14 dating
c14_half_life = 5730 * sec_per_year
lambda_c14 = np.log(2) / c14_half_life

def carbon_dating_age(ratio):
    """Calculate age from C-14/C-12 ratio relative to modern."""
    return -np.log(ratio) / lambda_c14 / sec_per_year

# Medical isotopes table
medical_isotopes = [
    ('Mo-99/Tc-99m', 66.0, 6.0, 'Imaging'),
    ('I-131', 8.02 * 24, np.inf, 'Thyroid'),
    ('F-18', 1.83, np.inf, 'PET'),
    ('Co-60', 5.27 * 365 * 24, np.inf, 'Therapy'),
    ('Cs-137', 30.17 * 365 * 24, np.inf, 'Calibration')
]

# Multi-stage decay with numerical solution
def bateman_analytical_2stage(t, lambda1, lambda2, N10):
    """Analytical Bateman solution for 2-stage decay."""
    N1 = N10 * np.exp(-lambda1 * t)
    if abs(lambda2 - lambda1) > 1e-10:
        N2 = N10 * lambda1 / (lambda2 - lambda1) * (np.exp(-lambda1 * t) - np.exp(-lambda2 * t))
    else:
        N2 = N10 * lambda1 * t * np.exp(-lambda1 * t)
    return N1, N2

# Create visualization
fig = plt.figure(figsize=(12, 10))
gs = fig.add_gridspec(3, 3, hspace=0.35, wspace=0.35)

# Plot 1: Mo-99/Tc-99m populations
ax1 = fig.add_subplot(gs[0, 0])
ax1.plot(t_mo / sec_per_hour, sol_mo[:, 0], 'b-', lw=2, label='Mo-99')
ax1.plot(t_mo / sec_per_hour, sol_mo[:, 1], 'r-', lw=2, label='Tc-99m')
ax1.plot(t_mo / sec_per_hour, sol_mo[:, 2], 'g--', lw=1.5, label='Tc-99')
ax1.set_xlabel('Time (hours)')
ax1.set_ylabel('Relative Population')
ax1.set_title('Mo-99/Tc-99m Generator')
ax1.legend(fontsize=7)
ax1.grid(True, alpha=0.3)

# Plot 2: Activities
ax2 = fig.add_subplot(gs[0, 1])
ax2.plot(t_mo / sec_per_hour, A_mo[:, 0] / A_mo[0, 0], 'b-', lw=2, label='$A_1$ (Mo-99)')
ax2.plot(t_mo / sec_per_hour, A_mo[:, 1] / A_mo[0, 0], 'r-', lw=2, label='$A_2$ (Tc-99m)')
ax2.axvline(x=t_max_tc, color='gray', ls='--', alpha=0.5)
ax2.set_xlabel('Time (hours)')
ax2.set_ylabel('Relative Activity')
ax2.set_title('Decay Activities')
ax2.legend(fontsize=7)
ax2.grid(True, alpha=0.3)

# Plot 3: Activity ratio (transient equilibrium)
ax3 = fig.add_subplot(gs[0, 2])
ratio = A_mo[:, 1] / (A_mo[:, 0] + 1e-20)
ax3.plot(t_mo / sec_per_hour, ratio, 'purple', lw=2)
ax3.axhline(y=equilibrium_ratio, color='r', ls='--',
            label=f'Equilibrium: {equilibrium_ratio:.3f}')
ax3.set_xlabel('Time (hours)')
ax3.set_ylabel('$A_2/A_1$')
ax3.set_title('Activity Ratio')
ax3.legend(fontsize=7)
ax3.grid(True, alpha=0.3)
ax3.set_ylim([0, equilibrium_ratio * 1.2])

# Plot 4: U-238 daughters approach secular equilibrium
ax4 = fig.add_subplot(gs[1, 0])
for i in range(1, 4):
    ax4.plot(t_u / sec_per_day, A_u[:, i] / (A_u[0, 0] + 1e-20),
             lw=1.5, label=u238_chain.names[i])
ax4.set_xlabel('Time (days)')
ax4.set_ylabel('Activity / Initial $A_0$')
ax4.set_title('U-238 Daughters (Secular Equil.)')
ax4.legend(fontsize=7)
ax4.grid(True, alpha=0.3)
ax4.set_yscale('log')

# Plot 5: Decay curve fitting
ax5 = fig.add_subplot(gs[1, 1])
t_decay = np.linspace(0, 5, 100)  # In half-lives
y_decay = np.exp(-np.log(2) * t_decay)

# Add noise for fitting demonstration
np.random.seed(42)
t_data = np.array([0, 0.5, 1, 1.5, 2, 3, 4])
y_data = np.exp(-np.log(2) * t_data) + np.random.normal(0, 0.02, len(t_data))

ax5.plot(t_decay, y_decay, 'b-', lw=2, label='Theory')
ax5.scatter(t_data, y_data, color='red', s=50, zorder=5, label='Data')
ax5.axhline(y=0.5, color='gray', ls='--', alpha=0.5)
ax5.axvline(x=1, color='gray', ls='--', alpha=0.5)
ax5.set_xlabel('Time / $t_{1/2}$')
ax5.set_ylabel('$N/N_0$')
ax5.set_title('Radioactive Decay Curve')
ax5.legend(fontsize=8)
ax5.grid(True, alpha=0.3)

# Plot 6: Carbon-14 dating
ax6 = fig.add_subplot(gs[1, 2])
ratios = np.linspace(0.01, 1.0, 100)
ages = np.array([carbon_dating_age(r) for r in ratios])

ax6.plot(ages / 1000, ratios, 'g-', lw=2)
ax6.axhline(y=0.5, color='gray', ls='--', alpha=0.5)
ax6.axvline(x=5.73, color='gray', ls='--', alpha=0.5)
ax6.set_xlabel('Age (kyr)')
ax6.set_ylabel('$^{14}$C / $^{14}$C$_0$')
ax6.set_title('Carbon-14 Dating')
ax6.grid(True, alpha=0.3)
ax6.set_xlim([0, 50])

# Plot 7: Half-lives comparison
ax7 = fig.add_subplot(gs[2, 0])
isotope_names = [m[0].split('/')[0] if '/' in m[0] else m[0] for m in medical_isotopes]
half_lives_hr = [m[1] for m in medical_isotopes]
colors = ['blue', 'red', 'green', 'orange', 'purple']

ax7.barh(isotope_names, np.log10(half_lives_hr), color=colors, alpha=0.7)
ax7.set_xlabel('$\\log_{10}(t_{1/2} / \\mathrm{hours})$')
ax7.set_title('Medical Isotope Half-Lives')
ax7.grid(True, alpha=0.3, axis='x')

# Plot 8: Generator elution cycles
ax8 = fig.add_subplot(gs[2, 1])
t_cycle = np.linspace(0, 100, 1000)
elution_times = [0, 24, 48, 72]

# Simulate multiple elutions
N_mo = np.ones_like(t_cycle)
N_tc = np.zeros_like(t_cycle)

for i, t in enumerate(t_cycle):
    # Decay between elutions
    hours = t
    N_mo[i] = np.exp(-lambda_1 * hours * sec_per_hour)

    # Tc-99m builds up
    N_tc[i] = (lambda_1 / (lambda_2 - lambda_1)) * N_mo[i] * \
              (1 - np.exp(-(lambda_2 - lambda_1) * (hours % 24) * sec_per_hour))

ax8.plot(t_cycle, N_tc * lambda_2 / lambda_1, 'r-', lw=2)
for et in elution_times[1:]:
    ax8.axvline(x=et, color='blue', ls='--', alpha=0.5)
ax8.set_xlabel('Time (hours)')
ax8.set_ylabel('Tc-99m Activity')
ax8.set_title('Generator Elution Cycles')
ax8.grid(True, alpha=0.3)

# Plot 9: Specific activity
ax9 = fig.add_subplot(gs[2, 2])
A_range = np.logspace(0, 3, 100)  # Mass numbers

# Specific activity = lambda * N_A / A
N_A = 6.022e23  # Avogadro's number
t_half_1yr = sec_per_year

specific_act = np.log(2) / t_half_1yr * N_A / A_range / 1e12  # TBq/g

ax9.loglog(A_range, specific_act, 'b-', lw=2)
ax9.set_xlabel('Mass Number $A$')
ax9.set_ylabel('Specific Activity (TBq/g)')
ax9.set_title('Specific Activity ($t_{1/2}$ = 1 yr)')
ax9.grid(True, alpha=0.3, which='both')

plt.savefig('radioactive_decay_plot.pdf', bbox_inches='tight', dpi=150)
print(r'\begin{center}')
print(r'\includegraphics[width=\textwidth]{radioactive_decay_plot.pdf}')
print(r'\end{center}')
plt.close()

# Summary calculations
N_A_value = 6.022e23
specific_mo99 = np.log(2) / (66 * sec_per_hour) * N_A_value / 99 / 1e12
\end{pycode}

\section{Results and Analysis}

\subsection{Mo-99/Tc-99m Generator}

\begin{pycode}
print(r'\begin{table}[htbp]')
print(r'\centering')
print(r'\caption{Mo-99/Tc-99m Generator Characteristics}')
print(r'\begin{tabular}{lcc}')
print(r'\toprule')
print(r'Parameter & Value & Units \\')
print(r'\midrule')
print(f'Mo-99 half-life & {66.0:.1f} & hours \\\\')
print(f'Tc-99m half-life & {6.0:.1f} & hours \\\\')
print(f'Optimal elution time & {t_max_tc:.1f} & hours \\\\')
print(f'Equilibrium ratio $A_2/A_1$ & {equilibrium_ratio:.3f} & -- \\\\')
print(f'Time to 90\\% equilibrium & {-np.log(0.1)/lambda_2/sec_per_hour:.1f} & hours \\\\')
print(r'\bottomrule')
print(r'\end{tabular}')
print(r'\end{table}')
\end{pycode}

\begin{remark}
The Mo-99/Tc-99m generator reaches transient equilibrium because $\lambda_2 > \lambda_1$. The daughter activity exceeds the parent activity by the factor $\lambda_2/(\lambda_2 - \lambda_1) = \py{f"{equilibrium_ratio:.2f}"}$.
\end{remark}

\subsection{Medical Isotopes}

\begin{pycode}
print(r'\begin{table}[htbp]')
print(r'\centering')
print(r'\caption{Medical Radioisotopes}')
print(r'\begin{tabular}{lccl}')
print(r'\toprule')
print(r'Isotope & $t_{1/2}$ & Production & Application \\')
print(r'\midrule')
print(r'Tc-99m & 6.0 hr & Generator & SPECT imaging \\')
print(r'F-18 & 110 min & Cyclotron & PET imaging \\')
print(r'I-131 & 8.0 days & Reactor & Thyroid therapy \\')
print(r'Co-60 & 5.3 yr & Reactor & External beam therapy \\')
print(r'Ir-192 & 74 days & Reactor & Brachytherapy \\')
print(r'\bottomrule')
print(r'\end{tabular}')
print(r'\end{table}')
\end{pycode}

\subsection{Equilibrium Types}

\begin{theorem}[Secular Equilibrium]
When $\lambda_1 \ll \lambda_2$ (parent half-life much longer than daughter):
\begin{equation}
A_2 \approx A_1 \quad \text{for } t \gg t_{1/2,2}
\end{equation}
Example: U-238 series where U-238 ($t_{1/2} = 4.5 \times 10^9$ yr) decays to short-lived daughters.
\end{theorem}

\begin{theorem}[Transient Equilibrium]
When $\lambda_1 < \lambda_2$ but comparable:
\begin{equation}
\frac{A_2}{A_1} = \frac{\lambda_2}{\lambda_2 - \lambda_1}
\end{equation}
Example: Mo-99/Tc-99m where ratio $= \py{f"{equilibrium_ratio:.2f}"}$.
\end{theorem}

\section{Applications}

\begin{example}[Radiometric Dating]
Carbon-14 dating uses the decay:
\begin{equation}
^{14}\text{C} \to ^{14}\text{N} + \beta^- + \bar{\nu}_e
\end{equation}
with $t_{1/2} = 5730$ years. Measurable ages range from $\sim$300 to 50,000 years.
\end{example}

\begin{example}[Nuclear Waste Management]
The activity of fission products decreases as:
\begin{itemize}
    \item Short-term (1-100 yr): dominated by Cs-137, Sr-90
    \item Intermediate (100-1000 yr): Sm-151, Tc-99
    \item Long-term ($>$1000 yr): actinides (Pu, Am, Cm)
\end{itemize}
\end{example}

\section{Discussion}

The Bateman equations provide a complete description of radioactive decay chains:

\begin{enumerate}
    \item \textbf{Generator design}: Optimal elution times maximize daughter activity while balancing parent depletion.
    \item \textbf{Diagnostic imaging}: Short-lived isotopes minimize patient dose while providing adequate counts.
    \item \textbf{Dose calculations}: Activity profiles determine internal dosimetry for therapy planning.
    \item \textbf{Environmental monitoring}: Equilibrium assumptions simplify radon progeny measurements.
\end{enumerate}

\section{Conclusions}

This computational analysis demonstrates:
\begin{itemize}
    \item Tc-99m maximum activity at $t = \py{f"{t_max_tc:.1f}"}$ hours
    \item Transient equilibrium ratio: \py{f"{equilibrium_ratio:.3f}"}
    \item Specific activity of Mo-99: \py{f"{specific_mo99:.0f}"} TBq/g
    \item C-14 dating range: 300 -- 50,000 years
\end{itemize}

The Bateman equations enable precise prediction of radionuclide behavior for medical, industrial, and research applications.

\section{Further Reading}
\begin{itemize}
    \item Magill, J., Galy, J., \textit{Radioactivity Radionuclides Radiation}, Springer, 2005
    \item Loveland, W.D., Morrissey, D.J., Seaborg, G.T., \textit{Modern Nuclear Chemistry}, 2nd Ed., Wiley, 2017
    \item Cherry, S.R., Sorenson, J.A., Phelps, M.E., \textit{Physics in Nuclear Medicine}, 4th Ed., Elsevier, 2012
\end{itemize}

\end{document}
