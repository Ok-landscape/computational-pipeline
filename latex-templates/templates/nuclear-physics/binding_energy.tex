\documentclass[a4paper, 11pt]{article}
\usepackage[utf8]{inputenc}
\usepackage[T1]{fontenc}
\usepackage{amsmath, amssymb, amsthm}
\usepackage{graphicx}
\usepackage{booktabs}
\usepackage{siunitx}
\usepackage{subcaption}
\usepackage[makestderr]{pythontex}

\newtheorem{definition}{Definition}
\newtheorem{theorem}{Theorem}
\newtheorem{example}{Example}
\newtheorem{remark}{Remark}

\title{Nuclear Binding Energy: Semi-Empirical Mass Formula and Nuclear Stability}
\author{Nuclear Physics Computation Laboratory}
\date{\today}

\begin{document}
\maketitle

\begin{abstract}
This technical report presents comprehensive computational analysis of nuclear binding energies using the semi-empirical mass formula (SEMF). We implement the Bethe-Weizs\"acker model with volume, surface, Coulomb, asymmetry, and pairing terms to predict nuclear masses and stability. The analysis includes the valley of stability, Q-values for nuclear reactions, and separation energies. Applications span nuclear structure, stellar nucleosynthesis, and nuclear energy production.
\end{abstract}

\section{Theoretical Framework}

\begin{definition}[Nuclear Binding Energy]
The binding energy $B(A,Z)$ is the energy required to disassemble a nucleus into free nucleons:
\begin{equation}
B(A,Z) = [Zm_p + Nm_n - M(A,Z)]c^2
\end{equation}
where $A = Z + N$ is the mass number, $Z$ is the proton number, and $N$ is the neutron number.
\end{definition}

\begin{theorem}[Semi-Empirical Mass Formula]
The nuclear binding energy can be approximated as:
\begin{equation}
B(A,Z) = a_V A - a_S A^{2/3} - a_C \frac{Z(Z-1)}{A^{1/3}} - a_A \frac{(A-2Z)^2}{A} + \delta(A,Z)
\end{equation}
where the terms represent volume, surface, Coulomb, asymmetry, and pairing contributions.
\end{theorem}

\subsection{Physical Interpretation of Terms}

\begin{itemize}
    \item \textbf{Volume}: $a_V A$ --- saturated nuclear force, proportional to nucleon number
    \item \textbf{Surface}: $-a_S A^{2/3}$ --- surface nucleons have fewer neighbors
    \item \textbf{Coulomb}: $-a_C Z(Z-1)/A^{1/3}$ --- electrostatic repulsion of protons
    \item \textbf{Asymmetry}: $-a_A (N-Z)^2/A$ --- Pauli exclusion principle effect
    \item \textbf{Pairing}: $\delta(A,Z)$ --- spin-pairing effects
\end{itemize}

\begin{example}[Pairing Term]
The pairing term depends on nucleon parity:
\begin{equation}
\delta(A,Z) = \begin{cases}
+a_P A^{-1/2} & \text{even-even (Z, N both even)} \\
0 & \text{odd A} \\
-a_P A^{-1/2} & \text{odd-odd (Z, N both odd)}
\end{cases}
\end{equation}
\end{example}

\section{Computational Analysis}

\begin{pycode}
import numpy as np
import matplotlib.pyplot as plt
from scipy.optimize import minimize_scalar
plt.rc('text', usetex=True)
plt.rc('font', family='serif', size=10)

# SEMF parameters (MeV) - standard values
a_V = 15.75   # Volume coefficient
a_S = 17.8    # Surface coefficient
a_C = 0.711   # Coulomb coefficient
a_A = 23.7    # Asymmetry coefficient
a_P = 11.18   # Pairing coefficient

# Physical constants
m_p = 938.272  # Proton mass (MeV/c^2)
m_n = 939.565  # Neutron mass (MeV/c^2)
m_e = 0.511    # Electron mass (MeV/c^2)

def pairing_term(A, Z):
    """Calculate pairing term."""
    N = A - Z
    if A == 0:
        return 0
    if Z % 2 == 0 and N % 2 == 0:
        return a_P / np.sqrt(A)
    elif Z % 2 == 1 and N % 2 == 1:
        return -a_P / np.sqrt(A)
    else:
        return 0

def binding_energy(A, Z):
    """Calculate total binding energy using SEMF (MeV)."""
    if A <= 0 or Z <= 0 or Z > A:
        return 0

    B = (a_V * A
         - a_S * A**(2/3)
         - a_C * Z * (Z - 1) / A**(1/3)
         - a_A * (A - 2*Z)**2 / A
         + pairing_term(A, Z))
    return max(0, B)

def binding_per_nucleon(A, Z):
    """Binding energy per nucleon (MeV)."""
    if A <= 0:
        return 0
    return binding_energy(A, Z) / A

def nuclear_mass(A, Z):
    """Nuclear mass in MeV/c^2."""
    return Z * m_p + (A - Z) * m_n - binding_energy(A, Z)

def atomic_mass(A, Z):
    """Atomic mass in MeV/c^2."""
    return nuclear_mass(A, Z) + Z * m_e

def most_stable_Z(A):
    """Find most stable Z for given A using SEMF."""
    if A <= 0:
        return 0
    # Analytical formula from SEMF
    Z_opt = A / (2 + a_C * A**(2/3) / (2 * a_A))
    return int(round(Z_opt))

def separation_energy_n(A, Z):
    """Neutron separation energy S_n."""
    return binding_energy(A, Z) - binding_energy(A-1, Z)

def separation_energy_p(A, Z):
    """Proton separation energy S_p."""
    return binding_energy(A, Z) - binding_energy(A-1, Z-1)

def separation_energy_alpha(A, Z):
    """Alpha separation energy S_alpha."""
    return binding_energy(A, Z) - binding_energy(A-4, Z-2) - binding_energy(4, 2)

# Q-value calculations
def Q_alpha(A, Z):
    """Q-value for alpha decay."""
    parent_mass = nuclear_mass(A, Z)
    daughter_mass = nuclear_mass(A-4, Z-2)
    alpha_mass = nuclear_mass(4, 2)
    return parent_mass - daughter_mass - alpha_mass

def Q_beta_minus(A, Z):
    """Q-value for beta-minus decay."""
    parent_mass = atomic_mass(A, Z)
    daughter_mass = atomic_mass(A, Z+1)
    return parent_mass - daughter_mass

def Q_beta_plus(A, Z):
    """Q-value for beta-plus decay."""
    parent_mass = atomic_mass(A, Z)
    daughter_mass = atomic_mass(A, Z-1)
    return parent_mass - daughter_mass - 2*m_e

# Calculate binding energies for chart of nuclides
A_range = np.arange(1, 261)
Z_stable = np.array([most_stable_Z(A) for A in A_range])
B_per_A = np.array([binding_per_nucleon(A, Z) for A, Z in zip(A_range, Z_stable)])

# Find maximum B/A
max_idx = np.argmax(B_per_A)
A_max = A_range[max_idx]
Z_max = Z_stable[max_idx]
B_max = B_per_A[max_idx]

# Specific nuclei for analysis
key_nuclei = [
    ('H-2', 2, 1),
    ('He-4', 4, 2),
    ('Li-7', 7, 3),
    ('C-12', 12, 6),
    ('O-16', 16, 8),
    ('Fe-56', 56, 26),
    ('Ni-62', 62, 28),
    ('U-235', 235, 92),
    ('U-238', 238, 92)
]

# SEMF term contributions
A_terms = np.linspace(10, 250, 200)
Z_terms = A_terms / 2.2

volume = a_V * np.ones_like(A_terms)
surface = -a_S / A_terms**(1/3)
coulomb = -a_C * Z_terms * (Z_terms - 1) / A_terms**(4/3)
asymmetry = -a_A * (A_terms - 2*Z_terms)**2 / A_terms**2

# Magic numbers analysis
magic_numbers = [2, 8, 20, 28, 50, 82, 126]

# Create visualization
fig = plt.figure(figsize=(12, 10))
gs = fig.add_gridspec(3, 3, hspace=0.35, wspace=0.35)

# Plot 1: B/A vs A
ax1 = fig.add_subplot(gs[0, 0])
ax1.plot(A_range, B_per_A, 'b-', lw=1.5)
ax1.axvline(x=A_max, color='r', ls='--', alpha=0.5)
ax1.annotate(f'Max: A={A_max}', xy=(A_max, B_max),
             xytext=(A_max+30, B_max-0.5),
             arrowprops=dict(arrowstyle='->', color='red'),
             fontsize=8)
ax1.set_xlabel('Mass Number $A$')
ax1.set_ylabel('$B/A$ (MeV)')
ax1.set_title('Binding Energy per Nucleon')
ax1.grid(True, alpha=0.3)

# Plot 2: SEMF term contributions
ax2 = fig.add_subplot(gs[0, 1])
ax2.plot(A_terms, volume, 'b-', lw=1.5, label='Volume')
ax2.plot(A_terms, surface, 'g-', lw=1.5, label='Surface')
ax2.plot(A_terms, coulomb, 'r-', lw=1.5, label='Coulomb')
ax2.plot(A_terms, asymmetry, 'm-', lw=1.5, label='Asymmetry')
ax2.axhline(y=0, color='gray', ls='--', alpha=0.5)
ax2.set_xlabel('Mass Number $A$')
ax2.set_ylabel('Contribution to $B/A$ (MeV)')
ax2.set_title('SEMF Term Contributions')
ax2.legend(fontsize=7)
ax2.grid(True, alpha=0.3)

# Plot 3: Valley of stability
ax3 = fig.add_subplot(gs[0, 2])
A_grid = np.arange(1, 201)
Z_grid = np.arange(1, 101)
B_grid = np.zeros((len(Z_grid), len(A_grid)))

for i, Z in enumerate(Z_grid):
    for j, A in enumerate(A_grid):
        if Z <= A:
            B_grid[i, j] = binding_per_nucleon(A, Z)

c = ax3.contourf(A_grid, Z_grid, B_grid, levels=20, cmap='viridis')
plt.colorbar(c, ax=ax3, label='$B/A$ (MeV)')

# Plot valley of stability
A_valley = np.arange(1, 201)
Z_valley = np.array([most_stable_Z(A) for A in A_valley])
ax3.plot(A_valley, Z_valley, 'r-', lw=2, label='Valley')
ax3.plot(A_valley, A_valley/2, 'w--', alpha=0.5, label='N=Z')
ax3.set_xlabel('Mass Number $A$')
ax3.set_ylabel('Proton Number $Z$')
ax3.set_title('Nuclear Stability Chart')
ax3.legend(fontsize=7, loc='lower right')

# Plot 4: Key nuclei B/A
ax4 = fig.add_subplot(gs[1, 0])
names = [n[0] for n in key_nuclei]
B_values = [binding_per_nucleon(n[1], n[2]) for n in key_nuclei]
colors = plt.cm.viridis(np.linspace(0.2, 0.8, len(key_nuclei)))
bars = ax4.bar(range(len(names)), B_values, color=colors)
ax4.set_xticks(range(len(names)))
ax4.set_xticklabels(names, rotation=45, fontsize=7)
ax4.set_ylabel('$B/A$ (MeV)')
ax4.set_title('Binding Energies of Key Nuclei')
ax4.grid(True, alpha=0.3, axis='y')

# Plot 5: Separation energies
ax5 = fig.add_subplot(gs[1, 1])
A_sep = np.arange(10, 121)
Z_sep = np.array([most_stable_Z(A) for A in A_sep])

S_n = np.array([separation_energy_n(A, Z) for A, Z in zip(A_sep, Z_sep)])
S_p = np.array([separation_energy_p(A, Z) for A, Z in zip(A_sep, Z_sep)])

ax5.plot(A_sep, S_n, 'b-', lw=1.5, label='$S_n$')
ax5.plot(A_sep, S_p, 'r--', lw=1.5, label='$S_p$')

# Mark magic numbers
for N_magic in [20, 28, 50, 82]:
    ax5.axvline(x=2*N_magic, color='gray', ls=':', alpha=0.5)

ax5.set_xlabel('Mass Number $A$')
ax5.set_ylabel('Separation Energy (MeV)')
ax5.set_title('Nucleon Separation Energies')
ax5.legend(fontsize=8)
ax5.grid(True, alpha=0.3)
ax5.set_ylim([0, 15])

# Plot 6: Q-values for alpha decay
ax6 = fig.add_subplot(gs[1, 2])
A_alpha = np.arange(150, 261)
Z_alpha = np.array([most_stable_Z(A) for A in A_alpha])

Q_a = np.array([Q_alpha(A, Z) for A, Z in zip(A_alpha, Z_alpha)])

ax6.plot(A_alpha, Q_a, 'g-', lw=1.5)
ax6.axhline(y=0, color='r', ls='--', alpha=0.5)
ax6.fill_between(A_alpha, Q_a, 0, where=(Q_a > 0), alpha=0.3, color='green',
                  label='$\\alpha$-unstable')
ax6.set_xlabel('Mass Number $A$')
ax6.set_ylabel('$Q_\\alpha$ (MeV)')
ax6.set_title('Alpha Decay Q-Values')
ax6.legend(fontsize=8)
ax6.grid(True, alpha=0.3)

# Plot 7: Pairing effect
ax7 = fig.add_subplot(gs[2, 0])
A_pair = np.arange(20, 101)
B_even_even = np.array([binding_per_nucleon(A, A//2) for A in A_pair if A % 2 == 0])
B_odd_odd = np.array([binding_per_nucleon(A, A//2) for A in A_pair if A % 2 == 0])
B_odd = np.array([binding_per_nucleon(A, (A-1)//2) for A in A_pair if A % 2 == 1])

A_even = A_pair[::2]
A_odd = A_pair[1::2]

ax7.plot(A_even, B_even_even, 'bo-', ms=4, label='Even-Even')
ax7.plot(A_odd, B_odd, 'r^-', ms=4, label='Odd A')
ax7.set_xlabel('Mass Number $A$')
ax7.set_ylabel('$B/A$ (MeV)')
ax7.set_title('Pairing Effect')
ax7.legend(fontsize=8)
ax7.grid(True, alpha=0.3)

# Plot 8: Mass excess
ax8 = fig.add_subplot(gs[2, 1])
A_me = np.arange(1, 101)
Z_me = np.array([most_stable_Z(A) for A in A_me])

# Mass excess = M - A*u (in MeV)
u_MeV = 931.494  # atomic mass unit in MeV
mass_excess = np.array([atomic_mass(A, Z) - A * u_MeV for A, Z in zip(A_me, Z_me)])

ax8.plot(A_me, mass_excess, 'purple', lw=1.5)
ax8.axhline(y=0, color='gray', ls='--', alpha=0.5)
ax8.set_xlabel('Mass Number $A$')
ax8.set_ylabel('Mass Excess (MeV)')
ax8.set_title('Nuclear Mass Excess')
ax8.grid(True, alpha=0.3)

# Plot 9: Fission vs fusion
ax9 = fig.add_subplot(gs[2, 2])

# Energy release per nucleon
E_fission = (B_per_A[233] - B_per_A[116]) * 0.85  # Approximate fission products
E_fusion_DT = binding_per_nucleon(4, 2) - (binding_per_nucleon(2, 1) + binding_per_nucleon(3, 1))/2

reactions = ['D-T Fusion', 'U-235 Fission', 'He-4 Formation']
energies = [17.6/4, 200/235, 28.3/4]  # MeV per nucleon

ax9.bar(reactions, energies, color=['red', 'blue', 'green'], alpha=0.7)
ax9.set_ylabel('Energy per Nucleon (MeV)')
ax9.set_title('Nuclear Energy Release')
ax9.grid(True, alpha=0.3, axis='y')

plt.savefig('binding_energy_plot.pdf', bbox_inches='tight', dpi=150)
print(r'\begin{center}')
print(r'\includegraphics[width=\textwidth]{binding_energy_plot.pdf}')
print(r'\end{center}')
plt.close()

# Summary calculations
B_Fe56 = binding_per_nucleon(56, 26)
B_Ni62 = binding_per_nucleon(62, 28)
B_He4 = binding_per_nucleon(4, 2)
Q_U235_alpha = Q_alpha(235, 92)
\end{pycode}

\section{Results and Analysis}

\subsection{Binding Energy per Nucleon}

\begin{pycode}
print(r'\begin{table}[htbp]')
print(r'\centering')
print(r'\caption{Binding Energies of Key Nuclei}')
print(r'\begin{tabular}{lccccc}')
print(r'\toprule')
print(r'Nucleus & A & Z & B (MeV) & B/A (MeV) & Type \\')
print(r'\midrule')

for name, A, Z in key_nuclei:
    B = binding_energy(A, Z)
    BpA = binding_per_nucleon(A, Z)
    N = A - Z
    if Z % 2 == 0 and N % 2 == 0:
        ptype = 'e-e'
    elif Z % 2 == 1 and N % 2 == 1:
        ptype = 'o-o'
    else:
        ptype = 'odd'
    print(f'{name} & {A} & {Z} & {B:.1f} & {BpA:.3f} & {ptype} \\\\')

print(r'\bottomrule')
print(r'\end{tabular}')
print(r'\end{table}')
\end{pycode}

\subsection{Maximum Stability}

The binding energy curve reaches its maximum near the iron-nickel region:
\begin{itemize}
    \item Maximum $B/A$ at $A = \py{A_max}$: \py{f"{B_max:.3f}"} MeV
    \item Fe-56: \py{f"{B_Fe56:.3f}"} MeV/nucleon
    \item Ni-62: \py{f"{B_Ni62:.3f}"} MeV/nucleon (highest known $B/A$)
    \item He-4: \py{f"{B_He4:.3f}"} MeV/nucleon (exceptionally stable)
\end{itemize}

\begin{remark}
While Fe-56 is often cited as the most stable nucleus, Ni-62 actually has the highest binding energy per nucleon. Fe-56 is the most abundant end product of stellar nucleosynthesis due to the peak in the Fe-56 production cross section.
\end{remark}

\subsection{SEMF Coefficients}

\begin{pycode}
print(r'\begin{table}[htbp]')
print(r'\centering')
print(r'\caption{Semi-Empirical Mass Formula Parameters}')
print(r'\begin{tabular}{lcc}')
print(r'\toprule')
print(r'Term & Coefficient & Physical Origin \\')
print(r'\midrule')
print(f'Volume & $a_V = {a_V}$ MeV & Strong force saturation \\\\')
print(f'Surface & $a_S = {a_S}$ MeV & Surface tension \\\\')
print(f'Coulomb & $a_C = {a_C}$ MeV & Electrostatic repulsion \\\\')
print(f'Asymmetry & $a_A = {a_A}$ MeV & Fermi gas model \\\\')
print(f'Pairing & $a_P = {a_P}$ MeV & Spin coupling \\\\')
print(r'\bottomrule')
print(r'\end{tabular}')
print(r'\end{table}')
\end{pycode}

\section{Nuclear Reactions}

\begin{example}[Nuclear Fusion]
The fusion of deuterium and tritium releases:
\begin{equation}
\text{D} + \text{T} \to \alpha + n + 17.6 \text{ MeV}
\end{equation}
This represents $\sim$3.5 MeV per nucleon, the highest energy density achievable.
\end{example}

\begin{example}[Nuclear Fission]
The fission of U-235 releases approximately 200 MeV per event:
\begin{equation}
^{235}\text{U} + n \to \text{fission products} + 2.5n + 200 \text{ MeV}
\end{equation}
This is $\sim$0.85 MeV per nucleon, less than fusion but easier to achieve.
\end{example}

\begin{theorem}[Energy Release Criterion]
Energy is released in nuclear reactions when the products have higher $B/A$ than reactants:
\begin{itemize}
    \item Fusion is energetically favorable for $A < 56$
    \item Fission is energetically favorable for $A > 100$
\end{itemize}
\end{theorem}

\section{Discussion}

The SEMF successfully reproduces nuclear binding energies with an RMS error of $\sim$2-3 MeV for nuclei away from closed shells. Key insights include:

\begin{enumerate}
    \item \textbf{Liquid drop model}: The first three terms describe the nucleus as a charged liquid drop with surface tension.
    \item \textbf{Fermi gas}: The asymmetry term arises from the Pauli exclusion principle in a degenerate Fermi gas.
    \item \textbf{Shell effects}: Magic numbers cause deviations from SEMF predictions, requiring shell corrections.
    \item \textbf{Stability valley}: The competition between Coulomb and asymmetry terms determines the valley of stability.
\end{enumerate}

\section{Conclusions}

This computational analysis demonstrates:
\begin{itemize}
    \item Maximum binding energy per nucleon: \py{f"{B_max:.3f}"} MeV at $A = \py{A_max}$
    \item Volume term dominates: $a_V = \py{a_V}$ MeV
    \item Alpha decay Q-value for U-235: \py{f"{Q_U235_alpha:.2f}"} MeV
    \item Energy release in fusion exceeds fission by factor of $\sim$4 per nucleon
\end{itemize}

The SEMF provides a quantitative foundation for understanding nuclear stability, decay modes, and energy production in nuclear reactions.

\section{Further Reading}
\begin{itemize}
    \item Krane, K.S., \textit{Introductory Nuclear Physics}, Wiley, 1987
    \item Heyde, K., \textit{Basic Ideas and Concepts in Nuclear Physics}, 3rd Edition, IOP Publishing, 2004
    \item Wong, S.S.M., \textit{Introductory Nuclear Physics}, 2nd Edition, Wiley-VCH, 2004
\end{itemize}

\end{document}
