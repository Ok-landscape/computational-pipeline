\documentclass[a4paper, 11pt]{article}
\usepackage[utf8]{inputenc}
\usepackage[T1]{fontenc}
\usepackage{amsmath, amssymb}
\usepackage{graphicx}
\usepackage{booktabs}
\usepackage{siunitx}
\usepackage[makestderr]{pythontex}

\title{Ocean Currents: Geostrophic Flow and Wind-Driven Circulation}
\author{Physical Oceanography Templates}
\date{\today}

\begin{document}
\maketitle

\section{Introduction}
Ocean currents are driven by wind stress, density differences, and Earth's rotation. This template explores geostrophic balance, Ekman spiral dynamics, Sverdrup transport, and western boundary current intensification.

\section{Mathematical Framework}

\subsection{Geostrophic Balance}
Large-scale ocean flow balances Coriolis force and pressure gradient:
\begin{equation}
f \mathbf{k} \times \mathbf{u}_g = -\frac{1}{\rho_0} \nabla p
\end{equation}
where $f = 2\Omega \sin\phi$ is the Coriolis parameter.

Component form:
\begin{align}
u_g &= -\frac{1}{\rho_0 f} \frac{\partial p}{\partial y} \\
v_g &= \frac{1}{\rho_0 f} \frac{\partial p}{\partial x}
\end{align}

\subsection{Ekman Transport}
Wind stress drives Ekman transport perpendicular to wind direction:
\begin{equation}
\mathbf{M}_E = \frac{\boldsymbol{\tau} \times \mathbf{k}}{\rho_0 f}
\end{equation}

The Ekman spiral describes velocity decay with depth:
\begin{align}
u(z) &= V_0 e^{z/D_E} \cos\left(\frac{\pi}{4} + \frac{z}{D_E}\right) \\
v(z) &= V_0 e^{z/D_E} \sin\left(\frac{\pi}{4} + \frac{z}{D_E}\right)
\end{align}
where $D_E = \sqrt{2A_v/f}$ is the Ekman depth.

\subsection{Sverdrup Balance}
Wind stress curl drives meridional transport:
\begin{equation}
\beta V = \frac{1}{\rho_0} \nabla \times \boldsymbol{\tau}
\end{equation}
where $\beta = \partial f/\partial y$ is the planetary vorticity gradient.

\subsection{Western Boundary Current}
Munk's model includes lateral friction:
\begin{equation}
\beta v = A_H \nabla^4 \psi + \frac{1}{\rho_0 H} \left(\frac{\partial \tau_y}{\partial x} - \frac{\partial \tau_x}{\partial y}\right)
\end{equation}

\section{Environment Setup}

\begin{pycode}
import numpy as np
import matplotlib.pyplot as plt
from scipy.integrate import odeint

plt.rc('text', usetex=True)
plt.rc('font', family='serif')
np.random.seed(42)

def save_plot(filename, caption=""):
    plt.savefig(filename, bbox_inches='tight', dpi=150)
    print(r'\begin{figure}[htbp]')
    print(r'\centering')
    print(r'\includegraphics[width=0.9\textwidth]{' + filename + '}')
    if caption:
        print(r'\caption{' + caption + '}')
    print(r'\end{figure}')
    plt.close()

# Physical constants
Omega = 7.292e-5  # Earth's rotation rate (rad/s)
rho_0 = 1025      # Reference density (kg/m^3)
g = 9.81          # Gravity (m/s^2)
\end{pycode}

\section{Geostrophic Flow Computation}

\begin{pycode}
# Create pressure field for geostrophic calculation
nx, ny = 50, 50
x = np.linspace(0, 1000e3, nx)  # 1000 km domain
y = np.linspace(0, 1000e3, ny)
X, Y = np.meshgrid(x, y)

# Sea surface height anomaly (mesoscale eddy)
eta = 0.3 * np.exp(-((X - 500e3)**2 + (Y - 500e3)**2) / (200e3)**2)

# Coriolis parameter (mid-latitude)
lat = 30  # degrees
f = 2 * Omega * np.sin(np.radians(lat))
beta = 2 * Omega * np.cos(np.radians(lat)) / 6.371e6

# Geostrophic velocities from SSH gradient
dx = x[1] - x[0]
dy = y[1] - y[0]

deta_dx = np.gradient(eta, dx, axis=1)
deta_dy = np.gradient(eta, dy, axis=0)

u_g = -g / f * deta_dy
v_g = g / f * deta_dx

speed = np.sqrt(u_g**2 + v_g**2)

# Visualization
fig, axes = plt.subplots(2, 2, figsize=(12, 10))

# Plot 1: Sea surface height
im1 = axes[0, 0].contourf(X/1e3, Y/1e3, eta*100, levels=20, cmap='RdBu_r')
axes[0, 0].set_xlabel('x (km)')
axes[0, 0].set_ylabel('y (km)')
axes[0, 0].set_title('Sea Surface Height Anomaly (cm)')
plt.colorbar(im1, ax=axes[0, 0])

# Plot 2: Geostrophic velocity vectors
skip = 3
axes[0, 1].quiver(X[::skip, ::skip]/1e3, Y[::skip, ::skip]/1e3,
                  u_g[::skip, ::skip], v_g[::skip, ::skip],
                  speed[::skip, ::skip], cmap='viridis', alpha=0.8)
axes[0, 1].set_xlabel('x (km)')
axes[0, 1].set_ylabel('y (km)')
axes[0, 1].set_title('Geostrophic Velocity Field')

# Plot 3: Speed magnitude
im3 = axes[1, 0].contourf(X/1e3, Y/1e3, speed*100, levels=20, cmap='plasma')
axes[1, 0].contour(X/1e3, Y/1e3, eta*100, levels=10, colors='white', linewidths=0.5)
axes[1, 0].set_xlabel('x (km)')
axes[1, 0].set_ylabel('y (km)')
axes[1, 0].set_title('Current Speed (cm/s)')
plt.colorbar(im3, ax=axes[1, 0])

# Plot 4: Cross-section of velocity
mid_idx = ny // 2
axes[1, 1].plot(x/1e3, u_g[mid_idx, :]*100, 'b-', label='u (E-W)', linewidth=2)
axes[1, 1].plot(x/1e3, v_g[mid_idx, :]*100, 'r-', label='v (N-S)', linewidth=2)
axes[1, 1].axhline(y=0, color='k', linestyle='--', linewidth=0.5)
axes[1, 1].set_xlabel('x (km)')
axes[1, 1].set_ylabel('Velocity (cm/s)')
axes[1, 1].set_title(f'Cross-section at y = {y[mid_idx]/1e3:.0f} km')
axes[1, 1].legend()
axes[1, 1].grid(True, alpha=0.3)

plt.tight_layout()
save_plot('geostrophic_flow.pdf', 'Geostrophic flow around a mesoscale eddy')

max_speed = np.max(speed) * 100
\end{pycode}

\section{Ekman Spiral}

\begin{pycode}
# Ekman layer parameters
A_v = 0.01  # Vertical eddy viscosity (m^2/s)
D_E = np.sqrt(2 * A_v / f)  # Ekman depth

# Wind stress
tau_x = 0.1  # N/m^2 (eastward wind)
tau_y = 0.0

# Surface velocity
V_0 = tau_x / (rho_0 * np.sqrt(A_v * f / 2))

# Depth array
z = np.linspace(0, -5 * D_E, 100)

# Ekman spiral velocities
u_ekman = V_0 * np.exp(z / D_E) * np.cos(np.pi/4 + z/D_E)
v_ekman = V_0 * np.exp(z / D_E) * np.sin(np.pi/4 + z/D_E)

# Ekman transport
M_E_x = -tau_y / (rho_0 * f)
M_E_y = tau_x / (rho_0 * f)

# Visualization
fig, axes = plt.subplots(2, 2, figsize=(12, 10))

# Plot 1: Ekman spiral hodograph
axes[0, 0].plot(u_ekman*100, v_ekman*100, 'b-', linewidth=2)
axes[0, 0].scatter([u_ekman[0]*100], [v_ekman[0]*100], c='red', s=100, zorder=5, label='Surface')
axes[0, 0].scatter([0], [0], c='black', s=50, marker='x')
axes[0, 0].arrow(0, 0, 5, 0, head_width=0.3, head_length=0.2, fc='green', ec='green', label='Wind')
axes[0, 0].set_xlabel('u (cm/s)')
axes[0, 0].set_ylabel('v (cm/s)')
axes[0, 0].set_title('Ekman Spiral Hodograph')
axes[0, 0].axis('equal')
axes[0, 0].grid(True, alpha=0.3)
axes[0, 0].legend()

# Plot 2: Velocity profiles with depth
axes[0, 1].plot(u_ekman*100, z, 'b-', label='u', linewidth=2)
axes[0, 1].plot(v_ekman*100, z, 'r-', label='v', linewidth=2)
axes[0, 1].axvline(x=0, color='k', linestyle='--', linewidth=0.5)
axes[0, 1].axhline(y=-D_E, color='gray', linestyle=':', label=f'$D_E$ = {D_E:.1f} m')
axes[0, 1].set_xlabel('Velocity (cm/s)')
axes[0, 1].set_ylabel('Depth (m)')
axes[0, 1].set_title('Ekman Velocity Profiles')
axes[0, 1].legend()
axes[0, 1].grid(True, alpha=0.3)

# Plot 3: 3D Ekman spiral
from mpl_toolkits.mplot3d import Axes3D
ax3d = fig.add_subplot(2, 2, 3, projection='3d')
ax3d.plot(u_ekman*100, v_ekman*100, z, 'b-', linewidth=2)
ax3d.scatter([u_ekman[0]*100], [v_ekman[0]*100], [z[0]], c='red', s=100)
ax3d.set_xlabel('u (cm/s)')
ax3d.set_ylabel('v (cm/s)')
ax3d.set_zlabel('Depth (m)')
ax3d.set_title('3D Ekman Spiral')

# Plot 4: Cumulative transport
cumulative_u = np.cumsum(u_ekman) * np.abs(z[1] - z[0])
cumulative_v = np.cumsum(v_ekman) * np.abs(z[1] - z[0])
axes[1, 1].plot(cumulative_u, z, 'b-', label='$M_x$', linewidth=2)
axes[1, 1].plot(cumulative_v, z, 'r-', label='$M_y$', linewidth=2)
axes[1, 1].axvline(x=M_E_y, color='r', linestyle='--', alpha=0.5, label=f'$M_E$ = {M_E_y:.2f}')
axes[1, 1].set_xlabel('Cumulative Transport (m$^2$/s)')
axes[1, 1].set_ylabel('Depth (m)')
axes[1, 1].set_title('Ekman Transport Integration')
axes[1, 1].legend()
axes[1, 1].grid(True, alpha=0.3)

plt.tight_layout()
save_plot('ekman_spiral.pdf', 'Ekman spiral and transport in the surface boundary layer')
\end{pycode}

\section{Sverdrup Transport}

\begin{pycode}
# Basin setup for Sverdrup calculation
Lx = 6000e3  # Basin width (m)
Ly = 3000e3  # Basin length (m)
nx, ny = 100, 50

x_s = np.linspace(0, Lx, nx)
y_s = np.linspace(0, Ly, ny)
X_s, Y_s = np.meshgrid(x_s, y_s)

# Latitude range (20-50 degrees N)
lat_s = 20 + 30 * Y_s / Ly
f_s = 2 * Omega * np.sin(np.radians(lat_s))
beta_s = 2 * Omega * np.cos(np.radians(lat_s)) / 6.371e6

# Wind stress (idealized westerlies/trades)
tau_x_s = -0.1 * np.cos(2 * np.pi * Y_s / Ly)  # Sinusoidal wind pattern
tau_y_s = np.zeros_like(X_s)

# Wind stress curl
dtau_x_dy = np.gradient(tau_x_s, y_s[1] - y_s[0], axis=0)
curl_tau = -dtau_x_dy  # For tau_y = 0

# Sverdrup transport
V_sv = curl_tau / (rho_0 * beta_s)

# Integrate from eastern boundary to get stream function
psi = np.zeros_like(V_sv)
for i in range(nx-2, -1, -1):
    psi[:, i] = psi[:, i+1] - V_sv[:, i] * (x_s[1] - x_s[0])

# Calculate interior velocities
u_sv = -np.gradient(psi, y_s[1] - y_s[0], axis=0)
v_sv = np.gradient(psi, x_s[1] - x_s[0], axis=1)

fig, axes = plt.subplots(2, 2, figsize=(12, 10))

# Plot 1: Wind stress pattern
im1 = axes[0, 0].contourf(X_s/1e3, Y_s/1e3, tau_x_s, levels=20, cmap='RdBu_r')
axes[0, 0].set_xlabel('x (km)')
axes[0, 0].set_ylabel('y (km)')
axes[0, 0].set_title('Zonal Wind Stress (N/m$^2$)')
plt.colorbar(im1, ax=axes[0, 0])

# Plot 2: Wind stress curl
im2 = axes[0, 1].contourf(X_s/1e3, Y_s/1e3, curl_tau*1e7, levels=20, cmap='RdBu_r')
axes[0, 1].set_xlabel('x (km)')
axes[0, 1].set_ylabel('y (km)')
axes[0, 1].set_title('Wind Stress Curl ($\\times 10^{-7}$ N/m$^3$)')
plt.colorbar(im2, ax=axes[0, 1])

# Plot 3: Sverdrup stream function
levels = np.linspace(psi.min(), psi.max(), 20)
im3 = axes[1, 0].contourf(X_s/1e3, Y_s/1e3, psi/1e6, levels=20, cmap='coolwarm')
cs = axes[1, 0].contour(X_s/1e3, Y_s/1e3, psi/1e6, levels=15, colors='black', linewidths=0.5)
axes[1, 0].set_xlabel('x (km)')
axes[1, 0].set_ylabel('y (km)')
axes[1, 0].set_title('Stream Function (Sv)')
plt.colorbar(im3, ax=axes[1, 0])

# Plot 4: Meridional transport
axes[1, 1].plot(y_s/1e3, V_sv[:, nx//2]/1e6, 'b-', linewidth=2)
axes[1, 1].axhline(y=0, color='k', linestyle='--', linewidth=0.5)
axes[1, 1].set_xlabel('y (km)')
axes[1, 1].set_ylabel('V (Sv/m)')
axes[1, 1].set_title('Sverdrup Meridional Transport')
axes[1, 1].grid(True, alpha=0.3)

plt.tight_layout()
save_plot('sverdrup_transport.pdf', 'Sverdrup wind-driven circulation')

max_transport = np.max(np.abs(psi)) / 1e6
\end{pycode}

\section{Western Boundary Current}

\begin{pycode}
# Munk layer solution for western boundary current
# Simplified 1D model

# Parameters
A_H = 1e4  # Horizontal eddy viscosity (m^2/s)
H = 1000   # Layer depth (m)
L_basin = 5000e3  # Basin width (m)

# Munk layer width
delta_M = (A_H / beta)**0.333

# Interior Sverdrup transport (constant for simplicity)
psi_I = 30e6  # 30 Sv

# Distance from western boundary
x_wb = np.linspace(0, 500e3, 500)

# Munk solution (normalized)
xi = x_wb / delta_M
psi_wb = psi_I * (1 - np.exp(-xi/2) * (np.cos(np.sqrt(3)/2 * xi) +
                                        1/np.sqrt(3) * np.sin(np.sqrt(3)/2 * xi)))

# Velocity in western boundary layer
v_wb = np.gradient(psi_wb, x_wb[1] - x_wb[0]) / H

# Also compute Stommel solution (bottom friction)
r = 1e-7  # Bottom friction coefficient
delta_S = r / beta
psi_stommel = psi_I * (1 - np.exp(-x_wb / delta_S))
v_stommel = np.gradient(psi_stommel, x_wb[1] - x_wb[0]) / H

fig, axes = plt.subplots(2, 2, figsize=(12, 10))

# Plot 1: Stream function comparison
axes[0, 0].plot(x_wb/1e3, psi_wb/1e6, 'b-', linewidth=2, label='Munk')
axes[0, 0].plot(x_wb/1e3, psi_stommel/1e6, 'r--', linewidth=2, label='Stommel')
axes[0, 0].axhline(y=psi_I/1e6, color='gray', linestyle=':', label='Interior')
axes[0, 0].axvline(x=delta_M/1e3, color='b', linestyle=':', alpha=0.5)
axes[0, 0].axvline(x=delta_S/1e3, color='r', linestyle=':', alpha=0.5)
axes[0, 0].set_xlabel('Distance from western boundary (km)')
axes[0, 0].set_ylabel('$\\psi$ (Sv)')
axes[0, 0].set_title('Western Boundary Current Structure')
axes[0, 0].legend()
axes[0, 0].grid(True, alpha=0.3)

# Plot 2: Velocity profiles
axes[0, 1].plot(x_wb/1e3, v_wb*100, 'b-', linewidth=2, label='Munk')
axes[0, 1].plot(x_wb/1e3, v_stommel*100, 'r--', linewidth=2, label='Stommel')
axes[0, 1].set_xlabel('Distance from western boundary (km)')
axes[0, 1].set_ylabel('v (cm/s)')
axes[0, 1].set_title('Northward Velocity')
axes[0, 1].legend()
axes[0, 1].grid(True, alpha=0.3)
axes[0, 1].set_xlim(0, 200)

# Plot 3: 2D gyre circulation (simplified)
nx_gyre, ny_gyre = 100, 50
x_gyre = np.linspace(0, L_basin, nx_gyre)
y_gyre = np.linspace(0, Ly, ny_gyre)
X_gyre, Y_gyre = np.meshgrid(x_gyre, y_gyre)

# Simple subtropical gyre
psi_gyre = psi_I * np.sin(np.pi * Y_gyre / Ly)
# Western intensification
xi_gyre = X_gyre / delta_M
wb_factor = 1 - np.exp(-xi_gyre/2) * (np.cos(np.sqrt(3)/2 * xi_gyre) +
                                       1/np.sqrt(3) * np.sin(np.sqrt(3)/2 * xi_gyre))
wb_factor = np.clip(wb_factor, 0, 1)
psi_gyre = psi_gyre * wb_factor

im3 = axes[1, 0].contourf(X_gyre/1e3, Y_gyre/1e3, psi_gyre/1e6, levels=20, cmap='coolwarm')
cs = axes[1, 0].contour(X_gyre/1e3, Y_gyre/1e3, psi_gyre/1e6, levels=10, colors='black', linewidths=0.5)
axes[1, 0].set_xlabel('x (km)')
axes[1, 0].set_ylabel('y (km)')
axes[1, 0].set_title('Subtropical Gyre with Western Intensification')
plt.colorbar(im3, ax=axes[1, 0], label='$\\psi$ (Sv)')

# Plot 4: Cross-basin transport profile
mid_y = ny_gyre // 2
axes[1, 1].plot(x_gyre/1e3, psi_gyre[mid_y, :]/1e6, 'b-', linewidth=2)
axes[1, 1].axvline(x=delta_M/1e3, color='r', linestyle='--',
                   label=f'$\\delta_M$ = {delta_M/1e3:.0f} km')
axes[1, 1].set_xlabel('x (km)')
axes[1, 1].set_ylabel('$\\psi$ (Sv)')
axes[1, 1].set_title('Cross-basin Transport Profile')
axes[1, 1].legend()
axes[1, 1].grid(True, alpha=0.3)

plt.tight_layout()
save_plot('western_boundary.pdf', 'Western boundary current dynamics')

max_wbc_vel = np.max(v_wb) * 100
\end{pycode}

\section{Results Summary}

\subsection{Geostrophic Flow Statistics}
\begin{pycode}
print(r'\begin{table}[htbp]')
print(r'\centering')
print(r'\caption{Geostrophic flow around mesoscale eddy}')
print(r'\begin{tabular}{lr}')
print(r'\toprule')
print(r'Parameter & Value \\')
print(r'\midrule')
print(f"Maximum SSH anomaly & {np.max(eta)*100:.1f} cm \\\\")
print(f"Maximum current speed & {max_speed:.1f} cm/s \\\\")
print(f"Coriolis parameter & {f:.2e} s$^{{-1}}$ \\\\")
print(f"Eddy radius & 200 km \\\\")
print(r'\bottomrule')
print(r'\end{tabular}')
print(r'\end{table}')
\end{pycode}

\subsection{Ekman Layer Parameters}
\begin{pycode}
print(r'\begin{table}[htbp]')
print(r'\centering')
print(r'\caption{Ekman spiral characteristics}')
print(r'\begin{tabular}{lr}')
print(r'\toprule')
print(r'Parameter & Value \\')
print(r'\midrule')
print(f"Ekman depth & {D_E:.1f} m \\\\")
print(f"Surface velocity & {V_0*100:.1f} cm/s \\\\")
print(f"Wind stress & {tau_x:.2f} N/m$^2$ \\\\")
print(f"Ekman transport & {M_E_y:.2f} m$^2$/s \\\\")
print(f"Surface deflection & 45$^\\circ$ \\\\")
print(r'\bottomrule')
print(r'\end{tabular}')
print(r'\end{table}')
\end{pycode}

\subsection{Gyre Circulation}
\begin{pycode}
print(r'\begin{table}[htbp]')
print(r'\centering')
print(r'\caption{Wind-driven gyre parameters}')
print(r'\begin{tabular}{lr}')
print(r'\toprule')
print(r'Parameter & Value \\')
print(r'\midrule')
print(f"Maximum transport & {max_transport:.1f} Sv \\\\")
print(f"Munk layer width & {delta_M/1e3:.0f} km \\\\")
print(f"Stommel layer width & {delta_S/1e3:.0f} km \\\\")
print(f"Max WBC velocity & {max_wbc_vel:.1f} cm/s \\\\")
print(f"Basin width & {L_basin/1e3:.0f} km \\\\")
print(r'\bottomrule')
print(r'\end{tabular}')
print(r'\end{table}')
\end{pycode}

\subsection{Physical Summary}
\begin{itemize}
    \item Ekman depth: \py{f"{D_E:.1f}"} m
    \item Munk boundary layer: \py{f"{delta_M/1e3:.0f}"} km
    \item Maximum gyre transport: \py{f"{max_transport:.1f}"} Sv
    \item WBC velocity: \py{f"{max_wbc_vel:.1f}"} cm/s
\end{itemize}

\section{Conclusion}
This template demonstrates the fundamental dynamics of ocean circulation. Geostrophic balance governs large-scale flow, with velocities determined by pressure gradients and Earth's rotation. The Ekman spiral shows how wind stress drives surface currents deflected from the wind direction. Sverdrup theory relates interior transport to wind stress curl, while western boundary current intensification results from the need to close the gyre circulation and conserve vorticity.

\end{document}
