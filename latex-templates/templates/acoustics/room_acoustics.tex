% Room Acoustics Analysis with PythonTeX
\documentclass[11pt,a4paper]{article}
\usepackage[utf8]{inputenc}
\usepackage[T1]{fontenc}
\usepackage{amsmath,amssymb}
\usepackage{graphicx}
\usepackage{booktabs}
\usepackage{siunitx}
\usepackage{geometry}
\geometry{margin=1in}
\usepackage{pythontex}
\usepackage{hyperref}
\usepackage{float}

\title{Room Acoustics Analysis\\Reverberation and Sound Field Modeling}
\author{Acoustics Engineering Laboratory}
\date{\today}

\begin{document}
\maketitle

\begin{abstract}
This technical report presents computational analysis of room acoustics including reverberation time calculations using Sabine and Eyring equations, sound absorption modeling, and acoustic parameter optimization.
\end{abstract}

\section{Introduction}

Room acoustics determines sound perception quality in enclosed spaces. The reverberation time $T_{60}$ is the primary metric.

\begin{pycode}
import numpy as np
import matplotlib.pyplot as plt
plt.rcParams['text.usetex'] = True
plt.rcParams['font.family'] = 'serif'
plt.rcParams['font.size'] = 10

# Room parameters
room_length, room_width, room_height = 15.0, 10.0, 4.0
room_volume = room_length * room_width * room_height
floor_area = room_length * room_width
ceiling_area = floor_area
wall_area = 2 * (room_length + room_width) * room_height
total_surface = floor_area + ceiling_area + wall_area
c = 343  # m/s

frequencies = np.array([125, 250, 500, 1000, 2000, 4000])
materials = {
    'Concrete': np.array([0.01, 0.01, 0.02, 0.02, 0.02, 0.03]),
    'Carpet': np.array([0.08, 0.24, 0.57, 0.69, 0.71, 0.73]),
    'Acoustic Tiles': np.array([0.29, 0.44, 0.60, 0.77, 0.86, 0.84]),
    'Glass': np.array([0.35, 0.25, 0.18, 0.12, 0.07, 0.04]),
}

fig, ax = plt.subplots(figsize=(8, 5))
for material, alpha in materials.items():
    ax.semilogx(frequencies, alpha, 'o-', label=material, linewidth=1.5)
ax.set_xlabel('Frequency (Hz)')
ax.set_ylabel('Absorption Coefficient $\\alpha$')
ax.set_title('Sound Absorption Coefficients')
ax.legend(loc='upper right', fontsize=8)
ax.grid(True, alpha=0.3)
plt.tight_layout()
plt.savefig('absorption_coefficients.pdf', dpi=150, bbox_inches='tight')
plt.close()
\end{pycode}

\begin{figure}[H]
\centering
\includegraphics[width=0.85\textwidth]{absorption_coefficients.pdf}
\caption{Frequency-dependent absorption coefficients for common materials.}
\end{figure}

\section{Sabine Reverberation Time}

The Sabine equation: $T_{60} = \frac{0.161 V}{A}$

\begin{pycode}
alpha_walls = materials['Concrete']
alpha_floor = materials['Carpet']
alpha_ceiling = materials['Acoustic Tiles']

total_absorption = wall_area * alpha_walls + floor_area * alpha_floor + ceiling_area * alpha_ceiling
RT60_sabine = 0.161 * room_volume / total_absorption

fig, ax = plt.subplots(figsize=(8, 5))
ax.semilogx(frequencies, RT60_sabine, 'b-o', linewidth=2, label='Sabine $T_{60}$')
ax.axhline(y=0.5, color='g', linestyle='--', label='Optimal (Speech)')
ax.axhline(y=1.5, color='r', linestyle='--', label='Optimal (Music)')
ax.set_xlabel('Frequency (Hz)')
ax.set_ylabel('Reverberation Time $T_{60}$ (s)')
ax.set_title('Sabine Reverberation Time vs Frequency')
ax.legend()
ax.grid(True, alpha=0.3)
plt.tight_layout()
plt.savefig('rt60_sabine.pdf', dpi=150, bbox_inches='tight')
plt.close()
\end{pycode}

\begin{figure}[H]
\centering
\includegraphics[width=0.85\textwidth]{rt60_sabine.pdf}
\caption{Sabine reverberation time across frequency bands.}
\end{figure}

\section{Eyring Reverberation Time}

\begin{pycode}
alpha_avg = total_absorption / total_surface
RT60_eyring = 0.161 * room_volume / (-total_surface * np.log(1 - alpha_avg))

fig, ax = plt.subplots(figsize=(8, 5))
ax.semilogx(frequencies, RT60_sabine, 'b-o', linewidth=2, label='Sabine')
ax.semilogx(frequencies, RT60_eyring, 'r-s', linewidth=2, label='Eyring')
ax.set_xlabel('Frequency (Hz)')
ax.set_ylabel('$T_{60}$ (s)')
ax.set_title('Comparison of Sabine and Eyring')
ax.legend()
ax.grid(True, alpha=0.3)
plt.tight_layout()
plt.savefig('rt60_comparison.pdf', dpi=150, bbox_inches='tight')
plt.close()
\end{pycode}

\begin{figure}[H]
\centering
\includegraphics[width=0.85\textwidth]{rt60_comparison.pdf}
\caption{Comparison of Sabine and Eyring predictions.}
\end{figure}

\section{Sound Pressure Level Distribution}

\begin{pycode}
idx_1k = 3
A_1k = total_absorption[idx_1k]
alpha_avg_1k = alpha_avg[idx_1k]
R_room = A_1k / (1 - alpha_avg_1k)
L_W, Q = 90, 2
r = np.linspace(0.5, 15, 100)

direct_field = Q / (4 * np.pi * r**2)
reverb_field = 4 / R_room
L_p = L_W + 10 * np.log10(direct_field + reverb_field)
r_c = np.sqrt(Q * R_room / (16 * np.pi))

fig, ax = plt.subplots(figsize=(8, 5))
ax.plot(r, L_p, 'b-', linewidth=2, label='Total SPL')
ax.axvline(x=r_c, color='k', linestyle='--', label=f'$r_c$ = {r_c:.2f} m')
ax.set_xlabel('Distance from Source (m)')
ax.set_ylabel('Sound Pressure Level (dB)')
ax.set_title('SPL Distribution in Room')
ax.legend()
ax.grid(True, alpha=0.3)
plt.tight_layout()
plt.savefig('spl_distribution.pdf', dpi=150, bbox_inches='tight')
plt.close()
\end{pycode}

\begin{figure}[H]
\centering
\includegraphics[width=0.85\textwidth]{spl_distribution.pdf}
\caption{Sound pressure level as function of distance.}
\end{figure}

\section{Room Modes}

\begin{pycode}
modes = []
for nx in range(0, 5):
    for ny in range(0, 5):
        for nz in range(0, 5):
            if nx == 0 and ny == 0 and nz == 0:
                continue
            f = (c/2) * np.sqrt((nx/room_length)**2 + (ny/room_width)**2 + (nz/room_height)**2)
            if f <= 200:
                modes.append(f)
modes.sort()

fig, ax = plt.subplots(figsize=(10, 3))
ax.eventplot([modes], lineoffsets=0.5, linelengths=0.8, colors='blue')
ax.set_xlabel('Frequency (Hz)')
ax.set_title('Room Mode Distribution (0-200 Hz)')
ax.set_xlim([0, 200])
ax.set_ylim([0, 1])
ax.set_yticks([])
plt.tight_layout()
plt.savefig('room_modes.pdf', dpi=150, bbox_inches='tight')
plt.close()

f_schroeder = 2000 * np.sqrt(RT60_sabine[idx_1k] / room_volume)
\end{pycode}

\begin{figure}[H]
\centering
\includegraphics[width=0.9\textwidth]{room_modes.pdf}
\caption{Room mode distribution in low-frequency range.}
\end{figure}

\section{Clarity Indices}

\begin{pycode}
t_ir = np.linspace(0, 2, 1000)
decay_rate = 6.91 / RT60_sabine[idx_1k]
impulse_response = np.exp(-decay_rate * t_ir)

idx_50 = np.argmin(np.abs(t_ir - 0.050))
energy_early = np.trapz(impulse_response[:idx_50]**2, t_ir[:idx_50])
energy_late = np.trapz(impulse_response[idx_50:]**2, t_ir[idx_50:])
C50 = 10 * np.log10(energy_early / energy_late)

fig, ax = plt.subplots(figsize=(8, 5))
ax.plot(t_ir * 1000, 20*np.log10(impulse_response + 1e-10), 'b-', linewidth=1.5)
ax.axvline(x=50, color='g', linestyle='--', label='50 ms')
ax.set_xlabel('Time (ms)')
ax.set_ylabel('Level (dB)')
ax.set_title('Room Impulse Response Energy Decay')
ax.legend()
ax.grid(True, alpha=0.3)
ax.set_xlim([0, 500])
plt.tight_layout()
plt.savefig('impulse_response.pdf', dpi=150, bbox_inches='tight')
plt.close()
\end{pycode}

\begin{figure}[H]
\centering
\includegraphics[width=0.85\textwidth]{impulse_response.pdf}
\caption{Room impulse response energy decay curve.}
\end{figure}

\section{Optimization}

\begin{pycode}
ceiling_coverage = np.linspace(0, 1, 50)
RT60_optimized = []
for coverage in ceiling_coverage:
    alpha_opt = coverage * materials['Acoustic Tiles'] + (1 - coverage) * materials['Concrete']
    A_opt = wall_area * alpha_walls + floor_area * alpha_floor + ceiling_area * alpha_opt
    RT60_optimized.append(0.161 * room_volume / A_opt[idx_1k])
RT60_optimized = np.array(RT60_optimized)

target_RT60 = 0.8
optimal_idx = np.argmin(np.abs(RT60_optimized - target_RT60))
optimal_coverage = ceiling_coverage[optimal_idx]

fig, ax = plt.subplots(figsize=(8, 5))
ax.plot(ceiling_coverage * 100, RT60_optimized, 'b-', linewidth=2)
ax.axhline(y=target_RT60, color='r', linestyle='--', label=f'Target = {target_RT60} s')
ax.set_xlabel('Acoustic Tile Coverage (\\%)')
ax.set_ylabel('$T_{60}$ (s)')
ax.set_title('$T_{60}$ Optimization')
ax.legend()
ax.grid(True, alpha=0.3)
plt.tight_layout()
plt.savefig('rt60_optimization.pdf', dpi=150, bbox_inches='tight')
plt.close()
\end{pycode}

\begin{figure}[H]
\centering
\includegraphics[width=0.85\textwidth]{rt60_optimization.pdf}
\caption{$T_{60}$ optimization via ceiling treatment.}
\end{figure}

\section{Results}

\begin{pycode}
print(r'\begin{table}[H]')
print(r'\centering')
print(r'\caption{Room Acoustic Parameters}')
print(r'\begin{tabular}{@{}ccc@{}}')
print(r'\toprule')
print(r'Frequency (Hz) & $T_{60}$ Sabine (s) & $T_{60}$ Eyring (s) \\')
print(r'\midrule')
for i, f in enumerate(frequencies):
    print(f'{f} & {RT60_sabine[i]:.2f} & {RT60_eyring[i]:.2f} \\\\')
print(r'\bottomrule')
print(r'\end{tabular}')
print(r'\end{table}')
\end{pycode}

Key metrics: $r_c = \py{round(r_c, 2)}$ m, $C_{50} = \py{round(C50, 1)}$ dB, $f_s = \py{round(f_schroeder, 1)}$ Hz.

\section{Conclusions}

The room configuration provides acceptable reverberation for multipurpose use. Optimal ceiling coverage is \py{round(optimal_coverage * 100, 0)}\%.

\end{document}
