% Sound Propagation Analysis
\documentclass[11pt,a4paper]{article}
\usepackage[utf8]{inputenc}
\usepackage[T1]{fontenc}
\usepackage{amsmath,amssymb}
\usepackage{graphicx}
\usepackage{booktabs}
\usepackage{siunitx}
\usepackage{geometry}
\geometry{margin=1in}
\usepackage{pythontex}
\usepackage{hyperref}
\usepackage{float}

\title{Sound Propagation Analysis\\Wave Equations and Transmission Loss}
\author{Acoustic Engineering Department}
\date{\today}

\begin{document}
\maketitle

\begin{abstract}
Analysis of sound wave propagation through various media, including acoustic impedance, transmission coefficients, and transmission loss calculations.
\end{abstract}

\section{Introduction}

Sound propagation is governed by the acoustic wave equation and boundary conditions at material interfaces.

\begin{pycode}
import numpy as np
import matplotlib.pyplot as plt
plt.rcParams['text.usetex'] = True
plt.rcParams['font.family'] = 'serif'

c_air, rho_air = 343, 1.21
Z_air = rho_air * c_air

media = {
    'Air': {'c': 343, 'rho': 1.21},
    'Water': {'c': 1480, 'rho': 1000},
    'Steel': {'c': 5960, 'rho': 7850},
    'Concrete': {'c': 3400, 'rho': 2400},
    'Glass': {'c': 5200, 'rho': 2500},
}
for props in media.values():
    props['Z'] = props['rho'] * props['c']
\end{pycode}

\section{Acoustic Impedance}

$Z = \rho c$

\begin{pycode}
fig, ax = plt.subplots(figsize=(10, 5))
names = list(media.keys())
impedances = [media[n]['Z'] for n in names]
ax.bar(names, impedances, color=plt.cm.viridis(np.linspace(0, 0.8, len(names))))
ax.set_ylabel('Acoustic Impedance (Pa$\\cdot$s/m)')
ax.set_title('Characteristic Acoustic Impedance')
ax.set_yscale('log')
ax.grid(True, alpha=0.3, axis='y')
plt.tight_layout()
plt.savefig('impedance_comparison.pdf', dpi=150, bbox_inches='tight')
plt.close()
\end{pycode}

\begin{figure}[H]
\centering
\includegraphics[width=0.9\textwidth]{impedance_comparison.pdf}
\caption{Acoustic impedance comparison.}
\end{figure}

\section{Reflection and Transmission}

$R = \frac{Z_2 - Z_1}{Z_2 + Z_1}$

\begin{pycode}
materials = ['Water', 'Steel', 'Concrete', 'Glass']
Z1 = media['Air']['Z']
R_coeffs = [(media[m]['Z'] - Z1) / (media[m]['Z'] + Z1) for m in materials]
T_intensity = [1 - r**2 for r in R_coeffs]

fig, (ax1, ax2) = plt.subplots(1, 2, figsize=(12, 5))
ax1.bar(materials, R_coeffs, color='steelblue')
ax1.set_ylabel('Reflection Coefficient $R$')
ax1.set_title('Reflection at Air-Material Interface')
ax1.grid(True, alpha=0.3, axis='y')

ax2.bar(materials, T_intensity, color='coral')
ax2.set_ylabel('Intensity Transmission $T_I$')
ax2.set_title('Transmission')
ax2.grid(True, alpha=0.3, axis='y')
plt.tight_layout()
plt.savefig('reflection_transmission.pdf', dpi=150, bbox_inches='tight')
plt.close()
\end{pycode}

\begin{figure}[H]
\centering
\includegraphics[width=0.95\textwidth]{reflection_transmission.pdf}
\caption{Reflection and transmission coefficients.}
\end{figure}

\section{Mass Law Transmission Loss}

$TL = 20 \log_{10}(\pi f m / \rho c)$

\begin{pycode}
freq = np.logspace(1, 4, 200)
surface_masses = {'Gypsum (12mm)': 10, 'Concrete (100mm)': 240, 'Steel (3mm)': 24, 'Glass (6mm)': 15}

fig, ax = plt.subplots(figsize=(10, 6))
for name, m in surface_masses.items():
    TL = 20 * np.log10(np.pi * freq * m / (rho_air * c_air))
    ax.semilogx(freq, TL, linewidth=1.5, label=f'{name}')
ax.set_xlabel('Frequency (Hz)')
ax.set_ylabel('Transmission Loss (dB)')
ax.set_title('Mass Law Transmission Loss')
ax.legend(loc='lower right')
ax.grid(True, alpha=0.3, which='both')
ax.set_xlim([10, 10000])
plt.tight_layout()
plt.savefig('mass_law_tl.pdf', dpi=150, bbox_inches='tight')
plt.close()
\end{pycode}

\begin{figure}[H]
\centering
\includegraphics[width=0.9\textwidth]{mass_law_tl.pdf}
\caption{Mass law transmission loss predictions.}
\end{figure}

\section{Coincidence Effect}

\begin{pycode}
panels = {
    'Steel (3mm)': {'E': 200e9, 'rho': 7850, 'nu': 0.3, 'h': 0.003},
    'Aluminum (2mm)': {'E': 70e9, 'rho': 2700, 'nu': 0.33, 'h': 0.002},
    'Glass (6mm)': {'E': 70e9, 'rho': 2500, 'nu': 0.22, 'h': 0.006},
}

f_coincidence = {}
for name, p in panels.items():
    fc = (c_air**2 / (2 * np.pi)) * np.sqrt(12 * p['rho'] * (1 - p['nu']**2) / (p['E'] * p['h']**2))
    f_coincidence[name] = fc

fig, ax = plt.subplots(figsize=(10, 6))
for name, p in panels.items():
    m = p['rho'] * p['h']
    fc = f_coincidence[name]
    TL_mass = 20 * np.log10(np.pi * freq * m / (rho_air * c_air))
    dip = 15 * np.exp(-(freq - fc)**2 / (2 * (fc * 0.5)**2))
    TL = np.maximum(TL_mass - dip, 0)
    ax.semilogx(freq, TL, linewidth=1.5, label=name)
ax.set_xlabel('Frequency (Hz)')
ax.set_ylabel('Transmission Loss (dB)')
ax.set_title('TL with Coincidence Effect')
ax.legend(loc='lower right')
ax.grid(True, alpha=0.3, which='both')
ax.set_xlim([100, 10000])
plt.tight_layout()
plt.savefig('coincidence_tl.pdf', dpi=150, bbox_inches='tight')
plt.close()
\end{pycode}

\begin{figure}[H]
\centering
\includegraphics[width=0.9\textwidth]{coincidence_tl.pdf}
\caption{Transmission loss with coincidence dips.}
\end{figure}

\section{Double Wall TL}

\begin{pycode}
m1, m2 = 12, 12
air_gaps = [50, 100, 200]

fig, ax = plt.subplots(figsize=(10, 6))
for gap in air_gaps:
    ell = gap / 1000
    f0 = (1 / (2 * np.pi)) * np.sqrt((rho_air * c_air**2 / ell) * (1/m1 + 1/m2))
    TL_double = np.zeros_like(freq)
    for i, f in enumerate(freq):
        if f < f0:
            TL_double[i] = 20 * np.log10(np.pi * f * (m1 + m2) / (rho_air * c_air))
        else:
            TL_single = 20 * np.log10(np.pi * f * m1 / (rho_air * c_air))
            TL_double[i] = 2 * TL_single + 20 * np.log10(f / f0)
    ax.semilogx(freq, np.maximum(TL_double, 0), linewidth=1.5, label=f'Gap = {gap} mm')
ax.set_xlabel('Frequency (Hz)')
ax.set_ylabel('Transmission Loss (dB)')
ax.set_title('Double Wall Transmission Loss')
ax.legend(loc='lower right')
ax.grid(True, alpha=0.3, which='both')
ax.set_xlim([50, 5000])
plt.tight_layout()
plt.savefig('double_wall_tl.pdf', dpi=150, bbox_inches='tight')
plt.close()
\end{pycode}

\begin{figure}[H]
\centering
\includegraphics[width=0.9\textwidth]{double_wall_tl.pdf}
\caption{Double wall transmission loss.}
\end{figure}

\section{Atmospheric Absorption}

\begin{pycode}
def atm_absorption(f, T=20, h=50, p=101.325):
    T_K = T + 273.15
    C = -6.8346 * (273.16/T_K)**1.261 + 4.6151
    h_mol = h * (101.325/p) * 10**C
    f_rO = (p/101.325) * (24 + 4.04e4 * h_mol * (0.02 + h_mol)/(0.391 + h_mol))
    f_rN = (p/101.325) * (T_K/293.15)**(-0.5) * (9 + 280 * h_mol * np.exp(-4.170 * ((T_K/293.15)**(-1/3) - 1)))
    alpha = 8.686 * f**2 * ((1.84e-11 * (101.325/p) * (T_K/293.15)**0.5) +
            (T_K/293.15)**(-2.5) * (0.01275 * np.exp(-2239.1/T_K) / (f_rO + f**2/f_rO) +
            0.1068 * np.exp(-3352.0/T_K) / (f_rN + f**2/f_rN)))
    return alpha

freq_atm = np.logspace(2, 4.5, 100)
fig, ax = plt.subplots(figsize=(10, 6))
for T, h, label in [(20, 50, '20C, 50\\% RH'), (20, 20, '20C, 20\\% RH'), (0, 50, '0C, 50\\% RH')]:
    ax.loglog(freq_atm, atm_absorption(freq_atm, T, h) * 1000, linewidth=1.5, label=label)
ax.set_xlabel('Frequency (Hz)')
ax.set_ylabel('Absorption (dB/km)')
ax.set_title('Atmospheric Absorption')
ax.legend()
ax.grid(True, alpha=0.3, which='both')
plt.tight_layout()
plt.savefig('atmospheric_absorption.pdf', dpi=150, bbox_inches='tight')
plt.close()
\end{pycode}

\begin{figure}[H]
\centering
\includegraphics[width=0.9\textwidth]{atmospheric_absorption.pdf}
\caption{Atmospheric sound absorption.}
\end{figure}

\section{Spreading Loss}

\begin{pycode}
L_W = 100
distances = np.linspace(1, 100, 100)
freq_cases = [500, 2000, 8000]

fig, ax = plt.subplots(figsize=(10, 6))
for f in freq_cases:
    alpha = atm_absorption(f)
    L_p = L_W - 20 * np.log10(distances) - 11 - alpha * distances
    ax.plot(distances, L_p, linewidth=1.5, label=f'{f} Hz')
ax.set_xlabel('Distance (m)')
ax.set_ylabel('SPL (dB)')
ax.set_title('Sound Level vs Distance')
ax.legend()
ax.grid(True, alpha=0.3)
plt.tight_layout()
plt.savefig('spreading_loss.pdf', dpi=150, bbox_inches='tight')
plt.close()
\end{pycode}

\begin{figure}[H]
\centering
\includegraphics[width=0.9\textwidth]{spreading_loss.pdf}
\caption{Sound level decay with distance.}
\end{figure}

\section{Results}

\begin{pycode}
print(r'\begin{table}[H]')
print(r'\centering')
print(r'\caption{Acoustic Properties}')
print(r'\begin{tabular}{@{}lccc@{}}')
print(r'\toprule')
print(r'Material & $c$ (m/s) & $\rho$ (kg/m$^3$) & $Z$ (Pa$\cdot$s/m) \\')
print(r'\midrule')
for name, props in media.items():
    print(f"{name} & {props['c']} & {props['rho']} & {props['Z']:.0f} \\\\")
print(r'\bottomrule')
print(r'\end{tabular}')
print(r'\end{table}')
\end{pycode}

\section{Conclusions}

This analysis demonstrates key principles of sound propagation and transmission loss including mass law, coincidence effects, and atmospheric absorption.

\end{document}
