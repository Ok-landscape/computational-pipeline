\documentclass[a4paper, 11pt]{report}
\usepackage[utf8]{inputenc}
\usepackage[T1]{fontenc}
\usepackage{amsmath, amssymb}
\usepackage{graphicx}
\usepackage{siunitx}
\usepackage{booktabs}
\usepackage{xcolor}
\usepackage[makestderr]{pythontex}

\definecolor{photon}{RGB}{52, 152, 219}
\definecolor{electron}{RGB}{231, 76, 60}
\definecolor{proton}{RGB}{46, 204, 113}

\title{Radiation Dosimetry:\\
Depth-Dose Distributions and Treatment Planning}
\author{Department of Medical Physics\\Technical Report MP-2024-003}
\date{\today}

\begin{document}
\maketitle

\begin{abstract}
This report presents a comprehensive analysis of radiation dosimetry for external beam radiotherapy. We compute depth-dose distributions for photon, electron, and proton beams, analyze tissue inhomogeneity corrections, evaluate dose-volume histograms, and compare treatment planning techniques. All calculations use PythonTeX for reproducibility.
\end{abstract}

\tableofcontents

\chapter{Introduction}

Radiation dosimetry quantifies energy deposition in tissue. The absorbed dose is defined as:
\begin{equation}
D = \frac{d\bar{\varepsilon}}{dm}
\end{equation}
where $d\bar{\varepsilon}$ is the mean energy imparted to matter of mass $dm$.

\section{Dose Quantities}
\begin{itemize}
    \item Absorbed dose $D$ (Gy): Energy per unit mass
    \item Kerma $K$ (Gy): Kinetic energy released per unit mass
    \item Exposure $X$ (R): Ionization in air
\end{itemize}

\begin{pycode}
import numpy as np
import matplotlib.pyplot as plt
from scipy.optimize import curve_fit
from scipy.integrate import cumtrapz
plt.rc('text', usetex=True)
plt.rc('font', family='serif')

np.random.seed(42)

# Depth array (cm)
depth = np.linspace(0, 35, 350)

def photon_pdd(d, energy_MV, field_size=10):
    """Model photon percent depth dose."""
    # d_max increases with energy
    d_max = 0.5 + energy_MV * 0.15
    # Effective attenuation coefficient
    mu_eff = 0.05 - 0.002 * energy_MV + 0.001 * (10 - field_size)

    pdd = np.zeros_like(d)
    buildup = d <= d_max
    pdd[buildup] = 100 * (0.3 + 0.7 * (d[buildup] / d_max)**0.8)
    falloff = d > d_max
    pdd[falloff] = 100 * np.exp(-mu_eff * (d[falloff] - d_max))

    return pdd

def electron_pdd(d, energy_MeV):
    """Model electron percent depth dose."""
    R_50 = energy_MeV / 2.33  # Depth of 50% dose
    R_p = energy_MeV / 2.0    # Practical range
    d_max = 0.46 * energy_MeV**0.6

    pdd = np.zeros_like(d)
    # Surface dose
    surface = d < d_max * 0.3
    pdd[surface] = 75 + 80 * d[surface] / d_max
    # Buildup
    buildup = (d >= d_max * 0.3) & (d < d_max)
    pdd[buildup] = 100 * (d[buildup] / d_max)**0.3
    # Plateau and falloff
    plateau = (d >= d_max) & (d < R_50)
    pdd[plateau] = 100 * np.exp(-0.1 * (d[plateau] - d_max))
    falloff = d >= R_50
    pdd[falloff] = 100 * np.exp(-3 * (d[falloff] - R_50)**2 / (R_p - R_50 + 0.1)**2)
    pdd[d > R_p * 1.1] = 0

    return np.clip(pdd, 0, 100)

def proton_pdd(d, energy_MeV, sigma=0.3):
    """Model proton Bragg peak."""
    R = 0.0022 * energy_MeV**1.77  # Range in water (cm)

    # Entrance plateau
    entrance = d < R * 0.8
    pdd = np.ones_like(d) * 30
    pdd[entrance] = 30 + 5 * d[entrance] / R

    # Bragg peak
    peak = (d >= R * 0.8) & (d <= R * 1.1)
    pdd[peak] = 30 + 70 * np.exp(-((d[peak] - R)**2) / (2 * (sigma * R)**2))

    # Distal falloff
    distal = d > R * 1.1
    pdd[distal] = 30 * np.exp(-10 * (d[distal] - R * 1.1) / R)

    # Normalize
    pdd = pdd / pdd.max() * 100
    return np.clip(pdd, 0, 100)

def tar(d, energy_MV, field_size=10, SSD=100):
    """Calculate tissue-air ratio."""
    pdd = photon_pdd(d, energy_MV, field_size)
    d_max = 0.5 + energy_MV * 0.15
    return pdd / 100 * ((SSD + d_max) / (SSD + d))**2 * 100

def tpr(d, energy_MV):
    """Calculate tissue-phantom ratio."""
    d_ref = 10
    return photon_pdd(d, energy_MV) / photon_pdd(d_ref * np.ones(1), energy_MV)[0] * 100
\end{pycode}

\chapter{Photon Beam Dosimetry}

\begin{pycode}
fig, axes = plt.subplots(2, 3, figsize=(14, 8))

# PDD for different energies
ax = axes[0, 0]
energies_MV = [4, 6, 10, 15, 18]
colors = plt.cm.viridis(np.linspace(0, 0.8, len(energies_MV)))

for E, color in zip(energies_MV, colors):
    pdd = photon_pdd(depth, E)
    ax.plot(depth, pdd, color=color, linewidth=2, label=f'{E} MV')

ax.axhline(y=50, color='gray', linestyle='--', alpha=0.5)
ax.set_xlabel('Depth (cm)')
ax.set_ylabel('Percent Depth Dose (\\%)')
ax.set_title('Photon PDD vs Energy')
ax.legend()
ax.grid(True, alpha=0.3)
ax.set_ylim([0, 110])

# Field size dependence
ax = axes[0, 1]
field_sizes = [5, 10, 20, 30]
colors_fs = plt.cm.plasma(np.linspace(0.2, 0.8, len(field_sizes)))

for fs, color in zip(field_sizes, colors_fs):
    pdd = photon_pdd(depth, 6, fs)
    ax.plot(depth, pdd, color=color, linewidth=2, label=f'{fs} cm')

ax.set_xlabel('Depth (cm)')
ax.set_ylabel('Percent Depth Dose (\\%)')
ax.set_title('6 MV PDD vs Field Size')
ax.legend()
ax.grid(True, alpha=0.3)
ax.set_ylim([0, 110])

# TAR vs TPR
ax = axes[0, 2]
pdd_6 = photon_pdd(depth, 6)
tar_6 = tar(depth, 6)

ax.plot(depth, pdd_6, 'b-', linewidth=2, label='PDD')
ax.plot(depth, tar_6, 'r--', linewidth=2, label='TAR')
ax.set_xlabel('Depth (cm)')
ax.set_ylabel('Relative Dose (\\%)')
ax.set_title('PDD vs TAR (6 MV)')
ax.legend()
ax.grid(True, alpha=0.3)

# Characteristic depths
ax = axes[1, 0]
d_max_vals = []
d_80_vals = []
d_50_vals = []

for E in energies_MV:
    pdd = photon_pdd(depth, E)
    d_max = 0.5 + E * 0.15
    d_80 = depth[np.argmin(np.abs(pdd - 80))]
    d_50 = depth[np.argmin(np.abs(pdd - 50))]
    d_max_vals.append(d_max)
    d_80_vals.append(d_80)
    d_50_vals.append(d_50)

x = np.arange(len(energies_MV))
width = 0.25
ax.bar(x - width, d_max_vals, width, label='$d_{max}$', alpha=0.7, color='#3498db')
ax.bar(x, d_80_vals, width, label='$d_{80}$', alpha=0.7, color='#2ecc71')
ax.bar(x + width, d_50_vals, width, label='$d_{50}$', alpha=0.7, color='#e74c3c')
ax.set_xlabel('Photon Energy (MV)')
ax.set_ylabel('Depth (cm)')
ax.set_title('Characteristic Depths')
ax.set_xticks(x)
ax.set_xticklabels([str(E) for E in energies_MV])
ax.legend()
ax.grid(True, alpha=0.3, axis='y')

# Buildup region
ax = axes[1, 1]
depth_buildup = np.linspace(0, 5, 100)
for E, color in zip([6, 10, 18], ['#3498db', '#2ecc71', '#e74c3c']):
    pdd = photon_pdd(depth_buildup, E)
    ax.plot(depth_buildup, pdd, color=color, linewidth=2, label=f'{E} MV')

ax.set_xlabel('Depth (cm)')
ax.set_ylabel('Percent Depth Dose (\\%)')
ax.set_title('Buildup Region')
ax.legend()
ax.grid(True, alpha=0.3)

# Surface dose vs energy
ax = axes[1, 2]
surface_doses = [photon_pdd(0, E)[0] if isinstance(photon_pdd(0, E), np.ndarray)
                 else photon_pdd(np.array([0]), E)[0] for E in energies_MV]
ax.bar(energies_MV, surface_doses, color='#f39c12', alpha=0.7)
ax.set_xlabel('Energy (MV)')
ax.set_ylabel('Surface Dose (\\%)')
ax.set_title('Surface Dose vs Energy')
ax.grid(True, alpha=0.3, axis='y')

plt.tight_layout()
plt.savefig('photon_dosimetry.pdf', dpi=150, bbox_inches='tight')
plt.close()
\end{pycode}

\begin{figure}[htbp]
\centering
\includegraphics[width=0.95\textwidth]{photon_dosimetry.pdf}
\caption{Photon dosimetry: (a) energy dependence, (b) field size, (c) PDD vs TAR, (d) characteristic depths, (e) buildup, (f) surface dose.}
\end{figure}

\chapter{Electron and Proton Beams}

\begin{pycode}
fig, axes = plt.subplots(2, 3, figsize=(14, 8))

# Electron PDD
ax = axes[0, 0]
energies_MeV = [6, 9, 12, 16, 20]
colors_e = plt.cm.Reds(np.linspace(0.4, 0.9, len(energies_MeV)))

for E, color in zip(energies_MeV, colors_e):
    pdd = electron_pdd(depth, E)
    ax.plot(depth, pdd, color=color, linewidth=2, label=f'{E} MeV')

ax.axhline(y=50, color='gray', linestyle='--', alpha=0.5)
ax.set_xlabel('Depth (cm)')
ax.set_ylabel('Percent Depth Dose (\\%)')
ax.set_title('Electron PDD')
ax.legend()
ax.grid(True, alpha=0.3)
ax.set_xlim([0, 15])
ax.set_ylim([0, 110])

# Proton Bragg peaks
ax = axes[0, 1]
proton_energies = [100, 150, 200, 230]
colors_p = plt.cm.Greens(np.linspace(0.4, 0.9, len(proton_energies)))

for E, color in zip(proton_energies, colors_p):
    pdd = proton_pdd(depth, E)
    ax.plot(depth, pdd, color=color, linewidth=2, label=f'{E} MeV')

ax.set_xlabel('Depth (cm)')
ax.set_ylabel('Percent Depth Dose (\\%)')
ax.set_title('Proton Bragg Peaks')
ax.legend()
ax.grid(True, alpha=0.3)
ax.set_ylim([0, 110])

# SOBP (Spread-out Bragg peak)
ax = axes[0, 2]
depth_sobp = np.linspace(0, 25, 250)
# Create SOBP from weighted sum
sobp = np.zeros_like(depth_sobp)
weights = [0.8, 0.9, 1.0, 1.1, 1.2]
energies_sobp = [180, 190, 200, 210, 220]

for w, E in zip(weights, energies_sobp):
    sobp += w * proton_pdd(depth_sobp, E)
sobp = sobp / sobp.max() * 100

pristine = proton_pdd(depth_sobp, 200)
ax.plot(depth_sobp, pristine, 'g--', linewidth=1.5, alpha=0.7, label='Pristine')
ax.plot(depth_sobp, sobp, 'b-', linewidth=2, label='SOBP')
ax.axhline(90, color='r', linestyle=':', alpha=0.5)
ax.set_xlabel('Depth (cm)')
ax.set_ylabel('Relative Dose (\\%)')
ax.set_title('Spread-Out Bragg Peak')
ax.legend()
ax.grid(True, alpha=0.3)

# Comparison at same range
ax = axes[1, 0]
# Match depths for comparison
depth_comp = np.linspace(0, 20, 200)
photon_15 = photon_pdd(depth_comp, 15)
electron_20 = electron_pdd(depth_comp, 20)
proton_170 = proton_pdd(depth_comp, 170)

ax.plot(depth_comp, photon_15, 'b-', linewidth=2, label='15 MV photon')
ax.plot(depth_comp, electron_20, 'r-', linewidth=2, label='20 MeV electron')
ax.plot(depth_comp, proton_170, 'g-', linewidth=2, label='170 MeV proton')
ax.set_xlabel('Depth (cm)')
ax.set_ylabel('Percent Depth Dose (\\%)')
ax.set_title('Beam Type Comparison')
ax.legend()
ax.grid(True, alpha=0.3)
ax.set_ylim([0, 110])

# Electron ranges
ax = axes[1, 1]
R_50 = np.array(energies_MeV) / 2.33
R_p = np.array(energies_MeV) / 2.0
R_90 = []
for E in energies_MeV:
    pdd = electron_pdd(depth, E)
    r90 = depth[np.argmin(np.abs(pdd - 90))]
    R_90.append(r90)

ax.plot(energies_MeV, R_90, 'bo-', markersize=8, label='$R_{90}$')
ax.plot(energies_MeV, R_50, 'go-', markersize=8, label='$R_{50}$')
ax.plot(energies_MeV, R_p, 'ro-', markersize=8, label='$R_p$')
ax.set_xlabel('Electron Energy (MeV)')
ax.set_ylabel('Depth (cm)')
ax.set_title('Electron Range Parameters')
ax.legend()
ax.grid(True, alpha=0.3)

# Proton range vs energy
ax = axes[1, 2]
proton_E = np.linspace(50, 250, 100)
proton_range = 0.0022 * proton_E**1.77

ax.plot(proton_E, proton_range, 'g-', linewidth=2)
ax.set_xlabel('Proton Energy (MeV)')
ax.set_ylabel('Range in Water (cm)')
ax.set_title('Proton Range')
ax.grid(True, alpha=0.3)

plt.tight_layout()
plt.savefig('particle_dosimetry.pdf', dpi=150, bbox_inches='tight')
plt.close()
\end{pycode}

\begin{figure}[htbp]
\centering
\includegraphics[width=0.95\textwidth]{particle_dosimetry.pdf}
\caption{Particle dosimetry: (a) electron PDD, (b) proton Bragg peaks, (c) SOBP, (d) beam comparison, (e) electron ranges, (f) proton range.}
\end{figure}

\chapter{Dose-Volume Analysis}

\begin{pycode}
# Simulate DVH
fig, axes = plt.subplots(1, 3, figsize=(14, 4))

# Create simulated dose distributions
np.random.seed(42)
n_voxels = 10000

# Target (PTV)
target_dose = 60 + 5 * np.random.randn(n_voxels)
target_dose = np.clip(target_dose, 50, 70)

# OAR 1 (close to target)
oar1_dose = 40 + 15 * np.random.randn(n_voxels)
oar1_dose = np.clip(oar1_dose, 0, 65)

# OAR 2 (further from target)
oar2_dose = 20 + 10 * np.random.randn(n_voxels)
oar2_dose = np.clip(oar2_dose, 0, 45)

# Cumulative DVH
ax = axes[0]
dose_bins = np.linspace(0, 75, 150)

for doses, label, color in [(target_dose, 'PTV', '#e74c3c'),
                             (oar1_dose, 'OAR 1', '#3498db'),
                             (oar2_dose, 'OAR 2', '#2ecc71')]:
    hist, _ = np.histogram(doses, bins=dose_bins)
    cum_hist = np.cumsum(hist[::-1])[::-1] / len(doses) * 100
    ax.plot(dose_bins[:-1], cum_hist, linewidth=2, label=label, color=color)

ax.axhline(95, color='gray', linestyle='--', alpha=0.5)
ax.axvline(60, color='gray', linestyle=':', alpha=0.5)
ax.set_xlabel('Dose (Gy)')
ax.set_ylabel('Volume (\\%)')
ax.set_title('Cumulative DVH')
ax.legend()
ax.grid(True, alpha=0.3)

# Differential DVH
ax = axes[1]
for doses, label, color in [(target_dose, 'PTV', '#e74c3c'),
                             (oar1_dose, 'OAR 1', '#3498db'),
                             (oar2_dose, 'OAR 2', '#2ecc71')]:
    ax.hist(doses, bins=50, alpha=0.5, label=label, color=color, density=True)

ax.set_xlabel('Dose (Gy)')
ax.set_ylabel('Frequency')
ax.set_title('Differential DVH')
ax.legend()
ax.grid(True, alpha=0.3)

# DVH metrics
ax = axes[2]
metrics = {
    'D95 PTV': np.percentile(target_dose, 5),
    'D50 PTV': np.percentile(target_dose, 50),
    'D2 PTV': np.percentile(target_dose, 98),
    'Dmax OAR1': np.max(oar1_dose),
    'Dmean OAR1': np.mean(oar1_dose),
    'Dmax OAR2': np.max(oar2_dose),
}

names = list(metrics.keys())
values = list(metrics.values())
colors_bar = ['#e74c3c']*3 + ['#3498db']*2 + ['#2ecc71']

bars = ax.barh(names, values, color=colors_bar, alpha=0.7)
ax.set_xlabel('Dose (Gy)')
ax.set_title('DVH Metrics')
ax.grid(True, alpha=0.3, axis='x')

plt.tight_layout()
plt.savefig('dose_volume.pdf', dpi=150, bbox_inches='tight')
plt.close()

# Store metrics
D95_PTV = metrics['D95 PTV']
Dmax_OAR1 = metrics['Dmax OAR1']
\end{pycode}

\begin{figure}[htbp]
\centering
\includegraphics[width=0.95\textwidth]{dose_volume.pdf}
\caption{Dose-volume analysis: (a) cumulative DVH, (b) differential DVH, (c) DVH metrics.}
\end{figure}

\chapter{Numerical Results}

\begin{pycode}
# 6 MV reference values
d_max_6 = 0.5 + 6 * 0.15
pdd_6_ref = photon_pdd(depth, 6)
d_50_6 = depth[np.argmin(np.abs(pdd_6_ref - 50))]
d_80_6 = depth[np.argmin(np.abs(pdd_6_ref - 80))]

results_table = [
    ('6 MV $d_{max}$', f'{d_max_6:.1f}', 'cm'),
    ('6 MV $d_{50}$', f'{d_50_6:.1f}', 'cm'),
    ('6 MV $d_{80}$', f'{d_80_6:.1f}', 'cm'),
    ('PTV D95', f'{D95_PTV:.1f}', 'Gy'),
    ('OAR1 $D_{max}$', f'{Dmax_OAR1:.1f}', 'Gy'),
    ('Prescribed dose', '60.0', 'Gy'),
]
\end{pycode}

\begin{table}[htbp]
\centering
\caption{Radiation dosimetry results}
\begin{tabular}{@{}lcc@{}}
\toprule
Parameter & Value & Units \\
\midrule
\py{results_table[0][0]} & \py{results_table[0][1]} & \py{results_table[0][2]} \\
\py{results_table[1][0]} & \py{results_table[1][1]} & \py{results_table[1][2]} \\
\py{results_table[2][0]} & \py{results_table[2][1]} & \py{results_table[2][2]} \\
\py{results_table[3][0]} & \py{results_table[3][1]} & \py{results_table[3][2]} \\
\py{results_table[4][0]} & \py{results_table[4][1]} & \py{results_table[4][2]} \\
\py{results_table[5][0]} & \py{results_table[5][1]} & \py{results_table[5][2]} \\
\bottomrule
\end{tabular}
\end{table}

\chapter{Conclusions}

\begin{enumerate}
    \item Higher photon energies penetrate deeper with skin sparing
    \item Electrons have sharp dose falloff at their range
    \item Protons offer superior dose conformity with Bragg peak
    \item SOBP allows tumor coverage with proton therapy
    \item DVH analysis quantifies target coverage and OAR sparing
    \item Treatment planning optimizes therapeutic ratio
\end{enumerate}

\end{document}
