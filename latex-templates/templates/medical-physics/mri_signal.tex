\documentclass[a4paper, 11pt]{report}
\usepackage[utf8]{inputenc}
\usepackage[T1]{fontenc}
\usepackage{amsmath, amssymb}
\usepackage{graphicx}
\usepackage{siunitx}
\usepackage{booktabs}
\usepackage{xcolor}
\usepackage[makestderr]{pythontex}

\definecolor{t1}{RGB}{231, 76, 60}
\definecolor{t2}{RGB}{46, 204, 113}
\definecolor{pd}{RGB}{52, 152, 219}

\title{Magnetic Resonance Imaging:\\
Signal Formation, Contrast Mechanisms, and K-Space}
\author{Department of Medical Physics\\Technical Report MP-2024-002}
\date{\today}

\begin{document}
\maketitle

\begin{abstract}
This report presents a comprehensive analysis of MRI signal formation and image reconstruction. We implement the Bloch equations, analyze T1 and T2 contrast mechanisms, demonstrate k-space encoding, compare pulse sequences, and evaluate image artifacts. All simulations use PythonTeX for reproducibility.
\end{abstract}

\tableofcontents

\chapter{Introduction}

MRI signal arises from nuclear magnetic resonance. The Bloch equations describe magnetization dynamics:
\begin{align}
\frac{dM_x}{dt} &= \gamma(M_y B_z - M_z B_y) - \frac{M_x}{T_2} \\
\frac{dM_y}{dt} &= \gamma(M_z B_x - M_x B_z) - \frac{M_y}{T_2} \\
\frac{dM_z}{dt} &= \gamma(M_x B_y - M_y B_x) - \frac{M_z - M_0}{T_1}
\end{align}

\section{Relaxation Times}
\begin{itemize}
    \item $T_1$: Spin-lattice (longitudinal) relaxation
    \item $T_2$: Spin-spin (transverse) relaxation
    \item $T_2^*$: Effective transverse relaxation (includes inhomogeneity)
\end{itemize}

\begin{pycode}
import numpy as np
import matplotlib.pyplot as plt
from scipy.fft import fft2, ifft2, fftshift
plt.rc('text', usetex=True)
plt.rc('font', family='serif')

np.random.seed(42)

# Tissue parameters at 1.5T (T1, T2, T2* in ms, rho relative)
tissues = {
    'White Matter': {'T1': 780, 'T2': 90, 'T2s': 70, 'rho': 0.70},
    'Gray Matter': {'T1': 920, 'T2': 100, 'T2s': 75, 'rho': 0.80},
    'CSF': {'T1': 4000, 'T2': 2000, 'T2s': 1500, 'rho': 1.00},
    'Fat': {'T1': 260, 'T2': 80, 'T2s': 60, 'rho': 0.90},
    'Muscle': {'T1': 870, 'T2': 50, 'T2s': 35, 'rho': 0.75},
    'Blood': {'T1': 1200, 'T2': 150, 'T2s': 100, 'rho': 0.85}
}

def T1_recovery(t, T1):
    return 1 - np.exp(-t/T1)

def T2_decay(t, T2):
    return np.exp(-t/T2)

def spin_echo_signal(TR, TE, T1, T2, rho):
    return rho * (1 - np.exp(-TR/T1)) * np.exp(-TE/T2)

def gradient_echo_signal(TR, TE, T1, T2s, rho, flip_angle):
    alpha = np.radians(flip_angle)
    E1 = np.exp(-TR/T1)
    return rho * np.sin(alpha) * (1 - E1) / (1 - np.cos(alpha)*E1) * np.exp(-TE/T2s)

def inversion_recovery(TI, T1, T2, rho, TR=5000, TE=10):
    return np.abs(rho * (1 - 2*np.exp(-TI/T1) + np.exp(-TR/T1)) * np.exp(-TE/T2))
\end{pycode}

\chapter{Relaxation Mechanisms}

\begin{pycode}
fig, axes = plt.subplots(2, 3, figsize=(14, 8))

t = np.linspace(0, 4000, 500)
colors = ['#3498db', '#2ecc71', '#e74c3c', '#f39c12', '#9b59b6', '#1abc9c']

# T1 recovery curves
ax = axes[0, 0]
for (name, params), color in zip(tissues.items(), colors):
    M_z = T1_recovery(t, params['T1'])
    ax.plot(t, M_z, color=color, linewidth=1.5, label=name)

ax.axhline(0.63, color='gray', linestyle='--', alpha=0.5)
ax.set_xlabel('Time (ms)')
ax.set_ylabel('$M_z/M_0$')
ax.set_title('T1 Recovery')
ax.legend(fontsize=7, loc='lower right')
ax.grid(True, alpha=0.3)

# T2 decay curves
ax = axes[0, 1]
t_short = np.linspace(0, 500, 500)
for (name, params), color in zip(tissues.items(), colors):
    M_xy = T2_decay(t_short, params['T2'])
    ax.plot(t_short, M_xy, color=color, linewidth=1.5, label=name)

ax.axhline(0.37, color='gray', linestyle='--', alpha=0.5)
ax.set_xlabel('Time (ms)')
ax.set_ylabel('$M_{xy}/M_0$')
ax.set_title('T2 Decay')
ax.legend(fontsize=7)
ax.grid(True, alpha=0.3)

# T1 vs T2 scatter
ax = axes[0, 2]
for (name, params), color in zip(tissues.items(), colors):
    ax.scatter(params['T1'], params['T2'], s=100, c=color, label=name)
ax.set_xlabel('T1 (ms)')
ax.set_ylabel('T2 (ms)')
ax.set_title('T1 vs T2 at 1.5T')
ax.legend(fontsize=7)
ax.grid(True, alpha=0.3)
ax.set_xscale('log')
ax.set_yscale('log')

# Pulse sequence comparison
ax = axes[1, 0]
# T1-weighted
TR_T1, TE_T1 = 500, 15
signals_T1 = {name: spin_echo_signal(TR_T1, TE_T1, p['T1'], p['T2'], p['rho'])
              for name, p in tissues.items()}

# T2-weighted
TR_T2, TE_T2 = 3000, 100
signals_T2 = {name: spin_echo_signal(TR_T2, TE_T2, p['T1'], p['T2'], p['rho'])
              for name, p in tissues.items()}

# PD-weighted
TR_PD, TE_PD = 3000, 15
signals_PD = {name: spin_echo_signal(TR_PD, TE_PD, p['T1'], p['T2'], p['rho'])
              for name, p in tissues.items()}

x = np.arange(len(tissues))
width = 0.25
tissue_names = list(tissues.keys())

bars1 = ax.bar(x - width, [signals_T1[n] for n in tissue_names], width,
               label='T1w', color='#e74c3c', alpha=0.7)
bars2 = ax.bar(x, [signals_T2[n] for n in tissue_names], width,
               label='T2w', color='#2ecc71', alpha=0.7)
bars3 = ax.bar(x + width, [signals_PD[n] for n in tissue_names], width,
               label='PDw', color='#3498db', alpha=0.7)

ax.set_xlabel('Tissue')
ax.set_ylabel('Signal (a.u.)')
ax.set_title('Contrast Comparison')
ax.set_xticks(x)
ax.set_xticklabels([n.split()[0] for n in tissue_names], fontsize=8, rotation=45)
ax.legend()
ax.grid(True, alpha=0.3, axis='y')

# Inversion recovery for T1 nulling
ax = axes[1, 1]
TI = np.linspace(0, 3000, 300)
for (name, params), color in zip(list(tissues.items())[:4], colors[:4]):
    signal = inversion_recovery(TI, params['T1'], params['T2'], params['rho'])
    ax.plot(TI, signal, color=color, linewidth=1.5, label=name)

# Mark CSF null point
T1_csf = tissues['CSF']['T1']
TI_null_csf = T1_csf * np.log(2)
ax.axvline(TI_null_csf, color='gray', linestyle='--', alpha=0.5)

ax.set_xlabel('TI (ms)')
ax.set_ylabel('Signal (a.u.)')
ax.set_title('Inversion Recovery (FLAIR)')
ax.legend(fontsize=7)
ax.grid(True, alpha=0.3)

# Flip angle optimization for GRE
ax = axes[1, 2]
flip_angles = np.linspace(1, 90, 100)
TR_gre = 50
TE_gre = 5

for (name, params), color in zip(list(tissues.items())[:4], colors[:4]):
    signals = [gradient_echo_signal(TR_gre, TE_gre, params['T1'], params['T2s'],
                                    params['rho'], alpha) for alpha in flip_angles]
    ax.plot(flip_angles, signals, color=color, linewidth=1.5, label=name)

ax.set_xlabel('Flip Angle (degrees)')
ax.set_ylabel('Signal (a.u.)')
ax.set_title(f'GRE Signal (TR={TR_gre} ms)')
ax.legend(fontsize=7)
ax.grid(True, alpha=0.3)

plt.tight_layout()
plt.savefig('mri_relaxation.pdf', dpi=150, bbox_inches='tight')
plt.close()
\end{pycode}

\begin{figure}[htbp]
\centering
\includegraphics[width=0.95\textwidth]{mri_relaxation.pdf}
\caption{MRI relaxation: (a) T1 recovery, (b) T2 decay, (c) T1-T2 relationship, (d) contrast comparison, (e) inversion recovery, (f) flip angle optimization.}
\end{figure}

\chapter{K-Space and Image Reconstruction}

\section{Spatial Encoding}
The MRI signal is collected in k-space:
\begin{equation}
S(k_x, k_y) = \iint \rho(x, y) e^{-i2\pi(k_x x + k_y y)} dx \, dy
\end{equation}

\begin{pycode}
# Create brain phantom
size = 128
phantom = np.zeros((size, size))
y, x = np.ogrid[-size//2:size//2, -size//2:size//2]

# Brain outline
mask = (x/45)**2 + (y/50)**2 <= 1
phantom[mask] = 0.8

# Ventricles
mask = ((x+10)/8)**2 + (y/20)**2 <= 1
phantom[mask] = 1.0
mask = ((x-10)/8)**2 + (y/20)**2 <= 1
phantom[mask] = 1.0

# Gray matter structures
mask = ((x+25)/10)**2 + (y/15)**2 <= 1
phantom[mask] = 0.6
mask = ((x-25)/10)**2 + (y/15)**2 <= 1
phantom[mask] = 0.6

# Lesion
mask = ((x-15)/5)**2 + ((y+20)/5)**2 <= 1
phantom[mask] = 0.9

fig, axes = plt.subplots(2, 3, figsize=(14, 8))

# K-space
kspace = fftshift(fft2(phantom))
kspace_mag = np.log(np.abs(kspace) + 1)

ax = axes[0, 0]
im = ax.imshow(phantom, cmap='gray')
ax.set_title('Object (Image Space)')
ax.set_xlabel('x')
ax.set_ylabel('y')
plt.colorbar(im, ax=ax, fraction=0.046)

ax = axes[0, 1]
im = ax.imshow(kspace_mag, cmap='gray')
ax.set_title('K-Space (Log Magnitude)')
ax.set_xlabel('$k_x$')
ax.set_ylabel('$k_y$')
plt.colorbar(im, ax=ax, fraction=0.046)

ax = axes[0, 2]
im = ax.imshow(np.angle(kspace), cmap='twilight')
ax.set_title('K-Space Phase')
ax.set_xlabel('$k_x$')
ax.set_ylabel('$k_y$')
plt.colorbar(im, ax=ax, fraction=0.046)

# Partial k-space effects
# Low frequency only (low resolution)
kspace_low = kspace.copy()
mask = np.zeros((size, size), dtype=bool)
mask[size//2-16:size//2+16, size//2-16:size//2+16] = True
kspace_low[~mask] = 0
recon_low = np.abs(ifft2(fftshift(kspace_low)))

ax = axes[1, 0]
im = ax.imshow(recon_low, cmap='gray')
ax.set_title('Low Frequency Only')
ax.set_xlabel('x')
ax.set_ylabel('y')
plt.colorbar(im, ax=ax, fraction=0.046)

# High frequency only (edges)
kspace_high = kspace.copy()
kspace_high[mask] = 0
recon_high = np.abs(ifft2(fftshift(kspace_high)))

ax = axes[1, 1]
im = ax.imshow(recon_high, cmap='gray')
ax.set_title('High Frequency Only')
ax.set_xlabel('x')
ax.set_ylabel('y')
plt.colorbar(im, ax=ax, fraction=0.046)

# Partial Fourier (half k-space)
kspace_partial = kspace.copy()
kspace_partial[:size//2, :] = 0
recon_partial = np.abs(ifft2(fftshift(kspace_partial)))

ax = axes[1, 2]
im = ax.imshow(recon_partial, cmap='gray')
ax.set_title('Partial Fourier')
ax.set_xlabel('x')
ax.set_ylabel('y')
plt.colorbar(im, ax=ax, fraction=0.046)

plt.tight_layout()
plt.savefig('mri_kspace.pdf', dpi=150, bbox_inches='tight')
plt.close()
\end{pycode}

\begin{figure}[htbp]
\centering
\includegraphics[width=0.95\textwidth]{mri_kspace.pdf}
\caption{K-space encoding: (a) image space, (b) k-space magnitude, (c) k-space phase, (d) low frequency, (e) high frequency, (f) partial Fourier.}
\end{figure}

\chapter{Image Artifacts}

\begin{pycode}
fig, axes = plt.subplots(2, 3, figsize=(14, 8))

# Gibbs ringing (truncation artifact)
kspace_trunc = kspace.copy()
kspace_trunc[:20, :] = 0
kspace_trunc[-20:, :] = 0
kspace_trunc[:, :20] = 0
kspace_trunc[:, -20:] = 0
recon_gibbs = np.abs(ifft2(fftshift(kspace_trunc)))

ax = axes[0, 0]
im = ax.imshow(recon_gibbs, cmap='gray')
ax.set_title('Gibbs Ringing')
ax.set_xlabel('x')
ax.set_ylabel('y')
plt.colorbar(im, ax=ax, fraction=0.046)

# Motion artifact
kspace_motion = kspace.copy()
for i in range(0, size, 8):
    kspace_motion[:, i] *= np.exp(1j * 0.3 * i)
recon_motion = np.abs(ifft2(fftshift(kspace_motion)))

ax = axes[0, 1]
im = ax.imshow(recon_motion, cmap='gray')
ax.set_title('Motion Artifact')
ax.set_xlabel('x')
ax.set_ylabel('y')
plt.colorbar(im, ax=ax, fraction=0.046)

# Aliasing (wrap-around)
kspace_alias = kspace[::2, ::2]  # Undersample
recon_alias = np.abs(ifft2(fftshift(kspace_alias)))

ax = axes[0, 2]
im = ax.imshow(recon_alias, cmap='gray')
ax.set_title('Aliasing (FOV/2)')
ax.set_xlabel('x')
ax.set_ylabel('y')
plt.colorbar(im, ax=ax, fraction=0.046)

# Chemical shift artifact simulation
phantom_fat = np.zeros((size, size))
mask = ((x-30)/8)**2 + (y/15)**2 <= 1
phantom_fat[mask] = 0.5

# Fat shifted by 3 pixels
phantom_shifted = np.roll(phantom_fat, 3, axis=1)
combined = phantom + phantom_shifted

ax = axes[1, 0]
im = ax.imshow(combined, cmap='gray')
ax.set_title('Chemical Shift')
ax.set_xlabel('x')
ax.set_ylabel('y')
plt.colorbar(im, ax=ax, fraction=0.046)

# Susceptibility artifact
susceptibility_map = np.zeros((size, size))
mask = ((x+20)/5)**2 + ((y-20)/5)**2 <= 1
susceptibility_map[mask] = 1
susceptibility_blur = ndimage.gaussian_filter(susceptibility_map, 5)
recon_suscept = phantom * (1 - 0.5*susceptibility_blur)

from scipy import ndimage

ax = axes[1, 1]
im = ax.imshow(recon_suscept, cmap='gray')
ax.set_title('Susceptibility Artifact')
ax.set_xlabel('x')
ax.set_ylabel('y')
plt.colorbar(im, ax=ax, fraction=0.046)

# SNR comparison
ax = axes[1, 2]
noise_levels = [0, 0.05, 0.1, 0.2]
center_profile = phantom[size//2, :]

for noise in noise_levels:
    if noise == 0:
        profile = center_profile
        label = 'Original'
    else:
        noisy_kspace = kspace + noise * np.max(np.abs(kspace)) * \
                       (np.random.randn(*kspace.shape) + 1j*np.random.randn(*kspace.shape))
        recon_noisy = np.abs(ifft2(fftshift(noisy_kspace)))
        profile = recon_noisy[size//2, :]
        label = f'SNR $\\approx$ {int(1/noise)}'
    ax.plot(profile, label=label, alpha=0.7)

ax.set_xlabel('Position')
ax.set_ylabel('Signal')
ax.set_title('Noise Effects')
ax.legend(fontsize=8)
ax.grid(True, alpha=0.3)

plt.tight_layout()
plt.savefig('mri_artifacts.pdf', dpi=150, bbox_inches='tight')
plt.close()

# Calculate contrast values
wm_gm_T1 = signals_T1['White Matter'] - signals_T1['Gray Matter']
wm_gm_T2 = signals_T2['White Matter'] - signals_T2['Gray Matter']
csf_gm_T2 = signals_T2['CSF'] - signals_T2['Gray Matter']
\end{pycode}

\begin{figure}[htbp]
\centering
\includegraphics[width=0.95\textwidth]{mri_artifacts.pdf}
\caption{MRI artifacts: (a) Gibbs ringing, (b) motion, (c) aliasing, (d) chemical shift, (e) susceptibility, (f) noise effects.}
\end{figure}

\chapter{Numerical Results}

\begin{pycode}
results_table = [
    ('T1-weighted parameters', f'TR={TR_T1}, TE={TE_T1}', 'ms'),
    ('T2-weighted parameters', f'TR={TR_T2}, TE={TE_T2}', 'ms'),
    ('PD-weighted parameters', f'TR={TR_PD}, TE={TE_PD}', 'ms'),
    ('WM-GM contrast (T1w)', f'{wm_gm_T1:.3f}', 'a.u.'),
    ('WM-GM contrast (T2w)', f'{wm_gm_T2:.3f}', 'a.u.'),
    ('CSF-GM contrast (T2w)', f'{csf_gm_T2:.3f}', 'a.u.'),
]
\end{pycode}

\begin{table}[htbp]
\centering
\caption{MRI signal parameters and contrast}
\begin{tabular}{@{}lcc@{}}
\toprule
Parameter & Value & Units \\
\midrule
\py{results_table[0][0]} & \py{results_table[0][1]} & \py{results_table[0][2]} \\
\py{results_table[1][0]} & \py{results_table[1][1]} & \py{results_table[1][2]} \\
\py{results_table[2][0]} & \py{results_table[2][1]} & \py{results_table[2][2]} \\
\py{results_table[3][0]} & \py{results_table[3][1]} & \py{results_table[3][2]} \\
\py{results_table[4][0]} & \py{results_table[4][1]} & \py{results_table[4][2]} \\
\py{results_table[5][0]} & \py{results_table[5][1]} & \py{results_table[5][2]} \\
\bottomrule
\end{tabular}
\end{table}

\chapter{Conclusions}

\begin{enumerate}
    \item T1-weighted imaging provides anatomical detail
    \item T2-weighted imaging highlights pathology
    \item K-space center determines contrast, periphery determines resolution
    \item Motion during acquisition causes ghosting artifacts
    \item Chemical shift causes fat-water misregistration
    \item Higher field strength increases SNR but also artifacts
\end{enumerate}

\end{document}
