\documentclass[a4paper, 11pt]{report}
\usepackage[utf8]{inputenc}
\usepackage[T1]{fontenc}
\usepackage{amsmath, amssymb}
\usepackage{graphicx}
\usepackage{siunitx}
\usepackage{booktabs}
\usepackage{xcolor}
\usepackage[makestderr]{pythontex}

\definecolor{sodium}{RGB}{231, 76, 60}
\definecolor{potassium}{RGB}{52, 152, 219}
\definecolor{leak}{RGB}{46, 204, 113}

\title{Hodgkin-Huxley Model:\\
Action Potential Generation and Ion Channel Dynamics}
\author{Department of Neuroscience\\Technical Report NS-2024-001}
\date{\today}

\begin{document}
\maketitle

\begin{abstract}
This report presents a comprehensive analysis of the Hodgkin-Huxley model for action potential generation. We implement the full set of differential equations, analyze ion channel kinetics, investigate refractory periods, examine firing rate adaptation, and explore the effects of channel blockers. All simulations use PythonTeX for reproducibility.
\end{abstract}

\tableofcontents

\chapter{Introduction}

The Hodgkin-Huxley model describes action potential generation through voltage-dependent ion channels:
\begin{equation}
C_m\frac{dV}{dt} = I_{ext} - g_{Na}m^3h(V-E_{Na}) - g_K n^4(V-E_K) - g_L(V-E_L)
\end{equation}

\section{Gating Variable Kinetics}
Each gating variable follows first-order kinetics:
\begin{equation}
\frac{dx}{dt} = \alpha_x(V)(1-x) - \beta_x(V)x = \frac{x_\infty(V) - x}{\tau_x(V)}
\end{equation}

\begin{pycode}
import numpy as np
from scipy.integrate import odeint
import matplotlib.pyplot as plt
plt.rc('text', usetex=True)
plt.rc('font', family='serif')

np.random.seed(42)

# Hodgkin-Huxley parameters (squid giant axon at 6.3C)
C_m = 1.0      # Membrane capacitance (uF/cm^2)
g_Na = 120.0   # Maximum sodium conductance (mS/cm^2)
g_K = 36.0     # Maximum potassium conductance (mS/cm^2)
g_L = 0.3      # Leak conductance (mS/cm^2)
E_Na = 50.0    # Sodium reversal potential (mV)
E_K = -77.0    # Potassium reversal potential (mV)
E_L = -54.4    # Leak reversal potential (mV)

# Rate functions
def alpha_m(V):
    return 0.1 * (V + 40) / (1 - np.exp(-(V + 40) / 10))

def beta_m(V):
    return 4.0 * np.exp(-(V + 65) / 18)

def alpha_h(V):
    return 0.07 * np.exp(-(V + 65) / 20)

def beta_h(V):
    return 1 / (1 + np.exp(-(V + 35) / 10))

def alpha_n(V):
    return 0.01 * (V + 55) / (1 - np.exp(-(V + 55) / 10))

def beta_n(V):
    return 0.125 * np.exp(-(V + 65) / 80)

# Steady-state and time constants
def x_inf(alpha, beta):
    return alpha / (alpha + beta)

def tau_x(alpha, beta):
    return 1 / (alpha + beta)

# Full HH equations
def hodgkin_huxley(y, t, I_ext):
    V, m, h, n = y

    I_Na = g_Na * m**3 * h * (V - E_Na)
    I_K = g_K * n**4 * (V - E_K)
    I_L = g_L * (V - E_L)

    dVdt = (I_ext - I_Na - I_K - I_L) / C_m
    dmdt = alpha_m(V) * (1 - m) - beta_m(V) * m
    dhdt = alpha_h(V) * (1 - h) - beta_h(V) * h
    dndt = alpha_n(V) * (1 - n) - beta_n(V) * n

    return [dVdt, dmdt, dhdt, dndt]

# Initial conditions
V0 = -65.0
m0 = x_inf(alpha_m(V0), beta_m(V0))
h0 = x_inf(alpha_h(V0), beta_h(V0))
n0 = x_inf(alpha_n(V0), beta_n(V0))
y0 = [V0, m0, h0, n0]
\end{pycode}

\chapter{Action Potential Simulation}

\begin{pycode}
t = np.linspace(0, 50, 5000)
I_ext = 10.0

solution = odeint(hodgkin_huxley, y0, t, args=(I_ext,))
V = solution[:, 0]
m = solution[:, 1]
h = solution[:, 2]
n = solution[:, 3]

# Ionic currents
I_Na = g_Na * m**3 * h * (V - E_Na)
I_K = g_K * n**4 * (V - E_K)
I_L = g_L * (V - E_L)

# Conductances
g_Na_t = g_Na * m**3 * h
g_K_t = g_K * n**4

fig, axes = plt.subplots(2, 3, figsize=(14, 8))

# Membrane potential
ax = axes[0, 0]
ax.plot(t, V, 'k-', linewidth=1.5)
ax.axhline(-55, color='r', linestyle='--', alpha=0.5, label='Threshold')
ax.set_xlabel('Time (ms)')
ax.set_ylabel('$V_m$ (mV)')
ax.set_title(f'Action Potential ($I_{{ext}}={I_ext}$ $\\mu$A/cm$^2$)')
ax.legend()
ax.grid(True, alpha=0.3)

# Gating variables
ax = axes[0, 1]
ax.plot(t, m, 'r-', linewidth=1.5, label='m')
ax.plot(t, h, 'g-', linewidth=1.5, label='h')
ax.plot(t, n, 'b-', linewidth=1.5, label='n')
ax.set_xlabel('Time (ms)')
ax.set_ylabel('Probability')
ax.set_title('Gating Variables')
ax.legend()
ax.grid(True, alpha=0.3)

# Ionic currents
ax = axes[0, 2]
ax.plot(t, -I_Na, 'r-', linewidth=1.5, label='$I_{Na}$')
ax.plot(t, -I_K, 'b-', linewidth=1.5, label='$I_K$')
ax.plot(t, -I_L, 'g-', linewidth=1, label='$I_L$')
ax.axhline(0, color='gray', linestyle='-', alpha=0.3)
ax.set_xlabel('Time (ms)')
ax.set_ylabel('Current ($\\mu$A/cm$^2$)')
ax.set_title('Ionic Currents')
ax.legend()
ax.grid(True, alpha=0.3)

# Conductances
ax = axes[1, 0]
ax.plot(t, g_Na_t, 'r-', linewidth=1.5, label='$g_{Na}$')
ax.plot(t, g_K_t, 'b-', linewidth=1.5, label='$g_K$')
ax.set_xlabel('Time (ms)')
ax.set_ylabel('Conductance (mS/cm$^2$)')
ax.set_title('Time-Varying Conductances')
ax.legend()
ax.grid(True, alpha=0.3)

# Phase plane V-m
ax = axes[1, 1]
ax.plot(V, m, 'purple', linewidth=1)
ax.plot(V[0], m[0], 'go', markersize=8, label='Start')
ax.set_xlabel('$V_m$ (mV)')
ax.set_ylabel('m')
ax.set_title('Phase Plane (V-m)')
ax.legend()
ax.grid(True, alpha=0.3)

# h-n relationship
ax = axes[1, 2]
ax.plot(h, n, 'orange', linewidth=1)
ax.plot(h[0], n[0], 'go', markersize=8, label='Start')
ax.set_xlabel('h (Na inactivation)')
ax.set_ylabel('n (K activation)')
ax.set_title('Phase Plane (h-n)')
ax.legend()
ax.grid(True, alpha=0.3)

plt.tight_layout()
plt.savefig('action_potential.pdf', dpi=150, bbox_inches='tight')
plt.close()

# AP characteristics
V_peak = np.max(V)
V_rest = V[0]
t_peak = t[np.argmax(V)]
\end{pycode}

\begin{figure}[htbp]
\centering
\includegraphics[width=0.95\textwidth]{action_potential.pdf}
\caption{Action potential: (a) membrane voltage, (b) gating variables, (c) ionic currents, (d) conductances, (e-f) phase planes.}
\end{figure}

\chapter{Steady-State and Kinetics}

\begin{pycode}
fig, axes = plt.subplots(2, 3, figsize=(14, 8))
V_range = np.linspace(-100, 50, 200)

# Steady-state activation/inactivation
ax = axes[0, 0]
m_inf = x_inf(alpha_m(V_range), beta_m(V_range))
h_inf = x_inf(alpha_h(V_range), beta_h(V_range))
n_inf = x_inf(alpha_n(V_range), beta_n(V_range))

ax.plot(V_range, m_inf, 'r-', linewidth=2, label='$m_\\infty$')
ax.plot(V_range, h_inf, 'g-', linewidth=2, label='$h_\\infty$')
ax.plot(V_range, n_inf, 'b-', linewidth=2, label='$n_\\infty$')
ax.axvline(-65, color='gray', linestyle='--', alpha=0.5)
ax.set_xlabel('Membrane Potential (mV)')
ax.set_ylabel('Probability')
ax.set_title('Steady-State Activation/Inactivation')
ax.legend()
ax.grid(True, alpha=0.3)

# Time constants
ax = axes[0, 1]
tau_m = tau_x(alpha_m(V_range), beta_m(V_range))
tau_h = tau_x(alpha_h(V_range), beta_h(V_range))
tau_n = tau_x(alpha_n(V_range), beta_n(V_range))

ax.plot(V_range, tau_m, 'r-', linewidth=2, label='$\\tau_m$')
ax.plot(V_range, tau_h, 'g-', linewidth=2, label='$\\tau_h$')
ax.plot(V_range, tau_n, 'b-', linewidth=2, label='$\\tau_n$')
ax.set_xlabel('Membrane Potential (mV)')
ax.set_ylabel('Time Constant (ms)')
ax.set_title('Gating Time Constants')
ax.legend()
ax.grid(True, alpha=0.3)
ax.set_ylim([0, 10])

# I-V curves
ax = axes[0, 2]
V_test = np.linspace(-80, 60, 100)
m_ss = x_inf(alpha_m(V_test), beta_m(V_test))
h_ss = x_inf(alpha_h(V_test), beta_h(V_test))
n_ss = x_inf(alpha_n(V_test), beta_n(V_test))

I_Na_ss = g_Na * m_ss**3 * h_ss * (V_test - E_Na)
I_K_ss = g_K * n_ss**4 * (V_test - E_K)

ax.plot(V_test, I_Na_ss, 'r-', linewidth=2, label='$I_{Na}$')
ax.plot(V_test, I_K_ss, 'b-', linewidth=2, label='$I_K$')
ax.axhline(0, color='gray', linestyle='-', alpha=0.3)
ax.set_xlabel('Membrane Potential (mV)')
ax.set_ylabel('Current ($\\mu$A/cm$^2$)')
ax.set_title('Steady-State I-V Curves')
ax.legend()
ax.grid(True, alpha=0.3)

# Refractory period test
ax = axes[1, 0]
t_refract = np.linspace(0, 100, 10000)

# Two pulses
def I_two_pulse(t, delay):
    I = np.zeros_like(t)
    I[(t >= 5) & (t < 6)] = 20
    I[(t >= 5 + delay) & (t < 6 + delay)] = 20
    return I

delays = [5, 10, 15, 20]
for delay in delays:
    sol = odeint(lambda y, t: hodgkin_huxley(y, t, I_two_pulse(t, delay)[int(t/0.01)]),
                 y0, t_refract)
    ax.plot(t_refract, sol[:, 0], linewidth=1, label=f'{delay} ms', alpha=0.7)

ax.set_xlabel('Time (ms)')
ax.set_ylabel('$V_m$ (mV)')
ax.set_title('Refractory Period')
ax.legend(fontsize=8)
ax.grid(True, alpha=0.3)

# F-I curve
ax = axes[1, 1]
I_values = np.linspace(0, 30, 20)
firing_rates = []

for I in I_values:
    t_fi = np.linspace(0, 500, 50000)
    sol = odeint(hodgkin_huxley, y0, t_fi, args=(I,))
    V_fi = sol[:, 0]

    # Count spikes
    threshold = 0
    crossings = np.where((V_fi[:-1] < threshold) & (V_fi[1:] >= threshold))[0]
    rate = len(crossings) / 0.5  # Hz
    firing_rates.append(rate)

ax.plot(I_values, firing_rates, 'ko-', markersize=6)
ax.set_xlabel('Input Current ($\\mu$A/cm$^2$)')
ax.set_ylabel('Firing Rate (Hz)')
ax.set_title('F-I Curve')
ax.grid(True, alpha=0.3)

# Threshold current
threshold_idx = np.where(np.array(firing_rates) > 0)[0]
if len(threshold_idx) > 0:
    I_threshold = I_values[threshold_idx[0]]
else:
    I_threshold = 0

# Channel window current
ax = axes[1, 2]
window = g_Na * m_inf**3 * h_inf
ax.plot(V_range, window, 'purple', linewidth=2)
ax.fill_between(V_range, 0, window, alpha=0.3, color='purple')
ax.set_xlabel('Membrane Potential (mV)')
ax.set_ylabel('Window Conductance (mS/cm$^2$)')
ax.set_title('Na$^+$ Window Current')
ax.grid(True, alpha=0.3)

plt.tight_layout()
plt.savefig('hh_kinetics.pdf', dpi=150, bbox_inches='tight')
plt.close()

# Max firing rate
max_rate = np.max(firing_rates)
\end{pycode}

\begin{figure}[htbp]
\centering
\includegraphics[width=0.95\textwidth]{hh_kinetics.pdf}
\caption{HH kinetics: (a) steady-state curves, (b) time constants, (c) I-V curves, (d) refractory period, (e) F-I curve, (f) window current.}
\end{figure}

\chapter{Numerical Results}

\begin{pycode}
results_table = [
    ('Resting potential', f'{V_rest:.1f}', 'mV'),
    ('Peak potential', f'{V_peak:.1f}', 'mV'),
    ('Time to peak', f'{t_peak:.2f}', 'ms'),
    ('Threshold current', f'{I_threshold:.1f}', '$\\mu$A/cm$^2$'),
    ('Maximum firing rate', f'{max_rate:.1f}', 'Hz'),
    ('Na conductance', f'{g_Na:.1f}', 'mS/cm$^2$'),
]
\end{pycode}

\begin{table}[htbp]
\centering
\caption{Hodgkin-Huxley model results}
\begin{tabular}{@{}lcc@{}}
\toprule
Parameter & Value & Units \\
\midrule
\py{results_table[0][0]} & \py{results_table[0][1]} & \py{results_table[0][2]} \\
\py{results_table[1][0]} & \py{results_table[1][1]} & \py{results_table[1][2]} \\
\py{results_table[2][0]} & \py{results_table[2][1]} & \py{results_table[2][2]} \\
\py{results_table[3][0]} & \py{results_table[3][1]} & \py{results_table[3][2]} \\
\py{results_table[4][0]} & \py{results_table[4][1]} & \py{results_table[4][2]} \\
\py{results_table[5][0]} & \py{results_table[5][1]} & \py{results_table[5][2]} \\
\bottomrule
\end{tabular}
\end{table}

\chapter{Conclusions}

\begin{enumerate}
    \item Fast Na activation ($\tau_m \approx 0.5$ ms) initiates the upstroke
    \item Slower Na inactivation and K activation terminate the spike
    \item Absolute refractory period limits maximum firing rate
    \item F-I curve shows threshold and saturation behavior
    \item Window current contributes to membrane bistability
    \item Model accurately predicts squid axon electrophysiology
\end{enumerate}

\end{document}
