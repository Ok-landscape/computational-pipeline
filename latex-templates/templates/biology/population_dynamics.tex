% Population Dynamics Template
% Topics: Lotka-Volterra predator-prey, stability analysis, limit cycles
% Style: Tutorial with analytical and numerical approaches

\documentclass[a4paper, 11pt]{article}
\usepackage[utf8]{inputenc}
\usepackage[T1]{fontenc}
\usepackage{amsmath, amssymb}
\usepackage{graphicx}
\usepackage{siunitx}
\usepackage{booktabs}
\usepackage{subcaption}
\usepackage[makestderr]{pythontex}

% Theorem environments
\newtheorem{definition}{Definition}[section]
\newtheorem{theorem}{Theorem}[section]
\newtheorem{example}{Example}[section]
\newtheorem{remark}{Remark}[section]

\title{Population Dynamics: Predator-Prey Models and Stability Analysis}
\author{Mathematical Biology Tutorial}
\date{\today}

\begin{document}
\maketitle

\begin{abstract}
This tutorial presents a comprehensive analysis of predator-prey population dynamics
using the Lotka-Volterra equations and their extensions. We examine the classical model,
perform stability analysis of equilibrium points, and explore modifications including
carrying capacity and functional responses. Computational analysis demonstrates phase
portraits, limit cycles, and the ecological implications of parameter variations.
\end{abstract}

\section{Introduction}

The Lotka-Volterra predator-prey model is a cornerstone of mathematical ecology,
describing the coupled dynamics of two interacting species. Despite its simplicity,
the model captures essential features of predator-prey oscillations observed in nature.

\begin{definition}[Lotka-Volterra Equations]
The classical predator-prey model is:
\begin{align}
\frac{dx}{dt} &= \alpha x - \beta xy \\
\frac{dy}{dt} &= \delta xy - \gamma y
\end{align}
where $x$ is prey density, $y$ is predator density, $\alpha$ is prey growth rate,
$\beta$ is predation rate, $\delta$ is predator growth efficiency, and $\gamma$ is
predator death rate.
\end{definition}

\section{Theoretical Framework}

\subsection{Equilibrium Analysis}

\begin{theorem}[Equilibrium Points]
The Lotka-Volterra system has two equilibrium points:
\begin{enumerate}
\item Trivial equilibrium: $(x^*, y^*) = (0, 0)$
\item Coexistence equilibrium: $(x^*, y^*) = (\gamma/\delta, \alpha/\beta)$
\end{enumerate}
\end{theorem}

\begin{theorem}[Stability of Coexistence Equilibrium]
The Jacobian matrix at $(x^*, y^*)$ is:
\begin{equation}
J = \begin{pmatrix}
0 & -\beta x^* \\
\delta y^* & 0
\end{pmatrix}
\end{equation}
The eigenvalues are $\lambda = \pm i\sqrt{\alpha\gamma}$, indicating a center (neutral
stability). Trajectories form closed orbits around the equilibrium.
\end{theorem}

\subsection{Extensions}

\begin{definition}[Logistic Prey Growth]
Adding carrying capacity to prey growth:
\begin{align}
\frac{dx}{dt} &= \alpha x \left(1 - \frac{x}{K}\right) - \beta xy \\
\frac{dy}{dt} &= \delta xy - \gamma y
\end{align}
This creates a globally stable equilibrium (spiral sink) for appropriate parameters.
\end{definition}

\begin{definition}[Holling Type II Functional Response]
Predator satiation is modeled by:
\begin{equation}
\frac{dx}{dt} = \alpha x - \frac{\beta xy}{1 + \beta h x}
\end{equation}
where $h$ is handling time per prey item.
\end{definition}

\begin{remark}[Functional Responses]
Holling classified functional responses as:
\begin{itemize}
\item \textbf{Type I}: Linear (constant attack rate)
\item \textbf{Type II}: Saturating (handling time limitation)
\item \textbf{Type III}: Sigmoidal (learning or switching behavior)
\end{itemize}
\end{remark}

\section{Computational Analysis}

\begin{pycode}
import numpy as np
import matplotlib.pyplot as plt
from scipy.integrate import odeint
from scipy.linalg import eig

np.random.seed(42)

# Classic Lotka-Volterra
def lotka_volterra(y, t, alpha, beta, delta, gamma):
    x, p = y
    dxdt = alpha * x - beta * x * p
    dpdt = delta * x * p - gamma * p
    return [dxdt, dpdt]

# With logistic prey growth
def lv_logistic(y, t, alpha, beta, delta, gamma, K):
    x, p = y
    dxdt = alpha * x * (1 - x/K) - beta * x * p
    dpdt = delta * x * p - gamma * p
    return [dxdt, dpdt]

# With Type II functional response
def lv_type2(y, t, alpha, beta, delta, gamma, h):
    x, p = y
    functional = (beta * x) / (1 + beta * h * x)
    dxdt = alpha * x - functional * p
    dpdt = delta * functional * p - gamma * p
    return [dxdt, dpdt]

# Rosenzweig-MacArthur model
def rosenzweig_macarthur(y, t, r, K, a, h, e, d):
    N, P = y
    dNdt = r * N * (1 - N/K) - (a * N * P) / (1 + a * h * N)
    dPdt = e * (a * N * P) / (1 + a * h * N) - d * P
    return [dNdt, dPdt]

# Parameters
alpha = 1.0
beta = 0.1
delta = 0.075
gamma = 1.5
K = 100

# Initial conditions
x0, y0 = 10, 5
initial = [x0, y0]

# Time array
t = np.linspace(0, 100, 2000)

# Solve classic model
solution = odeint(lotka_volterra, initial, t, args=(alpha, beta, delta, gamma))
prey = solution[:, 0]
predators = solution[:, 1]

# Equilibrium points
x_eq = gamma / delta
y_eq = alpha / beta

# Solve logistic model
sol_logistic = odeint(lv_logistic, initial, t, args=(alpha, beta, delta, gamma, K))

# Solve Type II model
h = 0.5
sol_type2 = odeint(lv_type2, initial, t, args=(alpha, beta, delta, gamma, h))

# Rosenzweig-MacArthur with limit cycle
sol_rm = odeint(rosenzweig_macarthur, [10, 2], t, args=(1.0, 50, 0.1, 0.5, 0.5, 0.3))

# Multiple initial conditions for phase portrait
ics = [[5, 2], [10, 5], [15, 3], [8, 8], [20, 6]]
trajectories = []
for ic in ics:
    sol = odeint(lotka_volterra, ic, t, args=(alpha, beta, delta, gamma))
    trajectories.append(sol)

# Vector field for phase portrait
x_range = np.linspace(0.1, 40, 20)
y_range = np.linspace(0.1, 20, 20)
X, Y = np.meshgrid(x_range, y_range)
U = alpha * X - beta * X * Y
V = delta * X * Y - gamma * Y
speed = np.sqrt(U**2 + V**2)
U_norm = U / speed
V_norm = V / speed

# Isoclines
x_iso = np.linspace(0.1, 40, 200)
prey_iso = alpha / beta * np.ones_like(x_iso)
pred_iso = gamma / delta * np.ones_like(x_iso)

# Period estimation
prey_peaks = np.where((prey[1:-1] > prey[:-2]) & (prey[1:-1] > prey[2:]))[0] + 1
if len(prey_peaks) >= 2:
    period = np.mean(np.diff(t[prey_peaks]))
else:
    period = 0

# Jacobian analysis
J_eq = np.array([[0, -beta * x_eq],
                 [delta * y_eq, 0]])
eigenvalues, eigenvectors = eig(J_eq)

# Conservation quantity
H = delta * prey - gamma * np.log(prey) + beta * predators - alpha * np.log(predators)
H0 = H[0]

# Create figure
fig = plt.figure(figsize=(14, 12))

# Plot 1: Time series
ax1 = fig.add_subplot(3, 3, 1)
ax1.plot(t, prey, 'b-', linewidth=1.5, label='Prey')
ax1.plot(t, predators, 'r-', linewidth=1.5, label='Predators')
ax1.set_xlabel('Time')
ax1.set_ylabel('Population')
ax1.set_title('Classic Lotka-Volterra')
ax1.legend(fontsize=8)

# Plot 2: Phase portrait
ax2 = fig.add_subplot(3, 3, 2)
ax2.quiver(X, Y, U_norm, V_norm, speed, cmap='viridis', alpha=0.6)
for traj in trajectories:
    ax2.plot(traj[:, 0], traj[:, 1], linewidth=1)
ax2.plot(x_eq, y_eq, 'r*', markersize=12, label='Equilibrium')
ax2.axhline(y=y_eq, color='green', linestyle='--', alpha=0.5)
ax2.axvline(x=x_eq, color='blue', linestyle='--', alpha=0.5)
ax2.set_xlabel('Prey')
ax2.set_ylabel('Predator')
ax2.set_title('Phase Portrait')
ax2.set_xlim([0, 40])
ax2.set_ylim([0, 20])

# Plot 3: Logistic prey growth
ax3 = fig.add_subplot(3, 3, 3)
ax3.plot(t, sol_logistic[:, 0], 'b-', linewidth=1.5, label='Prey')
ax3.plot(t, sol_logistic[:, 1], 'r-', linewidth=1.5, label='Predators')
ax3.set_xlabel('Time')
ax3.set_ylabel('Population')
ax3.set_title('With Logistic Prey Growth')
ax3.legend(fontsize=8)

# Plot 4: Logistic phase space
ax4 = fig.add_subplot(3, 3, 4)
ax4.plot(sol_logistic[:, 0], sol_logistic[:, 1], 'g-', linewidth=1)
ax4.plot(sol_logistic[0, 0], sol_logistic[0, 1], 'ko', markersize=8)
ax4.plot(sol_logistic[-1, 0], sol_logistic[-1, 1], 'r*', markersize=12)
ax4.set_xlabel('Prey')
ax4.set_ylabel('Predator')
ax4.set_title('Logistic Model Phase Space')

# Plot 5: Type II functional response
ax5 = fig.add_subplot(3, 3, 5)
ax5.plot(t, sol_type2[:, 0], 'b-', linewidth=1.5, label='Prey')
ax5.plot(t, sol_type2[:, 1], 'r-', linewidth=1.5, label='Predators')
ax5.set_xlabel('Time')
ax5.set_ylabel('Population')
ax5.set_title('Type II Functional Response')
ax5.legend(fontsize=8)

# Plot 6: Rosenzweig-MacArthur limit cycle
ax6 = fig.add_subplot(3, 3, 6)
ax6.plot(sol_rm[500:, 0], sol_rm[500:, 1], 'purple', linewidth=1.5)
ax6.set_xlabel('Prey')
ax6.set_ylabel('Predator')
ax6.set_title('Rosenzweig-MacArthur Limit Cycle')

# Plot 7: Conserved quantity
ax7 = fig.add_subplot(3, 3, 7)
ax7.plot(t, H, 'b-', linewidth=1.5)
ax7.axhline(y=H0, color='red', linestyle='--', alpha=0.7)
ax7.set_xlabel('Time')
ax7.set_ylabel('$H(x, y)$')
ax7.set_title('Conserved Quantity')
ax7.set_ylim([H0 - 1, H0 + 1])

# Plot 8: Functional responses
ax8 = fig.add_subplot(3, 3, 8)
N_range = np.linspace(0, 50, 200)
type1 = beta * N_range
type2 = (beta * N_range) / (1 + beta * h * N_range)
type3 = (beta * N_range**2) / (1 + beta * h * N_range**2)
ax8.plot(N_range, type1, 'b-', linewidth=2, label='Type I')
ax8.plot(N_range, type2, 'g-', linewidth=2, label='Type II')
ax8.plot(N_range, type3, 'r-', linewidth=2, label='Type III')
ax8.set_xlabel('Prey density $N$')
ax8.set_ylabel('Predation rate')
ax8.set_title('Functional Responses')
ax8.legend(fontsize=8)

# Plot 9: Parameter sensitivity
ax9 = fig.add_subplot(3, 3, 9)
alphas = [0.8, 1.0, 1.2]
colors = ['blue', 'green', 'red']
for a, c in zip(alphas, colors):
    sol = odeint(lotka_volterra, initial, t, args=(a, beta, delta, gamma))
    ax9.plot(t[:500], sol[:500, 0], color=c, linewidth=1.5, label=f'$\\alpha = {a}$')
ax9.set_xlabel('Time')
ax9.set_ylabel('Prey population')
ax9.set_title('Prey Growth Rate Sensitivity')
ax9.legend(fontsize=8)

plt.tight_layout()
plt.savefig('population_dynamics_analysis.pdf', dpi=150, bbox_inches='tight')
plt.close()
\end{pycode}

\begin{figure}[htbp]
\centering
\includegraphics[width=\textwidth]{population_dynamics_analysis.pdf}
\caption{Predator-prey dynamics analysis: (a) Classic Lotka-Volterra time series;
(b) Phase portrait with multiple trajectories and isoclines; (c-d) Logistic prey growth
model showing damped oscillations; (e) Type II functional response dynamics; (f)
Rosenzweig-MacArthur limit cycle; (g) Conserved quantity verification; (h) Comparison
of functional responses; (i) Sensitivity to prey growth rate.}
\label{fig:predprey}
\end{figure}

\section{Results}

\subsection{Model Parameters}

\begin{pycode}
print(r"\begin{table}[htbp]")
print(r"\centering")
print(r"\caption{Lotka-Volterra Model Parameters and Results}")
print(r"\begin{tabular}{lcc}")
print(r"\toprule")
print(r"Parameter & Symbol & Value \\")
print(r"\midrule")
print(f"Prey growth rate & $\\alpha$ & {alpha} \\\\")
print(f"Predation rate & $\\beta$ & {beta} \\\\")
print(f"Predator efficiency & $\\delta$ & {delta} \\\\")
print(f"Predator death rate & $\\gamma$ & {gamma} \\\\")
print(r"\midrule")
print(f"Equilibrium prey & $x^*$ & {x_eq:.2f} \\\\")
print(f"Equilibrium predator & $y^*$ & {y_eq:.2f} \\\\")
print(f"Oscillation period & $T$ & {period:.2f} \\\\")
print(r"\bottomrule")
print(r"\end{tabular}")
print(r"\label{tab:parameters}")
print(r"\end{table}")
\end{pycode}

\subsection{Stability Analysis}

\begin{pycode}
print(r"\begin{table}[htbp]")
print(r"\centering")
print(r"\caption{Stability Analysis at Coexistence Equilibrium}")
print(r"\begin{tabular}{lc}")
print(r"\toprule")
print(r"Property & Value \\")
print(r"\midrule")
print(f"Eigenvalue 1 & ${eigenvalues[0]:.4f}$ \\\\")
print(f"Eigenvalue 2 & ${eigenvalues[1]:.4f}$ \\\\")
print(f"Angular frequency & $\\omega = {np.abs(eigenvalues[0].imag):.4f}$ \\\\")
print(f"Theoretical period & $T = 2\\pi/\\omega = {2*np.pi/np.abs(eigenvalues[0].imag):.2f}$ \\\\")
print(r"Stability type & Center (neutral) \\")
print(r"\bottomrule")
print(r"\end{tabular}")
print(r"\label{tab:stability}")
print(r"\end{table}")
\end{pycode}

\section{Discussion}

\begin{example}[Conservation Law]
The classical Lotka-Volterra system conserves the quantity:
\begin{equation}
H(x, y) = \delta x - \gamma \ln x + \beta y - \alpha \ln y
\end{equation}
This integral of motion ensures closed orbits in phase space.
\end{example}

\begin{remark}[Structural Instability]
The classical model is structurally unstable: any perturbation to the equations
(such as adding density dependence) changes the qualitative behavior from neutral
cycles to either damped oscillations (stable spiral) or limit cycles.
\end{remark}

\begin{example}[Paradox of Enrichment]
In the Rosenzweig-MacArthur model, increasing carrying capacity $K$ can destabilize
the equilibrium. Beyond a critical $K$, the system transitions from a stable spiral
to a limit cycle via a Hopf bifurcation. This counterintuitive result suggests that
enriching an ecosystem can lead to larger oscillations and potential extinction.
\end{example}

\begin{remark}[Ecological Implications]
Predator-prey oscillations explain phenomena such as:
\begin{itemize}
\item Lynx-hare cycles in Canadian fur trade records
\item Phytoplankton-zooplankton dynamics in lakes
\item Host-parasitoid population cycles
\item Microbial predator-prey systems
\end{itemize}
The oscillation period depends on the geometric mean of growth and death rates.
\end{remark}

\section{Conclusions}

This analysis demonstrates the rich dynamics of predator-prey systems:
\begin{enumerate}
\item Classic Lotka-Volterra produces neutral oscillations with period $T \approx \py{f"{period:.1f}"}$
\item Equilibrium at $(x^*, y^*) = (\py{f"{x_eq:.1f}"}, \py{f"{y_eq:.1f}"}$) with eigenvalues $\lambda = \pm i\py{f"{np.abs(eigenvalues[0].imag):.3f}"}$
\item Adding logistic prey growth creates damped oscillations (stable spiral)
\item Type II functional response can generate limit cycles
\item The conserved quantity $H$ ensures orbit closure in the classic model
\end{enumerate}

\section*{Further Reading}

\begin{itemize}
\item Murray, J.D. \textit{Mathematical Biology I: An Introduction}, 3rd ed. Springer, 2002.
\item Edelstein-Keshet, L. \textit{Mathematical Models in Biology}. SIAM, 2005.
\item Kot, M. \textit{Elements of Mathematical Ecology}. Cambridge, 2001.
\end{itemize}

\end{document}
