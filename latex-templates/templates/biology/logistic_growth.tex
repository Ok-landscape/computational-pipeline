% Logistic Growth Model Template
% Topics: Carrying capacity, Allee effect, competition, harvesting
% Style: Research article with ecological applications

\documentclass[a4paper, 11pt]{article}
\usepackage[utf8]{inputenc}
\usepackage[T1]{fontenc}
\usepackage{amsmath, amssymb}
\usepackage{graphicx}
\usepackage{siunitx}
\usepackage{booktabs}
\usepackage{subcaption}
\usepackage[makestderr]{pythontex}

% Theorem environments
\newtheorem{definition}{Definition}[section]
\newtheorem{theorem}{Theorem}[section]
\newtheorem{example}{Example}[section]
\newtheorem{remark}{Remark}[section]

\title{Logistic Growth Models: Density Dependence and Population Regulation}
\author{Quantitative Ecology Research}
\date{\today}

\begin{document}
\maketitle

\begin{abstract}
This study presents a comprehensive analysis of logistic population growth models and
their extensions. We examine the classic logistic equation, the Allee effect (positive
density dependence at low populations), interspecific competition, and sustainable
harvesting strategies. Computational analysis demonstrates population dynamics under
various parameter regimes and identifies optimal management strategies for harvested
populations.
\end{abstract}

\section{Introduction}

The logistic growth model represents a fundamental advance over exponential growth by
incorporating density-dependent regulation through carrying capacity. Extensions of this
model address important ecological phenomena including Allee effects and species competition.

\begin{definition}[Logistic Growth]
The logistic growth equation describes population dynamics with density-dependent regulation:
\begin{equation}
\frac{dN}{dt} = rN\left(1 - \frac{N}{K}\right)
\end{equation}
where $r$ is the intrinsic growth rate and $K$ is the carrying capacity.
\end{definition}

\section{Theoretical Framework}

\subsection{Classic Logistic Model}

\begin{theorem}[Logistic Solution]
The analytical solution of the logistic equation is:
\begin{equation}
N(t) = \frac{K}{1 + \left(\frac{K - N_0}{N_0}\right)e^{-rt}}
\end{equation}
The population approaches $K$ asymptotically with an inflection point at $N = K/2$.
\end{theorem}

\begin{remark}[Maximum Sustainable Yield]
The growth rate $dN/dt$ is maximized when $N = K/2$, yielding the maximum sustainable
yield (MSY): $\text{MSY} = rK/4$.
\end{remark}

\subsection{Allee Effect}

\begin{definition}[Allee Effect]
The Allee effect describes reduced per capita growth rate at low population densities due
to difficulties in mate finding, reduced group defense, or inbreeding. A strong Allee
effect creates a critical population threshold $A$ below which extinction occurs:
\begin{equation}
\frac{dN}{dt} = rN\left(1 - \frac{N}{K}\right)\left(\frac{N}{A} - 1\right)
\end{equation}
\end{definition}

\subsection{Competition Models}

\begin{theorem}[Lotka-Volterra Competition]
For two competing species:
\begin{align}
\frac{dN_1}{dt} &= r_1 N_1 \left(1 - \frac{N_1 + \alpha_{12} N_2}{K_1}\right) \\
\frac{dN_2}{dt} &= r_2 N_2 \left(1 - \frac{N_2 + \alpha_{21} N_1}{K_2}\right)
\end{align}
where $\alpha_{ij}$ is the competition coefficient (effect of species $j$ on species $i$).
\end{theorem}

\subsection{Harvesting}

\begin{definition}[Harvesting Strategies]
Common harvesting models include:
\begin{itemize}
\item \textbf{Constant harvest}: $dN/dt = rN(1 - N/K) - H$
\item \textbf{Proportional harvest}: $dN/dt = rN(1 - N/K) - qEN$
\item \textbf{Threshold harvest}: Harvest only when $N > N_{threshold}$
\end{itemize}
where $H$ is harvest rate, $q$ is catchability, and $E$ is effort.
\end{definition}

\section{Computational Analysis}

\begin{pycode}
import numpy as np
import matplotlib.pyplot as plt
from scipy.integrate import odeint
from scipy.optimize import fsolve

np.random.seed(42)

# Model definitions
def logistic(N, t, r, K):
    return r * N * (1 - N/K)

def allee(N, t, r, K, A):
    if N <= 0:
        return 0
    return r * N * (1 - N/K) * (N/A - 1)

def logistic_harvest(N, t, r, K, H):
    return r * N * (1 - N/K) - H

def competition(y, t, r1, K1, r2, K2, alpha12, alpha21):
    N1, N2 = y
    dN1 = r1 * N1 * (1 - (N1 + alpha12*N2)/K1)
    dN2 = r2 * N2 * (1 - (N2 + alpha21*N1)/K2)
    return [dN1, dN2]

# Parameters
r = 0.5
K = 1000
N0 = 50
A = 100  # Allee threshold

# Time arrays
t_short = np.linspace(0, 30, 500)
t_long = np.linspace(0, 50, 500)

# Basic logistic
N_logistic = odeint(logistic, N0, t_long, args=(r, K))[:, 0]

# Exponential comparison
N_exp = N0 * np.exp(r * t_long)

# Different growth rates
r_values = [0.3, 0.5, 0.8, 1.0]
N_r_vary = {}
for r_val in r_values:
    N_r_vary[r_val] = odeint(logistic, N0, t_long, args=(r_val, K))[:, 0]

# Allee effect
N_allee_above = odeint(allee, 150, t_long, args=(r, K, A))[:, 0]
N_allee_below = odeint(allee, 80, t_long, args=(r, K, A))[:, 0]
N_allee_threshold = odeint(allee, A, t_long, args=(r, K, A))[:, 0]

# Harvesting analysis
H_values = [0, 30, 60, 90, 120]
N_harvest = {}
for H in H_values:
    N_harvest[H] = odeint(logistic_harvest, 800, t_long, args=(r, K, H))[:, 0]

# MSY calculation
MSY = r * K / 4
N_MSY = K / 2

# Competition outcomes
# Coexistence
N_coex = odeint(competition, [100, 100], t_long, args=(0.5, 1000, 0.5, 1000, 0.5, 0.5))
# Competitive exclusion
N_excl = odeint(competition, [100, 100], t_long, args=(0.5, 1000, 0.4, 800, 1.2, 1.5))

# Growth rate vs population
N_range = np.linspace(0, K, 200)
dN_dt_logistic = r * N_range * (1 - N_range/K)
dN_dt_allee = r * N_range * (1 - N_range/K) * (N_range/A - 1)

# Per capita growth rate
per_capita_logistic = r * (1 - N_range/K)
per_capita_allee = np.where(N_range > 0, r * (1 - N_range/K) * (N_range/A - 1), 0)

# Create figure
fig = plt.figure(figsize=(14, 12))

# Plot 1: Logistic vs exponential
ax1 = fig.add_subplot(3, 3, 1)
ax1.plot(t_long, N_logistic, 'b-', linewidth=2, label='Logistic')
ax1.plot(t_long, np.clip(N_exp, 0, 2500), 'r--', linewidth=2, label='Exponential')
ax1.axhline(y=K, color='gray', linestyle='--', alpha=0.7, label=f'K = {K}')
ax1.set_xlabel('Time')
ax1.set_ylabel('Population $N$')
ax1.set_title('Logistic vs Exponential Growth')
ax1.legend(fontsize=8)
ax1.set_ylim([0, 1500])

# Plot 2: Growth rate vs population
ax2 = fig.add_subplot(3, 3, 2)
ax2.plot(N_range, dN_dt_logistic, 'b-', linewidth=2, label='Logistic')
ax2.plot(N_range, dN_dt_allee, 'r-', linewidth=2, label='Allee')
ax2.axvline(x=K/2, color='blue', linestyle=':', alpha=0.7)
ax2.axvline(x=A, color='red', linestyle=':', alpha=0.7)
ax2.axhline(y=0, color='black', linewidth=0.5)
ax2.set_xlabel('Population $N$')
ax2.set_ylabel('$dN/dt$')
ax2.set_title('Population Growth Rate')
ax2.legend(fontsize=8)

# Plot 3: Effect of growth rate
ax3 = fig.add_subplot(3, 3, 3)
colors = plt.cm.viridis(np.linspace(0, 0.8, len(r_values)))
for i, r_val in enumerate(r_values):
    ax3.plot(t_long, N_r_vary[r_val], color=colors[i], linewidth=2, label=f'r = {r_val}')
ax3.axhline(y=K, color='gray', linestyle='--', alpha=0.7)
ax3.set_xlabel('Time')
ax3.set_ylabel('Population $N$')
ax3.set_title('Effect of Growth Rate')
ax3.legend(fontsize=8)

# Plot 4: Allee effect
ax4 = fig.add_subplot(3, 3, 4)
ax4.plot(t_long, N_allee_above, 'g-', linewidth=2, label=f'$N_0 = 150 > A$')
ax4.plot(t_long, N_allee_below, 'r-', linewidth=2, label=f'$N_0 = 80 < A$')
ax4.plot(t_long, N_allee_threshold, 'orange', linewidth=2, label=f'$N_0 = A$')
ax4.axhline(y=K, color='gray', linestyle='--', alpha=0.7)
ax4.axhline(y=A, color='black', linestyle=':', alpha=0.7)
ax4.set_xlabel('Time')
ax4.set_ylabel('Population $N$')
ax4.set_title(f'Allee Effect (A = {A})')
ax4.legend(fontsize=8)
ax4.set_ylim([0, 1200])

# Plot 5: Harvesting
ax5 = fig.add_subplot(3, 3, 5)
colors_h = plt.cm.plasma(np.linspace(0, 0.8, len(H_values)))
for i, H in enumerate(H_values):
    ax5.plot(t_long, N_harvest[H], color=colors_h[i], linewidth=2, label=f'H = {H}')
ax5.axhline(y=0, color='black', linewidth=0.5)
ax5.set_xlabel('Time')
ax5.set_ylabel('Population $N$')
ax5.set_title('Constant Harvest Rate')
ax5.legend(fontsize=8, loc='upper right')
ax5.set_ylim([0, 1000])

# Plot 6: Yield curve
ax6 = fig.add_subplot(3, 3, 6)
N_eq = np.linspace(0, K, 200)
yield_curve = r * N_eq * (1 - N_eq/K)
ax6.plot(N_eq, yield_curve, 'b-', linewidth=2)
ax6.axvline(x=N_MSY, color='red', linestyle='--', alpha=0.7)
ax6.axhline(y=MSY, color='red', linestyle='--', alpha=0.7)
ax6.scatter([N_MSY], [MSY], s=100, c='red', zorder=5, label=f'MSY = {MSY:.0f}')
ax6.set_xlabel('Equilibrium population $N^*$')
ax6.set_ylabel('Sustainable yield')
ax6.set_title('Maximum Sustainable Yield')
ax6.legend(fontsize=8)

# Plot 7: Competition - coexistence
ax7 = fig.add_subplot(3, 3, 7)
ax7.plot(t_long, N_coex[:, 0], 'b-', linewidth=2, label='Species 1')
ax7.plot(t_long, N_coex[:, 1], 'r-', linewidth=2, label='Species 2')
ax7.set_xlabel('Time')
ax7.set_ylabel('Population')
ax7.set_title('Competition: Coexistence')
ax7.legend(fontsize=8)

# Plot 8: Competition - exclusion
ax8 = fig.add_subplot(3, 3, 8)
ax8.plot(t_long, N_excl[:, 0], 'b-', linewidth=2, label='Species 1')
ax8.plot(t_long, N_excl[:, 1], 'r-', linewidth=2, label='Species 2')
ax8.set_xlabel('Time')
ax8.set_ylabel('Population')
ax8.set_title('Competition: Exclusion')
ax8.legend(fontsize=8)

# Plot 9: Per capita growth rate
ax9 = fig.add_subplot(3, 3, 9)
ax9.plot(N_range, per_capita_logistic, 'b-', linewidth=2, label='Logistic')
ax9.plot(N_range, per_capita_allee, 'r-', linewidth=2, label='Allee')
ax9.axhline(y=0, color='black', linewidth=0.5)
ax9.axvline(x=A, color='red', linestyle=':', alpha=0.7)
ax9.set_xlabel('Population $N$')
ax9.set_ylabel('Per capita growth rate')
ax9.set_title('Density Dependence')
ax9.legend(fontsize=8)

plt.tight_layout()
plt.savefig('logistic_growth_analysis.pdf', dpi=150, bbox_inches='tight')
plt.close()

# Calculate key values
t_inflection = np.log((K - N0) / N0) / r
doubling_time = np.log(2) / r
\end{pycode}

\begin{figure}[htbp]
\centering
\includegraphics[width=\textwidth]{logistic_growth_analysis.pdf}
\caption{Logistic growth analysis: (a) Comparison with exponential growth; (b) Population
growth rate showing inflection points; (c) Effect of intrinsic growth rate; (d) Allee
effect dynamics; (e) Constant harvest scenarios; (f) Maximum sustainable yield curve;
(g-h) Competition outcomes; (i) Per capita growth rate showing density dependence.}
\label{fig:logistic}
\end{figure}

\section{Results}

\subsection{Model Parameters}

\begin{pycode}
print(r"\begin{table}[htbp]")
print(r"\centering")
print(r"\caption{Logistic Growth Model Parameters and Key Values}")
print(r"\begin{tabular}{lcc}")
print(r"\toprule")
print(r"Parameter & Symbol & Value \\")
print(r"\midrule")
print(f"Intrinsic growth rate & $r$ & {r} \\\\")
print(f"Carrying capacity & $K$ & {K} \\\\")
print(f"Initial population & $N_0$ & {N0} \\\\")
print(f"Allee threshold & $A$ & {A} \\\\")
print(r"\midrule")
print(f"MSY population & $N_{{MSY}}$ & {N_MSY:.0f} \\\\")
print(f"Maximum sustainable yield & MSY & {MSY:.1f} \\\\")
print(f"Time to inflection & $t_{{infl}}$ & {t_inflection:.2f} \\\\")
print(f"Doubling time & $t_d$ & {doubling_time:.2f} \\\\")
print(r"\bottomrule")
print(r"\end{tabular}")
print(r"\label{tab:parameters}")
print(r"\end{table}")
\end{pycode}

\subsection{Harvesting Outcomes}

\begin{pycode}
print(r"\begin{table}[htbp]")
print(r"\centering")
print(r"\caption{Equilibrium Populations Under Different Harvest Rates}")
print(r"\begin{tabular}{ccc}")
print(r"\toprule")
print(r"Harvest rate $H$ & Equilibrium $N^*$ & Sustainable? \\")
print(r"\midrule")

for H in H_values:
    N_final = N_harvest[H][-1]
    if N_final > 1:
        sustainable = "Yes"
    else:
        sustainable = "No (collapse)"
    print(f"{H} & {N_final:.0f} & {sustainable} \\\\")

print(r"\bottomrule")
print(r"\end{tabular}")
print(r"\label{tab:harvest}")
print(r"\end{table}")
\end{pycode}

\section{Discussion}

\begin{example}[Maximum Sustainable Yield]
For fisheries management, the MSY occurs at $N = K/2$:
\begin{equation}
\text{MSY} = r \cdot \frac{K}{2} \cdot \left(1 - \frac{K/2}{K}\right) = \frac{rK}{4} = \py{f"{MSY:.1f}"}
\end{equation}
Harvesting at rates exceeding MSY leads to population collapse.
\end{example}

\begin{remark}[Allee Effect and Conservation]
The Allee effect has critical implications for conservation:
\begin{itemize}
\item Small populations face extinction risk even without external threats
\item Minimum viable population size must exceed the Allee threshold
\item Reintroduction programs must establish populations above this threshold
\item Habitat fragmentation increases Allee effect risks
\end{itemize}
\end{remark}

\begin{example}[Competition Outcomes]
The outcome of Lotka-Volterra competition depends on $\alpha$ values:
\begin{itemize}
\item Coexistence: $\alpha_{12} < K_1/K_2$ and $\alpha_{21} < K_2/K_1$
\item Species 1 wins: $\alpha_{12} < K_1/K_2$ and $\alpha_{21} > K_2/K_1$
\item Species 2 wins: $\alpha_{12} > K_1/K_2$ and $\alpha_{21} < K_2/K_1$
\item Unstable: Both inequalities reversed
\end{itemize}
\end{example}

\section{Conclusions}

This analysis demonstrates key aspects of logistic population dynamics:
\begin{enumerate}
\item The population approaches carrying capacity $K = \py{f"{K}"}$ with inflection at $K/2$
\item MSY of $\py{f"{MSY:.1f}"}$ occurs when harvesting maintains population at $K/2$
\item The Allee effect creates an extinction threshold at $A = \py{f"{A}"}$
\item Competition outcomes depend on relative competition coefficients
\item Per capita growth rate decreases linearly with population in the logistic model
\end{enumerate}

\section*{Further Reading}

\begin{itemize}
\item Gotelli, N.J. \textit{A Primer of Ecology}, 4th ed. Sinauer Associates, 2008.
\item Courchamp, F. et al. \textit{Allee Effects in Ecology and Conservation}. Oxford, 2008.
\item Clark, C.W. \textit{Mathematical Bioeconomics}, 3rd ed. Wiley-Interscience, 2010.
\end{itemize}

\end{document}
