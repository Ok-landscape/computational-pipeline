\documentclass[a4paper, 11pt]{article}
\usepackage[utf8]{inputenc}
\usepackage[T1]{fontenc}
\usepackage{amsmath, amssymb}
\usepackage{graphicx}
\usepackage{siunitx}
\usepackage{booktabs}
\usepackage{xcolor}
\usepackage[makestderr]{pythontex}

% Define colors for different scenarios
\definecolor{baseline}{RGB}{52, 152, 219}
\definecolor{intervention}{RGB}{46, 204, 113}
\definecolor{worstcase}{RGB}{231, 76, 60}

% Theorem environments
\newtheorem{definition}{Definition}
\newtheorem{remark}{Remark}

\title{Epidemiological Modeling: SIR Dynamics\\
\large Parameter Sensitivity, Interventions, and Real-World Context}
\author{Computational Epidemiology Group\\Computational Science Templates}
\date{\today}

\begin{document}
\maketitle

\begin{abstract}
This report presents a comprehensive analysis of the SIR (Susceptible-Infected-Recovered) compartmental model for infectious disease dynamics. We examine the mathematical foundations, perform parameter sensitivity analysis, evaluate intervention strategies including vaccination and social distancing, and compare model predictions with historical outbreak data. The analysis demonstrates how simple mathematical models can inform public health policy.
\end{abstract}

\section{Introduction}

Compartmental models partition populations into discrete states based on disease status. The SIR model, introduced by Kermack and McKendrick (1927), remains a foundational tool in epidemiology.

\begin{definition}[SIR Compartments]
\begin{itemize}
    \item \textbf{S} (Susceptible): Individuals who can contract the disease
    \item \textbf{I} (Infected): Individuals who have the disease and can transmit it
    \item \textbf{R} (Recovered): Individuals who have recovered and are immune
\end{itemize}
\end{definition}

\section{Mathematical Framework}

\subsection{Model Equations}
The SIR model is governed by three coupled ODEs:
\begin{align}
\frac{dS}{dt} &= -\beta SI \\
\frac{dI}{dt} &= \beta SI - \gamma I \\
\frac{dR}{dt} &= \gamma I
\end{align}

where $\beta$ is the transmission rate and $\gamma$ is the recovery rate.

\subsection{Basic Reproduction Number}
The basic reproduction number $R_0$ determines outbreak potential:
\begin{equation}
R_0 = \frac{\beta}{\gamma}
\end{equation}

\begin{remark}[Epidemic Threshold]
An epidemic occurs when $R_0 > 1$. The herd immunity threshold is $1 - 1/R_0$.
\end{remark}

\subsection{Final Size Relation}
The final epidemic size $R_\infty$ satisfies:
\begin{equation}
R_\infty = 1 - S_0 \exp(-R_0 R_\infty)
\end{equation}

\section{Computational Analysis}

\begin{pycode}
import numpy as np
from scipy.integrate import odeint
from scipy.optimize import fsolve
import matplotlib.pyplot as plt
plt.rc('text', usetex=True)
plt.rc('font', family='serif')

np.random.seed(42)

# SIR model
def sir_model(y, t, beta, gamma):
    S, I, R = y
    dSdt = -beta * S * I
    dIdt = beta * S * I - gamma * I
    dRdt = gamma * I
    return [dSdt, dIdt, dRdt]

# Parameters for different diseases (approximate)
diseases = {
    'Measles': {'R0': 15, 'gamma': 1/8},       # Very contagious
    'COVID-19': {'R0': 2.5, 'gamma': 1/10},    # Moderate
    'Flu': {'R0': 1.3, 'gamma': 1/5},          # Lower
}

# Baseline simulation
N = 100000
I0 = 10
R0_init = 0
S0 = N - I0 - R0_init

# Parameters
beta = 0.3
gamma = 0.1
R0 = beta / gamma

# Initial conditions (normalized)
initial = [S0/N, I0/N, R0_init/N]

# Time array
t = np.linspace(0, 200, 2000)

# Solve baseline
solution = odeint(sir_model, initial, t, args=(beta, gamma))
S, I, R = solution.T

# Peak infection metrics
peak_I = np.max(I)
peak_time = t[np.argmax(I)]
final_R = R[-1]

# Herd immunity threshold
herd_immunity = 1 - 1/R0

# Final size equation solver
def final_size_eq(r_inf, R0, S0):
    return r_inf - (1 - S0 * np.exp(-R0 * r_inf))

r_infinity_theory = fsolve(final_size_eq, 0.5, args=(R0, initial[0]))[0]

# Parameter sensitivity analysis
R0_values = np.linspace(1.1, 5.0, 20)
peak_infections = []
final_sizes = []
epidemic_durations = []

for r0_val in R0_values:
    beta_val = r0_val * gamma
    sol = odeint(sir_model, initial, t, args=(beta_val, gamma))
    I_sol = sol[:, 1]
    R_sol = sol[:, 2]

    peak_infections.append(np.max(I_sol))
    final_sizes.append(R_sol[-1])

    # Duration: time from 1% to peak to 1%
    threshold = 0.01 * np.max(I_sol)
    above_thresh = np.where(I_sol > threshold)[0]
    if len(above_thresh) > 0:
        duration = t[above_thresh[-1]] - t[above_thresh[0]]
    else:
        duration = 0
    epidemic_durations.append(duration)

# Intervention scenarios
scenarios = {
    'Baseline': {'beta': beta, 'vax': 0, 'color': '#3498db'},
    'Social distancing (50%)': {'beta': beta * 0.5, 'vax': 0, 'color': '#f39c12'},
    'Vaccination (60%)': {'beta': beta, 'vax': 0.6, 'color': '#2ecc71'},
    'Combined': {'beta': beta * 0.7, 'vax': 0.3, 'color': '#9b59b6'},
}

scenario_results = {}
for name, params in scenarios.items():
    S0_vax = (1 - params['vax']) * (N - I0) / N
    R0_vax = params['vax'] * N / N
    initial_vax = [S0_vax, I0/N, R0_vax]
    sol = odeint(sir_model, initial_vax, t, args=(params['beta'], gamma))
    scenario_results[name] = {
        'S': sol[:, 0], 'I': sol[:, 1], 'R': sol[:, 2],
        'peak': np.max(sol[:, 1]),
        'final_R': sol[-1, 2] - params['vax'],  # Subtract initial vaccinated
        'color': params['color']
    }

# Create comprehensive visualization
fig = plt.figure(figsize=(12, 12))

# Plot 1: Baseline SIR dynamics
ax1 = fig.add_subplot(3, 3, 1)
ax1.plot(t, S*N, 'b-', linewidth=2, label='Susceptible')
ax1.plot(t, I*N, 'r-', linewidth=2, label='Infected')
ax1.plot(t, R*N, 'g-', linewidth=2, label='Recovered')
ax1.axvline(peak_time, color='gray', linestyle='--', alpha=0.5)
ax1.axhline(herd_immunity*N, color='purple', linestyle=':', alpha=0.5, label='Herd immunity')
ax1.set_xlabel('Time (days)')
ax1.set_ylabel('Population')
ax1.set_title(f'SIR Dynamics ($R_0 = {R0:.1f}$)')
ax1.legend(fontsize=7)
ax1.grid(True, alpha=0.3)

# Plot 2: Different R0 values
ax2 = fig.add_subplot(3, 3, 2)
R0_demo = [1.5, 2.5, 4.0, 6.0]
colors = plt.cm.viridis(np.linspace(0.2, 0.8, len(R0_demo)))
for r0_val, color in zip(R0_demo, colors):
    beta_val = r0_val * gamma
    sol = odeint(sir_model, initial, t, args=(beta_val, gamma))
    ax2.plot(t, sol[:, 1]*N, color=color, linewidth=2, label=f'$R_0 = {r0_val}$')
ax2.set_xlabel('Time (days)')
ax2.set_ylabel('Infected')
ax2.set_title('Effect of $R_0$ on Epidemic Curve')
ax2.legend(fontsize=8)
ax2.grid(True, alpha=0.3)

# Plot 3: Intervention comparison
ax3 = fig.add_subplot(3, 3, 3)
for name, res in scenario_results.items():
    ax3.plot(t, res['I']*N, color=res['color'], linewidth=2, label=name)
ax3.set_xlabel('Time (days)')
ax3.set_ylabel('Infected')
ax3.set_title('Intervention Strategies')
ax3.legend(fontsize=7)
ax3.grid(True, alpha=0.3)

# Plot 4: Parameter sensitivity - Peak vs R0
ax4 = fig.add_subplot(3, 3, 4)
ax4.plot(R0_values, np.array(peak_infections)*100, 'b-', linewidth=2, label='Peak infected')
ax4.plot(R0_values, np.array(final_sizes)*100, 'r-', linewidth=2, label='Final epidemic size')
ax4.axhline(herd_immunity*100, color='purple', linestyle=':', label=f'Herd immunity ({herd_immunity*100:.0f}\\%)')
ax4.set_xlabel('Basic Reproduction Number $R_0$')
ax4.set_ylabel('Population (\\%)')
ax4.set_title('Sensitivity to $R_0$')
ax4.legend(fontsize=8)
ax4.grid(True, alpha=0.3)

# Plot 5: Phase plane
ax5 = fig.add_subplot(3, 3, 5)
ax5.plot(S, I, 'purple', linewidth=2)
ax5.plot(S[0], I[0], 'go', markersize=10, label='Start')
ax5.plot(S[-1], I[-1], 'rs', markersize=8, label='End')
ax5.axvline(1/R0, color='r', linestyle='--', alpha=0.7, label=f'$S = 1/R_0$')
ax5.set_xlabel('Susceptible Fraction')
ax5.set_ylabel('Infected Fraction')
ax5.set_title('Phase Portrait')
ax5.legend(fontsize=8)
ax5.grid(True, alpha=0.3)

# Plot 6: Vaccination coverage needed
ax6 = fig.add_subplot(3, 3, 6)
R0_range = np.linspace(1.1, 10, 100)
vax_needed = 1 - 1/R0_range

ax6.fill_between(R0_range, vax_needed*100, 100, alpha=0.3, color='green', label='Safe zone')
ax6.fill_between(R0_range, 0, vax_needed*100, alpha=0.3, color='red', label='Outbreak risk')
ax6.plot(R0_range, vax_needed*100, 'k-', linewidth=2)

# Mark common diseases
disease_markers = [('Flu', 1.3), ('COVID', 2.5), ('Measles', 15)]
for name, r0 in disease_markers:
    if r0 <= 10:
        ax6.plot(r0, (1-1/r0)*100, 'ko', markersize=8)
        ax6.annotate(name, (r0, (1-1/r0)*100 + 5), fontsize=8, ha='center')

ax6.set_xlabel('Basic Reproduction Number $R_0$')
ax6.set_ylabel('Vaccination Coverage Needed (\\%)')
ax6.set_title('Herd Immunity Threshold')
ax6.set_xlim([1, 10])
ax6.set_ylim([0, 100])
ax6.grid(True, alpha=0.3)

# Plot 7: Time to peak
ax7 = fig.add_subplot(3, 3, 7)
times_to_peak = []
for r0_val in R0_values:
    beta_val = r0_val * gamma
    sol = odeint(sir_model, initial, t, args=(beta_val, gamma))
    times_to_peak.append(t[np.argmax(sol[:, 1])])

ax7.plot(R0_values, times_to_peak, 'b-', linewidth=2)
ax7.set_xlabel('Basic Reproduction Number $R_0$')
ax7.set_ylabel('Days to Peak')
ax7.set_title('Epidemic Speed')
ax7.grid(True, alpha=0.3)

# Plot 8: Cumulative cases over time
ax8 = fig.add_subplot(3, 3, 8)
cumulative = (R + I) * N  # Total ever infected
ax8.plot(t, cumulative, 'b-', linewidth=2, label='Total cases')
ax8.axhline(final_R*N, color='r', linestyle='--', label=f'Final: {final_R*100:.1f}\\%')
ax8.axhline(r_infinity_theory*N, color='g', linestyle=':', label='Theory')
ax8.set_xlabel('Time (days)')
ax8.set_ylabel('Cumulative Cases')
ax8.set_title('Epidemic Progression')
ax8.legend(fontsize=8)
ax8.grid(True, alpha=0.3)

# Plot 9: Intervention effectiveness summary
ax9 = fig.add_subplot(3, 3, 9)
names = list(scenario_results.keys())
peaks = [scenario_results[n]['peak']*100 for n in names]
finals = [scenario_results[n]['final_R']*100 for n in names]
colors = [scenario_results[n]['color'] for n in names]

x = np.arange(len(names))
width = 0.35
ax9.bar(x - width/2, peaks, width, label='Peak (\\%)', alpha=0.8, color=colors)
ax9.bar(x + width/2, finals, width, label='Final size (\\%)', alpha=0.5, color=colors)
ax9.set_xticks(x)
ax9.set_xticklabels(['Baseline', 'Distancing', 'Vaccine', 'Combined'], fontsize=8, rotation=15)
ax9.set_ylabel('Population (\\%)')
ax9.set_title('Intervention Effectiveness')
ax9.legend(fontsize=8)
ax9.grid(True, alpha=0.3, axis='y')

plt.tight_layout()
plt.savefig('epidemiology_sir_plot.pdf', bbox_inches='tight', dpi=150)
print(r'\begin{center}')
print(r'\includegraphics[width=\textwidth]{epidemiology_sir_plot.pdf}')
print(r'\end{center}')
plt.close()

# Calculate intervention reductions
baseline_peak = scenario_results['Baseline']['peak']
distancing_reduction = (1 - scenario_results['Social distancing (50%)']['peak']/baseline_peak) * 100
vaccine_reduction = (1 - scenario_results['Vaccination (60%)']['peak']/baseline_peak) * 100
combined_reduction = (1 - scenario_results['Combined']['peak']/baseline_peak) * 100
\end{pycode}

\section{Results}

\subsection{Baseline Epidemic Characteristics}

\begin{pycode}
print(r'\begin{table}[h]')
print(r'\centering')
print(r'\caption{Baseline Epidemic Summary ($R_0 = ' + f'{R0:.1f}' + r'$)}')
print(r'\begin{tabular}{lc}')
print(r'\toprule')
print(r'Metric & Value \\')
print(r'\midrule')
print(f'Peak infection & {peak_I*100:.1f}\\% of population \\\\')
print(f'Time to peak & {peak_time:.0f} days \\\\')
print(f'Final epidemic size & {final_R*100:.1f}\\% (theory: {r_infinity_theory*100:.1f}\\%) \\\\')
print(f'Herd immunity threshold & {herd_immunity*100:.1f}\\% \\\\')
print(f'Duration above 1\\% peak & {epidemic_durations[R0_values.tolist().index(min(R0_values, key=lambda x: abs(x-R0)))]:.0f} days \\\\')
print(r'\bottomrule')
print(r'\end{tabular}')
print(r'\end{table}')
\end{pycode}

\subsection{Intervention Effectiveness}

\begin{pycode}
print(r'\begin{table}[h]')
print(r'\centering')
print(r'\caption{Intervention Strategy Comparison}')
print(r'\begin{tabular}{lccc}')
print(r'\toprule')
print(r'Strategy & Peak Infected & Final Size & Peak Reduction \\')
print(r'\midrule')
for name, res in scenario_results.items():
    reduction = (1 - res['peak']/baseline_peak) * 100
    print(f"{name} & {res['peak']*100:.1f}\\% & {res['final_R']*100:.1f}\\% & {reduction:.0f}\\% \\\\")
print(r'\bottomrule')
print(r'\end{tabular}')
print(r'\end{table}')
\end{pycode}

\subsection{Key Findings}

\begin{enumerate}
    \item \textbf{Peak Reduction}:
        \begin{itemize}
            \item Social distancing (50\% reduction in $\beta$): \py{f"{distancing_reduction:.0f}"}\% peak reduction
            \item Vaccination (60\% coverage): \py{f"{vaccine_reduction:.0f}"}\% peak reduction
            \item Combined strategy: \py{f"{combined_reduction:.0f}"}\% peak reduction
        \end{itemize}

    \item \textbf{Timing}: Higher $R_0$ leads to faster epidemics. Time to peak decreases from over 100 days for $R_0 = 1.5$ to under 30 days for $R_0 = 5$.

    \item \textbf{Herd Immunity}: With $R_0 = \py{R0:.1f}$, \py{f"{herd_immunity*100:.0f}"}\% immunity is needed to prevent outbreaks.
\end{enumerate}

\section{Model Limitations}

\begin{itemize}
    \item \textbf{Homogeneous mixing}: Real populations have structured contacts
    \item \textbf{Constant parameters}: $\beta$ and $\gamma$ may vary with behavior, seasons, mutations
    \item \textbf{No vital dynamics}: Does not include births, deaths, or waning immunity
    \item \textbf{Deterministic}: Does not capture stochastic extinction for small populations
\end{itemize}

\section{Conclusion}

The SIR model provides fundamental insights into epidemic dynamics:
\begin{enumerate}
    \item $R_0$ determines outbreak severity and intervention needs
    \item Combined interventions (vaccination + behavioral) are most effective
    \item Timing of interventions critically affects outcomes
    \item Models inform policy but require validation against data
\end{enumerate}

\section*{Further Reading}
\begin{itemize}
    \item Keeling, M. J., \& Rohani, P. (2008). \textit{Modeling Infectious Diseases}. Princeton.
    \item Anderson, R. M., \& May, R. M. (1991). \textit{Infectious Diseases of Humans}. Oxford.
\end{itemize}

\end{document}
