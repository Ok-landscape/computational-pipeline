% Evolutionary Dynamics Template
% Topics: Fitness landscapes, natural selection, genetic drift, mutation-selection balance
% Style: Textbook chapter with theoretical foundations

\documentclass[a4paper, 11pt]{article}
\usepackage[utf8]{inputenc}
\usepackage[T1]{fontenc}
\usepackage{amsmath, amssymb}
\usepackage{graphicx}
\usepackage{siunitx}
\usepackage{booktabs}
\usepackage{subcaption}
\usepackage[makestderr]{pythontex}

% Theorem environments
\newtheorem{definition}{Definition}[section]
\newtheorem{theorem}{Theorem}[section]
\newtheorem{example}{Example}[section]
\newtheorem{remark}{Remark}[section]

\title{Evolutionary Dynamics: Selection, Drift, and Fitness Landscapes}
\author{Theoretical Population Genetics}
\date{\today}

\begin{document}
\maketitle

\begin{abstract}
This chapter explores the fundamental forces driving evolutionary change in populations.
We analyze natural selection under various fitness schemes, examine the stochastic effects
of genetic drift in finite populations, and investigate the interplay between mutation
and selection. Computational simulations illustrate fitness landscapes, fixation
probabilities, and the dynamics of allele frequency change under different evolutionary
scenarios.
\end{abstract}

\section{Introduction}

Evolution operates through the interplay of deterministic forces (selection, mutation) and
stochastic processes (genetic drift). Understanding these dynamics requires both analytical
theory and computational simulation.

\begin{definition}[Fitness]
The fitness $w$ of a genotype is its relative reproductive success. For a diploid organism
with alleles $A$ and $a$, genotype fitnesses are denoted $w_{AA}$, $w_{Aa}$, and $w_{aa}$.
\end{definition}

\section{Theoretical Framework}

\subsection{Natural Selection}

\begin{theorem}[Change in Allele Frequency Under Selection]
For a population with allele frequency $p$ (for allele $A$) and genotype fitnesses
$w_{AA}$, $w_{Aa}$, $w_{aa}$, the change in one generation is:
\begin{equation}
\Delta p = \frac{p q [p(w_{AA} - w_{Aa}) + q(w_{Aa} - w_{aa})]}{\bar{w}}
\end{equation}
where $q = 1 - p$ and $\bar{w} = p^2 w_{AA} + 2pq w_{Aa} + q^2 w_{aa}$ is the mean fitness.
\end{theorem}

\subsection{Selection Schemes}

\begin{definition}[Types of Selection]
Different selection regimes produce distinct dynamics:
\begin{itemize}
\item \textbf{Directional}: One homozygote has highest fitness; allele goes to fixation
\item \textbf{Overdominance}: Heterozygote advantage; stable polymorphism
\item \textbf{Underdominance}: Heterozygote disadvantage; unstable equilibrium
\item \textbf{Frequency-dependent}: Fitness depends on allele frequency
\end{itemize}
\end{definition}

\subsection{Genetic Drift}

\begin{theorem}[Wright-Fisher Model]
In a finite population of size $N$, allele frequencies change stochastically. The
variance in allele frequency change per generation is:
\begin{equation}
\text{Var}(\Delta p) = \frac{p(1-p)}{2N}
\end{equation}
The probability of eventual fixation for a neutral allele at initial frequency $p_0$ is $p_0$.
\end{theorem}

\begin{remark}[Effective Population Size]
Real populations often have $N_e < N$ due to variance in reproductive success, unequal
sex ratios, and population fluctuations.
\end{remark}

\subsection{Mutation-Selection Balance}

\begin{theorem}[Equilibrium Frequency]
For a deleterious recessive allele with mutation rate $\mu$ and selection coefficient $s$:
\begin{equation}
\hat{q} \approx \sqrt{\frac{\mu}{s}}
\end{equation}
For a deleterious dominant allele: $\hat{q} \approx \mu/s$.
\end{theorem}

\section{Computational Analysis}

\begin{pycode}
import numpy as np
import matplotlib.pyplot as plt
from scipy.integrate import odeint

np.random.seed(42)

def selection_dynamics(p, t, w_AA, w_Aa, w_aa):
    """Continuous-time selection dynamics."""
    if p <= 0 or p >= 1:
        return 0
    q = 1 - p
    w_bar = p**2 * w_AA + 2*p*q * w_Aa + q**2 * w_aa
    dp = p * q * (p*(w_AA - w_Aa) + q*(w_Aa - w_aa)) / w_bar
    return dp

def wright_fisher(p0, N, generations):
    """Wright-Fisher genetic drift simulation."""
    p = np.zeros(generations)
    p[0] = p0
    for i in range(1, generations):
        if p[i-1] <= 0 or p[i-1] >= 1:
            p[i] = p[i-1]
        else:
            n_A = np.random.binomial(2*N, p[i-1])
            p[i] = n_A / (2*N)
    return p

def fixation_probability_simulation(p0, N, s, n_sims=1000):
    """Estimate fixation probability by simulation."""
    fixed = 0
    for _ in range(n_sims):
        p = p0
        while 0 < p < 1:
            # Selection
            w_bar = (1 + s) * p**2 + p*(1-p) + (1-p)**2
            p_sel = ((1 + s) * p**2 + p*(1-p)) / w_bar
            # Drift
            n_A = np.random.binomial(2*N, p_sel)
            p = n_A / (2*N)
        if p >= 1:
            fixed += 1
    return fixed / n_sims

# Parameters
generations = 200
t = np.linspace(0, generations, 1000)

# Selection scenarios
# Directional selection
p_dir = odeint(selection_dynamics, 0.1, t, args=(1.0, 0.95, 0.8))
# Overdominance
p_over = odeint(selection_dynamics, 0.1, t, args=(0.8, 1.0, 0.9))
# Underdominance (two initial conditions)
p_under1 = odeint(selection_dynamics, 0.4, t, args=(1.0, 0.8, 1.0))
p_under2 = odeint(selection_dynamics, 0.6, t, args=(1.0, 0.8, 1.0))
# Frequency-dependent (simplified)
p_freq = odeint(selection_dynamics, 0.3, t, args=(0.9, 1.0, 0.9))

# Genetic drift simulations
N_values = [20, 100, 500]
n_reps = 20
drift_results = {}
for N in N_values:
    drift_results[N] = [wright_fisher(0.5, N, generations) for _ in range(n_reps)]

# Fitness landscape
p_range = np.linspace(0.001, 0.999, 200)
w_bar_dir = p_range**2 * 1.0 + 2*p_range*(1-p_range) * 0.95 + (1-p_range)**2 * 0.8
w_bar_over = p_range**2 * 0.8 + 2*p_range*(1-p_range) * 1.0 + (1-p_range)**2 * 0.9
w_bar_under = p_range**2 * 1.0 + 2*p_range*(1-p_range) * 0.8 + (1-p_range)**2 * 1.0

# Mutation-selection balance
mu = 1e-5
s_values = np.logspace(-4, -1, 50)
q_eq_recessive = np.sqrt(mu / s_values)
q_eq_dominant = mu / s_values

# Fixation probability (theoretical)
N_fix = 100
s_fix = np.linspace(-0.1, 0.1, 50)
# Kimura's formula: P_fix = (1 - exp(-4Ns*p0)) / (1 - exp(-4Ns))
p0_fix = 1 / (2 * N_fix)  # New mutation
P_fix_theory = np.zeros_like(s_fix)
for i, s in enumerate(s_fix):
    if abs(s) < 1e-6:
        P_fix_theory[i] = p0_fix
    else:
        P_fix_theory[i] = (1 - np.exp(-4*N_fix*s*p0_fix)) / (1 - np.exp(-4*N_fix*s))

# Create figure
fig = plt.figure(figsize=(14, 12))

# Plot 1: Selection dynamics
ax1 = fig.add_subplot(3, 3, 1)
ax1.plot(t, p_dir, 'b-', linewidth=2, label='Directional')
ax1.plot(t, p_over, 'g-', linewidth=2, label='Overdominance')
ax1.plot(t, p_under1, 'r-', linewidth=1.5, label='Underdominance')
ax1.plot(t, p_under2, 'r--', linewidth=1.5)
ax1.set_xlabel('Generation')
ax1.set_ylabel('Allele frequency $p$')
ax1.set_title('Selection Dynamics')
ax1.legend(fontsize=8)
ax1.set_ylim([0, 1])

# Plot 2: Fitness landscape
ax2 = fig.add_subplot(3, 3, 2)
ax2.plot(p_range, w_bar_dir, 'b-', linewidth=2, label='Directional')
ax2.plot(p_range, w_bar_over, 'g-', linewidth=2, label='Overdominance')
ax2.plot(p_range, w_bar_under, 'r-', linewidth=2, label='Underdominance')
ax2.set_xlabel('Allele frequency $p$')
ax2.set_ylabel('Mean fitness $\\bar{w}$')
ax2.set_title('Fitness Landscape')
ax2.legend(fontsize=8)

# Plot 3: Genetic drift (small N)
ax3 = fig.add_subplot(3, 3, 3)
gen = np.arange(generations)
for traj in drift_results[20]:
    ax3.plot(gen, traj, linewidth=0.8, alpha=0.5)
ax3.set_xlabel('Generation')
ax3.set_ylabel('Allele frequency $p$')
ax3.set_title(f'Genetic Drift (N = 20)')
ax3.set_ylim([0, 1])

# Plot 4: Genetic drift (medium N)
ax4 = fig.add_subplot(3, 3, 4)
for traj in drift_results[100]:
    ax4.plot(gen, traj, linewidth=0.8, alpha=0.5)
ax4.set_xlabel('Generation')
ax4.set_ylabel('Allele frequency $p$')
ax4.set_title(f'Genetic Drift (N = 100)')
ax4.set_ylim([0, 1])

# Plot 5: Genetic drift (large N)
ax5 = fig.add_subplot(3, 3, 5)
for traj in drift_results[500]:
    ax5.plot(gen, traj, linewidth=0.8, alpha=0.5)
ax5.set_xlabel('Generation')
ax5.set_ylabel('Allele frequency $p$')
ax5.set_title(f'Genetic Drift (N = 500)')
ax5.set_ylim([0, 1])

# Plot 6: Mutation-selection balance
ax6 = fig.add_subplot(3, 3, 6)
ax6.loglog(s_values, q_eq_recessive, 'b-', linewidth=2, label='Recessive')
ax6.loglog(s_values, q_eq_dominant, 'r-', linewidth=2, label='Dominant')
ax6.set_xlabel('Selection coefficient $s$')
ax6.set_ylabel('Equilibrium frequency $\\hat{q}$')
ax6.set_title(f'Mutation-Selection Balance ($\\mu = {mu}$)')
ax6.legend()

# Plot 7: Fixation probability
ax7 = fig.add_subplot(3, 3, 7)
ax7.plot(s_fix, P_fix_theory, 'b-', linewidth=2)
ax7.axhline(y=p0_fix, color='gray', linestyle='--', alpha=0.7, label='Neutral')
ax7.axvline(x=0, color='gray', linestyle=':', alpha=0.5)
ax7.set_xlabel('Selection coefficient $s$')
ax7.set_ylabel('Fixation probability')
ax7.set_title(f'Fixation Probability (N = {N_fix})')
ax7.legend(fontsize=8)

# Plot 8: Hardy-Weinberg frequencies
ax8 = fig.add_subplot(3, 3, 8)
ax8.plot(p_range, p_range**2, 'b-', linewidth=2, label='$p^2$ (AA)')
ax8.plot(p_range, 2*p_range*(1-p_range), 'g-', linewidth=2, label='$2pq$ (Aa)')
ax8.plot(p_range, (1-p_range)**2, 'r-', linewidth=2, label='$q^2$ (aa)')
ax8.set_xlabel('Allele frequency $p$')
ax8.set_ylabel('Genotype frequency')
ax8.set_title('Hardy-Weinberg Equilibrium')
ax8.legend(fontsize=8)

# Plot 9: Variance in allele frequency
ax9 = fig.add_subplot(3, 3, 9)
N_range = np.logspace(1, 4, 100)
p_test = 0.5
var_deltap = p_test * (1 - p_test) / (2 * N_range)
ax9.loglog(N_range, var_deltap, 'b-', linewidth=2)
ax9.set_xlabel('Population size $N$')
ax9.set_ylabel('Var($\\Delta p$)')
ax9.set_title('Drift Variance vs Population Size')

plt.tight_layout()
plt.savefig('evolutionary_dynamics_analysis.pdf', dpi=150, bbox_inches='tight')
plt.close()

# Calculate equilibrium for overdominance
p_eq_over = (1.0 - 0.9) / (2*1.0 - 0.8 - 0.9)
\end{pycode}

\begin{figure}[htbp]
\centering
\includegraphics[width=\textwidth]{evolutionary_dynamics_analysis.pdf}
\caption{Evolutionary dynamics analysis: (a) Selection dynamics under different fitness
schemes; (b) Fitness landscapes showing mean fitness as function of allele frequency;
(c-e) Genetic drift in populations of different sizes; (f) Mutation-selection balance;
(g) Fixation probability as function of selection coefficient; (h) Hardy-Weinberg
genotype frequencies; (i) Drift variance versus population size.}
\label{fig:evolution}
\end{figure}

\section{Results}

\subsection{Selection Equilibria}

\begin{pycode}
print(r"\begin{table}[htbp]")
print(r"\centering")
print(r"\caption{Selection Equilibria for Different Fitness Schemes}")
print(r"\begin{tabular}{lcccc}")
print(r"\toprule")
print(r"Selection Type & $w_{AA}$ & $w_{Aa}$ & $w_{aa}$ & Equilibrium $\hat{p}$ \\")
print(r"\midrule")
print(r"Directional & 1.00 & 0.95 & 0.80 & 1.00 (fixation) \\")
print(f"Overdominance & 0.80 & 1.00 & 0.90 & {p_eq_over:.2f} \\\\")
print(r"Underdominance & 1.00 & 0.80 & 1.00 & 0.00 or 1.00 \\")
print(r"\bottomrule")
print(r"\end{tabular}")
print(r"\label{tab:equilibria}")
print(r"\end{table}")
\end{pycode}

\subsection{Drift Statistics}

\begin{pycode}
print(r"\begin{table}[htbp]")
print(r"\centering")
print(r"\caption{Genetic Drift Statistics After " + str(generations) + r" Generations}")
print(r"\begin{tabular}{cccc}")
print(r"\toprule")
print(r"$N$ & Fixed & Lost & Polymorphic \\")
print(r"\midrule")

for N in N_values:
    fixed = sum(1 for traj in drift_results[N] if traj[-1] >= 0.99)
    lost = sum(1 for traj in drift_results[N] if traj[-1] <= 0.01)
    poly = n_reps - fixed - lost
    print(f"{N} & {fixed} & {lost} & {poly} \\\\")

print(r"\bottomrule")
print(r"\end{tabular}")
print(r"\label{tab:drift}")
print(r"\end{table}")
\end{pycode}

\section{Discussion}

\begin{example}[Overdominance and Stable Polymorphism]
Heterozygote advantage (overdominance) maintains genetic variation in populations.
The equilibrium allele frequency for our example is:
\begin{equation}
\hat{p} = \frac{w_{Aa} - w_{aa}}{2w_{Aa} - w_{AA} - w_{aa}} = \frac{1.0 - 0.9}{2(1.0) - 0.8 - 0.9} = \py{f"{p_eq_over:.2f}"}
\end{equation}
Classic examples include sickle-cell anemia and MHC diversity.
\end{example}

\begin{remark}[Effective Population Size]
The variance in allele frequency change is $\text{Var}(\Delta p) = pq/(2N_e)$. Factors
reducing $N_e$ below census size include:
\begin{itemize}
\item Unequal sex ratios
\item Variance in reproductive success
\item Population bottlenecks
\item Overlapping generations
\end{itemize}
\end{remark}

\section{Conclusions}

This analysis demonstrates the fundamental forces of evolutionary change:
\begin{enumerate}
\item Selection drives systematic allele frequency change toward fitness peaks
\item Overdominance equilibrium at $\hat{p} = \py{f"{p_eq_over:.2f}"}$ maintains polymorphism
\item Genetic drift is stronger in small populations, with variance $\propto 1/N$
\item Mutation-selection balance maintains deleterious alleles at low frequencies
\item Fixation probability depends on both selection and population size
\end{enumerate}

\section*{Further Reading}

\begin{itemize}
\item Hartl, D.L. \& Clark, A.G. \textit{Principles of Population Genetics}, 4th ed. Sinauer, 2007.
\item Gillespie, J.H. \textit{Population Genetics: A Concise Guide}, 2nd ed. Johns Hopkins, 2004.
\item Nowak, M.A. \textit{Evolutionary Dynamics}. Harvard University Press, 2006.
\end{itemize}

\end{document}
