\documentclass[12pt]{article}
\usepackage{amsmath, amssymb, amsthm}
\usepackage{geometry}
\usepackage{sagetex}
\usepackage{float}
\usepackage{graphicx}

% Interaction with Beamer overlays (if converting to slides)
% This resolves conflicts where Sage plots on specific slides
% might not render correctly without explicit visibility control
% Fixed: Changed from [1] to [2] to accept both slide number and plot arguments
\newcommand{\sagevisible}[2]{\visible<#1->{\sageplot{#2}}}

\title{Explorations in Partition Theory and Topology}
\author{Department of Mathematics}

\begin{document}
\maketitle

\section{Integer Partitions}

The partition function $p(n)$ represents the number of distinct ways of representing $n$ as a sum of natural numbers. For large $n$, this computation is non-trivial.

Using SageMath, we calculate the number of partitions for $n=100$:

\begin{sageblock}
n = 100
partitions = number_of_partitions(n)
factors = factor(partitions)
\end{sageblock}

The number of partitions of 100 is:
\[ p(100) = \sage{partitions} \]

This number can be factored into primes as:
\[ \sage{factors} \]

This demonstrates Sage's ability to handle arbitrary-precision integers, which would overflow standard types in C++ or basic Python.

\section{Topological Surfaces}

We visualize the behavior of the function $f(x, y) = \sin(x^2 + y^2)$ near the origin.

\begin{sagesilent}
var('x, y')
# We use sagesilent because we don't want the plotting code
# to appear in the final text, only the image.
p = plot3d(sin(x^2 + y^2), (x, -2, 2), (y, -2, 2),
           adaptive=True, color='automatic')
\end{sagesilent}

\begin{figure}[H]
    \centering
    % SageTeX handles the file naming automatically (e.g., plot-0.png)
    \sageplot[width=0.8\textwidth]{p}
    \caption{3D Plot of $f(x, y)$ computed by SageMath.}
\end{figure}

\section{Additional Symbolic Computation}

SageMath excels at symbolic manipulation. Here we demonstrate polynomial factorization and calculus:

\begin{sageblock}
# Polynomial factorization
poly = x^4 - 1
factored = factor(poly)

# Symbolic integration
integral_result = integrate(sin(x)^2, x)
\end{sageblock}

The polynomial $x^4 - 1$ factors as:
\[ \sage{factored} \]

The indefinite integral of $\sin^2(x)$ is:
\[ \int \sin^2(x) \, dx = \sage{integral_result} + C \]

\end{document}
