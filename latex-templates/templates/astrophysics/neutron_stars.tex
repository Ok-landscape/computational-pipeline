% Neutron Star Physics
\documentclass[11pt,a4paper]{article}
\usepackage[utf8]{inputenc}
\usepackage[T1]{fontenc}
\usepackage{amsmath,amssymb}
\usepackage{graphicx}
\usepackage{booktabs}
\usepackage{siunitx}
\usepackage{geometry}
\geometry{margin=1in}
\usepackage{pythontex}
\usepackage{hyperref}
\usepackage{float}

\title{Neutron Star Physics\\Equation of State and Structure}
\author{Nuclear Astrophysics Group}
\date{\today}

\begin{document}
\maketitle

\begin{abstract}
Analysis of neutron star structure including mass-radius relations, equation of state, and magnetic field properties.
\end{abstract}

\section{Introduction}

Neutron stars are ultra-dense remnants of massive stars.

\begin{pycode}
import numpy as np
import matplotlib.pyplot as plt
from scipy.integrate import odeint
plt.rcParams['text.usetex'] = True
plt.rcParams['font.family'] = 'serif'

G = 6.674e-11
c = 2.998e8
M_sun = 1.989e30
hbar = 1.055e-34
m_n = 1.675e-27
\end{pycode}

\section{Equation of State}

\begin{pycode}
# Polytropic EOS
K = 1e11  # Polytropic constant
Gamma = 2.0  # Adiabatic index

rho = np.logspace(14, 16, 100)  # kg/m^3
P = K * rho**Gamma

fig, ax = plt.subplots(figsize=(10, 6))
ax.loglog(rho, P, 'b-', linewidth=2)
ax.set_xlabel('Density (kg/m$^3$)')
ax.set_ylabel('Pressure (Pa)')
ax.set_title('Polytropic Equation of State')
ax.grid(True, alpha=0.3, which='both')
plt.tight_layout()
plt.savefig('eos.pdf', dpi=150, bbox_inches='tight')
plt.close()
\end{pycode}

\begin{figure}[H]
\centering
\includegraphics[width=0.9\textwidth]{eos.pdf}
\caption{Polytropic equation of state.}
\end{figure}

\section{TOV Equation}

\begin{pycode}
def tov(y, r, K, Gamma):
    P, m = y
    if P <= 0:
        return [0, 0]
    rho = (P / K)**(1/Gamma)
    eps = rho * c**2 + P / (Gamma - 1)
    dPdr = -G * (eps + P) * (m + 4*np.pi*r**3*P/c**2) / (r**2 * (1 - 2*G*m/(r*c**2)))
    dmdr = 4 * np.pi * r**2 * eps / c**2
    return [dPdr, dmdr]

# Solve for different central densities
rho_c_range = np.logspace(14.5, 15.5, 20)
masses = []
radii = []

for rho_c in rho_c_range:
    P_c = K * rho_c**Gamma
    r = np.linspace(1, 20000, 10000)
    y0 = [P_c, 0]
    sol = odeint(tov, y0, r, args=(K, Gamma))
    P_sol = sol[:, 0]
    m_sol = sol[:, 1]

    idx = np.where(P_sol > 0)[0]
    if len(idx) > 0:
        R = r[idx[-1]]
        M = m_sol[idx[-1]]
        masses.append(M / M_sun)
        radii.append(R / 1000)

fig, ax = plt.subplots(figsize=(10, 6))
ax.plot(radii, masses, 'b-o', linewidth=1.5, markersize=4)
ax.set_xlabel('Radius (km)')
ax.set_ylabel('Mass ($M_\\odot$)')
ax.set_title('Mass-Radius Relation')
ax.grid(True, alpha=0.3)
ax.set_xlim([8, 16])
ax.set_ylim([0, 3])
plt.tight_layout()
plt.savefig('mass_radius.pdf', dpi=150, bbox_inches='tight')
plt.close()
\end{pycode}

\begin{figure}[H]
\centering
\includegraphics[width=0.9\textwidth]{mass_radius.pdf}
\caption{Neutron star mass-radius relation.}
\end{figure}

\section{Density Profile}

\begin{pycode}
rho_c = 1e15
P_c = K * rho_c**Gamma
r = np.linspace(1, 12000, 5000)
sol = odeint(tov, [P_c, 0], r, args=(K, Gamma))
P_profile = sol[:, 0]
rho_profile = np.where(P_profile > 0, (P_profile / K)**(1/Gamma), 0)

fig, ax = plt.subplots(figsize=(10, 6))
ax.plot(r / 1000, rho_profile / 1e15, 'b-', linewidth=2)
ax.set_xlabel('Radius (km)')
ax.set_ylabel('Density ($10^{15}$ kg/m$^3$)')
ax.set_title('Neutron Star Density Profile')
ax.grid(True, alpha=0.3)
plt.tight_layout()
plt.savefig('density_profile.pdf', dpi=150, bbox_inches='tight')
plt.close()
\end{pycode}

\begin{figure}[H]
\centering
\includegraphics[width=0.9\textwidth]{density_profile.pdf}
\caption{Internal density profile.}
\end{figure}

\section{Magnetic Field}

\begin{pycode}
# Pulsar spindown
P = np.logspace(-3, 1, 100)  # Period in seconds
P_dot = np.logspace(-20, -10, 100)  # Period derivative

PP, PP_dot = np.meshgrid(P, P_dot)
B_surf = 3.2e19 * np.sqrt(PP * PP_dot)  # Surface magnetic field in Gauss

fig, ax = plt.subplots(figsize=(10, 8))
levels = [1e8, 1e10, 1e12, 1e14, 1e16]
cs = ax.contour(np.log10(PP), np.log10(PP_dot), np.log10(B_surf), levels=np.log10(levels))
ax.clabel(cs, fmt='$10^{%.0f}$ G')
ax.set_xlabel('log$_{10}$ Period (s)')
ax.set_ylabel('log$_{10}$ $\\dot{P}$ (s/s)')
ax.set_title('Pulsar Magnetic Field')
ax.grid(True, alpha=0.3)
plt.tight_layout()
plt.savefig('magnetic_field.pdf', dpi=150, bbox_inches='tight')
plt.close()
\end{pycode}

\begin{figure}[H]
\centering
\includegraphics[width=0.9\textwidth]{magnetic_field.pdf}
\caption{P-$\dot{P}$ diagram with magnetic field lines.}
\end{figure}

\section{Spin-down Age}

\begin{pycode}
P_values = np.array([0.001, 0.01, 0.1, 1.0])  # Periods
P_dot_values = np.array([1e-15, 1e-14, 1e-13, 1e-12])  # Period derivatives

tau_c = P_values / (2 * P_dot_values)  # Characteristic age

fig, ax = plt.subplots(figsize=(10, 6))
ax.loglog(P_values, tau_c / (365.25 * 24 * 3600), 'b-o', linewidth=1.5, markersize=8)
ax.set_xlabel('Period (s)')
ax.set_ylabel('Characteristic Age (years)')
ax.set_title('Pulsar Spin-down Age')
ax.grid(True, alpha=0.3, which='both')
plt.tight_layout()
plt.savefig('spindown_age.pdf', dpi=150, bbox_inches='tight')
plt.close()
\end{pycode}

\begin{figure}[H]
\centering
\includegraphics[width=0.9\textwidth]{spindown_age.pdf}
\caption{Characteristic age vs period.}
\end{figure}

\section{Compactness}

\begin{pycode}
compactness = np.array(masses) * M_sun * G / (np.array(radii) * 1000 * c**2)

fig, ax = plt.subplots(figsize=(10, 6))
ax.plot(masses, compactness, 'b-o', linewidth=1.5, markersize=4)
ax.axhline(y=0.5, color='r', linestyle='--', label='Black hole limit')
ax.set_xlabel('Mass ($M_\\odot$)')
ax.set_ylabel('Compactness $GM/Rc^2$')
ax.set_title('Neutron Star Compactness')
ax.legend()
ax.grid(True, alpha=0.3)
plt.tight_layout()
plt.savefig('compactness.pdf', dpi=150, bbox_inches='tight')
plt.close()
\end{pycode}

\begin{figure}[H]
\centering
\includegraphics[width=0.9\textwidth]{compactness.pdf}
\caption{Compactness parameter.}
\end{figure}

\section{Results}

\begin{pycode}
M_max = max(masses)
R_at_max = radii[masses.index(M_max)]
print(r'\begin{table}[H]')
print(r'\centering')
print(r'\caption{Neutron Star Properties}')
print(r'\begin{tabular}{@{}lc@{}}')
print(r'\toprule')
print(r'Property & Value \\')
print(r'\midrule')
print(f'Maximum mass & {M_max:.2f} $M_\\odot$ \\\\')
print(f'Radius at max mass & {R_at_max:.1f} km \\\\')
print(f'Central density & $10^{{15}}$ kg/m$^3$ \\\\')
print(r'\bottomrule')
print(r'\end{tabular}')
print(r'\end{table}')
\end{pycode}

\section{Conclusions}

Neutron star structure depends critically on the equation of state of ultra-dense matter.

\end{document}
