% Galaxy Dynamics
\documentclass[11pt,a4paper]{article}
\usepackage[utf8]{inputenc}
\usepackage[T1]{fontenc}
\usepackage{amsmath,amssymb}
\usepackage{graphicx}
\usepackage{booktabs}
\usepackage{siunitx}
\usepackage{geometry}
\geometry{margin=1in}
\usepackage{pythontex}
\usepackage{hyperref}
\usepackage{float}

\title{Galaxy Dynamics\\Rotation Curves and Dark Matter}
\author{Extragalactic Astronomy Group}
\date{\today}

\begin{document}
\maketitle

\begin{abstract}
Analysis of galaxy dynamics including rotation curves, dark matter profiles, and scaling relations.
\end{abstract}

\section{Introduction}

Galaxy rotation curves provide evidence for dark matter.

\begin{pycode}
import numpy as np
import matplotlib.pyplot as plt
plt.rcParams['text.usetex'] = True
plt.rcParams['font.family'] = 'serif'

G = 4.302e-6  # kpc (km/s)^2 / M_sun
\end{pycode}

\section{Rotation Curve Components}

\begin{pycode}
r = np.linspace(0.1, 30, 200)  # kpc

# Bulge (Hernquist profile)
M_b = 1e10
a_b = 0.5
v_bulge = np.sqrt(G * M_b * r / (r + a_b)**2)

# Disk (exponential)
M_d = 5e10
R_d = 3.5
y = r / (2 * R_d)
v_disk = np.sqrt(G * M_d * r**2 / R_d**3 * (0.5 - y + y**2 * np.exp(-2*y)))

# Dark matter halo (NFW)
M_h = 1e12
c = 10
r_s = 20
x = r / r_s
v_halo = np.sqrt(G * M_h * (np.log(1 + x) - x/(1 + x)) / (r * (np.log(1 + c) - c/(1 + c))))

# Total
v_total = np.sqrt(v_bulge**2 + v_disk**2 + v_halo**2)

fig, ax = plt.subplots(figsize=(10, 6))
ax.plot(r, v_bulge, '--', label='Bulge', linewidth=1.5)
ax.plot(r, v_disk, '--', label='Disk', linewidth=1.5)
ax.plot(r, v_halo, '--', label='Dark Matter Halo', linewidth=1.5)
ax.plot(r, v_total, 'k-', label='Total', linewidth=2)
ax.set_xlabel('Radius (kpc)')
ax.set_ylabel('Rotation Velocity (km/s)')
ax.set_title('Galaxy Rotation Curve Decomposition')
ax.legend()
ax.grid(True, alpha=0.3)
ax.set_xlim([0, 30])
ax.set_ylim([0, 300])
plt.tight_layout()
plt.savefig('rotation_curve.pdf', dpi=150, bbox_inches='tight')
plt.close()
\end{pycode}

\begin{figure}[H]
\centering
\includegraphics[width=0.9\textwidth]{rotation_curve.pdf}
\caption{Rotation curve decomposition showing different components.}
\end{figure}

\section{NFW Profile}

$\rho(r) = \frac{\rho_s}{(r/r_s)(1+r/r_s)^2}$

\begin{pycode}
r_nfw = np.logspace(-1, 2, 100)
rho_s = 1e7  # M_sun / kpc^3

for c in [5, 10, 20]:
    r_s = 20 / c * 10
    x = r_nfw / r_s
    rho = rho_s / (x * (1 + x)**2)
    plt.loglog(r_nfw, rho, label=f'c = {c}')

plt.xlabel('Radius (kpc)')
plt.ylabel('Density ($M_\\odot$/kpc$^3$)')
plt.title('NFW Dark Matter Density Profile')
plt.legend()
plt.grid(True, alpha=0.3, which='both')
plt.tight_layout()
plt.savefig('nfw_profile.pdf', dpi=150, bbox_inches='tight')
plt.close()
\end{pycode}

\begin{figure}[H]
\centering
\includegraphics[width=0.9\textwidth]{nfw_profile.pdf}
\caption{NFW density profiles for different concentrations.}
\end{figure}

\section{Tully-Fisher Relation}

$M_B = a \log_{10} V_{max} + b$

\begin{pycode}
# Generate mock data
np.random.seed(42)
V_max = np.random.uniform(100, 300, 50)
M_B = -7.5 * np.log10(V_max) + 3.5 + np.random.normal(0, 0.3, 50)

fig, ax = plt.subplots(figsize=(10, 6))
ax.scatter(np.log10(V_max), M_B, alpha=0.7)
# Fit line
coeffs = np.polyfit(np.log10(V_max), M_B, 1)
V_fit = np.linspace(100, 300, 100)
ax.plot(np.log10(V_fit), np.polyval(coeffs, np.log10(V_fit)), 'r-', linewidth=2)
ax.set_xlabel('log$_{10}$ $V_{max}$ (km/s)')
ax.set_ylabel('$M_B$ (mag)')
ax.set_title('Tully-Fisher Relation')
ax.invert_yaxis()
ax.grid(True, alpha=0.3)
plt.tight_layout()
plt.savefig('tully_fisher.pdf', dpi=150, bbox_inches='tight')
plt.close()
\end{pycode}

\begin{figure}[H]
\centering
\includegraphics[width=0.9\textwidth]{tully_fisher.pdf}
\caption{Tully-Fisher relation for spiral galaxies.}
\end{figure}

\section{Dark Matter Fraction}

\begin{pycode}
r_frac = np.linspace(0.1, 30, 100)
M_baryon = M_b * r_frac**2 / (r_frac + a_b)**2 + M_d * (1 - (1 + r_frac/R_d) * np.exp(-r_frac/R_d))
x_frac = r_frac / r_s
M_dm = M_h * (np.log(1 + x_frac) - x_frac/(1 + x_frac)) / (np.log(1 + c) - c/(1 + c))
f_dm = M_dm / (M_dm + M_baryon)

fig, ax = plt.subplots(figsize=(10, 6))
ax.plot(r_frac, f_dm, 'b-', linewidth=2)
ax.axhline(y=0.5, color='r', linestyle='--')
ax.set_xlabel('Radius (kpc)')
ax.set_ylabel('Dark Matter Fraction')
ax.set_title('Enclosed Dark Matter Fraction')
ax.grid(True, alpha=0.3)
ax.set_xlim([0, 30])
ax.set_ylim([0, 1])
plt.tight_layout()
plt.savefig('dm_fraction.pdf', dpi=150, bbox_inches='tight')
plt.close()
\end{pycode}

\begin{figure}[H]
\centering
\includegraphics[width=0.9\textwidth]{dm_fraction.pdf}
\caption{Dark matter fraction vs radius.}
\end{figure}

\section{Velocity Dispersion}

\begin{pycode}
# Stellar velocity dispersion profile
r_sigma = np.linspace(0.1, 10, 100)
sigma_0 = 200  # km/s
r_e = 2  # Effective radius
sigma = sigma_0 * (1 + r_sigma / r_e)**(-0.5)

fig, ax = plt.subplots(figsize=(10, 6))
ax.plot(r_sigma, sigma, 'b-', linewidth=2)
ax.set_xlabel('Radius (kpc)')
ax.set_ylabel('Velocity Dispersion (km/s)')
ax.set_title('Stellar Velocity Dispersion Profile')
ax.grid(True, alpha=0.3)
plt.tight_layout()
plt.savefig('velocity_dispersion.pdf', dpi=150, bbox_inches='tight')
plt.close()
\end{pycode}

\begin{figure}[H]
\centering
\includegraphics[width=0.9\textwidth]{velocity_dispersion.pdf}
\caption{Velocity dispersion profile.}
\end{figure}

\section{Mass Modeling}

\begin{pycode}
# Enclosed mass
M_enc = (v_total * r)**2 / (G * r) * r

fig, ax = plt.subplots(figsize=(10, 6))
ax.semilogy(r, M_enc, 'b-', linewidth=2)
ax.set_xlabel('Radius (kpc)')
ax.set_ylabel('Enclosed Mass ($M_\\odot$)')
ax.set_title('Enclosed Mass Profile')
ax.grid(True, alpha=0.3, which='both')
plt.tight_layout()
plt.savefig('enclosed_mass.pdf', dpi=150, bbox_inches='tight')
plt.close()
\end{pycode}

\begin{figure}[H]
\centering
\includegraphics[width=0.9\textwidth]{enclosed_mass.pdf}
\caption{Enclosed mass profile.}
\end{figure}

\section{Results}

\begin{pycode}
v_max = np.max(v_total)
r_max = r[np.argmax(v_total)]
print(r'\begin{table}[H]')
print(r'\centering')
print(r'\caption{Galaxy Model Parameters}')
print(r'\begin{tabular}{@{}lc@{}}')
print(r'\toprule')
print(r'Property & Value \\')
print(r'\midrule')
print(f'Maximum velocity & {v_max:.0f} km/s \\\\')
print(f'Radius at $V_{{max}}$ & {r_max:.1f} kpc \\\\')
print(f'Bulge mass & {M_b:.1e} $M_\\odot$ \\\\')
print(f'Disk mass & {M_d:.1e} $M_\\odot$ \\\\')
print(f'Halo mass & {M_h:.1e} $M_\\odot$ \\\\')
print(r'\bottomrule')
print(r'\end{tabular}')
print(r'\end{table}')
\end{pycode}

\section{Conclusions}

Galaxy rotation curves require dark matter halos to explain observations.

\end{document}
