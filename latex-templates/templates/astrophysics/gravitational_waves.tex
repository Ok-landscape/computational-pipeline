% Gravitational Wave Physics
\documentclass[11pt,a4paper]{article}
\usepackage[utf8]{inputenc}
\usepackage[T1]{fontenc}
\usepackage{amsmath,amssymb}
\usepackage{graphicx}
\usepackage{booktabs}
\usepackage{siunitx}
\usepackage{geometry}
\geometry{margin=1in}
\usepackage{pythontex}
\usepackage{hyperref}
\usepackage{float}

\title{Gravitational Wave Physics\\Strain, Detection, and Binary Systems}
\author{Gravitational Wave Astronomy Group}
\date{\today}

\begin{document}
\maketitle

\begin{abstract}
Analysis of gravitational wave generation, propagation, and detection including chirp mass calculations and LIGO sensitivity.
\end{abstract}

\section{Introduction}

Gravitational waves are ripples in spacetime caused by accelerating masses.

\begin{pycode}
import numpy as np
import matplotlib.pyplot as plt
plt.rcParams['text.usetex'] = True
plt.rcParams['font.family'] = 'serif'

G = 6.674e-11
c = 2.998e8
M_sun = 1.989e30
pc = 3.086e16  # parsec in meters
\end{pycode}

\section{Chirp Mass}

$\mathcal{M} = \frac{(m_1 m_2)^{3/5}}{(m_1 + m_2)^{1/5}}$

\begin{pycode}
m1_range = np.linspace(1, 50, 50)
m2 = 30  # Fixed second mass

M_chirp = (m1_range * m2)**(3/5) / (m1_range + m2)**(1/5)

fig, ax = plt.subplots(figsize=(10, 6))
for m2_val in [10, 20, 30, 40]:
    M_c = (m1_range * m2_val)**(3/5) / (m1_range + m2_val)**(1/5)
    ax.plot(m1_range, M_c, linewidth=1.5, label=f'$m_2$ = {m2_val} $M_\\odot$')
ax.set_xlabel('$m_1$ ($M_\\odot$)')
ax.set_ylabel('Chirp Mass ($M_\\odot$)')
ax.set_title('Chirp Mass for Binary Systems')
ax.legend()
ax.grid(True, alpha=0.3)
plt.tight_layout()
plt.savefig('chirp_mass.pdf', dpi=150, bbox_inches='tight')
plt.close()
\end{pycode}

\begin{figure}[H]
\centering
\includegraphics[width=0.9\textwidth]{chirp_mass.pdf}
\caption{Chirp mass for different binary configurations.}
\end{figure}

\section{Gravitational Wave Frequency}

\begin{pycode}
# Orbital frequency to GW frequency
M_total = 60 * M_sun  # Total mass
r_sep = np.logspace(6, 8, 100) * 1000  # Separation in meters

f_orb = np.sqrt(G * M_total / r_sep**3) / (2 * np.pi)
f_gw = 2 * f_orb  # GW frequency is twice orbital

fig, ax = plt.subplots(figsize=(10, 6))
ax.loglog(r_sep / 1000, f_gw, 'b-', linewidth=2)
ax.set_xlabel('Separation (km)')
ax.set_ylabel('GW Frequency (Hz)')
ax.set_title('Gravitational Wave Frequency vs Separation')
ax.grid(True, alpha=0.3, which='both')
plt.tight_layout()
plt.savefig('gw_frequency.pdf', dpi=150, bbox_inches='tight')
plt.close()
\end{pycode}

\begin{figure}[H]
\centering
\includegraphics[width=0.9\textwidth]{gw_frequency.pdf}
\caption{GW frequency dependence on binary separation.}
\end{figure}

\section{Strain Amplitude}

$h = \frac{4}{D}\left(\frac{G\mathcal{M}}{c^2}\right)^{5/3}\left(\frac{\pi f}{c}\right)^{2/3}$

\begin{pycode}
D = 400 * 1e6 * pc  # Distance (400 Mpc)
M_c = 30 * M_sun     # Chirp mass
f_range = np.logspace(0, 3, 100)

h = (4 / D) * (G * M_c / c**2)**(5/3) * (np.pi * f_range / c)**(2/3)

fig, ax = plt.subplots(figsize=(10, 6))
ax.loglog(f_range, h, 'b-', linewidth=2)
ax.set_xlabel('Frequency (Hz)')
ax.set_ylabel('Strain $h$')
ax.set_title(f'GW Strain at D = 400 Mpc')
ax.grid(True, alpha=0.3, which='both')
plt.tight_layout()
plt.savefig('gw_strain.pdf', dpi=150, bbox_inches='tight')
plt.close()
\end{pycode}

\begin{figure}[H]
\centering
\includegraphics[width=0.9\textwidth]{gw_strain.pdf}
\caption{Gravitational wave strain amplitude.}
\end{figure}

\section{Inspiral Waveform}

\begin{pycode}
# Simple inspiral model
M_c_kg = 30 * M_sun
t_merge = 1.0  # Time to merger
t = np.linspace(0, t_merge - 0.01, 10000)
tau = t_merge - t  # Time to coalescence

# Frequency evolution
f_t = (1 / np.pi) * (5 / (256 * tau))**(3/8) * (G * M_c_kg / c**3)**(-5/8)
f_t = np.clip(f_t, 10, 1000)

# Phase
phi_t = -2 * (tau / (5 * G * M_c_kg / c**3))**(5/8)
h_t = np.sin(phi_t)

fig, (ax1, ax2) = plt.subplots(2, 1, figsize=(12, 8), sharex=True)
ax1.plot(t, f_t, 'b-', linewidth=1)
ax1.set_ylabel('Frequency (Hz)')
ax1.set_title('Inspiral Waveform')
ax1.set_yscale('log')
ax1.grid(True, alpha=0.3)

ax2.plot(t, h_t, 'b-', linewidth=0.5)
ax2.set_xlabel('Time (s)')
ax2.set_ylabel('Strain (arb. units)')
ax2.set_xlim([0.8, 1])
ax2.grid(True, alpha=0.3)
plt.tight_layout()
plt.savefig('inspiral_waveform.pdf', dpi=150, bbox_inches='tight')
plt.close()
\end{pycode}

\begin{figure}[H]
\centering
\includegraphics[width=0.9\textwidth]{inspiral_waveform.pdf}
\caption{Binary inspiral frequency and waveform evolution.}
\end{figure}

\section{LIGO Sensitivity}

\begin{pycode}
# Simplified LIGO noise curve
f_ligo = np.logspace(0.5, 4, 500)
S_n = 1e-47 * ((f_ligo / 100)**(-4) + 2 * (1 + (f_ligo / 100)**2))
h_n = np.sqrt(S_n * f_ligo)

fig, ax = plt.subplots(figsize=(10, 6))
ax.loglog(f_ligo, np.sqrt(S_n), 'b-', linewidth=2, label='LIGO Sensitivity')
ax.set_xlabel('Frequency (Hz)')
ax.set_ylabel('Strain Noise ($1/\\sqrt{\\mathrm{Hz}}$)')
ax.set_title('LIGO Sensitivity Curve')
ax.legend()
ax.grid(True, alpha=0.3, which='both')
ax.set_xlim([10, 3000])
plt.tight_layout()
plt.savefig('ligo_sensitivity.pdf', dpi=150, bbox_inches='tight')
plt.close()
\end{pycode}

\begin{figure}[H]
\centering
\includegraphics[width=0.9\textwidth]{ligo_sensitivity.pdf}
\caption{LIGO detector sensitivity curve.}
\end{figure}

\section{Energy Radiated}

\begin{pycode}
# Energy in GWs
eta = 0.25  # Symmetric mass ratio
M_total_energy = 60 * M_sun
E_rad = eta * M_total_energy * c**2 * 0.1  # ~10% radiated

distances = np.logspace(7, 10, 100) * pc
L_gw = E_rad / 0.1  # Peak luminosity over 0.1 s

fig, ax = plt.subplots(figsize=(10, 6))
ax.loglog(distances / (1e6 * pc), np.sqrt(L_gw * G / (c**3 * distances**2)), 'b-', linewidth=2)
ax.set_xlabel('Distance (Mpc)')
ax.set_ylabel('Strain')
ax.set_title('Detectable Strain vs Distance')
ax.grid(True, alpha=0.3, which='both')
plt.tight_layout()
plt.savefig('strain_distance.pdf', dpi=150, bbox_inches='tight')
plt.close()
\end{pycode}

\begin{figure}[H]
\centering
\includegraphics[width=0.9\textwidth]{strain_distance.pdf}
\caption{GW strain as function of source distance.}
\end{figure}

\section{Merger Rate}

\begin{pycode}
# Merger rate density
z = np.linspace(0, 2, 100)
R_0 = 100  # Local rate per Gpc^3 per year
R_z = R_0 * (1 + z)**1.5  # Simple evolution

fig, ax = plt.subplots(figsize=(10, 6))
ax.plot(z, R_z, 'b-', linewidth=2)
ax.set_xlabel('Redshift $z$')
ax.set_ylabel('Merger Rate (Gpc$^{-3}$ yr$^{-1}$)')
ax.set_title('Binary Black Hole Merger Rate')
ax.grid(True, alpha=0.3)
plt.tight_layout()
plt.savefig('merger_rate.pdf', dpi=150, bbox_inches='tight')
plt.close()
\end{pycode}

\begin{figure}[H]
\centering
\includegraphics[width=0.9\textwidth]{merger_rate.pdf}
\caption{Merger rate evolution with redshift.}
\end{figure}

\section{Results}

\begin{pycode}
M_c_example = (30 * 30)**(3/5) / (60)**(1/5)
print(r'\begin{table}[H]')
print(r'\centering')
print(r'\caption{GW150914-like Parameters}')
print(r'\begin{tabular}{@{}lc@{}}')
print(r'\toprule')
print(r'Parameter & Value \\')
print(r'\midrule')
print(f'Chirp mass & {M_c_example:.1f} $M_\\odot$ \\\\')
print(f'Energy radiated & {E_rad/M_sun/c**2:.1f} $M_\\odot c^2$ \\\\')
print(f'Peak frequency & $\\sim$250 Hz \\\\')
print(f'Peak strain & $\\sim 10^{{-21}}$ \\\\')
print(r'\bottomrule')
print(r'\end{tabular}')
print(r'\end{table}')
\end{pycode}

\section{Conclusions}

Gravitational wave astronomy provides unique insights into compact binary systems and strong-field gravity.

\end{document}
