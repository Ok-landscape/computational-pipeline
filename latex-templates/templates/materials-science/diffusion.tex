\documentclass[11pt,a4paper]{article}

% Document Setup
\usepackage[utf8]{inputenc}
\usepackage[T1]{fontenc}
\usepackage{lmodern}
\usepackage[margin=1in]{geometry}
\usepackage{amsmath,amssymb}
\usepackage{siunitx}
\usepackage{booktabs}
\usepackage{float}
\usepackage{caption}
\usepackage{subcaption}
\usepackage{hyperref}

% PythonTeX Setup
\usepackage[makestderr]{pythontex}

\title{Diffusion in Materials Science: Computational Analysis}
\author{Materials Engineering Laboratory}
\date{\today}

\begin{document}
\maketitle

\begin{abstract}
This report presents computational analysis of diffusion phenomena in materials science. We examine Fick's first and second laws, analytical solutions including error function profiles, numerical simulation of concentration evolution, and the Kirkendall effect in binary diffusion couples. Python-based computations provide quantitative analysis with dynamic visualization.
\end{abstract}

\tableofcontents
\newpage

\section{Introduction to Diffusion}

Diffusion is the net movement of atoms or molecules from regions of high concentration to regions of low concentration. This fundamental transport phenomenon governs numerous materials processes including:
\begin{itemize}
    \item Heat treatment and phase transformations
    \item Oxidation and corrosion mechanisms
    \item Semiconductor doping and device fabrication
    \item Sintering and powder metallurgy
\end{itemize}

% Initialize Python environment
\begin{pycode}
import numpy as np
import matplotlib.pyplot as plt
from scipy.special import erf, erfc
from scipy.integrate import odeint
from mpl_toolkits.mplot3d import Axes3D

plt.rcParams['figure.figsize'] = (8, 5)
plt.rcParams['font.size'] = 10
plt.rcParams['text.usetex'] = True

# Physical constants and parameters
k_B = 8.617e-5  # Boltzmann constant (eV/K)

def save_fig(filename):
    plt.savefig(filename, dpi=150, bbox_inches='tight')
    plt.close()
\end{pycode}

\section{Fick's Laws of Diffusion}

\subsection{Fick's First Law}

Fick's first law describes steady-state diffusion where the flux is proportional to the concentration gradient:
\begin{equation}
J = -D \frac{\partial C}{\partial x}
\end{equation}
where $J$ is the diffusion flux (\si{\mol\per\square\meter\per\second}), $D$ is the diffusion coefficient (\si{\square\meter\per\second}), and $C$ is concentration.

\begin{pycode}
# Steady-state diffusion through a membrane
D = 1e-10  # m^2/s (typical for metals)
L = 0.01  # membrane thickness (m)
C1 = 100  # concentration at x=0 (mol/m^3)
C2 = 10   # concentration at x=L (mol/m^3)

# Calculate steady-state flux
J_ss = -D * (C2 - C1) / L

# Position array
x = np.linspace(0, L*1000, 100)  # convert to mm
C_ss = C1 + (C2 - C1) * x / (L*1000)

fig, (ax1, ax2) = plt.subplots(1, 2, figsize=(10, 4))

# Concentration profile
ax1.plot(x, C_ss, 'b-', linewidth=2)
ax1.set_xlabel('Position (mm)')
ax1.set_ylabel('Concentration (mol/m$^3$)')
ax1.set_title("Steady-State Concentration Profile")
ax1.grid(True, alpha=0.3)
ax1.fill_between(x, 0, C_ss, alpha=0.3)

# Flux visualization
x_flux = np.linspace(0, L*1000, 10)
ax2.quiver(x_flux, np.ones_like(x_flux)*50, np.ones_like(x_flux)*2,
           np.zeros_like(x_flux), scale=30, color='red', width=0.01)
ax2.set_xlim(0, L*1000)
ax2.set_ylim(0, 100)
ax2.set_xlabel('Position (mm)')
ax2.set_ylabel('Concentration (mol/m$^3$)')
ax2.set_title(f'Diffusion Flux: J = {J_ss:.2e} mol/m$^2$/s')
ax2.grid(True, alpha=0.3)

plt.tight_layout()
save_fig('ficks_first_law.pdf')
\end{pycode}

\begin{figure}[H]
\centering
\includegraphics[width=\textwidth]{ficks_first_law.pdf}
\caption{Fick's first law: (a) steady-state concentration profile, (b) constant flux through membrane.}
\end{figure}

The calculated steady-state flux is $J = \py{f"{J_ss:.3e}"}$ \si{\mol\per\square\meter\per\second}.

\subsection{Fick's Second Law}

For non-steady-state diffusion, Fick's second law describes the time evolution of concentration:
\begin{equation}
\frac{\partial C}{\partial t} = D \frac{\partial^2 C}{\partial x^2}
\end{equation}

\section{Analytical Solutions}

\subsection{Error Function Solutions}

For semi-infinite solid with constant surface concentration:
\begin{equation}
\frac{C(x,t) - C_0}{C_s - C_0} = 1 - \text{erf}\left(\frac{x}{2\sqrt{Dt}}\right)
\end{equation}

\begin{pycode}
# Error function solution parameters
D = 5e-11  # m^2/s
C0 = 0     # Initial concentration
Cs = 1     # Surface concentration (normalized)

# Time points
times = [1, 10, 100, 1000]  # hours
colors = plt.cm.viridis(np.linspace(0, 1, len(times)))

x = np.linspace(0, 5e-3, 200)  # depth in meters

fig, (ax1, ax2) = plt.subplots(1, 2, figsize=(12, 5))

# Concentration profiles at different times
for t_hr, color in zip(times, colors):
    t = t_hr * 3600  # convert to seconds
    C = Cs * erfc(x / (2 * np.sqrt(D * t)))
    ax1.plot(x*1000, C, color=color, linewidth=2, label=f't = {t_hr} hr')

ax1.set_xlabel('Depth (mm)')
ax1.set_ylabel('Normalized Concentration $C/C_s$')
ax1.set_title('Concentration Profiles (Error Function Solution)')
ax1.legend()
ax1.grid(True, alpha=0.3)

# Diffusion penetration depth vs time
t_range = np.linspace(1, 1000, 100) * 3600
penetration = 2 * np.sqrt(D * t_range)

ax2.plot(t_range/3600, penetration*1000, 'b-', linewidth=2)
ax2.set_xlabel('Time (hours)')
ax2.set_ylabel('Penetration Depth $2\\sqrt{Dt}$ (mm)')
ax2.set_title('Diffusion Penetration Depth')
ax2.grid(True, alpha=0.3)

plt.tight_layout()
save_fig('error_function_solution.pdf')
\end{pycode}

\begin{figure}[H]
\centering
\includegraphics[width=\textwidth]{error_function_solution.pdf}
\caption{Error function solution showing concentration profiles at different times and penetration depth evolution.}
\end{figure}

\subsection{Thin-Film (Gaussian) Solution}

For an instantaneous planar source at $x=0$:
\begin{equation}
C(x,t) = \frac{M}{\sqrt{4\pi Dt}} \exp\left(-\frac{x^2}{4Dt}\right)
\end{equation}

\begin{pycode}
# Gaussian solution parameters
D = 1e-10  # m^2/s
M = 1e-6   # surface mass (mol/m^2)

times = [10, 100, 1000, 10000]  # seconds
x = np.linspace(-2e-3, 2e-3, 300)

fig, ax = plt.subplots(figsize=(9, 6))
colors = plt.cm.plasma(np.linspace(0.2, 0.9, len(times)))

for t, color in zip(times, colors):
    C = (M / np.sqrt(4 * np.pi * D * t)) * np.exp(-x**2 / (4 * D * t))
    ax.plot(x*1000, C*1e6, color=color, linewidth=2, label=f't = {t} s')

ax.set_xlabel('Position (mm)')
ax.set_ylabel('Concentration ($\\mu$mol/m$^3$)')
ax.set_title('Gaussian Diffusion from Thin-Film Source')
ax.legend()
ax.grid(True, alpha=0.3)

save_fig('gaussian_solution.pdf')
\end{pycode}

\begin{figure}[H]
\centering
\includegraphics[width=0.8\textwidth]{gaussian_solution.pdf}
\caption{Gaussian diffusion profile evolution from an instantaneous thin-film source.}
\end{figure}

\section{Temperature Dependence of Diffusion}

The diffusion coefficient follows the Arrhenius relationship:
\begin{equation}
D = D_0 \exp\left(-\frac{Q}{RT}\right)
\end{equation}

\begin{pycode}
# Arrhenius parameters for various systems
systems = {
    'C in Fe-$\\alpha$': {'D0': 6.2e-7, 'Q': 80},
    'C in Fe-$\\gamma$': {'D0': 2.3e-5, 'Q': 148},
    'Fe in Fe-$\\alpha$': {'D0': 2.0e-4, 'Q': 251},
    'Cu in Cu': {'D0': 7.8e-5, 'Q': 211},
    'Al in Al': {'D0': 1.7e-4, 'Q': 142}
}

R = 8.314  # J/(mol*K)
T = np.linspace(300, 1400, 200)

fig, (ax1, ax2) = plt.subplots(1, 2, figsize=(12, 5))

# Arrhenius plot
for name, params in systems.items():
    D = params['D0'] * np.exp(-params['Q']*1000 / (R * T))
    ax1.semilogy(1000/T, D, linewidth=2, label=name)

ax1.set_xlabel('1000/T (K$^{-1}$)')
ax1.set_ylabel('D (m$^2$/s)')
ax1.set_title('Arrhenius Plot of Diffusion Coefficients')
ax1.legend(fontsize=8)
ax1.grid(True, alpha=0.3)

# Q vs D0 comparison
D0_vals = [p['D0'] for p in systems.values()]
Q_vals = [p['Q'] for p in systems.values()]
names = list(systems.keys())

ax2.scatter(D0_vals, Q_vals, s=100, c=range(len(systems)), cmap='viridis')
for i, name in enumerate(names):
    ax2.annotate(name.replace('$\\alpha$', 'a').replace('$\\gamma$', 'g'),
                 (D0_vals[i], Q_vals[i]), xytext=(5, 5),
                 textcoords='offset points', fontsize=8)
ax2.set_xscale('log')
ax2.set_xlabel('$D_0$ (m$^2$/s)')
ax2.set_ylabel('Q (kJ/mol)')
ax2.set_title('Activation Energy vs Pre-exponential Factor')
ax2.grid(True, alpha=0.3)

plt.tight_layout()
save_fig('arrhenius_diffusion.pdf')

# Calculate specific values for table
T_test = 1000  # K
D_table = []
for name, params in systems.items():
    D = params['D0'] * np.exp(-params['Q']*1000 / (R * T_test))
    D_table.append((name.replace('$\\alpha$', 'alpha').replace('$\\gamma$', 'gamma'),
                   params['D0'], params['Q'], D))
\end{pycode}

\begin{figure}[H]
\centering
\includegraphics[width=\textwidth]{arrhenius_diffusion.pdf}
\caption{Temperature dependence of diffusion: Arrhenius behavior for various material systems.}
\end{figure}

\begin{table}[H]
\centering
\caption{Diffusion Parameters for Various Systems}
\begin{tabular}{lccc}
\toprule
System & $D_0$ (\si{\square\meter\per\second}) & Q (\si{\kilo\joule\per\mol}) & D at 1000 K \\
\midrule
\py{'C in Fe-alpha'} & \py{f"{6.2e-7:.2e}"} & 80 & \py{f"{6.2e-7 * np.exp(-80000/(8.314*1000)):.2e}"} \\
\py{'C in Fe-gamma'} & \py{f"{2.3e-5:.2e}"} & 148 & \py{f"{2.3e-5 * np.exp(-148000/(8.314*1000)):.2e}"} \\
\py{'Fe in Fe-alpha'} & \py{f"{2.0e-4:.2e}"} & 251 & \py{f"{2.0e-4 * np.exp(-251000/(8.314*1000)):.2e}"} \\
\py{'Cu in Cu'} & \py{f"{7.8e-5:.2e}"} & 211 & \py{f"{7.8e-5 * np.exp(-211000/(8.314*1000)):.2e}"} \\
\py{'Al in Al'} & \py{f"{1.7e-4:.2e}"} & 142 & \py{f"{1.7e-4 * np.exp(-142000/(8.314*1000)):.2e}"} \\
\bottomrule
\end{tabular}
\end{table}

\section{Numerical Simulation of Diffusion}

\subsection{Finite Difference Method}

We solve Fick's second law numerically using explicit finite differences:
\begin{equation}
C_i^{n+1} = C_i^n + \frac{D \Delta t}{(\Delta x)^2}\left(C_{i+1}^n - 2C_i^n + C_{i-1}^n\right)
\end{equation}

\begin{pycode}
# Numerical diffusion simulation
D = 1e-10  # m^2/s
L = 0.01   # domain length (m)
nx = 100   # spatial points
dx = L / (nx - 1)
dt = 0.5 * dx**2 / D  # stability criterion

# Initial condition: step function
x = np.linspace(0, L, nx)
C = np.zeros(nx)
C[x < L/2] = 1.0

# Store profiles at different times
t_save = [0, 1000, 5000, 20000]
profiles = {0: C.copy()}

# Time evolution
t = 0
for step in range(20001):
    C_new = C.copy()
    for i in range(1, nx-1):
        C_new[i] = C[i] + D * dt / dx**2 * (C[i+1] - 2*C[i] + C[i-1])
    C = C_new
    t += dt
    if step in t_save[1:]:
        profiles[step] = C.copy()

# Plot results
fig, (ax1, ax2) = plt.subplots(1, 2, figsize=(12, 5))

colors = plt.cm.viridis(np.linspace(0, 1, len(t_save)))
for step, color in zip(t_save, colors):
    t_actual = step * dt
    ax1.plot(x*1000, profiles[step], color=color, linewidth=2,
             label=f't = {t_actual:.0f} s')

ax1.set_xlabel('Position (mm)')
ax1.set_ylabel('Concentration (normalized)')
ax1.set_title('Numerical Diffusion Simulation')
ax1.legend()
ax1.grid(True, alpha=0.3)

# Conservation of mass check
mass = [np.trapz(profiles[step], x) for step in t_save]
steps_plot = [s * dt for s in t_save]
ax2.plot(steps_plot, mass, 'bo-', linewidth=2, markersize=8)
ax2.axhline(mass[0], color='r', linestyle='--', label='Initial mass')
ax2.set_xlabel('Time (s)')
ax2.set_ylabel('Total Mass (mol/m$^2$)')
ax2.set_title('Mass Conservation Check')
ax2.legend()
ax2.grid(True, alpha=0.3)

plt.tight_layout()
save_fig('numerical_diffusion.pdf')

# Stability parameter
Fo = D * dt / dx**2
\end{pycode}

\begin{figure}[H]
\centering
\includegraphics[width=\textwidth]{numerical_diffusion.pdf}
\caption{Finite difference solution of Fick's second law with mass conservation verification.}
\end{figure}

The Fourier number (stability criterion) is $Fo = \py{f"{Fo:.3f}"}$, which satisfies $Fo \leq 0.5$ for stability.

\section{Kirkendall Effect}

The Kirkendall effect demonstrates unequal diffusion rates in binary diffusion couples, resulting in marker movement and void formation.

\begin{pycode}
# Kirkendall effect simulation
# Binary diffusion couple: A-B system
L = 0.02   # 20 mm total
nx = 200
x = np.linspace(0, L, nx)
dx = L / (nx - 1)

# Diffusion coefficients (A diffuses faster than B)
D_A = 2e-10  # m^2/s
D_B = 5e-11  # m^2/s

# Initial concentrations
C_A = np.zeros(nx)
C_B = np.zeros(nx)
C_A[x < L/2] = 1.0  # Pure A on left
C_B[x >= L/2] = 1.0  # Pure B on right

# Time stepping
dt = 0.25 * dx**2 / max(D_A, D_B)
n_steps = 10000

# Track marker position (initially at interface)
marker_pos = L/2

# Store results
times = [0, 2500, 5000, 10000]
results = {0: {'C_A': C_A.copy(), 'C_B': C_B.copy(), 'marker': marker_pos}}

for step in range(1, n_steps+1):
    # Update A
    C_A_new = C_A.copy()
    for i in range(1, nx-1):
        C_A_new[i] = C_A[i] + D_A * dt / dx**2 * (C_A[i+1] - 2*C_A[i] + C_A[i-1])

    # Update B
    C_B_new = C_B.copy()
    for i in range(1, nx-1):
        C_B_new[i] = C_B[i] + D_B * dt / dx**2 * (C_B[i+1] - 2*C_B[i] + C_B[i-1])

    C_A = C_A_new
    C_B = C_B_new

    # Calculate marker velocity (proportional to flux difference)
    i_marker = int(marker_pos / dx)
    if 0 < i_marker < nx-1:
        J_A = -D_A * (C_A[i_marker+1] - C_A[i_marker-1]) / (2*dx)
        J_B = -D_B * (C_B[i_marker+1] - C_B[i_marker-1]) / (2*dx)
        v_marker = -(J_A - J_B) / (C_A[i_marker] + C_B[i_marker] + 1e-10)
        marker_pos += v_marker * dt * 1000  # scale for visibility

    if step in times:
        results[step] = {'C_A': C_A.copy(), 'C_B': C_B.copy(), 'marker': marker_pos}

# Plot results
fig, axes = plt.subplots(2, 2, figsize=(12, 10))

for idx, (step, ax) in enumerate(zip(times, axes.flat)):
    data = results[step]
    t = step * dt
    ax.plot(x*1000, data['C_A'], 'b-', linewidth=2, label='Species A')
    ax.plot(x*1000, data['C_B'], 'r-', linewidth=2, label='Species B')
    ax.axvline(L*500, color='k', linestyle='--', alpha=0.5, label='Initial interface')
    ax.axvline(data['marker']*1000, color='g', linestyle='-', linewidth=2, label='Marker')
    ax.set_xlabel('Position (mm)')
    ax.set_ylabel('Concentration')
    ax.set_title(f't = {t:.0f} s')
    ax.legend(fontsize=8)
    ax.grid(True, alpha=0.3)

plt.tight_layout()
save_fig('kirkendall_effect.pdf')

marker_shift = (results[n_steps]['marker'] - L/2) * 1000
\end{pycode}

\begin{figure}[H]
\centering
\includegraphics[width=\textwidth]{kirkendall_effect.pdf}
\caption{Kirkendall effect in a binary diffusion couple showing marker movement due to unequal diffusivities.}
\end{figure}

The marker shifted by \py{f"{marker_shift:.3f}"} mm toward the faster-diffusing side, demonstrating the Kirkendall effect.

\section{Interdiffusion Coefficient}

The interdiffusion (chemical) coefficient can be determined using the Matano-Boltzmann analysis:
\begin{equation}
\tilde{D} = -\frac{1}{2t}\left(\frac{dx}{dC}\right)_{C^*} \int_0^{C^*} x \, dC
\end{equation}

\begin{pycode}
# Matano-Boltzmann analysis
# Generate synthetic interdiffusion profile
x = np.linspace(-5, 5, 1000)  # mm
t = 3600  # 1 hour

# Interdiffusion profile (error function)
D_tilde = 1e-10  # m^2/s
C = 0.5 * erfc(x*1e-3 / (2*np.sqrt(D_tilde * t)))

# Matano analysis at different concentrations
C_star = [0.2, 0.4, 0.6, 0.8]
D_calc = []

fig, (ax1, ax2) = plt.subplots(1, 2, figsize=(12, 5))

ax1.plot(x, C, 'b-', linewidth=2)
ax1.set_xlabel('Position (mm)')
ax1.set_ylabel('Concentration')
ax1.set_title('Interdiffusion Profile')
ax1.grid(True, alpha=0.3)

for C_s in C_star:
    # Find position at C*
    idx = np.argmin(np.abs(C - C_s))
    x_s = x[idx]

    # Calculate dx/dC
    dCdx = np.gradient(C, x*1e-3)[idx]
    dxdC = 1/dCdx if dCdx != 0 else 0

    # Integrate
    integral = np.trapz(x[:idx]*1e-3, C[:idx])

    # Calculate D
    D = -dxdC * integral / (2*t)
    D_calc.append(D)

    ax1.axhline(C_s, color='r', linestyle='--', alpha=0.5)
    ax1.plot(x_s, C_s, 'ro', markersize=8)

ax2.plot(C_star, [d*1e10 for d in D_calc], 'bo-', linewidth=2, markersize=10)
ax2.axhline(D_tilde*1e10, color='r', linestyle='--', label=f'True D = {D_tilde:.0e}')
ax2.set_xlabel('Concentration')
ax2.set_ylabel('$\\tilde{D}$ ($\\times 10^{-10}$ m$^2$/s)')
ax2.set_title('Matano-Boltzmann Analysis')
ax2.legend()
ax2.grid(True, alpha=0.3)

plt.tight_layout()
save_fig('matano_analysis.pdf')
\end{pycode}

\begin{figure}[H]
\centering
\includegraphics[width=\textwidth]{matano_analysis.pdf}
\caption{Matano-Boltzmann analysis for determining concentration-dependent interdiffusion coefficient.}
\end{figure}

\section{3D Diffusion Visualization}

\begin{pycode}
# 3D visualization of diffusion evolution
D = 1e-10
x = np.linspace(0, 10, 100)
t = np.linspace(0.1, 100, 100)
X, T = np.meshgrid(x, t)

# Concentration field C(x,t)
C = erfc(X*1e-3 / (2*np.sqrt(D * T * 3600)))

fig = plt.figure(figsize=(10, 7))
ax = fig.add_subplot(111, projection='3d')

surf = ax.plot_surface(X, T, C, cmap='viridis', alpha=0.8)
ax.set_xlabel('Position (mm)')
ax.set_ylabel('Time (hours)')
ax.set_zlabel('Concentration')
ax.set_title('3D Visualization of Diffusion Evolution')

fig.colorbar(surf, shrink=0.5, aspect=10, label='Concentration')

save_fig('diffusion_3d.pdf')
\end{pycode}

\begin{figure}[H]
\centering
\includegraphics[width=0.8\textwidth]{diffusion_3d.pdf}
\caption{Three-dimensional visualization of concentration evolution during diffusion.}
\end{figure}

\section{Conclusions}

This analysis demonstrates key aspects of diffusion in materials science:
\begin{enumerate}
    \item Fick's laws provide the mathematical framework for both steady-state and transient diffusion
    \item Error function solutions enable analytical calculation of penetration profiles
    \item Temperature dependence follows Arrhenius behavior with characteristic activation energies
    \item Numerical methods allow simulation of complex boundary conditions
    \item The Kirkendall effect demonstrates the physical consequences of unequal diffusivities
    \item Matano-Boltzmann analysis enables experimental determination of diffusion coefficients
\end{enumerate}

\end{document}
