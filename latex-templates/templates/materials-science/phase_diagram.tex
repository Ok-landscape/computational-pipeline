\documentclass[11pt,a4paper]{article}

% Document Setup
\usepackage[utf8]{inputenc}
\usepackage[T1]{fontenc}
\usepackage{lmodern}
\usepackage[margin=1in]{geometry}
\usepackage{amsmath,amssymb}
\usepackage{siunitx}
\usepackage{booktabs}
\usepackage{float}
\usepackage{caption}
\usepackage{hyperref}

% PythonTeX Setup
\usepackage[makestderr]{pythontex}

\title{Binary Phase Diagrams: Computational Analysis}
\author{Materials Engineering Laboratory}
\date{\today}

\begin{document}
\maketitle

\begin{abstract}
This report presents computational analysis of binary phase diagrams in materials science. We examine isomorphous and eutectic systems, the lever rule for phase fraction calculations, cooling curve analysis, and Gibbs phase rule applications. Python-based computations provide quantitative analysis with dynamic visualization.
\end{abstract}

\tableofcontents
\newpage

\section{Introduction to Phase Diagrams}

Phase diagrams are graphical representations of the equilibrium phases present in a material system as a function of temperature, pressure, and composition. They are essential for:
\begin{itemize}
    \item Predicting microstructure evolution during solidification
    \item Designing heat treatment processes
    \item Understanding alloy behavior and properties
    \item Optimizing casting and welding procedures
\end{itemize}

% Initialize Python environment
\begin{pycode}
import numpy as np
import matplotlib.pyplot as plt
from scipy.optimize import fsolve
from scipy.interpolate import interp1d

plt.rcParams['figure.figsize'] = (8, 5)
plt.rcParams['font.size'] = 10
plt.rcParams['text.usetex'] = True

def save_fig(filename):
    plt.savefig(filename, dpi=150, bbox_inches='tight')
    plt.close()
\end{pycode}

\section{Gibbs Phase Rule}

The Gibbs phase rule determines the degrees of freedom in a system:
\begin{equation}
F = C - P + 2
\end{equation}
where $F$ is degrees of freedom, $C$ is number of components, and $P$ is number of phases.

For a binary system ($C=2$) at constant pressure:
\begin{equation}
F = 3 - P
\end{equation}

\begin{pycode}
# Gibbs phase rule visualization
phases = [1, 2, 3]
freedom = [3 - p for p in phases]

fig, ax = plt.subplots(figsize=(8, 5))

bars = ax.bar(phases, freedom, color=['green', 'blue', 'red'], alpha=0.7)
ax.set_xlabel('Number of Phases (P)')
ax.set_ylabel('Degrees of Freedom (F)')
ax.set_title('Gibbs Phase Rule for Binary System at Constant Pressure')
ax.set_xticks(phases)

for bar, f in zip(bars, freedom):
    ax.text(bar.get_x() + bar.get_width()/2, bar.get_height() + 0.1,
            f'F = {f}', ha='center', fontsize=12)

ax.grid(True, alpha=0.3, axis='y')
save_fig('gibbs_phase_rule.pdf')
\end{pycode}

\begin{figure}[H]
\centering
\includegraphics[width=0.7\textwidth]{gibbs_phase_rule.pdf}
\caption{Gibbs phase rule showing degrees of freedom for different numbers of phases.}
\end{figure}

\section{Isomorphous System}

An isomorphous system exhibits complete solid solubility. The Cu-Ni system is a classic example.

\begin{pycode}
# Cu-Ni isomorphous phase diagram
# Liquidus and solidus curves (simplified model)
x_Ni = np.linspace(0, 100, 200)  # wt% Ni

# Melting points: Cu = 1085 C, Ni = 1455 C
T_Cu = 1085
T_Ni = 1455

# Liquidus curve (quadratic model)
T_liquidus = T_Cu + (T_Ni - T_Cu) * (x_Ni/100) + 20 * (x_Ni/100) * (1 - x_Ni/100)

# Solidus curve (offset from liquidus)
T_solidus = T_Cu + (T_Ni - T_Cu) * (x_Ni/100) - 30 * (x_Ni/100) * (1 - x_Ni/100)

fig, ax = plt.subplots(figsize=(10, 7))

ax.plot(x_Ni, T_liquidus, 'b-', linewidth=2, label='Liquidus')
ax.plot(x_Ni, T_solidus, 'r-', linewidth=2, label='Solidus')

# Fill regions
ax.fill_between(x_Ni, T_liquidus, 1500, alpha=0.3, color='yellow', label='Liquid (L)')
ax.fill_between(x_Ni, T_solidus, 900, alpha=0.3, color='lightblue', label='Solid ($\\alpha$)')
ax.fill_between(x_Ni, T_solidus, T_liquidus, alpha=0.3, color='lightgreen', label='L + $\\alpha$')

# Tie line example
x_0 = 40  # Overall composition
T_tie = 1250
idx_liq = np.argmin(np.abs(T_liquidus - T_tie))
idx_sol = np.argmin(np.abs(T_solidus - T_tie))
x_L = x_Ni[idx_liq]
x_alpha = x_Ni[idx_sol]

ax.plot([x_L, x_alpha], [T_tie, T_tie], 'k-', linewidth=2)
ax.plot(x_0, T_tie, 'ko', markersize=10)
ax.plot(x_L, T_tie, 'bo', markersize=8)
ax.plot(x_alpha, T_tie, 'ro', markersize=8)

ax.annotate(f'$C_0$ = {x_0}\\%', (x_0, T_tie), xytext=(x_0+5, T_tie+30),
            fontsize=10, arrowprops=dict(arrowstyle='->', color='black'))

ax.set_xlabel('Composition (wt\\% Ni)')
ax.set_ylabel('Temperature ($^\\circ$C)')
ax.set_title('Cu-Ni Isomorphous Phase Diagram')
ax.legend(loc='upper left')
ax.set_xlim(0, 100)
ax.set_ylim(900, 1500)
ax.grid(True, alpha=0.3)

save_fig('isomorphous_diagram.pdf')
\end{pycode}

\begin{figure}[H]
\centering
\includegraphics[width=\textwidth]{isomorphous_diagram.pdf}
\caption{Cu-Ni isomorphous phase diagram with tie line construction at $T = 1250^{\circ}$C.}
\end{figure}

\section{The Lever Rule}

The lever rule calculates phase fractions in a two-phase region:
\begin{equation}
W_{\alpha} = \frac{C_L - C_0}{C_L - C_{\alpha}} \quad \text{and} \quad W_L = \frac{C_0 - C_{\alpha}}{C_L - C_{\alpha}}
\end{equation}

\begin{pycode}
# Lever rule calculation
C_0 = 40  # Overall composition
T_analysis = np.linspace(T_solidus[int(C_0*2)], T_liquidus[int(C_0*2)], 50)

# Interpolate liquidus and solidus
f_liquidus = interp1d(T_liquidus, x_Ni)
f_solidus = interp1d(T_solidus, x_Ni)

W_liquid = []
W_solid = []
temps = []

for T in T_analysis:
    # Find compositions at this temperature
    try:
        C_L = f_liquidus(T)
        C_alpha = f_solidus(T)
        if C_L > C_alpha:
            w_L = (C_0 - C_alpha) / (C_L - C_alpha)
            w_alpha = (C_L - C_0) / (C_L - C_alpha)
            W_liquid.append(w_L * 100)
            W_solid.append(w_alpha * 100)
            temps.append(T)
    except:
        pass

fig, (ax1, ax2) = plt.subplots(1, 2, figsize=(12, 5))

# Phase fractions vs temperature
ax1.plot(temps, W_liquid, 'b-', linewidth=2, label='Liquid')
ax1.plot(temps, W_solid, 'r-', linewidth=2, label='Solid ($\\alpha$)')
ax1.set_xlabel('Temperature ($^\\circ$C)')
ax1.set_ylabel('Phase Fraction (\\%)')
ax1.set_title(f'Phase Fractions during Cooling ($C_0$ = {C_0}\\% Ni)')
ax1.legend()
ax1.grid(True, alpha=0.3)

# Lever rule visualization
T_demo = 1250
C_L_demo = 35
C_alpha_demo = 48
C_0_demo = 40

ax2.plot([0, 100], [0, 0], 'k-', linewidth=3)
ax2.plot(C_L_demo, 0, 'b^', markersize=15, label=f'Liquid ({C_L_demo}\\%)')
ax2.plot(C_alpha_demo, 0, 'rs', markersize=15, label=f'Solid ({C_alpha_demo}\\%)')
ax2.plot(C_0_demo, 0, 'go', markersize=12, label=f'Overall ({C_0_demo}\\%)')

# Lever arms
ax2.annotate('', xy=(C_L_demo, 0.2), xytext=(C_0_demo, 0.2),
            arrowprops=dict(arrowstyle='<->', color='blue', lw=2))
ax2.text((C_L_demo+C_0_demo)/2, 0.3, f'{C_0_demo-C_L_demo}', ha='center', fontsize=12, color='blue')

ax2.annotate('', xy=(C_0_demo, -0.2), xytext=(C_alpha_demo, -0.2),
            arrowprops=dict(arrowstyle='<->', color='red', lw=2))
ax2.text((C_0_demo+C_alpha_demo)/2, -0.3, f'{C_alpha_demo-C_0_demo}', ha='center', fontsize=12, color='red')

W_L_calc = (C_0_demo - C_L_demo)/(C_alpha_demo - C_L_demo) * 100
W_alpha_calc = (C_alpha_demo - C_0_demo)/(C_alpha_demo - C_L_demo) * 100

ax2.set_xlim(20, 60)
ax2.set_ylim(-0.6, 0.6)
ax2.set_xlabel('Composition (wt\\% Ni)')
ax2.set_title(f'Lever Rule: $W_L$ = {W_L_calc:.1f}\\%, $W_\\alpha$ = {W_alpha_calc:.1f}\\%')
ax2.legend(loc='upper right')
ax2.set_yticks([])

save_fig('lever_rule.pdf')
\end{pycode}

\begin{figure}[H]
\centering
\includegraphics[width=\textwidth]{lever_rule.pdf}
\caption{Lever rule analysis: phase fractions during cooling and graphical representation.}
\end{figure}

\section{Eutectic System}

A eutectic system has limited solid solubility. The Pb-Sn system is a classic example.

\begin{pycode}
# Pb-Sn eutectic phase diagram
x_Sn = np.linspace(0, 100, 500)

# Key points
T_Pb = 327  # Melting point of Pb
T_Sn = 232  # Melting point of Sn
T_E = 183   # Eutectic temperature
x_E = 61.9  # Eutectic composition
x_alpha_max = 19  # Max solubility in alpha
x_beta_max = 97.5  # Max solubility in beta

# Liquidus curves
T_liq_left = T_Pb - (T_Pb - T_E) * (x_Sn / x_E)**1.2
T_liq_right = T_Sn - (T_Sn - T_E) * ((100 - x_Sn) / (100 - x_E))**1.2
T_liquidus = np.where(x_Sn <= x_E, T_liq_left, T_liq_right)

# Solidus curves (solvus)
T_sol_alpha = T_E + (T_Pb - T_E) * ((x_alpha_max - x_Sn) / x_alpha_max)**2
T_sol_beta = T_E + (T_Sn - T_E) * ((x_Sn - x_beta_max) / (100 - x_beta_max))**2

fig, ax = plt.subplots(figsize=(10, 8))

# Liquidus
ax.plot(x_Sn, T_liquidus, 'b-', linewidth=2.5, label='Liquidus')

# Solidus (solvus) lines
x_alpha = x_Sn[x_Sn <= x_alpha_max]
ax.plot(x_alpha, T_sol_alpha[x_Sn <= x_alpha_max], 'r-', linewidth=2)
x_beta = x_Sn[x_Sn >= x_beta_max]
ax.plot(x_beta, T_sol_beta[x_Sn >= x_beta_max], 'r-', linewidth=2)

# Eutectic isotherm
ax.plot([x_alpha_max, x_beta_max], [T_E, T_E], 'k-', linewidth=2)

# Labels for regions
ax.text(10, 280, 'L', fontsize=14, fontweight='bold')
ax.text(8, 100, '$\\alpha$', fontsize=14, fontweight='bold')
ax.text(88, 100, '$\\beta$', fontsize=14, fontweight='bold')
ax.text(50, 100, '$\\alpha + \\beta$', fontsize=14, fontweight='bold')
ax.text(30, 220, 'L + $\\alpha$', fontsize=12)
ax.text(75, 210, 'L + $\\beta$', fontsize=12)

# Eutectic point
ax.plot(x_E, T_E, 'ko', markersize=10)
ax.annotate(f'Eutectic\\n({x_E}\\%, {T_E}$^\\circ$C)',
            (x_E, T_E), xytext=(x_E-20, T_E-40), fontsize=10)

ax.set_xlabel('Composition (wt\\% Sn)')
ax.set_ylabel('Temperature ($^\\circ$C)')
ax.set_title('Pb-Sn Eutectic Phase Diagram')
ax.set_xlim(0, 100)
ax.set_ylim(0, 350)
ax.grid(True, alpha=0.3)

save_fig('eutectic_diagram.pdf')
\end{pycode}

\begin{figure}[H]
\centering
\includegraphics[width=\textwidth]{eutectic_diagram.pdf}
\caption{Pb-Sn eutectic phase diagram showing liquid, solid, and two-phase regions.}
\end{figure}

\section{Cooling Curves}

Cooling curves reveal phase transformations through thermal arrests.

\begin{pycode}
# Cooling curves for different compositions
compositions = [10, 40, 61.9, 80]  # wt% Sn
colors = ['blue', 'green', 'red', 'purple']

fig, axes = plt.subplots(2, 2, figsize=(12, 10))

for comp, color, ax in zip(compositions, colors, axes.flat):
    # Generate cooling curve
    time = np.linspace(0, 600, 1000)

    if comp < x_E:  # Hypoeutectic
        T_start = 350
        T_liq = T_Pb - (T_Pb - T_E) * (comp / x_E)**1.2

        # Cooling stages
        t1 = 100  # Start of solidification
        t2 = 300  # End of primary solidification
        t3 = 400  # End of eutectic solidification

        T = np.piecewise(time,
            [time < t1, (time >= t1) & (time < t2), (time >= t2) & (time < t3), time >= t3],
            [lambda t: T_start - (T_start - T_liq) * t/t1,
             lambda t: T_liq - (T_liq - T_E) * (t - t1)/(t2 - t1) * 0.8,
             T_E,
             lambda t: T_E - (t - t3) * 0.3])

    elif comp > x_E:  # Hypereutectic
        T_start = 280
        T_liq = T_Sn - (T_Sn - T_E) * ((100 - comp) / (100 - x_E))**1.2

        t1 = 80
        t2 = 250
        t3 = 350

        T = np.piecewise(time,
            [time < t1, (time >= t1) & (time < t2), (time >= t2) & (time < t3), time >= t3],
            [lambda t: T_start - (T_start - T_liq) * t/t1,
             lambda t: T_liq - (T_liq - T_E) * (t - t1)/(t2 - t1) * 0.8,
             T_E,
             lambda t: T_E - (t - t3) * 0.3])

    else:  # Eutectic
        T_start = 220
        t1 = 50
        t2 = 300

        T = np.piecewise(time,
            [time < t1, (time >= t1) & (time < t2), time >= t2],
            [lambda t: T_start - (T_start - T_E) * t/t1,
             T_E,
             lambda t: T_E - (t - t2) * 0.3])

    ax.plot(time, T, color=color, linewidth=2)
    ax.axhline(T_E, color='k', linestyle='--', alpha=0.5)
    ax.set_xlabel('Time (s)')
    ax.set_ylabel('Temperature ($^\\circ$C)')
    ax.set_title(f'{comp}\\% Sn')
    ax.grid(True, alpha=0.3)

    # Annotate thermal arrests
    if comp == 61.9:
        ax.annotate('Eutectic arrest', (175, T_E), xytext=(300, T_E+30),
                   arrowprops=dict(arrowstyle='->', color='black'))

plt.suptitle('Cooling Curves for Pb-Sn Alloys', fontsize=14, y=1.02)
plt.tight_layout()
save_fig('cooling_curves.pdf')
\end{pycode}

\begin{figure}[H]
\centering
\includegraphics[width=\textwidth]{cooling_curves.pdf}
\caption{Cooling curves for hypoeutectic (10\%, 40\%), eutectic (61.9\%), and hypereutectic (80\%) Pb-Sn alloys.}
\end{figure}

\section{Microstructure Evolution}

\begin{pycode}
# Microstructure composition analysis
# For 40 wt% Sn alloy at different stages

C_0 = 40  # Overall composition
stages = ['Just below liquidus', 'Above eutectic', 'Below eutectic']
T_stages = [260, 190, 100]

fig, axes = plt.subplots(1, 3, figsize=(14, 4))

for ax, stage, T in zip(axes, stages, T_stages):
    if T > T_E:
        # Two-phase: L + alpha
        W_alpha = 60  # Approximate
        W_L = 40
        sizes = [W_alpha, W_L]
        labels = ['$\\alpha$', 'L']
        colors_pie = ['lightblue', 'yellow']
    else:
        # Three phases present (but at equilibrium: alpha + beta)
        # Primary alpha + eutectic mixture
        W_primary = 55
        W_eutectic = 45
        sizes = [W_primary, W_eutectic]
        labels = ['Primary $\\alpha$', 'Eutectic ($\\alpha + \\beta$)']
        colors_pie = ['lightblue', 'lightgreen']

    ax.pie(sizes, labels=labels, colors=colors_pie, autopct='%1.1f%%',
           startangle=90)
    ax.set_title(f'{stage}\\n(T = {T}$^\\circ$C)')

plt.suptitle(f'Microstructure Evolution for {C_0}\\% Sn Alloy', fontsize=12)
plt.tight_layout()
save_fig('microstructure_evolution.pdf')
\end{pycode}

\begin{figure}[H]
\centering
\includegraphics[width=\textwidth]{microstructure_evolution.pdf}
\caption{Microstructure evolution during cooling of 40\% Sn alloy.}
\end{figure}

\section{Phase Fraction Calculations}

\begin{pycode}
# Complete phase fraction analysis at room temperature
compositions = np.linspace(0, 100, 100)
T_room = 25

# At room temperature: alpha + beta region
# Alpha: ~2% Sn, Beta: ~99% Sn
C_alpha = 2
C_beta = 99

W_alpha_arr = (C_beta - compositions) / (C_beta - C_alpha) * 100
W_beta_arr = (compositions - C_alpha) / (C_beta - C_alpha) * 100

fig, (ax1, ax2) = plt.subplots(1, 2, figsize=(12, 5))

ax1.plot(compositions, W_alpha_arr, 'b-', linewidth=2, label='$\\alpha$ phase')
ax1.plot(compositions, W_beta_arr, 'r-', linewidth=2, label='$\\beta$ phase')
ax1.axvline(x_E, color='k', linestyle='--', alpha=0.5, label='Eutectic')
ax1.set_xlabel('Composition (wt\\% Sn)')
ax1.set_ylabel('Phase Fraction (\\%)')
ax1.set_title('Equilibrium Phase Fractions at Room Temperature')
ax1.legend()
ax1.grid(True, alpha=0.3)

# Primary vs eutectic fractions
W_primary_alpha = np.maximum(0, (x_E - compositions) / (x_E - x_alpha_max) * 100)
W_primary_beta = np.maximum(0, (compositions - x_E) / (x_beta_max - x_E) * 100)
W_eutectic = 100 - W_primary_alpha - W_primary_beta

ax2.fill_between(compositions, 0, W_primary_alpha, alpha=0.5, label='Primary $\\alpha$')
ax2.fill_between(compositions, W_primary_alpha, W_primary_alpha + W_eutectic,
                 alpha=0.5, label='Eutectic')
ax2.fill_between(compositions, W_primary_alpha + W_eutectic, 100,
                 alpha=0.5, label='Primary $\\beta$')
ax2.set_xlabel('Composition (wt\\% Sn)')
ax2.set_ylabel('Microconstituent Fraction (\\%)')
ax2.set_title('Microconstituent Fractions')
ax2.legend()
ax2.grid(True, alpha=0.3)

plt.tight_layout()
save_fig('phase_fractions.pdf')
\end{pycode}

\begin{figure}[H]
\centering
\includegraphics[width=\textwidth]{phase_fractions.pdf}
\caption{Phase fractions and microconstituent analysis for Pb-Sn system.}
\end{figure}

\section{Summary Table}

\begin{table}[H]
\centering
\caption{Key Phase Diagram Parameters for Pb-Sn System}
\begin{tabular}{lcc}
\toprule
Parameter & Value & Units \\
\midrule
Eutectic temperature & \py{f"{T_E}"} & $^{\circ}$C \\
Eutectic composition & \py{f"{x_E}"} & wt\% Sn \\
Max solubility in $\alpha$ & \py{f"{x_alpha_max}"} & wt\% Sn \\
Max solubility in $\beta$ & \py{f"{x_beta_max}"} & wt\% Sn \\
Melting point of Pb & \py{f"{T_Pb}"} & $^{\circ}$C \\
Melting point of Sn & \py{f"{T_Sn}"} & $^{\circ}$C \\
\bottomrule
\end{tabular}
\end{table}

\section{Conclusions}

This analysis demonstrates key aspects of binary phase diagrams:
\begin{enumerate}
    \item The Gibbs phase rule determines degrees of freedom in multi-phase systems
    \item Isomorphous systems show complete solid solubility with smooth liquidus/solidus curves
    \item The lever rule enables quantitative calculation of phase fractions
    \item Eutectic systems exhibit invariant reactions at fixed temperature and composition
    \item Cooling curves reveal thermal arrests during phase transformations
    \item Microstructure depends on both equilibrium phases and solidification path
\end{enumerate}

\end{document}
